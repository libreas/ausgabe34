\documentclass[a4paper,
fontsize=11pt,
%headings=small,
oneside,
numbers=noperiodatend,
parskip=half-,
bibliography=totoc,
final
]{scrartcl}

\usepackage{synttree}
\usepackage{graphicx}
\setkeys{Gin}{width=.4\textwidth} %default pics size

\graphicspath{{./plots/}}
\usepackage[ngerman]{babel}
\usepackage[T1]{fontenc}
%\usepackage{amsmath}
\usepackage[utf8x]{inputenc}
\usepackage [hyphens]{url}
\usepackage{booktabs} 
\usepackage[left=2.4cm,right=2.4cm,top=2.3cm,bottom=2cm,includeheadfoot]{geometry}
\usepackage{eurosym}
\usepackage{multirow}
\usepackage[ngerman]{varioref}
\setcapindent{1em}
\renewcommand{\labelitemi}{--}
\usepackage{paralist}
\usepackage{pdfpages}
\usepackage{lscape}
\usepackage{float}
\usepackage{acronym}
\usepackage{eurosym}
\usepackage[babel]{csquotes}
\usepackage{longtable,lscape}
\usepackage{mathpazo}
\usepackage[normalem]{ulem} %emphasize weiterhin kursiv
\usepackage[flushmargin,ragged]{footmisc} % left align footnote
\usepackage{ccicons} 

%%%% fancy LIBREAS URL color 
\usepackage{xcolor}
\definecolor{libreas}{RGB}{112,0,0}

\usepackage{listings}

\urlstyle{same}  % don't use monospace font for urls

\usepackage[fleqn]{amsmath}

%adjust fontsize for part

\usepackage{sectsty}
\partfont{\large}

%Das BibTeX-Zeichen mit \BibTeX setzen:
\def\symbol#1{\char #1\relax}
\def\bsl{{\tt\symbol{'134}}}
\def\BibTeX{{\rm B\kern-.05em{\sc i\kern-.025em b}\kern-.08em
    T\kern-.1667em\lower.7ex\hbox{E}\kern-.125emX}}

\usepackage{fancyhdr}
\fancyhf{}
\pagestyle{fancyplain}
\fancyhead[R]{\thepage}

% make sure bookmarks are created eventough sections are not numbered!
% uncommend if sections are numbered (bookmarks created by default)
\makeatletter
\renewcommand\@seccntformat[1]{}
\makeatother


\usepackage{hyperxmp}
\usepackage[colorlinks, linkcolor=black,citecolor=black, urlcolor=libreas,
breaklinks= true,bookmarks=true,bookmarksopen=true]{hyperref}
%URLs hart brechen
\makeatletter 
\g@addto@macro\UrlBreaks{ 
  \do\a\do\b\do\c\do\d\do\e\do\f\do\g\do\h\do\i\do\j 
  \do\k\do\l\do\m\do\n\do\o\do\p\do\q\do\r\do\s\do\t 
  \do\u\do\v\do\w\do\x\do\y\do\z\do\&\do\1\do\2\do\3 
  \do\4\do\5\do\6\do\7\do\8\do\9\do\0} 
% \def\do@url@hyp{\do\-} 
\makeatother 

%meta
%meta

\fancyhead[L]{V. Nagel\\ %author
LIBREAS. Library Ideas, 34 (2018). % journal, issue, volume.
\href{http://nbn-resolving.de/}
{}} % urn 
% recommended use
%\href{http://nbn-resolving.de/}{\color{black}{urn:nbn:de...}}
\fancyhead[R]{\thepage} %page number
\fancyfoot[L] {\ccLogo \ccAttribution\ \href{https://creativecommons.org/licenses/by/4.0/}{\color{black}Creative Commons BY 4.0}}  %licence
\fancyfoot[R] {ISSN: 1860-7950}

\title{\LARGE{Feel the Elements}}% title
\subtitle{Im Rahmen des 11. BibCamps 2018 in Hamburg wurden neue Ideen rund um das
Bibliothekswesen gesät und weiterentwickelt}
\author{Vanessa Nagel} % author

\setcounter{page}{1}

\hypersetup{%
      pdftitle={Feel the Elements. Im Rahmen des 11. BibCamps 2018 in Hamburg wurden neue Ideen rund um das
Bibliothekswesen gesät und weiterentwickelt},
      pdfauthor={Vanessa Nagel},
      pdfcopyright={CC BY 4.0 International},
      pdfsubject={LIBREAS. Library Ideas, 34 (2018).},
      pdfkeywords={Bibliothekswissenschaft, Bibcamp, Studium, Anwendung},
      pdflicenseurl={https://creativecommons.org/licenses/by/4.0/},
      pdfcontacturl={http://libreas.eu},
      baseurl={http://libreas.eu},
      pdflang={de},
      pdfmetalang={de}
     }



\date{}
\begin{document}

\maketitle
\thispagestyle{fancyplain} 

%abstracts

%body
Unter dem Motto \enquote{Feel The Elements} fand am 13. und 14. Juli
2018 das 11. BibCamp statt, ein BarCamp für Bibliothekar*innen im
deutschsprachigen Raum. Die Hochschule für Angewandte Wissenschaften in
Hamburg präsentierte nach 2011 bereits zum zweiten Mal das BibCamp am
Kunst- und Mediencampus. Organisiert wurde die Veranstaltung von
Studierenden der Studiengänge Bibliotheks- und Informationsmanagement
sowie Medien und Information im Rahmen eines Wahlpflichtseminars unter
der Leitung von Dipl.-Bibl. Nicole Filbrandt. Parallel zum BibCamp fand
als besonderes Plus der alljährliche Rundgang Finkenau statt, bei dem
Abschluss- und Studienarbeiten aller Studiengänge der Fakultät Design,
Medien und Information präsentiert wurden und von den Teilnehmer*innen
des BibCamps abends besucht werden konnten.

In 13 verschiedenen Sessions wurde über Themen wie interne Kommunikation
und vernetztes Arbeiten sowie über Team- und Communitybuilding intensiv
diskutiert. Auch Wissensorganisation, Öffentlichkeitsarbeit und das
Berufsbild im Allgemeinen sowie die Zukunft der Bibliothek wurden
thematisiert. Darüber hinaus wurde über Projektmanagement gesprochen und
Erfolgsfaktoren für Projekte aus den Erfahrungen der Teilnehmenden
zusammengetragen. Ein hitzig diskutiertes und sich durch fast alle
Sessions durchziehendes Thema war zudem die Datenschutz-Grundverordnung
(DSGVO) und ihre Auswirkungen auf die Berufspraxis.

Besonders das Thema vernetztes Arbeiten anhand der Methode
\enquote{Working Out Loud} fand unter den Teilnehmer*innen großen
Anklang und wurde spontan an beiden Veranstaltungstagen als Session
angeboten. Die Methode basiert auf einem 12-Wochen-Programm, dem
sogenannten Circle Guide, und soll in kleinen Schritten ein soziales und
interdisziplinäres Netzwerk aufbauen. Dabei hat jeder ein eigenes Ziel,
auf das er individuell hinarbeitet, erhält Unterstützung durch den
Circle und nutzt so das gesamte Potenzial der Gemeinschaft. Dank
Querverbindungen und Rückmeldungen wird die Arbeit verbessert und die
Veröffentlichung von Zwischenergebnissen schafft eine Transparenz der
Arbeit. Die Referentinnen konnten das Thema -- auch durch eine
praktische Übung -- gut vermitteln, da sie bereits eigene Circle Guides
entwickelt haben, die für den Einsatz im Studium geeignet sind. Eine
Veröffentlichung ihrer Circle Guides ist geplant; sie werden
gegebenenfalls über den Hochschulserver Hannover zur Verfügung gestellt.

Ein weiteres beliebtes Thema unter den Teilnehmenden war das
\enquote{Communitybuilding mit Hintern und Schenkeln}, eine Form des
Team- und Communitybuildings durch Stadtradeln. Teilnehmende sind dabei
nicht nur die Belegschaft, sondern auch Leser*innen der Bibliothek. Das
Marketing erfolgt hierbei durch Mund-zu-Mund-Propaganda und vor allem
die Leser*innen tragen dazu bei -- unter anderem in sozialen Netzwerken.
Die Belegschaft und die Leser*innen können so ins Gespräch kommen und
Kontakte knüpfen. Zudem wird die Bibliothek im Stadtbild und in den
Medien sichtbarer. Das Stadtradeln wird von der Strava-App begleitet, so
dass jede*r Teilnehmende immer sehen kann, wer wie viel geradelt ist. In
Krefeld wird dieses Konzept bereits erfolgreich durchgeführt.

Auch das Book Bike trägt, im Rahmen des Leseförderungskonzeptes als
klassische Außenbibliotheksarbeit, zur Öffentlichkeitsarbeit bei. Bei
dem Book Bike handelt sich um ein Lastenfahrrad, in dem neben Büchern
auch Bastel-Equipment, Stühle und ein Sonnenschirm untergebracht sind.
Das Book Bike kann für ein paar Stunden an einen Ort bestellt und dort
verwendet werden. Die Idee stammt ursprünglich aus einem Jugendzentrum
in Dortmund. Derzeit befindet sich das Book Bike noch in der Testphase,
wobei überprüft wird, ob es sich lohnt, Sponsoren zu gewinnen um die
Idee weiter auszubauen. Insgesamt beteiligen sich bereits fünf Städte in
NRW, unter anderem hat bereits die Mediothek in Krefeld das Konzept
erfolgreich angewendet.

Aufgrund der Bildproduktion für die sozialen Netzwerke entbrannte
speziell in dieser Session eine Diskussion über die DSGVO. Bei der
Bildproduktion einer solchen Veranstaltung muss stets eine
DSGVO-konforme Einverständniserklärung vorhanden sein, die die
Teilnehmer*innen unterzeichnen müssen. Dies ist zwar mit einem gewissen
Aufwand verbunden, kann allerdings auch die Glaubwürdigkeit der
Bibliothek erhöhen. Durch einen korrekten Umgang mit der
Datenschutz-Grundverordnung wird gezeigt, dass man als positives
Beispiel \enquote{durch den digitalen Wandel} führt.

Abschließend lässt sich festhalten, dass auch in diesem Jahr das BibCamp
wieder für einen gelungenen Erfahrungsaustausch innerhalb der Community
sorgte. Viele Diskussionen und Gespräche rund um das Bibliothekswesen
sowie die Begegnung der Teilnehmer*innen auf Augenhöhe schufen eine
angenehme Atmosphäre. Dazu hat sicherlich auch die rege Beteiligung auf
Twitter und anderen Social Media-Kanälen beigetragen. Neu in diesem
Jahr: Vereinzelte Sessions wurden erstmalig mit einem Livestream über
den aktuell eingerichteten Instagram-Account begleitet. Wer jetzt Lust
bekommen hat: Das nächste BibCamp findet am 15. und 16. November 2019 an
der Technischen Hochschule in Köln statt.

%autor
\begin{center}\rule{0.5\linewidth}{\linethickness}\end{center}

\textbf{Vanessa Nagel}, geboren 1989 in Hamburg, studiert Bibliotheks-
und Informationsmanagement (B.A.) im fünften Semester am Department
Information der HAW Hamburg. Ihre thematischen Schwerpunkte liegen im
Bereich Öffentliche Bibliotheken, Social Media und IT.

\end{document}
