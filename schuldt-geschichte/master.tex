\documentclass[a4paper,
fontsize=11pt,
%headings=small,
oneside,
numbers=noperiodatend,
parskip=half-,
bibliography=totoc,
final
]{scrartcl}

\usepackage{synttree}
\usepackage{graphicx}
\setkeys{Gin}{width=.4\textwidth} %default pics size

\graphicspath{{./plots/}}
\usepackage[ngerman]{babel}
\usepackage[T1]{fontenc}
%\usepackage{amsmath}
\usepackage[utf8x]{inputenc}
\usepackage [hyphens]{url}
\usepackage{booktabs} 
\usepackage[left=2.4cm,right=2.4cm,top=2.3cm,bottom=2cm,includeheadfoot]{geometry}
\usepackage{eurosym}
\usepackage{multirow}
\usepackage[ngerman]{varioref}
\setcapindent{1em}
\renewcommand{\labelitemi}{--}
\usepackage{paralist}
\usepackage{pdfpages}
\usepackage{lscape}
\usepackage{float}
\usepackage{acronym}
\usepackage{eurosym}
\usepackage[babel]{csquotes}
\usepackage{longtable,lscape}
\usepackage{mathpazo}
\usepackage[normalem]{ulem} %emphasize weiterhin kursiv
\usepackage[flushmargin,ragged]{footmisc} % left align footnote
\usepackage{ccicons} 

%%%% fancy LIBREAS URL color 
\usepackage{xcolor}
\definecolor{libreas}{RGB}{112,0,0}

\usepackage{listings}

\urlstyle{same}  % don't use monospace font for urls

\usepackage[fleqn]{amsmath}

%adjust fontsize for part

\usepackage{sectsty}
\partfont{\large}

%Das BibTeX-Zeichen mit \BibTeX setzen:
\def\symbol#1{\char #1\relax}
\def\bsl{{\tt\symbol{'134}}}
\def\BibTeX{{\rm B\kern-.05em{\sc i\kern-.025em b}\kern-.08em
    T\kern-.1667em\lower.7ex\hbox{E}\kern-.125emX}}

\usepackage{fancyhdr}
\fancyhf{}
\pagestyle{fancyplain}
\fancyhead[R]{\thepage}

% make sure bookmarks are created eventough sections are not numbered!
% uncommend if sections are numbered (bookmarks created by default)
\makeatletter
\renewcommand\@seccntformat[1]{}
\makeatother


\usepackage{hyperxmp}
\usepackage[colorlinks, linkcolor=black,citecolor=black, urlcolor=libreas,
breaklinks= true,bookmarks=true,bookmarksopen=true]{hyperref}
\usepackage{breakurl}

%meta
%meta

\fancyhead[L]{K. Schuldt, B. Kaden\\ %author
LIBREAS. Library Ideas, 34 (2018). % journal, issue, volume.
\href{http://nbn-resolving.de/}
{}} % urn 
% recommended use
%\href{http://nbn-resolving.de/}{\color{black}{urn:nbn:de...}}
\fancyhead[R]{\thepage} %page number
\fancyfoot[L] {\ccLogo \ccAttribution\ \href{https://creativecommons.org/licenses/by/4.0/}{\color{black}Creative Commons BY 4.0}}  %licence
\fancyfoot[R] {ISSN: 1860-7950}

\title{\LARGE{Anmerkungen zur Geschichte der Frage: „Was ist Bibliothekswissenschaft“}}% title
\subtitle{Einführung in die historische Abteilung}
\author{Karsten Schuldt, Ben Kaden} % author

\setcounter{page}{1}

\hypersetup{%
      pdftitle={Anmerkungen zur Geschichte der Frage: „Was ist Bibliothekswissenschaft“ : Einführung in die historische Abteilung},
      pdfauthor={Karsten Schuldt, Ben Kaden},
      pdfcopyright={CC BY 4.0 International},
      pdfsubject={LIBREAS. Library Ideas, 34 (2018).},
      pdfkeywords={Bibliothekswissenschaft, Geschichte, Institut für Bibliothekswissenschaft, Humboldt-Universität zu Berlin},
      pdflicenseurl={https://creativecommons.org/licenses/by/4.0/},
      pdfcontacturl={http://libreas.eu},
      baseurl={http://libreas.eu},
      pdflang={de},
      pdfmetalang={de}
     }



\date{}
\begin{document}

\maketitle
\thispagestyle{fancyplain} 

%abstracts

%body
Als 1928 das Bibliothekswissenschaftliche Institut an der Berliner
Universität eingerichtet und dafür der Lehrstuhl für
Bibliothekhilfswissenschaften von der Universität Göttingen verlegt
wurde, geschah dies nicht ohne Kontext. Vielmehr gab es in den 1920er
und 1930er Jahren im deutschsprachigen Raum eine relativ intensive
Debatte darüber, (a) warum eine Ausbildung des Personals zumindest für
die Wissenschaftlichen Bibliotheken an Universitäten erfolgen und (b)
was der Inhalt einer Bibliothekswissenschaft sein sollte. Preussen ging
mit der Einrichtung des Instituts in Berlin voran -- genauso wie zuvor
mit der Einrichtung der ersten ordentlichen Professur für
Bibliothekswissenschaft in Göttingen, die Karl Dziatzko bekleidete. In
der Schweiz folgte die Universität Bern 1935 diesem Vorbild. Die darum
kreisende Debatte scheint heute vergessen zu sein, vermutlich jedoch
nicht, weil die damals gestellten Fragen geklärt wären.

Wir republizieren hier drei Beiträge dieser Debatte.\footnote{Gerne
  erwähnen wir, dass wir dazu auf Transkribus zurückgegriffen und sehr
  von der Arbeit, die hinter diesem Projekt steht, profitiert haben.
  (\url{https://transkribus.eu/Transkribus/})} Der Beitrag George
Leidingers von 1928 postuliert eine Bibliothekswissenschaft, die
explizit auf die Arbeit in Wissenschaftlichen Bibliotheken ausgerichtet
sein und die vier Abteilungen (1) Buchkunde, (2) Literaturkunde, (3) die
Lehre vom Bibliothekswesen der Vergangenheit (Geschichte des
Bibliothekswesens) und (4) die Lehre vom Bibliothekswesen der Gegenwart
umfassen soll. Sichtbar wird in diesem Beitrag, dass das, was später --
sowie auch heute noch -- als Bibliotheksmanagement den Hauptteil der
Lehre im Bibliothekswesen umfasst, zur Gründungszeit des Instituts in
Berlin stark in Geschichte und Buchkunde verankert war. Bibliotheken zu
leiten und zu entwickeln sei nur möglich, wenn auch Ursprung und
Entwicklung der Bibliotheken, der Literatur und des Buches und zwar von
den griechischen Wurzelen beginnend, bekannt seien. Diese Position
vertritt auch Fritz Milkau, dessen Einführung in sein einflussreiches
\enquote{Handbuch der Bibliothekswissenschaft, Band 1} (von insgesamt
drei Bänden) wir neu publizieren. Was Milkau ergänzt, ist eine
Geschichte der Diskussion zur Frage, was Inhalt der
Bibliothekswissenschaft sein soll. Er bemerkt, dass auch der Begriff
\enquote{Wissenschaft} eine Geschichte habe, in deren Kern die
Bedeutungsverschiebung von \enquote{alles Wissen zu einem Thema} hin zu
\enquote{systematisches Wissen zu einem Thema} steht. Diese Bemerkung
führt uns zurück zur Kardinalfrage, welche Bedeutung
\enquote{Wissenschaft} in Bibliothekswissenschaft in der Gegenwart
besitzt. Sind wir in ihr wieder zurückgekehrt zu einem sehr breiten,
post-systematischen Verständnis von Wissenschaft? Wenn ja, ist das gut?

Die Antrittsvorlesung von Hans Lutz in Bern (der leider kurz darauf
verstarb) fasst die zeitgenössischen Diskussionen sehr gut zusammen.
Zugleich war die Vorlesung auch als ein Programm seiner zukünftigen
Professur gedacht. Die Publikation des Textes in der damals einzigen
bibliothekarischen Zeitschrift der Schweiz sollte diese Systematik einer
Bibliothekswissenschaft zudem dem Bibliothekswesen des Landes
vorstellen. Auch Lutz betont eine Verankerung der
Bibliothekswissenschaft in der Bibliotheks- und Literaturgeschichte,
führt dies aber weiter zu konkreten Fragen wie dem Bibliotheksbau, der
Katalogisierung und dem Benutzungsdienst.

Sichtbar wird in den drei Texten, wie man zu der Zeit, als das Institut
in Berlin eingerichtet wurde, davon ausging, dass gehobenes
bibliothekarisches Personal ein umfassendes Wissen über
Bibliotheksmanagement (wenn auch nicht unter diesem Namen), Bibliotheks-
und Buchgeschichte und Literatur haben müsste und dass es Aufgabe einer
Bibliothekswissenschaft sei, dieses Wissen systematisch zu erarbeiten
und zu vermitteln und zwar immer aus dem konkreten Blickwinkel der
Bibliothek. Dies ermöglicht auch, wie Leidinger ausführt, eine
konkurrenzlose Koexistenz mit anderen mehr oder weniger
Medienwissenschaften wie eben der Buchkunde oder der aufblühenden
Zeitungswissenschaft.

Die Diskussion um die Frage, was Bibliothekswissenschaft ist und sein
soll, scheint heute, 90 Jahre später, quasi nicht mehr stattzufinden.
Sie brach aber nicht einfach ab oder wurde durch erfolgreiche
Verankerung des Fachs im Wissenschaftsgefüge aufgehoben. Die Situation
der Bibliothekswissenschaft blieb fast durchgängig prekär. Die
Diskussion um ihren Sinn und Inhalt rückte gerade deshalb schubweise
immer wieder neu und mit wenig nachhaltiger Klärung in den Vordergrund,
zuletzt auch in Beiträgen der am IBI entstandenen Publikationen
\emph{Bibliothekswissenschaft -- quo vadis?} (herausgegeben von Petra
Hauke, München, 2005) sowie \emph{Vom Wandel der Wissensorganisation im
Informationszeitalter} (herausgegeben von Petra Hauke und Konrad Umlauf,
Bad Honnef, 2006). Auch zuvor zeigt der Blick in Publikationen wie das
\enquote{Zentralblatt für Bibliothekswesen}, die \enquote{Zeitschrift
für Bibliothekswesen und Bibliographie} und \enquote{Bibliothek.
Forschung und Praxis}, dass die Frage bis in die frühen 1990er Jahre
immer wieder aufgegriffen wurde. Ähnlich wie auch später die teils unter
anderem am IBI mit harten Bandagen gefochtenen Diskussionen zur
richtigen Definition des Verhältnisses von \enquote{Information} und
\enquote{Wissen}, führten all diese Diskurse und Diskursversuche nie zu
einer verbindlichen und anerkannten Selbstverortung und auch eine für
Wissenschaften lange typische Bildung von Schulen ließ sich kaum
ableiten. Wo sie geschah, geschah dies eher zufällig anhand konkreter
Forschungs- und Arbeitsschwerpunkte. Leidingers Ansicht \enquote{die
Pflege der Bibliothekswissenschaft ist in erster Linie eine
Persönlichkeitsfrage} erwies sich als zutreffend. Aber nicht immer in
der von ihm intendierten Weise.

Die Ursache für die schwankende Stellung der Bibliothekswissenschaft
liegt vermutlich erstens im Fach selbst, das mit dem Abschluss den
Großteil der Absolventinnen und Absolventen in Praxisfelder und
weitgehend aus der Wissenschaft selbst entlässt. Und zweitens an der
geringen akademischen Integration sowohl was die Ressourcen als auch die
Verortung im Kontext eines übergeordneten Wissenschaftsgefüges betrifft.
Auch wenn das Berliner Institut einer Philosophischen Fakultät
zugeordnet ist, herrscht keinesfalls Konsens, ob die Disziplin überhaupt
zu den Geisteswissenschaften gezählt werden sollte. Die informatische
Wende und die lange ausgeprägte Fokussierung auf das
Bibliotheksmanagement haben zudem geisteswissenschaftliche und
insbesondere historische Aspekte weitgehend aus den Curricula gedrängt.
Dass das Fach lange Zeit auch stark von nicht genuin
bibliothekswissenschaftlich ausgebildeten Quereinsteigern geprägt wurde,
die oft entweder die Perspektive ihrer Herkunftsdisziplinen forcierten
oder aber gerade darauf zu achten schienen, sie nicht zu integrieren --
was ebenso zu blinden Flecken führte --, dürfte einer Konsolidierung des
Selbstverständnisses dieses lange Zeit auch noch durch die Heterogenität
von Bibliothek und Dokumentation zusätzlich diversifizierten Faches
ebenfalls im Wege gestanden haben. Entsprechend verwundert es nicht,
wenn die Frage, was eine Bibliothekswissenschaft ist beziehungsweise
sein kann oder sein sollte, bis heute nur jeweils aus einer subjektiven
Warte, aber nicht als allgemeiner Konsens beantwortet werden kann. Wir
hoffen, mit der Republizierung der Texte aus der Wissenschaftsgeschichte
des Fachs zu zeigen, dass eine solche Diskussion möglich ist und dass
sie weiterhin (wieder) sinnvoll wäre.

%autor
\begin{center}\rule{0.5\linewidth}{\linethickness}\end{center}

\textbf{Ben Kaden} ist Bibliothekswissenschaftler und arbeitet an der
Universitätsbibliothek der Humboldt-Universität zu Berlin.

\textbf{Karsten Schuldt} (Chur / Berlin) ist Wissenschaftlicher
Mitarbeiter am Schweizerischen Institut für Informationswissenschaft,
HTW Chur.

\end{document}
