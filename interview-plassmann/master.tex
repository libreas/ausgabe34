\documentclass[a4paper,
fontsize=11pt,
%headings=small,
oneside,
numbers=noperiodatend,
parskip=half-,
bibliography=totoc,
final
]{scrartcl}

\usepackage{synttree}
\usepackage{graphicx}
\setkeys{Gin}{width=.4\textwidth} %default pics size

\graphicspath{{./plots/}}
\usepackage[ngerman]{babel}
\usepackage[T1]{fontenc}
%\usepackage{amsmath}
\usepackage[utf8x]{inputenc}
\usepackage [hyphens]{url}
\usepackage{booktabs} 
\usepackage[left=2.4cm,right=2.4cm,top=2.3cm,bottom=2cm,includeheadfoot]{geometry}
\usepackage{eurosym}
\usepackage{multirow}
\usepackage[ngerman]{varioref}
\setcapindent{1em}
\renewcommand{\labelitemi}{--}
\usepackage{paralist}
\usepackage{pdfpages}
\usepackage{lscape}
\usepackage{float}
\usepackage{acronym}
\usepackage{eurosym}
\usepackage[babel]{csquotes}
\usepackage{longtable,lscape}
\usepackage{mathpazo}
\usepackage[normalem]{ulem} %emphasize weiterhin kursiv
\usepackage[flushmargin,ragged]{footmisc} % left align footnote
\usepackage{ccicons} 

%%%% fancy LIBREAS URL color 
\usepackage{xcolor}
\definecolor{libreas}{RGB}{112,0,0}

\usepackage{listings}

\urlstyle{same}  % don't use monospace font for urls

\usepackage[fleqn]{amsmath}

%adjust fontsize for part

\usepackage{sectsty}
\partfont{\large}

%Das BibTeX-Zeichen mit \BibTeX setzen:
\def\symbol#1{\char #1\relax}
\def\bsl{{\tt\symbol{'134}}}
\def\BibTeX{{\rm B\kern-.05em{\sc i\kern-.025em b}\kern-.08em
    T\kern-.1667em\lower.7ex\hbox{E}\kern-.125emX}}

\usepackage{fancyhdr}
\fancyhf{}
\pagestyle{fancyplain}
\fancyhead[R]{\thepage}

% make sure bookmarks are created eventough sections are not numbered!
% uncommend if sections are numbered (bookmarks created by default)
\makeatletter
\renewcommand\@seccntformat[1]{}
\makeatother


\usepackage{hyperxmp}
\usepackage[colorlinks, linkcolor=black,citecolor=black, urlcolor=libreas,
breaklinks= true,bookmarks=true,bookmarksopen=true]{hyperref}
%URLs hart brechen
\makeatletter 
\g@addto@macro\UrlBreaks{ 
  \do\a\do\b\do\c\do\d\do\e\do\f\do\g\do\h\do\i\do\j 
  \do\k\do\l\do\m\do\n\do\o\do\p\do\q\do\r\do\s\do\t 
  \do\u\do\v\do\w\do\x\do\y\do\z\do\&\do\1\do\2\do\3 
  \do\4\do\5\do\6\do\7\do\8\do\9\do\0} 
% \def\do@url@hyp{\do\-} 
\makeatother 

%meta
%meta

\fancyhead[L]{K. Schlebbe\\ %author
LIBREAS. Library Ideas, 34 (2018). % journal, issue, volume.
\href{http://nbn-resolving.de/}
{}} % urn 
% recommended use
%\href{http://nbn-resolving.de/}{\color{black}{urn:nbn:de...}}
\fancyhead[R]{\thepage} %page number
\fancyfoot[L] {\ccLogo \ccAttribution\ \href{https://creativecommons.org/licenses/by/4.0/}{\color{black}Creative Commons BY 4.0}}  %licence
\fancyfoot[R] {ISSN: 1860-7950}

\title{\LARGE{Interview mit Engelbert Plassmann}}% title
\author{Kirsten Schlebbe (Interviewerin)} % author

\setcounter{page}{1}

\hypersetup{%
      pdftitle={Interview mit Engelbert Plassmann},
      pdfauthor={Kirsten Schlebbe (Interviewerin)},
      pdfcopyright={CC BY 4.0 International},
      pdfsubject={LIBREAS. Library Ideas, 34 (2018).},
      pdfkeywords={Bibliothekswissenschaft, Institut für Bibliotheks- und Informationswissenschaft, Humboldt-Universität zu Berlin, Projektseminar, Buchprojekt, Studium},
      pdflicenseurl={https://creativecommons.org/licenses/by/4.0/},
      pdfcontacturl={http://libreas.eu},
      baseurl={http://libreas.eu},
      pdflang={de},
      pdfmetalang={de}
     }



\date{}
\begin{document}

\maketitle
\thispagestyle{fancyplain} 

%abstracts

%body
\textbf{KS: Vielen Dank, Herr Prof.~Plassmann, dass Sie sich die Zeit
für dieses Gespräch genommen haben. Sie haben zunächst Philosophie und
katholische Theologie studiert, anschließend ein Studium der
Rechtswissenschaft abgeschlossen und später auch bei Paul Mikat an der
Rechtswissenschaftlichen Fakultät promoviert. Im Anschluss haben Sie die
Referendarausbildung für den Höheren Bibliotheksdienst absolviert. Wie
kam es denn zu dem Wechsel in den bibliothekarischen Bereich?}

\textbf{EP}: Das will ich gerne schildern. Nach dem Ende des Studiums
der Theologie, das mir Freude gemacht hat und von dem ich bis heute
zehre, konnte ich mich nicht entschließen, in den kirchlichen Dienst zu
treten. Ich bin immer gerne Mitglied der christlichen
Glaubensgemeinschaft gewesen, aber beruflich in den kirchlichen Dienst
gehen, das wollte ich nicht. Also habe ich dann mit 25 Jahren ganz von
vorne mit Rechtswissenschaft angefangen, aus der Überlegung heraus, dass
dieses Studium mir später viele Möglichkeiten bietet und die Überlegung
sollte sich nicht als falsch erweisen.

Nach der ersten juristischen Staatsprüfung in Würzburg habe ich dann an
meiner Promotion aus dem Gebiet der Rechtsgeschichte gearbeitet. Und
schon in dieser Promotion, das sieht man am Titel der Arbeit, habe ich
meine beiden Studienfächer miteinander verbunden. Es war eine Arbeit zur
Geschichte des Staatskirchenrechts, in Deutschland immer ein
interessantes Thema. Die Promotion habe ich 1967 in Bochum an der damals
neu gegründeten Universität abgeschlossen, 1968 ist die Dissertation als
Verlagspublikation erschienen. Als ich das hinter mir hatte, wollte ich
nicht auf einen typischen Justizberuf zugehen, also Richter,
Staatsanwalt, Rechtsanwalt oder dergleichen werden. Eher schon
Verwaltungsbeamter. Und just zu diesem Zeitpunkt las ich in einer
juristischen Fachzeitschrift einen Artikel mit der Überschrift
\enquote{Die seltenen Berufe für Juristen}. Bei diesen Berufen waren
genannt: Der höhere Dienst an wissenschaftlichen Bibliotheken, der
höhere Archivdienst, der höhere Bankdienst in der Deutschen Bundesbank,
der höhere auswärtige Dienst sowie der Steuerberater und der
Wirtschaftsprüfer.

Einige kamen für mich nicht in Frage, da wäre eine vorherige Banklehre
nötig gewesen. Aber der höhere Dienst an wissenschaftlichen Bibliotheken
sprach mich ungemein an. Damals war für die Zulassung zum
Referendardienst noch die Promotion in einem wissenschaftlichen Fach
erforderlich, gleichgültig in welchem. Das habe ich mir nicht zweimal
durchgelesen, sondern mir gesagt: Wofür hast du dir schließlich die Mühe
gemacht? Ich besuchte Günther Pflug, den ersten Leiter,
Gründungsdirektor der UB Bochum, später -- bis zu seiner Pensionierung
1988 -- Generaldirektor der Deutschen Bibliothek in Frankfurt am Main
und einer der bekanntesten Bibliothekare des 20. Jahrhunderts in
Deutschland. Wir verstanden uns sofort und haben uns eine halbe Stunde
lang gut unterhalten. Danach war ich umgehend Wissenschaftliche
Hilfskraft bei ihm und kurze Zeit später Bibliotheksreferendar, mit
einer zweijährigen Ausbildung. Nach der Prüfung, die eine Zweite
Staatsprüfung war, so wie bei den Juristen oder Gymnasiallehrern, habe
ich in Bochum mit der praktischen Arbeit in der Bibliothek angefangen.
So bin ich zum Bibliothekswesen gekommen. Es hat also vor allem mit dem
genannten Aufsatz zu tun, der mir durch eine Fügung des Himmels genau im
richtigen Moment beschert worden ist.

\textbf{KS: Sie haben vorhin erwähnt, dass Sie bereits bei Ihrer
Promotion Ihre Studienfächer miteinander verbunden haben. Haben Ihre
Studienabschlüsse in Philosophie, Theologie und Rechtswissenschaft auch
Ihre Forschung und Lehre in der Bibliothekswissenschaft beeinflusst?}

\textbf{EP:} Ja, und zwar in hohem Maße. Wenn ich zuerst auf die
Forschung zu sprechen komme, braucht man sich eigentlich nur meine
persönliche Bibliographie anzusehen, da wird man eine ganze Menge von
Beiträgen finden, welche deutlich belegen, dass ich meine früheren
Studiengänge in der Forschung weitergeführt habe. Ich möchte
ausdrücklich darauf zu sprechen kommen, dass ich meine Antrittsvorlesung
an der Humboldt-Universität zu dem Thema \enquote{Bibliotheksgeschichte
und Verfassungsgeschichte} gehalten habe. Ich bin ein Freund solcher
interdisziplinären Verbindungen. Im Jahr darauf habe ich noch einmal
eine öffentliche Vorlesung im Senatssaal der Universität gehalten, mit
dem Thema \enquote{Eine Reichsbibliothek?}. Damals wurde viel über die
Aufgaben der Deutschen Bibliothek, heute Deutsche Nationalbibliothek,
diskutiert. Ob diese nicht erweitert werden sollten. Die DNB sammelt ja
nur die deutsche Literatur oder auf Deutschland bezogene Literatur.
Daher war die Frage: Sollte man es nicht machen, wie es in anderen
Ländern ist, in Frankreich mit der \emph{Bibliothèque Nationale} und in
vielen anderen Ländern auch? Zu diesem Thema habe ich damals Stellung
genommen. Dazu vielleicht einen Satz: Ich bin der Meinung, dass das bei
uns in Deutschland, so wie es sich historisch entwickelt hat, ganz gut
läuft mit der Deutschen Nationalbibliothek einerseits -- die haben wir
jetzt sogar an zwei Standorten, das hängt mit der neuesten deutschen
Geschichte im 20. Jahrhundert zusammen -- aber dazu eben die beiden
großen wissenschaftlichen Allgemeinbibliotheken: die Bayerische
Staatsbibliothek und hier unsere Staatsbibliothek zu Berlin.

Auch meine Beiträge im \emph{Lexikon des gesamten Buchwesens}
beantworten Ihre Frage. Dort steht eine ganze Portion von Artikeln, die
mit Patentwesen, Loseblattsammlungen und dergleichen zu tun haben. Da
ist die Rechtswissenschaft erkennbar. Auch etliche Rezensionen von
Büchern, die zu juristischen Bibliographien geschrieben worden sind.

Das gilt in entsprechendem Maße auch für mein anderes Studienfach, die
Theologie. Die zahlreichen Exkursionen, die ich mit Studenten gemacht
habe, haben sich immer auf Bibliotheken aller historischen Zeiten
bezogen, einschließlich der Besuche in bedeutenden Bibliotheken der
Gegenwart. Auch das 18. Jahrhundert hatten wir viel. Auf dem Programm
standen zahlreiche Klosterbibliotheken, die 1803 der Säkularisation zum
Opfer gefallen sind. Auch darüber habe ich einen öffentlichen Vortrag
hier gehalten, der in der blauen Schriftenreihe veröffentlicht worden
ist. Über die Exkursionen habe ich auch in den Fachzeitschriften eine
ganze Reihe von Berichten veröffentlicht. Und die wären ohne den
Hintergrund des Theologiestudiums nicht so geschrieben worden und ich
hätte es auch den Studenten nicht in dem Maße nahebringen können, wie
mir das nun möglich war. Soviel zu der Frage, ob meine Studienfächer die
Forschung beeinflusst haben.

Die Lehre haben sie natürlich auch beeinflusst. Wie ich glaube,
ebenfalls in hohem Maße. Und zwar lässt sich das in den alten
Vorlesungsverzeichnissen, in denen ich geblättert habe, gut
nachvollziehen. So habe ich regelmäßig, seit dem Beginn meiner
Lehrtätigkeit in Köln 1976, das Fach \enquote{Rechtskunde für
Bibliothekare} unterrichtet, ein nicht unwichtiges Fach. Ein
Bibliothekar muss wissen: Was ist im juristischen Sinn der Träger einer
Bibliothek? Eine Gemeinde, der Staat (das Land, gegebenenfalls der
Bund)? Eine Stiftung, eine kirchliche Einrichtung, eine freie
Einrichtung und so weiter? Welche rechtlichen Befugnisse gibt es? Welche
Anstellungsverhältnisse? Dann rechtliche Dinge wie die Ablieferung von
Dissertationen oder Pflichtablieferungen an die Nationalbibliothek und
an regionale Bibliotheken. Das Gebiet habe ich immer vertreten,
eigentlich jedes Semester. Da hat sich mein früheres Jurastudium mit
aller Deutlichkeit bemerkbar gemacht. Ich glaube im Übrigen, dass die
Studenten es auch in anderen Vorlesungen bemerken konnten. Zum Beispiel
in der häufig gehaltenen Vorlesung \enquote{Einführung in das
Bibliothekswesen der Bundesrepublik Deutschland} (eigentlich das, was in
meinen drei \enquote{dicken Büchern} aufbereitet ist), da habe ich immer
Wert auf die rechtlichen Zusammenhänge gelegt. Ich kenne eine ganze
Reihe von Bibliothekaren, jedenfalls in meiner Generation, in der
älteren, aber auch in der jüngeren Generation, die die Bibliothek zu
sehr aus einer Innensicht betrachten und darüber oft versäumen, über den
Träger und die rechtlichen Rahmenbedingungen nachzudenken, denen die
Bibliothek unterliegt. Die Bibliothek kommt aus den Beamten- und
Angestelltenverhältnissen, den rechtlichen, ja nicht raus. Und sie kommt
auch aus den Regeln für den Kauf von Büchern nicht raus und so weiter
und so fort.

Was die Theologie angeht, so hat natürlich auch sie ihren Niederschlag
in der Lehre gefunden, soweit es die Bibliotheksgeschichte angeht. Die
überragende Bedeutung der Kirche für die Entwicklung unserer ganzen
Schriftkultur, jedenfalls im romanisch-germanischen Bereich Europas: Was
haben wir den schreibenden Mönchen des Mittelalters zu verdanken, die
uns keineswegs nur die Heilige Schrift, sondern auch wichtige Teile der
weltlichen Literatur, die Schriften der griechischen und römischen
Dichter, Philosophen und Rechtsgelehrten überliefert haben! Unsere ganze
Kultur ist ja nicht zu denken ohne das. Und dann die Bedeutung der
kirchlichen Einrichtungen bis ins 18. Jahrhundert einschließlich, bis
die Säkularisation einen tiefen Einschnitt gemacht hat. Aber immerhin
nur so, dass heute bestimmte Bibliotheken, vor allem die
Staatsbibliothek in München, einen Schatz an älterer Literatur hütet,
der genau auf diese Tradition zurückgeht. Das sind Dinge, die natürlich
auch in meine Vorlesungen eingeflossen sind. Soviel zu dieser Frage. Ich
habe mich durchaus um die geistigen Verknüpfungen bemüht, weil ich es
immer als sinnvoll angesehen und mir gesagt habe: Wenn du schon diese
Voraussetzungen und Kenntnisse hast, dann gib sie auch weiter.

\textbf{KS: Wenn wir nun auf Ihre berufliche Entwicklung zu sprechen
kommen: Nach dem Abschluss der Referendarausbildung haben Sie einige
Jahre in der Bibliothekspraxis gearbeitet, auch auf leitenden
Positionen. Wie hat sich das entwickelt und wie kam es schließlich zum
Wechsel in den Hochschulbereich?}

\textbf{EP:} Ja, das ging ziemlich schnell. Nun war ich ja schon relativ
alt geworden \emph{{[}lacht{]},} ich war schon Mitte 30, als ich dann
schließlich anfing nach all diesen Vorentscheidungen und Vorbereitungen.
Ich war einige Jahre in Bochum als Referent der Abteilung
\enquote{Erwerbung und Koordinierung}, also letztlich für die
Anschaffung der Bücher zuständig. Die Fachreferenten machten die
Vorschläge, ich war Leiter der Stelle, die für den Kontakt zum
Buchhandel zuständig war. Ich hatte mehrere Mitarbeiterinnen,
Diplom-Bibliothekarinnen in der Vorakzession, die die üblichen Prüfungen
machten und mir das dann vorlegten. Ferner in der Akzession, der
Abteilung, die die eingehenden Bücher bearbeitete, und -- ganz wichtig
-- in der Zeitschriftenstelle, wo mit größter Sorgfalt die zwölfmal,
24-mal oder sechsmal im Jahr kommenden Neueingänge der einzelnen Hefte
zu verzeichnen waren -- das alles war mein Zuständigkeitsbereich, den
ich mit Vergnügen wahrgenommen habe. Der Kontakt mit den Buchhändlern
und Verlegern hat Spaß gemacht, auch bin ich regelmäßig zur Buchmesse
gefahren. Es war leider nur eine relativ kurze Zeit.

Zum Teil gleichzeitig, weil es keine Bibliothekare gab -- zwar Stellen
(!), aber keine Bibliothekare -- hatte ich den Aufbau der
Fachhochschulbibliothek Bochum zu leisten. 1971 wurde in den
westdeutschen Ländern generell die Fachhochschule als neuer
Hochschultypus eingeführt und aus verschiedenen Vorgängereinrichtungen
heraus gebildet-- Ingenieurschulen, höheren Fachschulen für Wirtschaft,
Sozialarbeit und so weiter. Die neuen Fachhochschulen wurden, jedenfalls
in Nordrhein-Westfalen, wo ich es erlebt habe, gut mit Mitteln zur
Ersteinrichtung versehen. Damals war im Wissenschaftsministerium in
Düsseldorf ein Referent für das Bibliothekswesen zuständig, der auch in
Berlin gut bekannt ist: Dr.~Antonius Jammers, später Generaldirektor der
Staatsbibliothek. Als Referent für das Bibliotheks- und
Dokumentationswesen im Ministerium hat Jammers es mit viel Geschick
erreicht, dass auch die Bibliotheken der neuen Fachhochschulen wirklich
gut ausgestattet wurden. Soweit ich mich dank der Kontakte mit Kollegen
in anderen Bundesländern erinnere, beneideten die uns darum, wie wir
unsere Fachhochschulbibliotheken aufbauen konnten: mit Mitteln, die
ihnen geradezu opulent erschienen. Kurz und gut, ich hatte zeitweise
neben meiner Arbeit an der UB die FHB Bochum aufzubauen. Mit dem Rektor
und dem Kanzler, die damals ganz neu an der Fachhochschule waren und
sich auch erst einfinden mussten, hatte ich glänzenden Kontakt. Beide
unterstützten mich, merkten auch, dass ich Rückenwind aus dem
Ministerium hatte und nutzten das.

Soweit meine Tätigkeiten in der bibliothekarischen Praxis im engeren
Sinne.

Ein weiterer wichtiger Abschnitt gehört noch dazu. 1973 wurde ich ins
Wissenschaftsministerium berufen. Jammers brauchte damals dringend
Unterstützung; es war ja die Zeit, in der die vielen neuen Hochschulen
gegründet wurden. In Nordrhein-Westfalen waren das 1972 die
Gesamthochschulen Duisburg, Essen, Paderborn, Siegen und Wuppertal, fünf
Hochschulen, die sich rasch entwickelten. Schon 1971 waren, wie erwähnt,
die Fachhochschulen gegründet worden, in Nordrhein-Westfalen die
stattliche Anzahl von zehn derartigen Einrichtungen.

An dieser Stelle kam mir mein vorgängiges Jurastudium zustatten. Als
Jurist arbeitet man sich in einem Ministerium wesentlich schneller ein
als jemand, der diese Voraussetzung nicht mitbringt. Ich hatte schon
erwähnt, dass ich für mich keinen typischen Justizberuf wollte, höherer
Verwaltungsbeamter wollte ich aber durchaus werden. Und so einer war ich
nun, und zwar für drei Jahre. Ich war gerne im Ministerium. Man konnte
vieles auf den Weg bringen, was man von einer einzelnen Bibliothek aus
überhaupt nicht kann.

Das fand nach drei Jahren ein Ende. Ich wollte wieder in die
bibliothekarische Arbeit im engeren Sinne zurück. Da gab es einen klaren
Schnitt, den das Ministerium, ohne es zu wollen, mir nahelegte. Als
nämlich drei Jahre vorbei waren und die Mehrarbeit im Bibliotheksreferat
erledigt war, sagte sich der Staatssekretär wohl: Ach, jetzt ziehen wir
den da ab, wir brauchen dringend an einer anderen Stelle Verstärkung.
Und was war diese Stelle? Das Referat für die Vergabe von
Studienplätzen! Es gab in Dortmund die Zentralstelle zur Vergabe von
Studienplätzen (ZVS), die es heute noch gibt. Und das dafür zuständige
Referat im Ministerium brauchte, weil der Ansturm auf die Hochschulen
begonnen hatte, Verstärkung. Auf eine Arbeit in diesem Referat hatte
ich, um das ganz banal auszudrücken, null Bock. Aber wirklich null. Mich
mit Widerspruchsbescheiden gegen Entscheidungen der Dortmunder Stelle zu
beschäftigen, warum \emph{der} Mediziner nicht drangekommen ist, aber
ein anderer... also nein. Doch konnte ich nicht einfach NEIN sagen. Ich
war ja Beamter des Ministeriums und der Staatssekretär hat natürlich ein
Direktionsrecht. Aber welch eine Fügung des Himmels kam mir zu Hilfe!

In Köln war im Studiengang Bibliothekswesen, dem einzigen in
Nordrhein-Westfalen, eine Dozentenstelle zu besetzen. Und ein Kollege,
dessen Name auch in Berlin, auch am IBI gut bekannt ist, Paul Kaegbein,
war damals gerade neuer Leiter in Köln geworden, wollte mich haben und
machte das im Ministerium deutlich. Nebenbei bemerkt: Es handelte sich
zu diesem Zeitpunkt noch nicht um die spätere Fachhochschule für
Bibliotheks- und Dokumentationswesen, sondern um deren
Vorgängereinrichtung, das Bibliothekar-Lehrinstitut des Landes
Nordrhein-Westfalen. Kaegbein kannte mich aus der
\enquote{Planungsgruppe Bibliothekswesen im Hochschulbereich
Nordrhein-Westfalen} beim Ministerium und war auf die Idee gekommen, ich
könnte in Köln passen. Kurz und gut, ich wurde nach Köln berufen und
damit war das Schreckgespenst \enquote{Zentralstelle zur Vergabe von
Studienplätzen} verschwunden \emph{{[}lacht{]}.}

So ist es gekommen, dass ich im Hochschulbereich gelandet bin. Die
Arbeit in der Lehre machte mir bald, das heißt nach einer gewissen Zeit,
großen Spaß. Erst neulich habe ich in den Unterlagen geblättert, die ich
mir damals zur Vorbereitung auf meine ersten Vorlesungen
zusammengestellt habe. \enquote{Das war nicht so ohne.} Es dauert ja
eine Weile, bis man Routine hat.

Zum Thema \emph{Arbeit in der Bibliothekspraxis} gab es noch ein
Zwischenspiel, das nicht unwichtig war und gut ein Jahr dauerte. 1976,
als der Wechsel in die Lehre schon bevorstand, wurde ich vom Ministerium
an das Hochschulbibliothekszentrum des Landes NRW (HBZ) delegiert.
Dessen erster Direktor war im Jahre 1973 Günther Pflug geworden. Als das
HBZ eigentlich noch nicht voll in Gang gekommen war, wurde Günther Pflug
1976 zum Generaldirektor der Deutschen Bibliothek in Frankfurt am Main
berufen. Die ehrenvolle Berufung konnte und wollte Pflug natürlich nicht
ablehnen, doch für das Ministerium in Düsseldorf war guter Rat teuer.
Jammers fand keinen, der die Aufgabe am HBZ zu übernehmen bereit gewesen
wäre, obwohl es um eine gut dotierte Stelle ging: B2 und damit über
Professoren- oder Oberstudiendirektoren-Gehältern liegend.

So delegierte das Ministerium mich nach Köln auf die vakant gewordene
Stelle. Ich habe damals dem Staatssekretär ausdrücklich gesagt, dass ich
die Aufgabe am HBZ nicht als Daueraufgabe betrachte; Jammers wusste das
sowieso. Ich war mir bewusst, dass meine Stärke nicht in dem Bereich
liegt, den das HBZ zu betreuen hat. Die Weiterentwicklung der frühen
Datenverarbeitung und der ganz frühen Informationstechnik war meine
Sache nicht. Jeder sollte sich selbst richtig einschätzen und mir war
klar, dass meine Möglichkeiten woanders liegen. So habe ich betont, dass
ich die Aufgabe einstweilen übernehme, weil ich mich als Beamter dazu
verpflichtet fühle und weil ich in der Lage bin, die dort anfallenden
Verwaltungsaufgaben zu erfüllen. Das Ganze zog sich dann immerhin über
15 Monate hin. Dann hatte das Ministerium in der Person des Kollegen
Dr.~Peter Rau einen guten Nachfolger für Pflug gefunden und ich konnte
mich in Köln ganz der Lehre widmen.

Ich fasse zusammen: Universitätsbibliothek Bochum,
Fachhochschulbibliothek Bochum (nebenamtliche Leitung),
Wissenschaftsministerium in Düsseldorf, Hochschulbibliothekszentrum in
Köln (kommissarische Leitung). Soweit die Praxiserfahrung, die ich von
1977 an in die Lehre einbringen konnte.

\textbf{KS: 1995 wurden Sie dann auf die Professur für
Bibliothekswissenschaft an der Philosophischen Fakultät I der
Humboldt-Universität zu Berlin berufen. Wie kam es zu dieser
Entwicklung?}

\textbf{EP:} Ja, da ist eigentlich die Weltgeschichte ursächlich,
genauer und etwas bescheidener gesagt, die Vereinigung Deutschlands. So
habe ich wiederholt formuliert, und zwar mit Bedacht; es gibt Fügungen,
die man ergreifen kann und soll. Im Jahre 1989, im Frühjahr wohlgemerkt,
war ich zum Vorsitzenden des Vereins Deutscher Bibliothekare gewählt
worden. Die wenigsten werden wohl geahnt haben, was sich im Laufe dieses
Jahres noch ereignen sollte, speziell am 9. November. Und dann kamen die
Ereignisse und ich war nun Vorsitzender des Vereins. Ich habe mir
gesagt: Daraus erwächst dir eine Verpflichtung, du musst jetzt umgehend
auf der anderen Seite der niedergegangenen Mauer Kontakte zu den
dortigen Kollegen knüpfen. Das ist auch gelungen, unter Mithilfe der
weiteren Vorstandsmitglieder. Natürlich erstreckten sich die Kontakte
auf Kollegen Ost, die einige Zeit später nicht mehr im Amt waren, und
auf Strukturen Ost, die noch im Jahre 1990 zerfielen. Immerhin gab es
bereits im Februar 1990, als man noch nicht wusste, dass Deutschland
schon im Oktober desselben Jahres vereinigt sein würde, eine größere
Versammlung in Warnemünde. Die Vorstände der Verbände West fuhren
dorthin, wir wollten den Kollegen entgegenkommen.

Spätestens in Warnemünde stellten wir fest, dass wir allesamt
ahnungslose Wessis waren. Ich vermute, dass die Kollegen im Osten weit
mehr Ahnung von unseren Verhältnissen hatten als umgekehrt. Folgendes
Beispiel: Am 4. und 5. November 1989 fand eine turnusmäßige Sitzung der
Arbeitsgemeinschaft der bibliothekarischen Ausbildungsstätten statt, und
zwar in den Räumen der Bayerischen Bibliotheksschule in München. Der
Kollege Rupert Hacker warf die Frage auf: Was machen wir denn mit den
Kolleginnen und Kollegen, die aus dem anderen Teil Deutschlands über die
Botschaften in Budapest und Prag hierher gekommen sind? Unter den vielen
Menschen, welche die Bundesrepublik auf diesem Wege erreicht hatten, gab
es natürlich auch Bibliothekare. Kein Mensch im Westen wusste, wie man
die Kollegen, wenn man sie anstellen wollte, tariflich eingruppieren
sollte. In der Bayerischen Bibliotheksschule saßen zehn oder zwölf
Personen, samt und sonders kundige und interessierte Kollegen mit vielen
Kontakten, die einander ratlos ansahen. Keiner konnte die Frage
beantworten. Ja, wozu bilden die denn in Leipzig aus? Wozu bilden die in
Sondershausen aus? Wozu an der Humboldt-Universität? Keiner wusste es.
Es wurde die Aufgabe gestellt, jeder sollte sich, so gut es geht, kundig
machen. Das ging schon bald besser als erwartet, weil sich einige Tage
später die Mauer öffnete.

Dies vorausgeschickt war ich von Anfang an durch den VDB-Vorsitz und
durch die Arbeitsgemeinschaft der bibliothekarischen Ausbildungsstätten
ziemlich eng in die Themen einbezogen, die mit der Vereinigung
Deutschlands zusammenhingen. Das führte unter anderem dazu, dass ich im
Frühjahr 1990 einen Besuch in Dresden in dem neu gebildeten Ministerium
machte, nachdem die ersten Landtagswahlen in den östlichen Bundesländern
stattgefunden und die Landtage und Landesregierungen sich konstituiert
hatten. Die Reise habe ich als VDB-Vorsitzender unternommen, um
festzustellen, wer auf Regierungsebene für die Bibliotheken zuständig
ist und um sodann die notwendigen Kontakte zu knüpfen und Informationen
auszutauschen. In Dresden empfing mich Herr Dr.~Rosenkranz, mit dem ich
mich sofort gut verstanden habe. Wir hörten einander aufmerksam zu. Nach
dem eingehenden Gespräch (und einem kurzen Gang durch die immer noch
nach dem Zweiten Weltkrieg aussehende Stadt) reiste ich wieder nach
Hause. In der Folgezeit habe ich auch das zuständige Ministerium des
Landes Sachsen-Anhalt aufgesucht (wegen Sondershausen) und die
zuständige Senatsverwaltung in Berlin (wegen der HU), natürlich auch die
betreffen Ausbildungsstätten selber.

Im Lauf des Jahres 1991, ein Jahr nach meinem Besuch in Dresden,
erreichte mich der Ruf, als Gründungsdekan für den
geisteswissenschaftlichen Fachbereich der neuen Fachhochschule für
Technik, Wirtschaft und Kultur (HTWK) nach Leipzig zu kommen. An eine
solche Entwicklung hatte ich 1990 nicht gedacht, aber es kam so und ging
natürlich letztlich auf meinen Besuch in Dresden zurück. Das Ministerium
suchte einen Gründungsdekan aus einem westlichen Bundesland, weil es bei
der Neustrukturierung der bibliothekarischen Studien viele Dinge zu
regeln galt, die man in der allgemein westlichen, also nicht nur
westdeutschen, sondern auch französischen, englischen, kurzum in der
ganzen westlichen Welt üblichen Form der Hochschulorganisation
durchführen wollte. Dass ich die Aufgabe bewältigen könnte, nahm man in
Dresden vielleicht auch deshalb an, weil ich vier Jahre lang Rektor der
Fachhochschule für Bibliotheks- und Dokumentationswesen in Köln gewesen
war (1986--1990). Ich habe es nicht bereut, den Ruf angenommen zu haben.
Die Jahre in Leipzig sollten die schönsten meiner beruflichen Laufbahn
werden, die interessantesten wurden es eh. Auf die gewissen Abstriche,
die ich in der Erinnerung an meine Humboldt-Zeit zu machen habe, komme
ich später noch zu sprechen \emph{{[}lacht{]}.}

Wie die einheimischen Kollegen von den drei Vorgängereinrichtungen in
Leipzig mitgezogen haben, war wunderbar, obwohl sich alle im Klaren sein
mussten, wie unsicher ihre berufliche Zukunft aussah. Die Einrichtungen,
aus denen der neue Fachbereich an der künftigen HTWK hervorgehen sollte,
waren die Fachschule für Bibliothekare und Buchhändler, die für
Öffentliche Bibiotheken und Buchhandel ausbildete; die Fachschule für
Wissenschaftliches Bibliothekswesen (ÖB und WB waren in der DDR strikt
getrennt); die Fachschule für Museologen. Für den ganzen Fachbereich
waren 18 Stellen für hauptamtlich Lehrende vorgesehen, etwa 50
\enquote{Lehrer} hatte es an den drei Vorgängereinrichtungen gegeben --
die übliche personelle Überbesetzung der DDR. Die letzteren waren zum
größeren Teil nicht promoviert und wussten, dass sie nur geringe Chancen
hatten, auf eine der wenigen Hochschullehrerstellen übernommen zu
werden.

In dieser Lage gab es für mich beziehungsweise für die Gründungs- und
Berufungskommission eine gewisse Entspannung aus politischen Gründen.
Das Ministerium in Dresden hat mir nämlich alsbald vermittelt, dass ich
mich um die politische Seite der Sache nicht zu kümmern habe, also um
die Frage, wer von jenen 50 Mitarbeitern bei der Stasi mitgemacht hat
und so weiter. Das hat das Ministerium selbst übernommen und mir fiel
ein Stein vom Herzen. Natürlich bemerkte ich nach einer gewissen Zeit,
wo in Leipzig der politische Hase lief, habe mich dann aber bemüht,
daraus nichts zu machen. Im Übrigen bemerkte ich in den vielen
persönlichen Gesprächen schon, welche Kollegen fachlich wirklich gut und
politisch unbelastet waren. Die Promovierten hatten ein Prae, das war
klar und konnte nicht anders sein. Letztlich hat die Gründungs- und
Berufungskommission, deren Vorsitzender ich war, über die Berufungen
entschieden. Sie hat ihre heikle Aufgabe gut gelöst und die Hochschule
auf die Beine gebracht. Der neue Fachbereich bekam bald Zuspruch; ich
kann mich genau erinnern, wie die ersten Studenten aus Bayern kamen
\emph{{[}lacht{]}.}

Über die Entwicklung des Fachbereichs habe ich die Fachwelt ausführlich
informiert, in einem ZfBB-Heft nach dem anderen. Kollege Lehmann, damals
Generaldirektor der Deutschen Bibliothek und Redaktor der
\emph{Zeitschrift für Bibliothekswesen und Bibliographie}, sagte mal zu
mir: Herr Plassmann, als Sie beim fünften oder sechsten Beitrag waren,
da dachte ich -- so lobenswert die Berichte sind -- irgendwann muss
dieser Fortsetzungsroman doch mal ein Ende haben! Dieses Wort gebrauchte
er \emph{{[}lacht{]}.} Ich konnte ihm nur antworten: Wenn Sie meine
Beiträge noch ein paar mal aufnehmen, findet die Sache einen guten
Abschluss. Er hat es noch ein paar mal geduldet, so sind insgesamt zehn
derartige Berichte erschienen. Damit ist alles nachvollziehbar, was wir
damals in Leipzig gemacht haben. Ich hatte erstklassige Hilfe durch
Prof.~Christian Uhlig, der als Gründungsprofessor für den Studiengang
Buchhandel gekommen war. Aus meiner früheren Arbeit in Bochum war mir
der Buchhandel nicht unbekannt, doch war mir klar, dass meine Kenntnisse
auf dem Gebiet bei Weitem nicht ausreichten, um diesen Studiengang
voranzubringen und in der akademischen Welt so zu platzieren, wie er es
verdiente. Christian Uhlig, gelernter Buchhändler und Volkswirt, war
akademischer Direktor an der Ruhr-Universität Bochum und arbeitete in
der Entwicklungspolitik, hatte aber seine Vergangenheit keineswegs
vergessen. Jedenfalls war er umstandslos bereit, als Gründungsprofessor
nach Leipzig zu kommen. Noch andere Namen sind zu nennen. Vor allem
Gottfried Rost von der Deutschen Bücherei, Mitglied meiner Kommission,
und Thorsten Seela haben mir durch Rat und Tat so gut zur Seite
gestanden, dass ich die Klippen und Untiefen des Umwandlungsprozesses
weit besser umschiffen konnte, als es ohne ihre Hilfe möglich gewesen
wäre.

Langer Rede kurzer Sinn: Meine Arbeit in Leipzig bleibt mir in sehr
schöner Erinnerung. An die Zeit als Gründungsdekan, vom Ministerium
ernannt, schloss sich noch eine kurze Zeit als gewählter Dekan an, die
ich allerdings nicht mehr zu Ende geführt habe.

Es erschien nämlich schon bald eine Stellenausschreibung der
Humboldt-Universität. Gesucht wurde ein C-4-Professor für das Institut
für Bibliothekswissenschaft. Ich glaubte, dass die Stellenbeschreibung
genau auf meine bisherigen Arbeiten passte und bewarb mich. Die
Berufungskommission hatte aber, wie ich feststellen musste, kein
Interesse an meiner Bewerbung. Ich hörte nichts von der Kommission,
erfuhr aber von meinem an der Humboldt studierenden Schwiegersohn von
einem Aushang, der besagte, dass zwei Probevorträge im Institut für
Bibliothekswissenschaft gehalten würden, der eine von Pamela Spence
Richards aus den USA, der andere von einem Kollegen aus Österreich; zwei
Probevorträge und nicht, wie üblich, drei. Da wusste ich Bescheid und
zog, um mir eine förmliche Absage zu ersparen, meine Bewerbung zurück.

Derweil wollte das Rektorat der HTWK mich auf Dauer in Leipzig halten,
auch das Ministerium in Dresden signalisierte mir diese Absicht. Doch
war ich der Meinung, dass das Kollegium in Leipzig seine Sache gut
alleine machen würde und keinen Professor aus dem Westen mehr brauchte.
So kehrte ich zum Sommersemester 1995 nach Köln zurück, wo ich den
hieran sehr interessierten Studenten neben dem normalen Stoff vieles von
dem, was sich in den östlichen Bundesländern tat, sozusagen \emph{live}
vermitteln konnte.

Dann kam es doch zur Berufung an die Humboldt. Ein denkwürdiger Anruf
von Walther Umstätter im Frühjahr 1995: \enquote{Sitzt du auf einem
Stuhl?} Meine Antwort: \enquote{Ja, ich sitze auf einem Stuhl.} Er habe
mir etwas Wichtiges mitzuteilen, fuhr der Kollege fort. \enquote{Würdest
du doch nach Berlin kommen?} Ich erwiderte: \enquote{Ja, ich würde schon
nach Berlin kommen, ich bin nicht beleidigt oder dergleichen; wenn ihr
mich wirklich haben wollt, komme ich. Nur eins: Nochmal bewerben werde
ich mich nicht.} Und Umstätter hat es geschafft, meine Berufung an die
Humboldt-Universität zu erreichen, auch ohne dass ich mich erneut
beworben hätte. In den drei Jahren in Leipzig hatte ich Spaß daran
gehabt, Neues auf den Weg zu bringen und freute mich nun, das auch in
Berlin zu versuchen.

\textbf{KS: Dort lag zu diesem Zeitpunkt die Zusammenlegung der
Institute von HU und FU noch nicht lange zurück. Welche Auswirkungen
hatte diese Situation, wie haben Sie die Atmosphäre am IBI damals
wahrgenommen?}

\textbf{EP:} Um es rundheraus zu sagen: Die Atmosphäre hätte besser sein
können \emph{{[}lacht{]}.} Es wäre ja fast ein Wunder gewesen, wenn es
anders gewesen wäre. Es kamen drei sehr verschiedenartige Gruppen von
Personen im Lehrkörper zusammen: Erstens \enquote{Humboldt alt},
zweitens \enquote{FU} und drittens die Neuberufenen, die von woanders
hergekommen waren. Diese drei Gruppen hatten jeweils ganz
unterschiedliche Vorgeschichten. Wobei ich den Unterschied zwischen den
Neuberufenen aus den westlichen Bundesländern und den
\enquote{FU-Leuten} als größer wahrgenommen habe als den Unterschied
zwischen den Kollegen West und Ost. Das alte West-Berlin war in einem
Maße ein Fall für sich, wie man sich das heute kaum noch vorstellen
kann.

Zu Mauerzeiten bin ich sehr oft in Berlin (West) gewesen, und, von zwei
Ausnahmen abgesehen, immer mit dem Flugzeug gereist. Wenn Sitzungen von
Vertretern bestimmter Sparten wie zum Beispiel der bibliothekarischen
Ausbildungsstätten aus verschiedenen Bundesländern stattfanden, so
wurden diese natürlich nach Hannover, Stuttgart oder Düsseldorf
anberaumt, aber auch oft -- jedenfalls öfter als man denken könnte --
nach Berlin (West). Das lag daran, dass Ministerien im Westen
Dienstreisen nach Berlin mit dem Flugzeug zahlten, weil Berlin
unterstützt werden sollte. Ich nahm dann in Düsseldorf morgens das erste
Flugzeug und kam abends mit dem letzten zurück. So kam ich öfter nach
Berlin und war mir klar, dass Berlin -- auch im engeren Bereich des
Bibliothekswesens -- etwas Besonderes ist.

An der FU kam noch Folgendes hinzu. Die dortige bibliothekarische
Ausbildung bestand aus zwei disparaten Teilen, erstens einem relativ
wenig genutzten universitären Studiengang, den man nur in Kombination
mit anderen Fächern studieren konnte, und zweitens einem eigentlichen
Fachhochschulstudiengang (Diplom-Studiengang), der nur sechs Semester
dauerte und in ÖB und WB aufgeteilt war. Deren Studenten bekamen alle
einen Abschluss, wie man ihn in den alten Bundesländern an der
Fachhochschule bekam. Und die Absolventen wurden an den Bibliotheken
auch entsprechend eingestellt. Die Regelung stammte letztlich aus der
Zeit der Begeisterung für die Gesamthochschule, an der universitäre und
FH-Studiengänge möglichst eng miteinander verzahnt werden sollten. Dass
das aus vielen Gründen, auf die wir hier nicht eingehen können, dann
nicht so gekommen ist, steht auf einem anderen Blatt. An der FU hatte es
sich aber erhalten und damit war die FU in unserem Kreise der
bibliothekarischen Ausbildungsstätten ein \emph{Unicum.} Beinahe hätte
ich gesagt, ein \emph{Unicum et Curiosum,} aber gut, so war das. Ein
wirkliches Unicum et Curiosum an der FU bestand darin, dass es in dem
sechsköpfigen hauptamtlichen Lehrkörper nicht eine einzige Frau gab,
obwohl in Berlin (West) genauso wie überall sonst in West und Ost 85 bis
90\,\% Studentinnen eingeschrieben waren. In Köln, Leipzig und an
anderen Orten bestand damals längst ein Viertel oder ein Drittel des
hauptamtlichen Lehrkörpers aus Frauen.

Nachdem das FU-Institut 1994 an die Humboldt transferiert worden war,
bestanden die disparaten Studiengänge zunächst an der HU weiter. Die
FU-Kollegen wussten von früher, dass ich, so wie die meisten Kollegen
aus den westlichen Bundesländern, diesen Zustand suboptimal fand. So
begegneten die FU-Kollegen den aus den alten Bundesländern Gekommenen
mit Vorbehalten; zwei FU-Kollegen nehme ich ausdrücklich aus, die haben
mich kollegial und sehr freundlich empfangen. Um es aus meiner Sicht
offen zu sagen: Ich habe das Konzept der verschiedenartigen Studiengänge
nach außen mit vertreten, war mir aber klar darüber, dass die
Humboldt-Universität das nicht auf Dauer mittragen würde. Und dass es
sich zum Nachteil unseres Instituts auswirken würde, wenn wir die von
der FU mitgebrachte Extrawurst an der HU unbedingt hätten beibehalten
wollen.

\textbf{KS: Wie haben sich denn die Ausrichtung und auch die
Fachidentität am Institut während Ihrer Zeit dort entwickelt?}

\textbf{EP:} Als die FH-äquivalenten Studiengänge schließlich
ausgelaufen waren und wir uns auf die universitären Studiengänge
konzentrieren konnten, brach eine gute Zeit am IBI an. Erstens war es
eine wunderbare Studentengeneration, zweitens nenne ich diese Jahre, die
ich hier erlebt habe, immer die Zeit der Re-Akademisierung des
Studiengangs. Dabei setze ich voraus, dass es in der DDR nicht das
gegeben hat, was wir unter einem akademischen Studiengang verstehen.
Hervorragende Einzelleistungen, wie zum Beispiel von Horst Kunze, möchte
ich davon ausnehmen. Aber die Strukturen als Ganzes waren eigentlich
nicht akademisch. Die Freiheit des einzelnen, der Instituts- und
Fakultätsräte und so weiter gab es in der DDR so nicht. Kurzum, die
Re-Akademisierung war eine schöne Periode der Institutsgeschichte; in
dem Vorwort \enquote{Gruß und Dank} in der Festschrift für Konrad Umlauf
habe ich das festgehalten.

Mit den letzten Jahren vor der Jahrhundertwende und den ersten Jahren
danach verbinde ich die Durchführung einer wachsenden Zahl von
Promotionsverfahren und die Erfindung des Promovendenkollegs, dem sich
Konrad Umlauf umgehend und mit hohem Einsatz anschloss. Die
Promotionsanwärter lernten sich dort gegenseitig kennen, kamen mit
mehreren Professoren persönlich ins Gespräch und das in aller Ruhe,
einen Freitagnachmittag und einen Sonnabendvormittag lang. Kurz nachdem
wir damit angefangen hatten, las ich in einer hochschulpolitischen
Zeitschrift: Der Wissenschaftsrat empfiehlt die Einrichtung von
Promovendenkollegs. Wir hatten also den richtigen Weg eingeschlagen.
Zweitens verbinde ich mit jener Zeit die Neustrukturierung und
Weiterführung des aus der DDR übernommenen Fernstudiums, im Zusammenhang
damit die Kooperation mit der Universität Koblenz-Landau. Drittens trat
das Institut mit öffentlichen Vorträgen von allgemeinerem Interesse
hervor, im Zusammenhang damit ist die Einführung des Berliner
bibliothekswissenschaftlichen Kolloquiums zu nennen, das geradezu ein
Markenzeichen des Instituts wurde. Viertens ist das Erscheinen größerer
Monographien von Institutsangehörigen zu nennen; auf diesem Felde haben
mehrere Kollegen sich um das fachliche Ansehen des Instituts große
Verdienste erworben, es war der Kern der Re-Akademisierung. Hier fällt
der Unterschied zur DDR frappant auf. Aus dem Institut kamen zur
DDR-Zeit \enquote{Lehrbriefe}, in denen das praktische Wissen des
Bibliothekars (auf schlechtem Papier) tradiert wurde. Ich habe mir die
neulich nochmal durchgesehen, man kann nur den Kopf schütteln. Die
Begründung der blauen Schriftenreihe des Instituts ist ein weiteres
sichtbares Zeichen des wiedererwachten wissenschaftlichen Lebens. Im
Jahre 2000 erschien der erste Band und nach einer allerdings
merkwürdigen Lücke von 2008 bis 2014 sind wir nun, so scheint mir, bei
Band 28. Mit einer eigenen Schriftenreihe zeigt ein Institut nach außen
hin nicht nur, welche Nachwuchsleute kommen, wenn dort zum Beispiel
Dissertationen veröffentlicht werden, sondern es zeigt auch ein Stück
seines eigenen Profils. Last but not least ist auf die Durchführung
zahlreicher Exkursionen mit Studenten des Instituts hinzuweisen, und auf
das Anknüpfen persönlicher Kontakte in auswärtigen und ausländischen
Bibliotheken.

Das 200. Jubiläum der Humboldt-Universität, insbesondere die seinerzeit
erschienene voluminöse Festschrift, hat die erfreuliche Entwicklung
vieler Fächer einer weiteren Öffentlichkeit bekannt gemacht. Der Beitrag
zu unserem Fach ist demgegenüber eine herbe Enttäuschung. In Band 6 der
Festschrift beschreiben zwei hiesige Kollegen die Entwicklung des
Instituts seit 1995 so, dass man das Institut nicht wiedererkennt. Schon
die Überschrift \enquote{Die Bibliothekswissenschaften in Berlin} (im
Plural!) ist falsch, es gibt doch nicht mehrere
Bibliothekswissenschaften! Und die Einrichtung heißt \enquote{Institut
für Bibliotheks- und Informationswissenschaft} (im Singular!). Das ist
das eine. Das andere: Nichts von dem, was ich vorhin aus lebendiger
Erinnerung über das wiedererwachte akademische Leben an unserem Institut
geschildert habe, kommt in dem Beitrag vor, aber auch gar nichts. Für
die Zeit von 1995 bis 2010 (Erscheinungsjahr der Festschrift) werden
einige formelle Dinge erwähnt, wie Wahlen abliefen, wer berufen wurde
und dergleichen. So wird auch mein Name erwähnt, nur einmal und so
nebenher, dazu mit einer abfällig gemeinten Wendung über mein
vorgerücktes Lebensalter; es sollte wohl zum Ausdruck gebracht werden:
Von dem kann ja nichts mehr kommen. Übrigens ist selbst dies unrichtig
angegeben, da steht nämlich: \enquote{der 59-jährige}. Ich war aber
schon 60 \emph{{[}lacht{]}.} Die Namen von Umstätter und Umlauf werden
überhaupt nicht genannt. Gerade diese beiden Kollegen haben viel und
nachhaltig zu der geschilderten erfreulichen Entwicklung, zur
Re-Akademisierung beigetragen.

Der Beitrag in der Festschrift stellt eine Verfälschung der Geschichte
dar und ist daher ein Ärgernis. Das hat unser Institut nicht verdient.
Ich bin dankbar, dass ich das dort gezeichnete schiefe Bild in diesem
Interview zurechtrücken konnte.

Zu dem von Ihnen angeschnittenen Thema \enquote{Fachidentität} gehört
natürlich auch das Dauerthema Printbuch beziehungsweise Printbibliothek
und Digitalbuch beziehungsweise Digitalbibliothek. Der (vermeintliche)
Gegensatz spielte vor 20 Jahren noch nicht die provokante Rolle, die ihm
heute vielfach zugeteilt wird. Darum habe ich hier die damalige
Re-Akademisierung hervorgehoben. Die war nach der DDR-Zeit eigentlich
das wichtigere.

\textbf{KS: Sie haben während Ihrer Zeit am Institut zahlreiche
Lehrveranstaltungen gehalten und Projekte durchgeführt. Gibt es
Veranstaltungen oder Projekte, die Ihnen besonders in Erinnerung
geblieben sind?}

\textbf{EP:} Ja. Leider ist die Webseite zu dem Projekt \enquote{Das
Buch und sein Haus} zurzeit nicht zugänglich. Das hängt damit zusammen,
dass sie dringend einer Bearbeitung bedurfte; ich habe eine Reihe von
Fotos neu aufgenommen und mich auch entschieden, die Gesamtstruktur ein
wenig zu verändern, hoffe aber, dass die Seite bald wieder
freigeschaltet werden kann. Die Präsentation des Projekts, an dem mir
viel lag und liegt, steht in unmittelbarem Zusammenhang mit meinen
Exkursionen. Die waren so eine Art Alleinstellungsmerkmal von mir.
Innerhalb von Deutschland gab es, soweit ich das aus dem Kopf sagen
kann, drei oder vier Exkursionen zu Bibliotheken im Rheinland und im
Ruhrgebiet, vier Exkursionen zu Bibliotheken in Bayern, zwei oder drei
Exkursionen zu Bibliotheken in Baden-Württemberg. Eine Exkursion haben
wir zu Bibliotheken im Elsass gemacht, wo wir von den französischen
Kollegen übrigens besonders freundlich empfangen und geführt wurden
(durchweg in deutscher Sprache). Und dann drei Exkursionen nach Polen,
davon eine unter Einschluss von Litauen. Eine nach Tschechien und
Österreich. In meiner Kölner Zeit hatte ich viele andere organisiert,
infolge der politischen Bedingungen in Europa damals vornehmlich in die
Niederlande, nach Belgien und Luxemburg. Bewusst habe ich die
\emph{highlights} in London und Paris, die British Library und die
Bibliothèque Nationale ausgelassen -- in der Erwägung, dass die jüngeren
Kollegen ohnehin mal dahin kommen, aber zum Beispiel ins Elsass, nach
Colmar oder Schlettstadt (Sélestat) nicht so leicht. Oder nach Polen,
nach Lublin oder Posen.

In fast jedem Semester an der HU habe ich die Exkursionen zu deutschen
und ausländischen Bibliotheken unternommen, darüber hinaus auch die
üblichen Berlin-Exkursionen. Aus allen sind Fotos für das Projekt
\enquote{Das Buch und sein Haus} hervorgegangen. Bei den Studenten
fanden die Angebote stets guten Zuspruch. Ich erinnere mich an volle
Omnibusse mit 30 oder 35 Studenten. Für die Studenten wurde das eigene
Studienfach viel anschaulicher und lebendiger, darüber hinaus lernten
sie sich untereinander besser kennen als an der Universität; Ergebnis
unter anderem: ein Ehepaar \emph{{[}lacht{]}.}

\textbf{KS: Gegen Ende der 1990er Jahre steuerte das Institut auf eine
Krise zu, die Schließung drohte. Wie haben Sie diese Entwicklungen
wahrgenommen?}

\textbf{EP:} Das habe ich natürlich mit tiefer Trauer wahrgenommen. Umso
mehr mit Trauer, als ich das Schicksal des Lehrstuhls Kaegbein noch
lebhaft in Erinnerung hatte. Dieser war in der alten Bundesrepublik der
einzige Lehrstuhl für Bibliothekswissenschaft auf Universitätsebene,
wenn ich vom Sonderfall FU absehe. Kaegbein ist im Jahre 1990 emeritiert
worden. Ich war viele Jahre hindurch Lehrbeauftragter an der Universität
zu Köln, auch dadurch mit Kaegbein verbunden, und habe die Festrede
gehalten. Damals, im Juli 1990, stand die Vereinigung Deutschlands bevor
und für uns Bibliothekare die Aussicht auf Erneuerung des Studiengangs
an der HU. Dass der Kölner Vorgang sich just zehn Jahre später an der HU
wiederholen würde, damit hatte ich allerdings nicht gerechnet. Schon gar
nicht damit, dass ich selber dann in der Rolle von Kaegbein sein würde:
Mit der Pensionierung des Professors wird der Studiengang geschlossen.
Umso trauriger hat es mich gestimmt.

Über die Entscheidungen des politischen Senats und des
Abgeordnetenhauses will ich nicht richten. Insgesamt kann ich höchstens
sagen, dass diese politischen Gremien nach meiner Auffassung die
Bildungspolitik generell etwas hintanstellen. Man braucht ja nur aktuell
den \emph{Tagesspiegel} aufzuschlagen und liest von der Lehrer-Misere in
Berlin. Das ist vielleicht ein Fall für sich, doch will ich es
wenigstens andeuten. Wie aber der Akademische Senat und das Präsidium
der Humboldt-Universität die politischen Entscheidungen umgesetzt haben
und nun ausgerechnet ein Fach, das in Deutschland allein gestellt ist,
schließen wollen, da hört mein Verständnis auf. Prof.~Mlynek, damals
Präsident der HU, will ich keinen persönlichen Vorwurf machen. Da ist
sicherlich vieles im Hintergrund geschehen, wovon ich nichts weiß. Aber
insgesamt muss ich schon sagen: Die Idee, den Studiengang Bibliotheks-
und Informationswissenschaft zu schließen, war irritierend.

\textbf{KS: Zum Glück ist es ja letztendlich nicht dazu gekommen, die
Schließung konnte abgewendet werden. Was hat denn aus Ihrer Sicht zur
\enquote{Rettung} des Instituts in den 2000er Jahren geführt?}

\textbf{EP:} Ich möchte hier deutlich die Studenten hervorheben. Die
haben mit großer Geduld und starkem Impetus die sechs Jahre überstanden,
bis schließlich der US-Professor kam \emph{{[}lacht{]}.} Ich sehe die
Studenten noch drüben in der \emph{BibLounge} sitzen und Pläne
schmieden. Immer wieder fiel ihnen etwas Neues ein. Wenn ich sie heute
sehen würde, dann würden mir wohl ihre Namen wieder einfallen. Viele von
ihnen sind auf meinen Exkursionen dabei gewesen. Ich weiß nicht, was bis
zur Berufung von Michael Seadle im Hintergrund gelaufen ist, dazu kann
ich nicht viel sagen. Natürlich haben die Kollegen und auch ich selbst
noch vieles zur Überbrückung der Übergangszeit getan. Die heikle Lage
des Instituts war für mich ein kräftiger Anstoß dazu, als Emeritus
weiter Vorlesungen zu halten, mich auch weiter an Prüfungen zu
beteiligen, am Promovendenkolleg mitzuarbeiten und so weiter. Ich möchte
meine eigene Rolle nicht übermäßig hervorkehren, zumal ich meine
Mitarbeit nach einigen Jahren zurückfahren musste. Spätestens 2003
bemerkte ich, dass ich durch die rasenden Fortschritte der
Informationstechnik bedingt nicht mehr so weitermachen konnte wie
bisher. So habe ich mich von diesem Jahre an auf das Gebiet der
Bibliotheksgeschichte zurückgezogen.

\textbf{KS: Wie Sie selbst beschreiben, haben Sie sich 2000 in einen
sehr aktiven Ruhestand verabschiedet, aber auch weiterhin Forschungs-,
Lehr und Prüfungstätigkeit an der Humboldt-Universität wahrgenommen. Wie
würden Sie die Entwicklung des Instituts in den letzten 18 Jahren aus
Ihrer Sicht beschreiben?}

\textbf{EP:} Sicherlich hat das Institut den Schritt von der Print- zur
Digitalbibliothek gemacht. Dafür steht der Name Michael Seadle
eindeutig. Und das war notwendig und richtig. Es war an allen
Studienstätten notwendig und richtig. Zum akademischen Leben, wie ich es
für die 1990er Jahre geschildert habe, kann ich jedoch für die spätere
Phase keine genauere Einschätzung bieten.

\textbf{KS: Das Institut feiert im kommenden Wintersemester 2018/19 sein
90-jähriges Bestehen. Was wünschen Sie dem IBI zum 90. Geburtstag?}

\textbf{EP:} Das will ich gerne beantworten. Ich wünsche dem Institut,
dass seine Angehörigen auch in Zukunft so umsichtig, so weitsichtig und
auch so tolerant sind, dass sie keine falschen Gegensätze zwischen
verschiedenen Themenfeldern aufkommen lassen, dass sie vielmehr
weiterhin das Wort UND sagen können. Um es genauer zu sagen:
Digitalbibliothek UND Printbibliothek, Öffentliche Bibliotheken UND
Wissenschaftliche Bibliotheken, für die Dozenten: Fachstudium UND
bibliothekarische Ausbildung, Theorie UND Praxis -- als früherer
Fachhochschullehrer sage ich das Letztere mit Nachdruck. Auf diesen
Feldern, die man sicherlich noch ergänzen könnte, sollte immer das Wort
UND seinen Platz haben. Dieses Wort bedeutet ja: Wir wollen Inklusion,
nicht Exklusion. Vielen herzlichen Dank!

\textbf{KS: Ich danke Ihnen für das Gespräch, Herr Prof.~Plassmann.}

%autor
\begin{center}\rule{0.5\linewidth}{\linethickness}\end{center}

\textbf{Prof.~Dr.~em. Engelbert Plassmann} war unter anderem Professor
an der FH Köln und Gründungsdekan des Fachbereichs Buch und Museum der
FH jetzt HTWK Leipzig und schließlich von 1995 bis 2000 Professor am
damaligen Institut für Bibliothekswissenschaft (IB) und bis 2009 unter
anderem am IB und IBI noch in der Lehre aktiv.

\textbf{Kirsten Schlebbe} ist wissenschaftliche Mitarbeiterin und
Dozentin am Institut für Bibliotheks- und Informationswissenschaft an
der Humboldt-Universität zu Berlin. Sie hat 2015 ihr Masterstudium am
Institut abgeschlossen. 2016 begann sie ihre Promotion zum digitalen
Informationsverhalten von Klein- und Vorschulkindern bei Prof.~Dr.~Elke
Greifeneder. Seit 2016 betreut sie das Berliner
Bibliothekswissenschaftliche Kolloquium.

\end{document}
