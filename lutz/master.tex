\documentclass[a4paper,
fontsize=11pt,
%headings=small,
oneside,
numbers=noperiodatend,
parskip=half-,
bibliography=totoc,
final
]{scrartcl}

\usepackage{synttree}
\usepackage{graphicx}
\setkeys{Gin}{width=.4\textwidth} %default pics size

\graphicspath{{./plots/}}
\usepackage[ngerman]{babel}
\usepackage[T1]{fontenc}
%\usepackage{amsmath}
\usepackage[utf8x]{inputenc}
\usepackage [hyphens]{url}
\usepackage{booktabs} 
\usepackage[left=2.4cm,right=2.4cm,top=2.3cm,bottom=2cm,includeheadfoot]{geometry}
\usepackage{eurosym}
\usepackage{multirow}
\usepackage[ngerman]{varioref}
\setcapindent{1em}
\renewcommand{\labelitemi}{--}
\usepackage{paralist}
\usepackage{pdfpages}
\usepackage{lscape}
\usepackage{float}
\usepackage{acronym}
\usepackage{eurosym}
\usepackage[babel]{csquotes}
\usepackage{longtable,lscape}
\usepackage{mathpazo}
\usepackage[normalem]{ulem} %emphasize weiterhin kursiv
\usepackage[flushmargin,ragged]{footmisc} % left align footnote
\usepackage{ccicons} 
\setcapindent{0pt} % no indentation in captions

%%%% fancy LIBREAS URL color 
\usepackage{xcolor}
\definecolor{libreas}{RGB}{112,0,0}

\usepackage{listings}

\urlstyle{same}  % don't use monospace font for urls

\usepackage[fleqn]{amsmath}

%adjust fontsize for part

\usepackage{sectsty}
\partfont{\large}

%Das BibTeX-Zeichen mit \BibTeX setzen:
\def\symbol#1{\char #1\relax}
\def\bsl{{\tt\symbol{'134}}}
\def\BibTeX{{\rm B\kern-.05em{\sc i\kern-.025em b}\kern-.08em
    T\kern-.1667em\lower.7ex\hbox{E}\kern-.125emX}}

\usepackage{fancyhdr}
\fancyhf{}
\pagestyle{fancyplain}
\fancyhead[R]{\thepage}

% make sure bookmarks are created eventough sections are not numbered!
% uncommend if sections are numbered (bookmarks created by default)
\makeatletter
\renewcommand\@seccntformat[1]{}
\makeatother


\usepackage{hyperxmp}
\usepackage[colorlinks, linkcolor=black,citecolor=black, urlcolor=libreas,
breaklinks= true,bookmarks=true,bookmarksopen=true]{hyperref}
%URLs hart brechen
\makeatletter 
\g@addto@macro\UrlBreaks{ 
  \do\a\do\b\do\c\do\d\do\e\do\f\do\g\do\h\do\i\do\j 
  \do\k\do\l\do\m\do\n\do\o\do\p\do\q\do\r\do\s\do\t 
  \do\u\do\v\do\w\do\x\do\y\do\z\do\&\do\1\do\2\do\3 
  \do\4\do\5\do\6\do\7\do\8\do\9\do\0} 
% \def\do@url@hyp{\do\-} 
\makeatother 

%meta
%meta

\fancyhead[L]{H. Lutz \\ %author
LIBREAS. Library Ideas, 34 (2018). % journal, issue, volume.
\href{http://nbn-resolving.de/}
{}} % urn 
% recommended use
%\href{http://nbn-resolving.de/}{\color{black}{urn:nbn:de...}}
\fancyhead[R]{\thepage} %page number
\fancyfoot[L] {\ccpd\ \href{https://creativecommons.org/publicdomain/mark/1.0/}{\color{black}Creative Commons Public Domain Mark 1.0}}  %licence
\fancyfoot[R] {ISSN: 1860-7950}

\title{\LARGE{Was ist Bibliothekswissenschaft}}% title
\author{Hans Lutz} % author

\setcounter{page}{1}

\hypersetup{%
      pdftitle={Was ist Bibliothekswissenschaft},
      pdfauthor={Hans Lutz},
      pdfcopyright={CC Public Domain Mark 1.0},
      pdfsubject={LIBREAS. Library Ideas, 34 (2018).},
      pdfkeywords={Bibliothekswissenschaft, Geschichte, Selbstverständnis, Berufsbild},
      pdflicenseurl={https://creativecommons.org/publicdomain/mark/1.0/},
      pdfcontacturl={http://libreas.eu},
      baseurl={http://libreas.eu},
      pdflang={de},
      pdfmetalang={de}
     }



\date{}
\begin{document}

\maketitle
\thispagestyle{fancyplain} 

%abstracts

%body
Dr.~Hans Lutz

Probevortrag, gehalten am 15. Juli 1935 vor der Philosophischen Fakultät
I der Berner Hochschule. (Für den Druck in Einzelheiten erweitert.)

Hochgeehrter Herr Dekan!

Hochansehnliche Versammlung!

Wenn es mir vergönnt ist, der erste Vertreter meines Faches zu sein, der
vor der hohen Fakultät das Wort ergreifen darf, -- welches Thema läge
wohl näher, als Ihnen Rechenschaft abzulegen über die Frage: Was ist
Bibliothekwissenschaft, welches sind ihre Aufgaben und Ziele?

Die Frage ist nicht leicht zu beantworten, ist doch das Wort
Bibliothekwissenschaft, wie auch die Sache selbst noch recht jung, beide
sind noch lebhaft umstritten, noch sind die Stimmen, auch aus
Bibliothekarkreisen, nicht verstummt, welche behaupten, dass es eine
Bibliothekwissenschaft überhaupt nicht gebe\footnote{Ueber die Frage der
  Bibliothekwissenschaft haben sich geäussert: \emph{Rullmann},
  Friedrich. Die Bibliothekseinrichtungskunde zum Theile einer
  gemeinsamen Organisation, die Bibliothekswissenschaft als solche einem
  besondern Universitätsstudium in Deutschland unterworfen. Freiburg i.
  B. 1874. \emph{Gräsel}, Arnim. Grundzüge der Bibliothekslehre, 1890,
  8. 7--10; Handbuch der Bibliothekslehre, 1902, S. 7--13.
  \emph{Eichler}, Ferdinand. Begriff und Aufgabe der
  Bibliothekswissenschaft, 1896; Bibliothekswissenschaft als
  Wertwissenschaft, Bibliothekspolitik als Weltpolitik, 1923.
  \emph{Harnack}, Adolf von. Die Professur für Bibliothekswissenschaften
  in Preussen. Vossische Zeitung, 24. Juli 1921, abgedruckt in:
  Erforschtes und Erlebtes, 1923. S. 218--223; Bibliothekswissenschaft
  als Wertwissenschaft\ldots{} ZfB 40, 1923, S. 529--537.
  \emph{Leidinger}, Georg. Was ist Bibliothekswissenschaft? ZfB 45,
  1928, S. 440--454. \emph{Milkau}, Fritz. Bibliothekwesen, in: Aus 50
  Jahren deutscher Wissenschaft (Festschrift Friedrich Schmidt-Ott 1930,
  S. 22-43; Zur Einführung, in: Handbuch der Bibliothekswissenschaft,
  Bd. 1, 1931, S. V--XI. \emph{Rath}, Erich von. Die Forschungsaufgaben
  der Bibliotheken, in: Forschungsinstitute, Bd. 1, 1930, S. 136--147,
  beschränkt sich in der Hauptsache auf die Wiegendruckforschung.}.

Ich möchte nun mein Thema so behandeln, dass ich Ihnen im ersten Teil
des Vortrages die äussere Entwicklung der Bibliothekwissenschaft
darlege, im zweiten ihre Arbeitsgebiete umreisse und im dritten die
quaestio iuris behandle.

\hypertarget{section}{%
\subsection{1.}\label{section}}

Zweifellos gab es schon in den assyrischen und ägyptischen Bibliotheken,
von den griechisch-römischen ganz zu schweigen, aus der Erfahrung
gewonnene Regeln, welche wir als Anfänge einer Bibliothekwissenschaft
ansprechen dürfen. Man kam mit wenigen Regeln aus, da die Bibliotheken
bis zum Jahr 1800, von ein paar Ausnahmen abgesehen, nicht zu
umfangreich waren und nach den Grundsätzen einer grossen Privatbücherei
verwaltet werden konnten. Als Benützer waren nur wenige Personen
zugelassen, eine systematische Aufstellung und das Gedächtnis des
Bibliothekars genügten für die Benützung, die Kataloge wurden meist
nachlässig geführt, Regeln brauchten daher nicht aufgezeichnet zu
werden. Einen wissenschaftlichen Bibliothekarberuf gab es kaum, die
Bibliothekare wie Leibniz und Lessing waren in erster Linie Gelehrte und
fühlten sich weniger berufen, die Bibliothek zu verwalten, als sie zu
gebrauchen. Für sie war die Bibliothek Werkzeug und Fundgrube für ihr
gelehrtes Streben, auf die Verwaltung verwandten sie gerade so viel
Kraft als nötig. In den Bibliotheken herrschte die beschauliche Ruhe
eines Archivs.

Das wurde im 19. Jahrhundert anders. Der Rückgang des Lateinischen, die
Ausbreitung der Wissenschaften und der Nationalliteraturen im Gefolge
der Aufklärung, das Aufblühen der Universitäten seit 1815, die
wachsenden Scharen der Benützer, die steigende Buchproduktion, gefördert
durch die Erfindung der Schnellpresse, liessen dann die Büchermassen so
rasch anwachsen, dass eine weitgehende Arbeitsteilung nötig wurde. Es
ist die Leistung des 19. Jahrhunderts, diese durchgeführt zu haben. Die
Büchereien spezialisierten sich in Nationalbibliotheken, welche die
gesamte Büchererzeugung eines Landes erfassen, in die
Universitätsbibliotheken, welche das wissenschaftliche Schrifttum
sammeln, an sie schliessen sich die Instituts- und Behördenbibliotheken
für Sondergebiete an, und die Lesegesellschaften und Volksbibliotheken
sorgen für die schöne Literatur.

Der Beamtenstab vergrößert sich, die Universitätsbibliotheken -- und
diese meine ich im Folgenden vor allem, wenn ich von Bibliotheken
spreche -- werden zunächst noch von einem Professor, meist einem
Philologen, im Nebenamt verwaltet, seit den 1870er Jahren entwickelt
sich ein selbständiger wissenschaftlicher Bibliothekarstand, dessen
Ehrgeiz es ist, mit dem anvertrauten Pfunde zu wuchern, die
Bücherschätze so liberal wie möglich zugänglich zu machen. Als erster
Direktor einer wissenschaftlichen Bibliothek im Hauptamt wurde auf dem
europäischen Festland 1867 an der Universitätsbibliothek Basel Wilhelm
Vischer eingesetzt. In Deutschland wird 1872 Karl Dziatzko der erste
hauptamtliche Direktor an der Universitätsbibliothek Breslau. Bis zur
Jahrhundertwende hat sich dann ein hauptamtlicher Bibliothekarstand an
allen grossen wissenschaftlichen Bibliotheken gebildet.

Dieser findet seinen Ausdruck in den \emph{Berufsvereinen} und
\emph{Kongressen}\footnote{Vgl. Krüss im Handbuch der
  Bibliothekswissenschaft. Bd. I., S. 717--732.}, welche hier zu
erwähnen sind, weil sie in der Hauptsache der fachlichen Ausbildung der
Mitglieder dienen, und berufständische und Besoldungsfragen fast ganz
zurücktreten. Der älteste Verein ist die American Library Association,
gegründet 1876, weitere Landesverbände wurden gegründet in England 1877,
Japan 1892, Oesterreich und Italien 1896, in der Schweiz 1897,
Deutschland 1900, Frankreich 1906 usw.

Sechs internationale Kongresse fanden vor dem Krieg statt, meist im
Anschluss an Weltausstellungen, der erste 1877 in London. Die
Nachkriegszeit brachte dann auf Anregung der Association des
bibliothécaires français im Jahre 1927 die Gründung des Internationalen
Verbandes der Bibliothekarvereine, mit einem ständigen Ausschuss, der
jährlich tagt, u.\,a. 1932 in Bern. Dem Verband gehören zur Zeit 34
Vereine in 25 Ländern an. Er veranlasste den ersten Weltkongress für
Bibliothekwesen und Bibliographie 1929 in Rom und Venedig. Der zweite
fand im Mai 1935 in Madrid statt.

Generalsekretär des Verbandes ist der Direktor der Völkerbundbibliothek
in Genf, Dr.~T. P. Sevensma. Damit sind enge Beziehungen hergestellt zu
den \emph{Bestrebungen des Völkerbundes} auf dem Gebiet des
Bibliothekwesens. Das Institut international de coopération
intellectuelle in Paris setzte gleich bei seiner Gründung 1922 eine
Unterkommission für Bibliographie ein, seit 1927 Comité des experts
bibliothécaires genannt, welche die internationale Zusammenarbeit der
Bibliotheken fördern soll. Sie setzt sich aus den Direktoren der grossen
Landesbibliotheken zusammen, die Schweiz ist von Anfang an durch Marcel
Godet vertreten.

Das \emph{Zeitschriftenwesen}\footnote{Vgl. Gräsel, Handbuch, 8. 20--45.}
entwickelt sich seit 1840 mit der Gründung von Naumanns «Serapeum» und
Petzholdts «Anzeiger für Literatur und Bibliothekwissenschaft», beide
sind bis 1870 und 1886 erschienen. Dann folgt 1876 «The Library
Journal», NewYork; 1884 das «Zentralblatt für Bibliothekswesen», «The
Library Chronicle», London, und das «Bulletin des bibliothèques et des
archives», welche alle bis heute laufen, zum Teil unter verändertem
Titel. Das «Schweizerische Gutenbergmuseum», 1915 ff. von K. J. Lüthi
herausgegeben, und die «Nachrichten der Vereinigung schweizerischer
Bibliothekare» N. F. 1928 ff. (in der Zeitschrift «Der Schweizer Sammler
und Familienforscher» erscheinend) geben kein vollständiges Bild von der
bibliothekwissenschaftlichen Arbeit in der Schweiz, weil mancher Aufsatz
in deutschen und französischen Zeitschriften erscheint.

An Reihen sind zu nennen: Die «Sammlung bibliothekwissenschaftlicher
Arbeiten» hg. v. Dziatzko 1886 ff., die «Beihefte zum Zentralblatt für
Bibliothekwesen» 1888 ff. und die «Publikationen der Vereinigung
schweizerischer Bibliothekare» 1907 ff. Eine jährliche Bibliographie
erscheint seit 1905, seit 1926 ist sie zu einer «Internationalen
Bibliographie des Buch- und Bibliothekwesens» erweitert worden. Mit
Milkaus «Handbuch der Bibliothekswissenschaft» 1931 ff. haben wir die
erste, grosse Zusammenfassung der bibliothekwissenschaftlichen Forschung
erhalten.

Dieser Entwicklung des Berufstandes folgt alsbald die \emph{Entwicklung
der bibliothekwissenschaftlichen Einrichtungen an den} \emph{Hochschulen
und ähnlichen Instituten}\footnote{Vgl. Gräsel, Handbuch, S. 457--492.}.
Im 17. und 18. Jahrhundert wurden an den Universitäten Vorlesungen unter
dem Titel bibliothecaria peritia u. dgl. gehalten, das ist Bibliographie
und Gelehrtengeschichte\footnote{Nach Friedrich \emph{Koldewey},
  Geschichte der klassischen Philologie auf der Universität Helmstädt,
  Braunschweig 1897, las Melchior Schmid (Schmidius, 1638--1697,
  Griechischprofessor seit 1669 und Verwalter der UB) über
  Literargeschichte (optimorum scriptorum notitia) und
  Bibliothekwissenschaft (bibliothecaria peritia) (S. 106--110). Justus
  Christoph Böhmer (1670--1732, Prof.~der Politik, Dogmatik, Moral und
  Eloquenz) liest z.\,B. S. S. 1716 über Bibliothekwesen, d.\,h.
  deutlich Literargeschichte und die neuesten Zeitschriften (solche
  Zeitungskollegien waren damals beliebt): «I. Chr. Böhmer\ldots{} nunc
  de praecipiis ephemeridibus eruditorum, de bibliothecarum vitarumque
  scriptoribus aget; quae ad historiam litterariam bonorumque varii
  generis librorum notitiam pertinent pluralia, ad certa et selectiora
  capita suo tempore revocaturas.» (S. 86.)}. 1821 wurde die Ecole des
Chartes in Paris gegründet. 1865 hielt der Bibliothekar Tommaso Gar
Vorlesungen über Bibliologie an der Universität Neapel, welche 1868 im
Druck erschienen sind. 1884 eröffnete die französische
Unterrichtsverwaltung an der Ecole des Chartes in Paris eine mündliche
und schriftliche Staatsprüfung für junge Leute zur Erlangung eines
Fähigkeitszeugnisses für den Dienst an französischen Universitäts- und
Departements-Fakultätsbibliotheken. In den Vereinigten Staaten hielt
Melvil Dewey seit 1887 regelmässig Lehrkurse über Bibliothekwissenschaft
am Columbia College, New-York, auch an andern Universitäten gibt es
solche. In Preussen wurde 1886 in Göttingen die erste und bis jetzt
einzige ordentliche Professur für Bibliothekhilfswissenschaften
errichtet und mit Dziatzko als erstem Vertreter besetzt, später wurde
sie an die Universität Berlin verlegt. 1928 wurde an der Berliner
Hochschule das Bibliothekwissenschaftliche Institut errichtet. An der
Universität Würzburg ist Bibliothekwissenschaft als Nebenfach bei
Doktorprüfungen anerkannt\footnote{ZfB 1933, S. 549.}. 1934 wurden in
Deutschland an 18 Universitäten und 5 technischen Hochschulen von 30
Dozenten 40 bibliothekwissenschaftliche Vorlesungen und Uebungen
angezeigt, solche über Paläographie sind hierin nicht
eingerechnet\footnote{Jahrbuch der Deutschen Bibliothekare, Jg. 25, S.
  288 und ZfB 1933, S. 397, 441, 490.}. In Oesterreich gibt es in Wien
und Graz Dozenturen. In der Schweiz erhielt Prof.~Gustav Binz 1922
zugleich mit seiner Wahl zum Oberbibliothekar der Universitätsbibliothek
Basel einen Lehrauftrag für englische Philologie und
Bibliothekwissenschaft, in Zürich hat Hermann Escher seit
Winter-Semester 1931/32 einen Lehrauftrag. In Bern las P. D. Hans Georg
Wirz im Sommer-Semester 1934 einstündig über «Die Bibliotheken der
Schweiz und des Auslandes in den letzten 150 Jahren».

Der grössere Teil der Vorlesungen betrafen die Einführung in die
Bibliographie. Teils in der mehr äusserlichen Art: Einführung in die
Benützung der Bibliothek (einstündig oder nur drei Stunden zu Anfang des
Semesters), teils in genau umrissenen Einführungen in die Bibliographie
der Geisteswissenschaften, der Naturwissenschaften, der Technik, selbst
der Germanistik, Romanistik, Anglistik. In diesen Fällen ist vermutlich
stets vorher mit dem betreffenden Ordinarius Abrede getroffen worden.
Der kleinere Teil befasst sich mit der Geschichte der Bibliotheken, des
Buches, des Einbandes, der Buchmalerei, der Handschriftenkunde, der
deutschen Zeitung.

Sie sehen, man kann die 1870/80er Jahre als den \emph{Wendepunkt}
betrachten. Der alte gemütliche Bibliothekarbetrieb verschwindet
langsam, und an seine Stelle tritt die \emph{moderne Bibliothek}, deren
Hauptkennzeichen das Bestreben ist, den Benützern die Bücher so leicht
wie möglich zugänglich zu machen. In den angelsächsischen Ländern nennt
man auf dem Gebiet der Erziehung die Bibliotheken in einem Atem mit
Schule und Kirche. Der Bibliothekarstand entwickelt sich in 60 Jahren
aus kleinen Anfängen zu einer weltumspannenden Organisation. Die
Bibliothekwissenschaft wird erst von dieser Zeit an zusammenhängend
bearbeitet.

\hypertarget{section-1}{%
\subsection{2.}\label{section-1}}

Was für \emph{Arbeitgebiete} umfasst die Bibliothekwissenschaft?

Man teilt sie seit langem in zwei Hauptgruppen: Bibliotheklehre und
Bibliothekkunde. Die Bibliotheklehre umfasst die Kenntnisse von der
Einrichtung und Verwaltung einer Bibliothek. Sie ist also
Betriebswissenschaft. Unter Bibliothekkunde versteht man die Geschichte
des Buches in weitestem Umfang und der Bibliotheken. Methodisch arbeitet
die Bibliothekwissenschaft regional, es interessiert sie zunächst das
Buch- und Bibliothekwesen des eigenen Landes oder Sprachgebiets; von
andern Ländern sind nur die grossen Erscheinungen von Wichtigkeit. Darin
sind wir schon durch die Sammlungen gebunden. Bibliothektechnisch hat
auch das Ausland andere Verhältnisse, die wir nicht ohne weiteres
übernehmen können. Neben historisch höchst individuellen Erzeugnissen
behandelt die Bibliothekwissenschaft Massenerscheinungen und arbeitet
dann meist quantitativ-statistisch.

Wir wenden uns kurz der \emph{Bibliotheklehre} zu.

Die \emph{Literatur} über die Bibliotheklehre kann man beginnen lassen
mit Gabriel Naudes «Advis pour dresser une bibliothèque» 1627; allein
brauchbare Arbeiten liefern erst Albrecht Christoph Kayser: «Ueber die
Manipulation bey der Einrichtung einer Bibliothek» 1790, und Martin
Schrettinger: «Versuch eines vollständigen Lehrbuches der
Bibliothek-Wissenschaft» 1808--29. Bekanntere Namen sind auch Friedrich
Adolf Ebert, Christian Molbech und Arnim Gräsel. Soweit sie unsere
Verhältnisse betreffen, sind sie jetzt alle überholt durch den zweiten
Band des Milkauschen Handbuchs der Bibliothekwissenschaft 1933, und für
die Volksbibliotheken durch Paul Ladewigs «Politik der Bücherei» 1934.
Die einschlägige englische und amerikanische Literatur kommt wegen der
andern Verhältnisse für uns weniger in Betracht.

Die Bibliotheklehre umfasst eine Anzahl Spezialfragen, die ich hier
gerade nur nennen kann, die aber alle zur Zeit eifrig bearbeitet werden
und im Flusse sind. Die Rationalisierungsbestrebungen der Nachkriegszeit
haben auch das Bibliothekwesen ergriffen, und es sind schon gute
Ergebnisse erzielt worden.

Da ist zunächst das \emph{Bibliothekgebäude}, das zwei Hauptprobleme
umschliesst: Die grösste Raumausnutzung im Magazin und die
zweckmässigste Aneinanderreihung der Benutzer- und Dienst\-räume. Für
beide Fragen ist die Schweizerische Landesbibliothek einstweilen der
Musterbau.

Erforderlich ist ferner Kenntnis der \emph{Erwerbungsarten}, der
kaufmännischen Seite überhaupt: Organisation des Buchhandels, des
Antiquariats, des Schriftentausches, des Pflichtexemplarwesens, der
Einbandarten, sowie der Bibliothekstatistik und des Registratur- und
Kontrollwesens.

Was soll der Bibliothekar sammeln? Wie vermehrt er am besten die
Bücherbestände? Die Buchproduktion ist fortwährend gestiegen, die
Durchschnittsbedeutung des Buches gegen frühere Zeiten gesunken. Da alle
Bibliotheken zu wenig Geld erhalten, um alles zu kaufen, was des
Sammelns und Erhaltens wert ist, so gilt es möglichst klug auszuwählen,
teils durch Beschränkung auf gewisse Gebiete, teils durch kritische
Sichtung des Angebotenen. Das erste geschieht durch genaue Kenntnis der
Benutzerwünsche, besonders dessen, was an der Universität gelehrt wird,
das zweite durch eine umfassende Kenntnis der Buchproduktion. Hierzu
muss der Bibliothekar alle wichtigen Nationalbibliographien,
Fachbibliographien und kritischen Zeitschriften kennen. In der
kritischen Einschätzung der Kritiken liegt ein guter Teil der Kunst des
bibliothekarischen Sammelns.

Wie \emph{stellt} man die Bücher zweckmässig \emph{auf}? Früher nach
einem wissenschaftlichen System, dass die Aufstellung im Magazin die
Gliederung der Wissenschaften repräsentierte, man ersparte sich damit
einen Sachkatalog. Heute genügt das nicht mehr, soll man nun nach
Gruppen, nach dem Alphabet, nach Formaten oder nach dem Numerus currens
aufstellen? Ueber die rationellste Lösung ist viel geschrieben worden,
was ist für jeden Fall das Zweckmässige?

Die Hauptarbeit des Bibliothekars ist gewöhnlich das
\emph{Katalogisieren}. Es ist der Hauptdienst, den die Bibliothekare der
Wissenschaft leisten, indem sie die gesamte Literatur ordnen und
bereitstellen, sowohl die gedruckten Bücher, wie die Handschriften.
Ueber das Aeussere der Kataloge: Band- oder Zettelform, Format der
Zettel, handschriftliche oder gedruckte Kataloge liegen genaue
Erfahrungen vor. Jede Art hat gewisse Vorzüge, wann muss man sie
anwenden?

Soll man z.\,B. Kataloge drucken? Im Allgemeinen die der kleinen
Büchereien und der Nationalbibliotheken. Warum? In Amerika bezieht man
gedruckte Katalogzettel von der Library of Congress in Washington und
spart damit viel bibliographische Arbeit und Fehler; warum haben die
europäischen Bibliotheken diese Einrichtung nicht nachgeahmt? Welche
Nachteile sind damit verbunden? Die Deutsche Bücherei empfahl wiederholt
den andern Bibliotheken, die von ihr gedruckten Titel für Katalogzettel
zu verwenden.

Der erste gedruckte Katalog erschien 1602 in München, dann folgen die
grossen geistigen Zentren. Die Schweiz folgt sehr spät, 1744 mit dem
Zürcher Katalog. Heute hat sich die Gattung zu den hundertbändigen
Riesenleistungen der Kataloge des Britischen Museums, der
Nationalbibliothek in Paris und des preussischen Gesamtkatalogs
entwickelt. Jeder wird nach seiner Vollendung etwa 2 Millionen Titel
umfassen\footnote{Vgl. Handbuch der Bibliothekswissenschaft II, S. 302
  ff.}.

Ueber die Regeln, nach denen wir die Bücher für den Katalog aufnehmen,
kann ich auch nur das Allgemeinste sagen. «Nichts erscheint leichter»,
sagt der beste deutsche Kenner der Katalogisierungsregeln, Rudolf
Kaiser\footnote{Handbuch II, S. 261.}, «als Büchertitel alphabetisch zu
ordnen, und das ist auch der Fall bei kleinen Bibliotheken. Dass aber
die Fachleute sich schon jahrhundertelang bemüht haben, Regeln für die
Ordnung grosser Büchermassen zu schaffen, ohne bis heute zu
übereinstimmenden Ansichten zu kommen, das wird dem Laien immer
unverständlich bleiben; er hat eben keine Vorstellung von der
unendlichen Mannigfaltigkeit wie von der Kompliziertheit der Titel,
besonders wenn sie in fremden Sprachen vorliegen.» Schon die
alphabetische Ordnung der Verfassernamen bereitet Schwierigkeiten; die
einfachsten Fälle kennen Sie vom Gebrauch des Adress- und
Telephonbuches: Soll man Müller unter Muller oder Mueller suchen, Von
Orelli unter V oder O, kommt j hinter i wie in den romanischen Ländern
oder stellt man beide durcheinander? Der Erste, der hier mit einer
gedruckten Instruktion Ordnung zu schaffen versuchte, war wieder
Dziatzko, 1886\footnote{Dziatzko, Carl. Instruction für die Ordnung der
  Titel im alphabetischen Zettelkatalog der Universität zu Breslau,
  1886.}. Die preussische Kataloginstruktion von 1908 versucht mit 241
Regeln alle möglichen Fälle zu ordnen, die neue vatikanische gar mit
500, aber die Fülle der Erscheinungen spottet jeder Kasuistik. Es ist
die Aufgabe der Bibliothekwissenschaft möglichst einfache und
folgerichtige Regeln herauszuarbeiten, zunächst in den einzelnen
Sprachgebieten. Die internationale Verbreitung der Bibliographien und
Kataloge drängt auf internationale Regeln; dem steht aber im Wege, dass
alle bisherigen Bibliothekkataloge umgearbeitet werden müssten, und
diese Riesenarbeit lohnt sich nicht.

Neben dem alphabetischen Katalog ist ein \emph{Sachkatalog} nötig. Kann
man ihn nicht entbehren, wenn man die Bücher im Magazin systematisch
aufstellt? Bei kleinen Bibliotheken geht das, grosse Bibliotheken
setzten einst ihren Stolz darein, dank ihrer differenzierten
Aufstellung, keinen Sachkatalog nötig zu haben. Heute geht das nicht
mehr. Soll man nun die Titel nach Schlagworten alphabetisch wie im
Konversationslexikon ordnen oder systematisch nach Wissenschaften? Wählt
man ein System, muss man es den eigenen Beständen anpassen oder gibt es
Universalsysteme, die man fix und fertig übernehmen kann, wie die
Anhänger der Dezimalklassifikation behaupten? Hierüber tobt der Streit
der Meinungen.

Der \emph{Benutzungsdienst} erfordert wieder seine besondern Kenntnisse.
Theoretisch wichtig ist die Frage Ausleihe- oder Präsenzsystem,
praktisch die der kurzen Auskunft, wo ein Leser sich rasch informieren
kann über eine Frage, die ihn beschäftigt. Hier könnten unsere
Bibliotheken mehr leisten, wir kennen uns alle viel zu wenig in den
Nachschlagewerken unserer Lesesääle aus.

Am wichtigsten ist aber der Fragenkreis der \emph{Bücherkunde} oder
Bibliographie, auf den wir hier stossen. Was macht der Bibliothekar,
wenn Literatur gesucht wird, welche die Bibliothek nicht besitzt? Da er
eine genaue Kenntnis der Organisation des Bibliothekwesens seines Landes
besitzen soll, kann er den Suchenden an die einschlägige Bücherei weisen
oder das Buch von dieser kommen lassen. Auch die Fragen des zentralen
Auskunftbüros, der Gesamtkatalogisierung und des internationalen
Leihverkehrs gehören hierher.

Wenn nun aber der Suchende die \emph{gesamte} Literatur über ein Thema
zusammenstellen will, oder wenn er gar nicht weiss, was über den
Gegenstand geschrieben worden ist, mit dem er sich beschäftigt, so ist
es die Aufgabe des Bibliothekars, ihm mit seinem bibliographischen
Wissen zu Hilfe zu kommen und ihm die Bibliographien nachzuweisen, in
denen er die gesuchten Titel findet.

Die Bibliographien, d.\,h. die Verzeichnisse der erschienenen Bücher,
neuerdings auch der Zeitschriftenaufsätze, sind der älteste Zweig der
Bibliothekliteratur\footnote{Vgl. Gg. Schneider, Handbuch der
  Bibliographie, 4. A., 8. 1--35.}. Der erste Versuch stammt von
Johannes Trithemius, Liber de scriptoribus ecclesiasticis, Basel 1494.
Der «Vater der Bibliographie» ist Konrad Gessner, der deutsche Plinius,
mit seiner Bibliotheca universalis, Zürich 1545, in welcher er die
gesamte lateinische, griechische und hebräische gelehrte Literatur
seiner Zeit zusammenstellte. Diese Literaturgattung blühte üppig im
Zeitalter der Polyhistoren in Verbindung mit der Gelehrtengeschichte bis
zu Meusel und Jöcher. Mit 1800 stirbt diese Gattung ab, und die Spezial-
und Nationalbibliographien beginnen mit Brunet, Ebert u.\,a. Der
Buchhandel und die Nationalbibliotheken übernehmen die nationalen
Grundbibliographien; die wissenschaftliche Bibliographie folgt im 19.
und 20. Jahrhundert der Spezialisierung der Wissenschaften nach, bis
schliesslich 1925 im Auftrag des Völkerbundes von Marcel Godet ein Index
bibliographicus geschaffen werden muss, damit man nur die Uebersicht
hat, nicht über die Einzelbibliographien, -- denn 1885 zählte die
Abteilung Bibliographie in der Pariser Nationalbibliothek bereits 39 000
Nummern -- sondern nur über die bibliographischen Periodica. Die zweite
Auflage 1931 zählt 1900 Titel auf.

Vom Bibliothekar wird vielfach gefordert, dass er Bibliographien
herstellt. Trotzdem stammt der grössere Teil der Bibliographien nicht
von Bibliothekaren. Woher kommt das? Die Tätigkeit des Bibliothekars
veranschaulichen wir vielleicht am besten, wenn wir ihn als einen
Verwalter eines geistigen Warenhauses bezeichnen, in dem möglichst alles
vorhanden sein soll, von den Werken über Elektronen bis zum Bau des
Weltalls, vom Gilgamesch bis zum Bestseller der letzten Woche, vom
erotischen Roman bis zur Bibel. Alles soll der Bibliothekar kennen und
bei Anschaffungen ein Werturteil darüber haben. Sie sehen, der
Bibliothekar wird durch seinen Beruf geradezu zur Oberflächlichkeit
erzogen, er kann das ja nicht alles gründlich wissen. Die Waren, die er
verwaltet, die Bücher, haben ihren Wert in ihrem geistigen Inhalt, und
den kann nur der Fachmann richtig einschätzen. Der Bibliothekar muss
meist auf die Urteile der Fachleute abstellen, d.\,h. auf die kritischen
Zeitschriften. Die Beziehungen zum Buch bleiben äusserlich, und die
Gefahr ist gross, dass er in blosser Routine versinkt.

Hieraus ergeben sich die Schwierigkeiten seiner Beziehungen zur
Bibliographie. Natürlich übersieht er am ehesten, was alles
bibliographisch geleistet wird. Mit Recht wird von ihm ein gediegenes
polyhistorisches Wissen gefordert, und die tägliche Erfahrung lehrt,
dass er mit seiner äusserlichen Bücherkenntnis auch dem Fachmann
wertvolle Hinweise geben kann. Aber die Aufgabe, selbst Bibliographien
herzustellen, kann er als Bibliothekar nur formal erfüllen; er kann nur
quantitativ zusammenstellen, was alles über einen Gegenstand geschrieben
worden ist. Im möglichst vollständigen Zusammentragen des Rohstoffes
haben die Bibliothekare der Wissenschaft viele Dienste geleistet. Aber
die höhere Art der Bibliographie, die wertende, kann er nicht ausüben,
diese ist Sache des Fachmannes, der den innern Wert der Literatur
richtig einschätzen kann, und diese wird bei der wachsenden
Bucherzeugung immer nötiger. Praktisch ist die Scheidung natürlich nicht
so scharf, weil der Bibliothekar auf irgend einem Gebiet auch Fachmann
ist; es kommt mir hier nur darauf an, die Grenze der
Bibliothekwissenschaft klar herauszuarbeiten.

Hieraus erhellt auch die Schwierigkeit für den Bibliothekar,
bibliographische Kurse abzuhalten. Selbstverständlich wird ein
Bibliothekar, der z.\,B. mit Helveticis zu tun hat, diese Literatur auch
gut kennen. Aber eigentlich ist es Aufgabe der Professoren, die
Studenten während des Studiums in die Bibliographie schrittweise
einzuführen, weil es dann eben von innen heraus geschieht, aus der
wissenschaftlichen Fragestellung heraus und in organischem Zusammenhang
mit den Forschungen des Dozenten selbst. Der Bibliothekar ist nur
Gehilfe. Wenn die Professoren diese wichtige Erziehungsarbeit selbst in
die Hand nehmen, wird sie der Bibliothekar ihnen gerne überlassen.
Einstweilen kann er noch gute Dienste leisten, da die bisherige
Erziehung zur Bücherkunde nicht genügt. Die Schwierigkeit liegt in der
doppelten Aufgabe der Hochschullehrer begründet, die zugleich Forscher
und Lehrer sein sollen, und mancher ist eben mehr Forscher als Lehrer.
Dr.~Carl Benziger schreibt über die bibliographische Ausbildung:
«Mehrjährige Beobachtung hat mir gezeigt, wie unselbständig und
unaufgeklärt im allgemeinen die Hochschulstudenten die Bibliothek
benutzen; ein Mangel an Orientierung durch die Lehrer kann hier nicht
ausser Frage gestellt werden. Statt gleich im ersten Semester dem
angehenden Juristen einige Stunden Quellenkunde und bibliographische
Einführung in die grundlegenden Handbücher und Hilfswerke zu geben,
lernt er sie meistens nur durch Zufall kennen, wenn er sich bereits mit
der Dissertation beschäftigt\footnote{Wünsche und Richtlinien für das
  Schweiz. Bibliothekwesen. In: Wissen und Leben 1913, Heft 23--24.}.»

Der Unterricht in der Bücherkunde ist nicht leicht, weil der Anfänger
ihren Nutzen nicht recht einsieht. Man muss ihn nach und nach in ihre
Verfeinerungen einführen, wenn ihn die Masse der Titel nicht kopfscheu
machen soll. Man beginne mit einer sorgfältigen Auswahl des Wichtigsten,
der Lehrbücher und Monographien, und führe ihn erst nach und nach in die
Spezialarbeiten und Zeitschriftenaufsätze ein\footnote{Vgl. Fick,
  Richard. Die bibliographische Schulung des Bibliothekars, ZfB 1928, S.
  551--61. Schneider, Georg. Die Bibliographie an den wissenschaftlichen
  Bibliotheken, Festschrift Ernst Kuhnert 1928, S. 322--26.}.

Wir kommen zum zweiten Gebiet, mit dem sich die Bibliothekwissenschaft
befasst, zur \emph{Buch- und Büchereigeschichte}.

Für die Geschichte der \emph{Schrift}, der verschiedenen Schreibstoffe:
Stein, Ton, Papyrus, Pergament usw., sowie über die verschiedenen
Schriftsysteme, Bilder-, Silben- und Buchstabenschrift übernimmt die
Bibliothekwissenschaft die fertigen Ergebnisse von der Völkerkunde und
den Sprachwissenschaften, die Entwicklung der antiken und
mittelalterlichen Schrift von der Paläographie, die der
\emph{Buchmalerei} von der Kunstwissenschaft. Der
Bibliothekwissenschafter fasst ihre Ergebnisse unter dem Gesichtspunkt
des Buches zusammen, gelegentlich arbeitet er Hand in Hand mit ihnen.

Auf eigenen Boden kommt die Bibliothekwissenschaft erst mit der
Erfindung der Buchdruckerkunst. Da ist zunächst das ganze schwierige
Fragengebiet um Gutenberg: Wer ist der Erfinder, worin besteht die
Erfindung, welches sind die Vorstufen des Buchdrucks mit beweglichen
Lettern, welches sind echte Erzeugnisse Gutenbergs usw. Diese Frage
können freilich nur Spezialisten lösen.

Daran schliesst sich die \emph{Wiegendruckforschung} an. Aus der Zeit
von etwa 1445 bis 1500 sind von 1100 Druckern 30--40 000 verschiedene
Drucke in 450 000 Exemplaren erhalten. Eine stattliche Anzahl von ihnen
gibt Erscheinungsjahr, Druckort und Drucker nicht an. Die Aufgabe ist,
sie alle ausfindig zu machen, zu vergleichen, sie nach allerhand
Anhaltspunkten, besonders aus dem Typenmaterial nach der
Proctor-Häblerschen Methode, aber auch nach den Holzschnitten,
Initialen, Satzart, Papier, Wasserzeichen usw. örtlich und zeitlich
festzulegen und sie einem bestimmten Drucker zuzuschreiben, und so einen
Stammbaum der Ausbreitung des Buchdrucks zu gewinnen. Diese Arbeit wird
zur Zeit im wissenschaftlichen Grossbetriebe durch die Berliner
Zentralstelle geleistet, welche den Gesamtkatalog der Wiegendrucke
herausgibt. Dieser hat zum Ziel, alle Wiegendrucke mit ihren Fundorten
zu verzeichnen. Damit wird eine wichtige Quelle geschaffen für das
geistige Leben jener Zeit, denn selbstverständlich entwickelt sich der
Buchdruck am schnellsten in den damaligen geistigen Zentren, Köln,
Strassburg, Basel, den süddeutschen Städten, Rom, Venedig, Paris
usw.\footnote{Ein hübsches Beispiel aus späterer Zeit: 1580 kommen die
  Jesuiten nach Freiburg im Uechtland, 1581 folgt die Einführung des
  Buchdrucks.}. Wann wurde da zuerst, und was wurde gedruckt? Wenn auch
die Inkunabelbestimmung so verfeinert ist, dass sie am besten
Spezialisten übernehmen, so bleibt doch dem Bibliothekar die Pflicht,
die einheimischen Inkunabeln zu kennen. Dann ist auch seine Aufgabe, aus
der Ortsgeschichte und aus den Archiven die nähern Umstände der
einheimischen Druckgeschichte zu erforschen. Für Basel hat in dieser
Hinsicht Stehlin\footnote{Stehlin, Karl. Regesten zur Geschichte des
  Buchdrucks bis 1500, in: Archiv für Geschichte des deutschen
  Buchhandels, Bd. 11--12, 1887--88.} alle archivalischen Notizen
mustergiltig zusammengestellt.

Die Bedeutung der Druckergeschichte lässt sich gut erläutern an den
Luther- und Reformationsdrucken. Sie hat der Lutherphilologie gute
Dienste geleistet durch die Klassifizierung der Drucke der einzelnen
Werke, ebenso ist sie aufschlussreich für die zeitliche und örtliche
Festlegung der Flugschriftenliteratur jener Zeit.

Die \emph{Geschichte des europäischen Buchdrucks seit 1500} ist durch
folgende Namen gekennzeichnet: Im 16. Jahrhundert durch Aldus Manutius
in Venedig, die Giunta in Florenz, die Familie Etienne in Paris und Genf
und Plantin in Antwerpen; im 17. Jahrhundert durch die Elzevier in
Holland und die Imprimerie royale in Paris; im 18. Jahrhundert durch
Breitkopf und Härtel in Leipzig, Unger in Berlin, Baskerville in London,
die Didot in Paris und Bodoni in Parma. Im 19. Jahrhundert führt die
Ausbildung der Maschinen zunächst zum Verfall, bis William Morris 1891
mit der Kelmscott Press einen neuen Aufschwung bringt.

Die Geschichte des Buchdrucks an den einzelnen Orten veranschauliche ich
an den wichtigsten Leistungen der Schweiz. Sie besitzt für ihre
Druckgeschichte verschiedene gute Leistungen. Der grösste Teil des
Feldes ist aber noch nicht abschliessend angebaut. Die Aufgabe ist, die
Druck- und Verlagkataloge zu rekonstruieren.

Für \emph{Basel} ist die Frühzeit gut bearbeitet, dagegen haben die
Basler Drucker seit 1550, die König, Brandmüller, Decker, Schorndorf,
Thurneysen, Mechel, Haas noch keinen Bearbeiter gefunden. Besser steht
\emph{Genf} da. Durch die Arbeit von E. H. Gaullieur: Etude sur la
typographie genevoise du 15\textsuperscript{e} jusqu'au
19\textsuperscript{e} siecle, 1855, haben wir einen guten Ueberblick
über die Drucker der Reformationszeit und der Aufklärung\footnote{Weigelt,
  Gertrude. Les éditions Fick. Gutenbergmuseum Jg. 21, 1935.}. In
\emph{Lausanne} blüht Grasset, der Drucker Hallers, und in
\emph{Yverdon} F. B. de Felice, von dem K. J. Lüthi\footnote{Gutenbergstube,
  Jg. 1, 1915 f.} eine Bibliographie zusammengestellt hat. Für die
Geschichte des \emph{Berner} Buchdrucks besitzen wir wertvolle
Vorarbeiten von Dr.~Adolf Fluri\footnote{Lüthi, Karl J. Versuch einer
  Bibliographie zur heimischen Druck- und Pressegeschichte.
  Gutenbergmuseum Jg. 10, 1924. Fluri, Ad.: Gutenbergmuseum Jg. 6, 1920
  ff., 16, 1930. Die Arbeit von A. M. Lacroix ist nur handschriftlich in
  Genf vorhanden.}. Sein Aufsatz über den ersten Berner Drucker Apiarius
ist durch seinen zu frühen Tod leider Bruchstück geblieben. Wertvoll ist
sein «Versuch einer Bibliographie der heidnischen Kirchengesangbücher».
Den Wert solcher mühseliger und entsagungsvoller bibliographischer
Vorarbeiten für die Forschung kann man etwa erkennen aus Paul Wernles
Kirchengeschichte der Schweiz im 18. Jahrhundert, der gerade die
Wichtigkeit dieser anonymen Kirchen- und Schulliteratur als Lesestoff
der breiten Schichten herausgestellt hat. Für die Obrigkeitliche
Buchdruckerei ist die Aufgabe noch zu lösen. Für \emph{Luzern} und die
Urschweiz haben R. und Fritz Blaser\footnote{Gutenbergstube Jg. 2, 1916,
  Gutenbergmuseum Jg. 18, 1932 ff.} tüchtige Arbeiten geliefert. Die
Tätigkeit Froschauers in \emph{Zürich} hat Rudolphi zusammengestellt.
Zwingli und Conrad Gessner sind dessen berühmteste Autoren. Max Rychner
hat einen allgemeinenen Ueberblick über den Verlag Orell Füssli und
seine Vorgänger gegeben, eine genaue wissenschaftliche Durchdringung
fehlt noch. Carl Benziger hat das Werk der Stiftsdruckerei
\emph{Einsiedeln} behandelt, ebenso hat die Tipografia elvetica in
\emph{Capolago}, welche 1830--53 für das Risorgimento eine wichtige
Rolle gespielt hat, von Rinaldo Caddeo in Mailand die abschliessende,
reich ausgestattete Geschichte und Bibliographie in zwei Bänden
erhalten. K. J. Lüthi\footnote{Gutenbergstube Jg. 3, 1917.} hat eine
Geschichte der \emph{romanischen} \emph{Bibelausgaben} des 16.--18.
Jahrhunderts geliefert, sowie Zusammenstellungen der \emph{hebräischen},
\emph{äthiopischen} und \emph{chinesischen} Drucke der Schweiz.

Die \emph{Buchillustration} der Schweiz hat stets Beachtung gefunden,
teils wegen ihrer Eigenart, teils, weil diese Bücher viel gesammelt
werden. Eine bequeme Uebersicht gibt F. C. Lonchamp: Manuel du
bibliophile suisse 1922.

Die \emph{Einbandforschung} steht erst in den Anfängen. Hier wartet noch
reiches Material in den Bibliotheken und Museen. Ueber die Schweizer
\emph{Ex-libris} schrieb L. Gerster «Die schweizerischen
Bibliothekzeichen» 1898 und Agnes Wegmann gibt in Zusammenarbeit mit der
Zentralbibliothek Zürich ein grosses Werk heraus.

An die Geschichte des Buchdrucks schliesst sich die Geschichte des
\emph{Buchhandels} an, für die wir für das deutsche Sprachgebiet auf das
grundlegende Werk von Fr. Kapp und J. Goldfriedrich «Geschichte des
deutschen Buchhandels» 1886 ff. angewiesen sind. Die «Festgabe zum
75jährigen Jubiläum des schweizerischen Buchhändlervereins 1849--1924»
ist nur eine Skizze.

Interessanter ist die Geschichte der grossen \emph{Verleger.} Das 18.
Jahrhundert bringt die endgiltige Trennung von Druckerei und Verlag, der
technischen Druckkunst und des Unternehmens mit Geistesgut. Wenn ich die
Namen Friedrich Nicolai, Göschen, Cotta, Hoffmann und Campe, Georg
Hirth, Albert Langen, Eugen Diederichs und Samuel Fischer herausgreife,
so ist die enge Verbundenheit der Buchgeschichte mit der Entwicklung der
deutschen Literatur deutlich. Die Verlagsgeschichte gibt wichtige
Aufschlüsse über die Interessen des Publikums. Jedes dichterische Werk
muss zuerst einen Verleger finden, der ihm den Weg zum Erfolg und in die
Literaturgeschichte öffnet. Wer würde es denken, dass Mommsens Römische
Geschichte von seinem Schwiegervater, dem Verleger Karl Reimer angeregt
worden ist? Salomon Hirzel gab den Anstoss zu Freytags Bildern aus der
deutschen Vergangenheit und zu Michael Bernays Jungem Goethe, Teubner
für die Kultur der Gegenwart, Oskar Beck für Iwan Müllers Handbuch der
klassischen Altertumswissenschaft. Für die Schweiz wäre für das 18.
Jahrhundert Orell, Gessner, Füssli und Co.~zu nennen, für die 1840er
Jahre das Literarische Comptoir in Zürich und Winterthur und die
gleichgesinnten Verlage, deren Tätigkeit Werner Näf und Hans Gustav
Keller dargestellt haben\footnote{Näf, W. Das literarische Comptoir
  Zürich und Winterthur. Neujahrsblatt der Literar. Gesellschaft in
  Bern, N. F. Heft 7, Bern 1929. Keller, Hans Gustav. Die politischen
  Verlagsanstalten und Druckereien in der Schweiz und ihre Bedeutung für
  die Vorgeschichte der deutschen Revolution von 1848. Diss. Bern 1935 =
  Berner Untersuchungen zur allg. Geschichte, H. 8.}.

Aber auch der Einfluss der Verleger auf das Publikum zeigt
aufschlussreiche Zusammenhänge. 1840--1865 wurden Scott (schon seit
1825), Marryat, Dickens, Bulwer, Dumas, Sue, Sand usw. in Zehntausenden
von Exemplaren verbreitet, weil die Verleger diesen Ausländern kein
Honorar zahlen mussten. Als 1867 der Schutz für die deutschen Klassiker
Lessing, Wieland, Herder, Goethe, Schiller fiel, war die Bahn frei für
die Klassikerausgaben, die sich dann bis zum Weltkrieg auf die Höhe der
Ausgaben des Georg Müller Verlages entwickelt haben. Reclams
Universalbibliothek hat seit 1867 Klassiker in 18 Millionen Nummern
verbreitet, die antiken Schriftsteller in 8 ½, philosophische Literatur
in 5 (Kant 800 000), Ibsen 4 ½ Millionen. Die Verlegergeschichte kann
für die Geschichte der führenden Geister nur äusserliche Daten liefern,
dafür erfasst sie geistige Massenerscheinungen, die Geschichte des
Publikumsgeschmackes\footnote{Schulze, Friedrich. Der deutsche
  Buchhandel und die geistigen Strömungen der letzten 100 Jahre. 1925.}.

Zu den historischen Aufgaben des Bibliothekars gehört auch die
\emph{Geschichte der Bibliotheken}. Vorbild ist immer noch die
Darstellung, welche Ebert 1822 von der Dresdner Bibliothek gegeben hat.
In der Schweiz haben wir Geschichten von der Universitätsbibliothek
Basel und den Stadtbibliotheken von Zürich und Bern. Die innere
Geschichte der Sammlungen ist noch nirgends geschrieben worden. Lessing
hat bei der Besprechung von Jakob Burckhards Historia bibliothecae
augustanae gesagt, die Hauptsache sei, zu zeigen, was sie der
Gelehrsamkeit und den Gelehrten genützt haben. Adalbert Wagner hat die
Privatbibliothek des Freiburger Humanisten Peter Falck rekonstruiert und
damit eine Quelle zur Geschichte des Humanismus erschlossen. Wieviel man
aus der Wiederherstellung der Bibliothek für die Geistesgeschichte des
Besitzers gewinnen kann, zeigt meisterhaft Walther Köhler: Huldrych
Zwinglis Bibliothek (84. Neujahrsblatt zum Besten des Waisenhauses in
Zürich 1921).

Adolf von \emph{Harnack} hat in einem Aufsatz: «Die Professur für
Bibliothekswissenschaften in Preussen»\footnote{Vossische Zeitung, 24.
  Juli 1921. Abgedruckt in Erforschtes und Erlebtes, 1923, S. 218--223.}
der Bibliothekwissenschaft eine grosse Aufgabe für die Zukunft gestellt.
Er meint, alle bisher aufgeführten Aufgaben genügten nicht, um eine
ordentliche Professur zu rechtfertigen, deren Aufgabe sei vielmehr eine
volkswirtschaftliche, nämlich der Volkswirtschaft mit Geistesgut. «Ihr
Objekt ist das gesamte heutige Buchwesen, einschliesslich der
Zeitschriften und Zeitungen, wissenschaftlich, pädagogisch, technisch
und kommerziell betrachtet, zunächst in Deutschland, dann auch in allen
Kulturstaaten.» «Ihr Inhaber, der natürlich die Bibliothektechnik und
-kunde beherrschen muss, \ldots{} muss erstlich \ldots{} die gesamte
Statistik des Buchwesens überschauen; er muss die Bedingungen der
Bücherproduktion kennen und in das Zeitschrifts- und Zeitungswesen
eingedrungen sein. Zweitens aber muss er das Volksbibliothekswesen
studiert und sich die Aufgaben des Volksbildungswesens, soweit es durch
Bibliotheken aufzubauen und zu erhalten ist, klargemacht haben \ldots{}
Die Hebung des Bildungsstandes der ganzen Nation, die Ueberwindung
überspannter parteipolitischer Gegensätze, \ldots{} die Uebermittlung
der Schätze unserer klassischen Literatur und die Einführung in die
echte populärwissenschaftliche Literatur ist von hieraus zu erreichen,
soweit sie sich erreichen lässt. Hier organisatorische Richtlinien
aufzustellen und mit den organisatorischen Kräften des Staats und der
Gemeinden in Verbindung zu treten, ist die Aufgabe des Professors für
Bibliothekswissenschaften.» Drittens ist ein Sachkenner gefordert, der
neben dem Buchhändler-Börsenverein unabhängig die Wissenschaft von der
Nationalökonomik des Buches selbständig zum Ausdruck bringt, «der die
Verhältnisse, Aufgaben und Schranken des Verlegerberufs und des
Buchhandels ebenso gründlich kennt wie die grossen Bedürfnisse der
Wissenschaft und Literatur in Bezug auf die Produktion und Verbreitung
des Buchs und der Zeitschrift unter den verschiedenen hier
einschlagenden Punkten. Und über Deutschland muss sein Blick
hinausreichen in die andern Länder, sowohl um zu lernen, was dort zu
lernen ist, als auch, um das deutsche Buch zu schützen».

Das ist freilich eine Aufgabe, die eine volle Arbeitskraft erfordert,
und die bis jetzt noch kaum in Angriff genommen ist.

Endlich wird gelegentlich behauptet, die Bibliothekare sollten sich auf
dem Gebiete der Gelehrtengeschichte und der Geschichte der
Universitäten, Akademien usw. betätigen\footnote{ZfB 1933, S. 528--35}.
Gewiss sollen sie für ihren Beruf gute Kenntnisse davon haben, aber hier
selbständig zu arbeiten ist nicht ihre Aufgabe. Sie können als solche
nur Vorarbeit leisten. Es ist nicht einzusehen, warum ein Bibliothekar
besonders befähigt sein sollte, etwa die Geschichte der Schweizerischen
Naturforschenden Gesellschaft zu schreiben. Beide Gebiete scheiden als
Arbeitsfelder der Bibliothekwissenschaft aus.

\hypertarget{section-2}{%
\subsection{3.}\label{section-2}}

Wir kommen endlich zur \emph{quaestio juris}.

a) Können wir von einer Bibliothekswissenschaft reden?

b) Gehört sie als Lehrfach an die Hochschulen?

Martin Schrettinger hat 1808 den Ausdruck Bibliothekwissenschaft geprägt
und versteht darunter alle Kenntnisse zur zweckmässigen Einrichtung
einer Bibliothek, also nur Bibliotheklehre. Ihm folgen die ältern
Vertreter Ebert, Molbech und Zoller. Friedrich Buhmann ist m. W. der
erste, der 1874 unter Bibliothekwissenschaft sowohl Bibliotheklehre wie
Bibliothekkunde versteht. Ihm schliessen sich Arnim Graesel 1890,
Ferdinand Eichler 1897 und Karl Dziatzko 1900 an\footnote{K. Dziatzko.
  Die Beziehungen des Bibliothekwesens zum Schulwesen und zur
  Philologie. Neue Jahrbücher f. d. klass. Altertum, Jg. 3, 1900 II, S.
  94--102.}. Eichler fasst unsern Gegenstand als Buch- und
Bibliothekwesen, Dziatzko als Schrift-, Buch- und Bibliothekwesen
zusammen. Dieser Begriff ist für die Gegenwart massgebend geworden
(Leidinger) und wird sich mit Milkaus Handbuch ganz durchsetzen.

Kann man diese Summe des Wissens als Wissenschaft bezeichnen? Das ist
zunächst eine Frage des Sprachgebrauchs. Schrettinger und seine
Zeitgenossen haben ihn ohne Bedenken verwendet. Die zweite Hälfte des
19. Jahrhunderts hat unter dem Einfluss des Positivismus nicht eine
strengere, sondern eine engere Auffassung von Wissenschaft gehabt; man
wollte als Wissenschaft nur gelten lassen, was sich auf
naturwissenschaftliche Gesetze zurückführen liesse. Daher sprach man vom
Zentralblatt für Bibliothek\emph{wesen} oder «ebenso schüchtern wie
unlogisch» (Milkau) von einer Professur für
Bibliothek\emph{hilfswissenschaften}. Heute spricht man von Finanz-,
Handels-, Kriegs-, Theater-, Zeitungs-, Betriebswissenschaften und
doziert sie auf den Hochschulen. Es liegt also kein Grund vor, dem
System der Erfahrungen über das Buch- und Büchereiwesen den Namen
Wissenschaft zu verweigern. Wer es grundsätzlich doch tun will, dem sei
es unbenommen.

Hinter dem Sprachgebrauch steht allerdings die Erkenntnistheorie. Die
Unsicherheit über den Begriff Wissenschaft wird andauern, bis eine
Kritik der historischen Vernunft geliefert worden ist. Wer unserem
Gebiet den Wissenschaftcharakter abstreitet, muss ihn auch der
Geschichtsforschung weigern. Und seit der Relativitätstheorie sind auch
die Naturwissenschaften nicht mehr «exakt». Was bleibt dann überhaupt
als Wissenschaft übrig? Und wie will man dieses Wissen nennen? Die
Kritiker, welche uns den Wissenschaftcharakter absprechen, wissen meist
nicht, um was es geht. Wir Bibliothekare haben keinen Grund, unsern nach
den anerkannten wissenschaftlichen Methoden erarbeiteten Kenntnissen die
Wissenschaftlichkeit abzuerkennen. Allfälligen Missverständnissen
vorzubeugen, bemerke ich noch, dass ich die Wissenschaft nicht
überschätze, wie die dem seelenlosen Positivismus verfallenen
Wissenschaftler der Vorkriegszeit, die vor lauter Hingabe an das Objekt
den Menschen vergassen. Die praktische Vernunft steht immer über der
theoretischen.

Inhaltlich hat die Bibliothekwissenschaft ein zweigeteiltes Feld: 1. Die
Betriebswissenschaft der Bibliotheken mit der Bibliographie und 2. die
Geschichte des Buch- und Bibliothekwesens. Ihr Feld ist schmal,
erfordert aber intensive Arbeit. Ihre Tätigkeit geht leicht in andere
Arbeitsgebiete: Paläographie, Kunstgeschichte, Statistik usw. über, sie
muss von ihnen lernen oder mit ihnen zusammenarbeiten. Aber das müssen
andere Wissenschaften wie Geschichte und Geographie auch. Sie ist nur
eine Hilfswissenschaft, die bestimmt ist, den andern Wissenschaften, vor
allem den historischen zu dienen.

Hat sie ein Recht, an der Hochschule gelehrt zu werden? Die wachsende
Zahl der Vorlesungen in Deutschland bejaht diese Frage ohne weiteres.
Die Technisierung der Wissenschaft, die zunehmende Ausdehnung und
Unübersichtlichkeit der literarischen Produktion steigert das Bedürfnis
bei den Studierenden, die nötige technische Schulung in der
Bibliographie zu erhalten, um sich in den Büchermassen zurecht zu
finden. Praktisch scheint mir dies die nötigste und zugleich dankbarste
Aufgabe des Bibliothekwissenschafters zu sein. Sie entspricht dem Beruf
des Bibliothekars, der in erster Linie ein Vermittlerberuf ist. Der
moderne Bibliothekar hat das Bestreben, die ihm anvertrauten Schätze so
viel wie möglich zugänglich zu machen; ein Buch, das nicht gelesen wird,
hat seinen Zweck verfehlt. Wenn er daneben noch Hörer findet, welche
sich für die Geschichte des Buches begeistern, um so besser. Ist doch
das Buch immer noch das wichtigste Handwerkzeug des Wissenschafters, bei
dem es sich wohl lohnt, seine Geschichte zu kennen\footnote{Die
  Bibliotheken gewinnen ihrerseits durch die Verbindung mit den
  Hochschulen. Die Beziehungen zu ihren wichtigsten Benützern gestalten
  sich enger und der Bibliothekbetrieb wird dadurch vor der Gefahr
  bewahrt, sich einem Sonderdasein zu ergeben.}.

Ich glaube mit meinen Ausführungen den Beweis erbracht zu haben, dass
Bibliothekwissenschaft ein streng wissenschaftlich betriebenes Fach ist,
das einen Platz an der Hochschule haben darf, zugleich aber auch, dass
ich über die Grenzen meiner Wissenschaft im klaren bin.

Ferdinand Eichler hatte 1923 behauptet, Bibliothekwissenschaft sei eine
Wissenschaft über den Wissenschaften. Dieser Ueberschätzung ist Adolf
von Harnack, der im Nebenamt 15 Jahre lang Generaldirektor der Kgl.
Bibliothek in Berlin war, entgegengetreten. Seine Ausführungen sind das
beste, was über Bibliothekwissenschaft gesagt worden ist. Er hat dem
Bibliothekwissenschafter ungefähr die Aufgabe zugewiesen, wie ich sie
umrissen habe und den Hauptton ganz auf die praktische Aufgabe gelegt.
«Was bleibt also der Bibliothekwissenschaft? Wenig oder viel, wie man es
nehmen will -- das Buch (das literarische Dokument) als \emph{solches},
wie es der Träger der Literatur und der Wissenschaft ist, das Buch mit
seiner generellen und doch höchst individuellen Naturgeschichte von
seiner Entstehung bis zu seinem Einband, das Buch nach seinen Fundorten
und seiner Verbreitung, das Buch als Gegenstand der Sammlung, weil
\emph{ein} Buch kein Buch ist. Die Bibliothekwissenschaft, von der
Bibliophilie belebt, ist die Summe der Kenntnisse von der Bibliothek und
dem Buch an sich -- man kann das \emph{auch} Wissenschaft nennen -- aus
welchen sich die Kunst, die Bücher zu sammeln, zu finden und zu
konservieren und den Interessenten zum Gebrauch darzubieten von selbst
ergibt. Der letztere Zweck ist der Haupt- und Endzweck dieser
«Wissenschaft», -- auf \emph{Dienstleistung} ist sie ganz und gar
eingestellt. Nennt sich der Geistliche «minister verbi divini», so soll
sich der Bibliothekar «minister verbi scripti et impressi» nennen und
zugleich «minister litterarum et artium studiosorum»\footnote{ZfB 1923,
  S. 532.}.

In diesem Sinne Harnacks Bibliothekwissenschaft zu treiben, soll mein
Bestreben sein.

%autor
\begin{center}\rule{0.5\linewidth}{\linethickness}\end{center}

\textbf{Dr.~Hans Lutz}, Bibliothekar an der Universität Basel,
schweizerische Landesbibliothek, ab 1936 Privatdozent für
Bibliothekswissenschaft (Universität Bern). († 1938)

\end{document}
