\documentclass[a4paper,
fontsize=11pt,
%headings=small,
oneside,
numbers=noperiodatend,
parskip=half-,
bibliography=totoc,
final
]{scrartcl}

\usepackage{synttree}
\usepackage{graphicx}
\setkeys{Gin}{width=.4\textwidth} %default pics size

\graphicspath{{./plots/}}
\usepackage[ngerman]{babel}
\usepackage[T1]{fontenc}
%\usepackage{amsmath}
\usepackage[utf8x]{inputenc}
\usepackage [hyphens]{url}
\usepackage{booktabs} 
\usepackage[left=2.4cm,right=2.4cm,top=2.3cm,bottom=2cm,includeheadfoot]{geometry}
\usepackage{eurosym}
\usepackage{multirow}
\usepackage[ngerman]{varioref}
\setcapindent{1em}
\renewcommand{\labelitemi}{--}
\usepackage{paralist}
\usepackage{pdfpages}
\usepackage{lscape}
\usepackage{float}
\usepackage{acronym}
\usepackage{eurosym}
\usepackage[babel]{csquotes}
\usepackage{longtable,lscape}
\usepackage{mathpazo}
\usepackage[normalem]{ulem} %emphasize weiterhin kursiv
\usepackage[flushmargin,ragged]{footmisc} % left align footnote
\usepackage{ccicons} 

%%%% fancy LIBREAS URL color 
\usepackage{xcolor}
\definecolor{libreas}{RGB}{112,0,0}

\usepackage{listings}

\urlstyle{same}  % don't use monospace font for urls

\usepackage[fleqn]{amsmath}

%adjust fontsize for part

\usepackage{sectsty}
\partfont{\large}

%Das BibTeX-Zeichen mit \BibTeX setzen:
\def\symbol#1{\char #1\relax}
\def\bsl{{\tt\symbol{'134}}}
\def\BibTeX{{\rm B\kern-.05em{\sc i\kern-.025em b}\kern-.08em
    T\kern-.1667em\lower.7ex\hbox{E}\kern-.125emX}}

\usepackage{fancyhdr}
\fancyhf{}
\pagestyle{fancyplain}
\fancyhead[R]{\thepage}

% make sure bookmarks are created eventough sections are not numbered!
% uncommend if sections are numbered (bookmarks created by default)
\makeatletter
\renewcommand\@seccntformat[1]{}
\makeatother


\usepackage{hyperxmp}
\usepackage[colorlinks, linkcolor=black,citecolor=black, urlcolor=libreas,
breaklinks= true,bookmarks=true,bookmarksopen=true]{hyperref}
%URLs hart brechen
\makeatletter 
\g@addto@macro\UrlBreaks{ 
  \do\a\do\b\do\c\do\d\do\e\do\f\do\g\do\h\do\i\do\j 
  \do\k\do\l\do\m\do\n\do\o\do\p\do\q\do\r\do\s\do\t 
  \do\u\do\v\do\w\do\x\do\y\do\z\do\&\do\1\do\2\do\3 
  \do\4\do\5\do\6\do\7\do\8\do\9\do\0} 
% \def\do@url@hyp{\do\-} 
\makeatother 

%meta
%meta

\fancyhead[L]{A. Vita, F. Härting\\ %author
LIBREAS. Library Ideas, 34 (2018). % journal, issue, volume.
\href{http://nbn-resolving.de/}
{}} % urn 
% recommended use
%\href{http://nbn-resolving.de/}{\color{black}{urn:nbn:de...}}
\fancyhead[R]{\thepage} %page number
\fancyfoot[L] {\ccLogo \ccAttribution\ \href{https://creativecommons.org/licenses/by/4.0/}{\color{black}Creative Commons BY 4.0}}  %licence
\fancyfoot[R] {ISSN: 1860-7950}

\title{\LARGE{Das Studium am IBI in der Gegenwart}}% title
\subtitle{Entwicklung der aktuellen Studienordnungen am IBI}
\author{Annemarie Vita, Felicitas Härting} % author

\setcounter{page}{1}

\hypersetup{%
      pdftitle={Das Studium am IBI in der Gegenwart: Eindrücke von Bachelor- und Master-Studierenden sowie frisch gebackenen Alumni},
      pdfauthor={Annemarie Vita, Felicitas Härting},
      pdfcopyright={CC BY 4.0 International},
      pdfsubject={LIBREAS. Library Ideas, 34 (2018).},
      pdfkeywords={Bibliothekswissenschaft, Institut für Bibliotheks- und Informationswissenschaft, Humboldt-Universität zu Berlin, Studium, Studierende, Befragung},
      pdflicenseurl={https://creativecommons.org/licenses/by/4.0/},
      pdfcontacturl={http://libreas.eu},
      baseurl={http://libreas.eu},
      pdflang={de},
      pdfmetalang={de}
     }



\date{}
\begin{document}

\maketitle
\thispagestyle{fancyplain} 

%abstracts
\begin{abstract}
Fragenentwicklung: Gesamtes Projektteam

Durchführung, Auswertung und Korrektur: Annemarie Vita, Felicitas
Härting
\end{abstract}

%body
Zum 90-jährigen Jubiläum des Instituts für Bibliotheks- und
Informationswissenschaft sollen natürlich auch die aktuellen
Studierenden zu Wort kommen: Warum studieren sie am Institut, was
verbinden sie mit dem IBI und wie sehen ihre Pläne für die Zeit nach dem
Abschluss aus? Das Projektteam \enquote{90 Jahre IBI} hat daher
Bachelor- und Master-Studierende, darunter auch Alumni, befragt.

\hypertarget{was-hat-dich-dazu-bewegt-dein-studium-am-ibi-zu-beginnen}{%
\subsubsection{Was hat dich dazu bewegt, dein Studium am IBI zu
beginnen?}\label{was-hat-dich-dazu-bewegt-dein-studium-am-ibi-zu-beginnen}}

Patricia: Ich wollte neben Geschichtswissenschaften etwas studieren, bei
dem ich einen \enquote{praktischen} Nutzen sehe. Nachdem ich ein
Praktikum in einer Bibliothek absolvierte, habe ich mich entschlossen,
mein Nebenfach Bibliotheks- und Informationswissenschaft zum Hauptfach
zu machen.

Nico: Mehr oder weniger Zufall. Das Studienfach klang interessant und da
ich keine richtige Idee hatte, was ich sonst studieren soll, habe ich
einfach angefangen. Es hat mir so gut gefallen, dass ich geblieben bin
und mittlerweile meinen Master mache.

Mandy: Mein Hauptfach ist Geschichte und ich habe mich schon immer dafür
interessiert. Während meines Freiwilligen Sozialen Jahrs habe ich in
einer Gedenkstätte gearbeitet und wusste, dass sich Geschichte, Archive
und Bibliotheken in einer einzigartigen Beziehung befinden. Innerhalb
einer Gedenkstätte befinden sie sich schon fast in einer perfekten
Symbiose. Ein passenderes Zweitfach als Bibliotheks- und
Informationswissenschaft gibt es für mich nicht.

Annika: Die Möglichkeit des Kombi-Bachelors und den Studiengang an einer
Universität zu studieren, die einen guten Ruf genießt. Außerdem habe ich
schon ein Jahr in Berlin gelebt und finde die Stadt toll.

Julia: Die Studieninhalte sind interessant, und nach Abschluss eröffnen
sich viele berufliche Einstiegsmöglichkeiten.

Sonja: Ich fand die Fachrichtung interessant und wollte zudem lieber an
einer Universität als an einer Fachhochschule studieren.

Hagen: In meiner Ausbildung habe ich gemerkt, wie Informationssoftware,
-praktiken und \mbox{-verwaltung} teilweise keinen hohen Reifegrad aufgewiesen
haben. Im Zuge des stetigen Anstiegs elektronischer Daten hat es mich
interessiert, wie zukünftig mit der wachsenden Menge umgegangen werden
muss.

Lars: Der Zufall führte mich in der zehnten Klasse zu einem Praktikum in
einer Bibliothek, das mir so viel Spaß gemacht hat, dass ich nach dem
Abitur meine Ausbildung zum Fachangestellten für Medien- und
Informationsdienste (FaMI) begann. Nach einem Jahr im Beruf habe ich das
Studium begonnen.

Andreas: Nach meiner FaMi-Ausbildung und einem Jahr Berufspraxis hat
mich zum einen der Wissensdurst gepackt. Zum anderen habe ich gemerkt,
dass für die wirklich interessanten Stellen ein Bachelor-Abschluss nötig
ist.

Lisa: Die Tätigkeit als FaMi war eintönig und eingeschränkt. Außerdem
wollte ich gerne in Berlin studieren.

\hypertarget{was-verbindest-du-mit-dem-ibi}{%
\subsubsection{Was verbindest du mit dem
IBI?}\label{was-verbindest-du-mit-dem-ibi}}

Patricia: Es ist ein Ort zum Lernen und Lehren, vor allem aber ein Ort,
an dem ich mich wohlfühle. Zu meinem Erstaunen herrscht hier keine
Anonymität und Gleichgültigkeit gegenüber den Studierenden, wie es
anderswo der Fall ist.

Nico: Neben meinem Studium arbeite ich als Studentische Hilfskraft am
Institut. Daher verbinde ich mit dem IBI nicht nur ein interessantes und
abwechslungsreiches Studium, sondern auch eine angenehme
Arbeitsatmosphäre mit hilfsbereiten Dozenten und Kollegen.

Mandy: Im IBI und dem Studium verbinden sich für mich der nostalgische
Charme des Analogen, und der neue Glanz der digitalen Welt.

Annika: Das alte, baufällige Gebäude, ausgesprochen freundliche und
motivierte Dozenten. Zu Beginn meines Studiums saßen viel zu viele
Studierende in den Kursen, am Ende zu wenige. Die BibLounge, mit ihren
tollen Sofas, gehört für mich dazu genauso wie lange Schlangen vor den
Toiletten.

Julia: Ein Elternteil hat früher am IBI studiert. Frau Dr.~Pannier und
Herr Heinz waren damals schon Dozenten.

Sonja: Vor allem das Gebäude, welches viel Charakter besitzt.

Hagen: Das IBI ist ein familiäres und offenes Institut, das sehr
effizient und kooperativ mit den Studierenden zusammenarbeitet. Für mich
verkörpert es aber auch einen Wandel. Das Portfolio scheint sich in
Richtung digitaler Informationswissenschaften zu entwickeln. Das begrüße
ich.

Lars: Eine kleine Fach-Community, die jeden offen und
leistungsunabhängig aufnimmt.

Andreas: Das IBI bietet mir die Möglichkeit beruflich aufzusteigen, ohne
die hohen Kosten von Fachhochschulen auf mich nehmen zu müssen.

Lisa: Das IBI hat einen guten Ruf, da eine gute Ausbildung geboten wird.
Darum kümmern sich viele tolle Dozenten, denen ich viel zu verdanken
habe. Im Laufe meines Studiums sind viele Freundschaften entstanden, die
auch heute noch bestehen.

\hypertarget{was-sind-deine-beruflichen-pluxe4ne-fuxfcr-die-zeit-nach-dem-abschluss}{%
\subsubsection{Was sind deine beruflichen Pläne für die Zeit nach dem
Abschluss?}\label{was-sind-deine-beruflichen-pluxe4ne-fuxfcr-die-zeit-nach-dem-abschluss}}

Patricia: Nach meinem Bachelor-Abschluss möchte ich den Master am IBI
machen. Ich strebe ein Archivreferendariat an, um als Archivarin zu
arbeiten. Ich kann mir jedoch auch gut vorstellen, in einer
Wissenschaftlichen Bibliothek zu arbeiten.

Nico: Das weiß ich noch nicht genau. Ein Abschluss in unserem Bereich
bringt vielfältige Möglichkeiten mit sich, sodass ich mich jetzt noch
nicht endgültig festlegen möchte. Vorstellbar wären Positionen in
verschiedensten Bibliotheken, in der freien Wirtschaft oder sogar als
Forschender und Dozent.

Mandy: Mein berufliches Ziel ist die Arbeit in einer Gedenkstätte oder
Bibliothek, vielleicht auch einem Archiv mit historischem Hintergrund.

Annika: Bisher sortiere ich noch Möglichkeiten und möchte noch mehr
Bereiche kennenlernen, bevor ich mich entscheide. Aber auf jeden Fall
möchte ich meinen Master machen.

Julia: Nach dem Studium möchte ich ein Jahr Work and Travel in
Neuseeland machen. Danach strebe ich ein duales Studium für den
gehobenen Archivdienst an der Archivschule Marburg an.

Sonja: Ich bin gegenüber verschiedenen Möglichkeiten recht
aufgeschlossen, würde aber gerne in einer Bibliothek arbeiten.

Hagen: Informationsmanagement in einer Firma.

Lars: Mein Zweitfach ist Englisch, und ich strebe den Master in
Amerikanistik an.

Andreas: Mein langfristiges Ziel ist es nach dem Abschluss die Leitung
einer kleinen Bibliothek zu übernehmen.

Lisa: Leitung einer Stadtbibliothek.

\hypertarget{was-wuxfcnscht-du-dem-ibi-zum-90.-geburtstag}{%
\subsubsection{Was wünscht du dem IBI zum 90.
Geburtstag?}\label{was-wuxfcnscht-du-dem-ibi-zum-90.-geburtstag}}

Patricia: Ich wünsche dem IBI, dass es noch lange besteht und weiterhin
ein Ort bleibt, an dem man gerne lernt und sich verbunden fühlt.

Nico: Weiterhin viel Tatendrang und Begeisterung von seinen Mitarbeitern
und Studierenden. Auf die nächsten 90 Jahre!

Mandy: Ich wünsche dem IBI eine gute Zukunft mit genügend Geldmitteln
und weniger Problemen als bisher. Und, dass es noch mindestens 90
weitere Jahre besteht!

Annika: Alles Gute, viele Gelder, eine schnelle Sanierung des Gebäudes,
weiterhin motivierte Angestellte und Dozenten, aber vor allem
Studierende mit Freude, Motivation und Potenzial.

Julia: Ich wünsche dem IBI, dass es sich mit dem neuen Master
\enquote{Information Science} nicht von der traditionellen
Bibliothekswissenschaft abwendet. Den zukünftigen Studierenden am IBI
wünsche ich, dass sie weiterhin die Möglichkeit haben an einer Kursfahrt
teilzunehmen, wie bisher von Frau Dr.~Hauke angeboten. Es lohnt sich!

Sonja: Weitere 90 Jahre -- mindestens!

Hagen: Alles Gute! Auf, dass es weitere 90 Jahre werden!

Lars: Alles Gute! Auf, dass es nur die ersten 90 Jahre gewesen sein
werden!

Andreas: Ich wünsche dem IBI für die Zukunft viele engagierte
Dozentinnen und Dozenten, und natürlich interessierte Studierende.

Lisa: Ich wünsche dem IBI weiterhin tolle Studierende, eine engagierte
Fachschaft, gewissenhafte Dozenten und, dass die Bibliothekswissenschaft
nicht ausstirbt.

%autor
\begin{center}\rule{0.5\linewidth}{\linethickness}\end{center}

\textbf{Annemarie Vita} hat die Umfrage für diesen Beitrag durchgeführt,
die verschiedenen Antworten zusammengefügt und überarbeitet. Sie
studiert zurzeit im siebten Bachelorsemester Bibliotheks- und
Informationswissenschaft am Institut der Humboldt-Universität zu Berlin.
Nach ihrem Abschluss strebt sie ein Volontariat im Bereich Digital
Rights Management an.

\textbf{Felicitas Härting} studiert zurzeit im zweiten Bachelorsemester
\enquote{Bibliotheks- und Informationswissenschaft} am Institut. 2013 hat sie
die Ausbildung zur \enquote{Fachangestellten für Medien- und
Informationsdienste} in der Stadtbibliothek Steglitz-Zehlendorf
absolviert und arbeitet seitdem auch dort. Nach ihrem Bachelorabschluss
würde sie gerne in einer Kinder- und Jugendbibliothek arbeiten, das
Referat für Jugendliteratur betreuen und Veranstaltungen für und mit
Kindern auf die Beine stellen.

\end{document}
