\documentclass[a4paper,
fontsize=11pt,
%headings=small,
oneside,
numbers=noperiodatend,
parskip=half-,
bibliography=totoc,
final
]{scrartcl}

\usepackage{synttree}
\usepackage{graphicx}
\setkeys{Gin}{width=.4\textwidth} %default pics size

\graphicspath{{./plots/}}
\usepackage[ngerman]{babel}
\usepackage[T1]{fontenc}
%\usepackage{amsmath}
\usepackage[utf8x]{inputenc}
\usepackage [hyphens]{url}
\usepackage{booktabs} 
\usepackage[left=2.4cm,right=2.4cm,top=2.3cm,bottom=2cm,includeheadfoot]{geometry}
\usepackage{eurosym}
\usepackage{multirow}
\usepackage[ngerman]{varioref}
\setcapindent{1em}
\renewcommand{\labelitemi}{--}
\usepackage{paralist}
\usepackage{pdfpages}
\usepackage{lscape}
\usepackage{float}
\usepackage{acronym}
\usepackage{eurosym}
\usepackage[babel]{csquotes}
\usepackage{longtable,lscape}
\usepackage{mathpazo}
\usepackage[normalem]{ulem} %emphasize weiterhin kursiv
\usepackage[flushmargin,ragged]{footmisc} % left align footnote
\usepackage{ccicons} 

%%%% fancy LIBREAS URL color 
\usepackage{xcolor}
\definecolor{libreas}{RGB}{112,0,0}

\usepackage{listings}

\urlstyle{same}  % don't use monospace font for urls

\usepackage[fleqn]{amsmath}

%adjust fontsize for part

\usepackage{sectsty}
\partfont{\large}

%Das BibTeX-Zeichen mit \BibTeX setzen:
\def\symbol#1{\char #1\relax}
\def\bsl{{\tt\symbol{'134}}}
\def\BibTeX{{\rm B\kern-.05em{\sc i\kern-.025em b}\kern-.08em
    T\kern-.1667em\lower.7ex\hbox{E}\kern-.125emX}}

\usepackage{fancyhdr}
\fancyhf{}
\pagestyle{fancyplain}
\fancyhead[R]{\thepage}

% make sure bookmarks are created eventough sections are not numbered!
% uncommend if sections are numbered (bookmarks created by default)
\makeatletter
\renewcommand\@seccntformat[1]{}
\makeatother


\usepackage{hyperxmp}
\usepackage[colorlinks, linkcolor=black,citecolor=black, urlcolor=libreas,
breaklinks= true,bookmarks=true,bookmarksopen=true]{hyperref}
%URLs hart brechen
\makeatletter 
\g@addto@macro\UrlBreaks{ 
  \do\a\do\b\do\c\do\d\do\e\do\f\do\g\do\h\do\i\do\j 
  \do\k\do\l\do\m\do\n\do\o\do\p\do\q\do\r\do\s\do\t 
  \do\u\do\v\do\w\do\x\do\y\do\z\do\&\do\1\do\2\do\3 
  \do\4\do\5\do\6\do\7\do\8\do\9\do\0} 
% \def\do@url@hyp{\do\-} 
\makeatother 

%meta
%meta

\fancyhead[L]{B. Kaden, K. Schuldt\\ %author
LIBREAS. Library Ideas, 34 (2018). % journal, issue, volume.
\href{http://nbn-resolving.de/}
{}} % urn 
% recommended use
%\href{http://nbn-resolving.de/}{\color{black}{urn:nbn:de...}}
\fancyhead[R]{\thepage} %page number
\fancyfoot[L] {\ccLogo \ccAttribution\ \href{https://creativecommons.org/licenses/by/3.0/}{\color{black}Creative Commons BY 3.0}}  %licence
\fancyfoot[R] {ISSN: 1860-7950}

\title{\LARGE{Bibliographie Geschichte der Berliner und restlichen deutschsprachigen Bibliothekswissenschaft (1946–2018)}}% title
\author{Ben Kaden, Karsten Schuldt} % author

\setcounter{page}{1}

\hypersetup{%
      pdftitle={Bibliographie Geschichte der Berliner und restlichen deutschsprachigen Bibliothekswissenschaft (1946–2018)},
      pdfauthor={Ben Kaden, Karsten Schuldt},
      pdfcopyright={CC BY 3.0 Unported},
      pdfsubject={LIBREAS. Library Ideas, 34 (2018).},
      pdfkeywords={Bibliothekswissenschaft, Geschichte, Bibliographie},
      pdflicenseurl={https://creativecommons.org/licenses/by/3.0/},
      pdfcontacturl={http://libreas.eu},
      baseurl={http://libreas.eu},
      pdflang={de},
      pdfmetalang={de}
     }



\date{}
\begin{document}

\maketitle
\thispagestyle{fancyplain} 

%abstracts

%body
\hypertarget{vorbemerkung}{%
\section{Vorbemerkung}\label{vorbemerkung}}

Die Bibliothekswissenschaft in (zumeist) Berlin -- aber auch manchmal
Köln, Göttingen, München oder Leipzig -- fragt sich seit je \enquote{quo
vadis}. Oder auch einfach was sie ist und sein will. Es braucht gar
nicht viel der Recherche, um zu entdecken, dass von ihrer Idee bis im
Prinzip in die Gegenwart zwar mit unterschiedlicher Konjunktur aber doch
regelmäßig Versuche einer Selbstverständigung darüber unternommen
werden, was dieses Fach eigentlich sein kann. Sie ist naturgemäß ein
wenig artifiziell und ergibt sich nicht zwangsläufig aus der Evolution
wissenschaftlicher Disziplinarität. Ihre engste Verwandtschaft hat sie
traditionell mit der Metaebene der Wissens- oder
Wissenschaftsorganisation. Naheliegend hätte man sie sich beispielsweise
als eine Art Ableger der Epistemologie denken, vorstellen und entwickeln
können. Dass dies nicht die konsequente Entwicklungslinie wurde, hat nun
vermutlich damit zu tun, dass sie im Herzen gar nicht so sehr
wissenschaftlich im engeren Sinne sein konnte und sollte.\footnote{Vgl.
  auch: Kaden, Ben: Zur Geschichte der Dauerfrage \enquote{Was ist
  Bibliothekswissenschaft?} In: LIBREAS.Tumblr. 14. August 2018
  \url{http://libreas.tumblr.com/post/176987186291/was-ist-bibliothekswissenschaft}}
Sondern damit, dass es ihr darum ging, die sich nicht zuletzt durch die
wachsende Menge an Druckwerken, in denen die Resultate der
expandierenden Wissenschaft und auch anderer Wissenskulturen in die
Büchersammlungen immer auch mit der Bedrohung einer Unbeherrschbarkeit
drängte, an Komplexität gewinnenden Aufgaben zu bewältigen. Die
Bibliothekswissenschaft war keine des kritischen Hinterfragens, auch
keine der Theoriekonstruktion und des Strebens nach neuen Erkenntnissen.
Vielmehr lag ihre Aufgabe in der Entwicklung praktischer und
praktikabler Lösungen für die sich wandelnde Überinstitution der
Bibliothek. Sie sollte, wenn man so will, Lösungen entwickeln, mit dem,
was man Publikations- oder auch Informationsflut, nie jedoch
Wissensflut, nannte und nennt, umzugehen, die Effizienz zu steigern
sowie Fähigkeiten und Anforderungsprofile für diejenigen entwickeln,
denen die Aufgabe der Organisation des buchgewordenen, später auch
zeitschriftenfixierten Wissens zukam. In diesem Sinne, also als eine Art
angewandte Wissenschaft, war die Bibliothekswissenschaft durch und durch
sehr modern. Zugleich wurde und wird deutlich, woran es ihr fehlte und
bis heute zumindest teilweise fehlt. Die kritische, hinterfragende,
abstrahierende, modell- und methodenbildende Erkenntnissuche und oft
sogar eine systematische, auf dem Konsens einer Community beruhende
Selbstorganisation und verbindende Forschungsagenda bleibt über weite
Strecken ein Desiderat. Ebenso leider auch über weite Strecken der Blick
auf die Ideen- und Wissenschaftsgeschichte, die in den Curricula des IBI
bestenfalls zufällig eine Rolle zu spielen scheint. Wir sind nicht in
der Position diese Fehlstellen zu schließen. Aber immerhin können und
möchten wir allen, die ein Interesse daran haben, wenigstens eine
Quellenliste zur Hand geben, die eine Auswahl, vielleicht sogar einen
Kanon, zu den Selbstbestimmungsdebatten der Bibliothekswissenschaft
benennt.

\hypertarget{bibliografie}{%
\section{Bibliografie}\label{bibliografie}}

\hypertarget{section}{%
\subsection*{2001--2018}\label{section}}

Hauke, Petra: Bibliothekswissenschaft -- quo vadis? : eine Disziplin
zwischen Traditionen und Visionen : Programme -- Modelle --
Forschungsaufgaben. München: KG Saur, 2005.

Kaden, Ben: Gegenwart, Zukunft und Ende der Bibliothekswissenschaft. In:
Hauke, Petra; Umlauf, Konrad: Vom Wandel der Wissensorganisation im
Informationszeitalter -- Festschrift für Walther Umstätter zum 65.
Geburtstag. Bad Honnef: Bock + Herchen, 2006, S. 29--48.\\
\href{https://doi.org/10.18452/2322}{https://doi.org/10.18452/2322}.

Schröter, Marcus: Bücher, Bildung, Bibliotheken -- Altes Buch und Neue
Medien an der Universität Rostock. Bibliothekswissenschaft für
Historiker. In: Bibliothek. Forschung und Praxis, Heft~1, 2005, S. 25--37.
\url{https://doi.org/10.1515/BFUP.2005.25}.

Umstätter, Walther: Bibliothekswissenschaft im Wandel, von den
geordneten Büchern zur Wissensorganisation. In: Bibliothek. Forschung
und Praxis, Heft~3, 2009, 327--331.
\url{https://doi.org/10.1515/bfup.2009.036}.

Wagner-Döbler, Roland et al.: Literaturversorgung auf fünf
Sondersammelgebieten 1991--2000. Bestandsstruktur und -entwicklung in
Wirtschaftswissenschaften, Mathematik, Sprach- und
Literaturwissenschaft, Bibliothekswissenschaft, Keltologie. In:
Bibliothek. Forschung und Praxis, Heft~2, 2003, S. 189--193.
\url{https://doi.org/10.1515/BFUP.2003.189}.

\hypertarget{section-1}{%
\subsection*{1991--2000}\label{section-1}}

Bornhöft, Margrit: Bibliothekswissenschaft in Deutschland -- Eine
Bestandsaufnahme. Aachen: Wissenschaftsverlag Mainz, 1999.

Rohde, Renate: Zur Geschichte der bibliothekswissenschaftlichen
Ausbildung in Berlin. In: Kunze, Horst (Hrsg.): Bibliothekswissenschaft
in Berlin. Wiesbaden: Harrassowitz, 1999, S. 11--46.

Strzolka, Rainer: Repertorium der Bibliothekswissenschaft. Hannover:
Koechert, 1996.

\hypertarget{section-2}{%
\subsection*{1986--1990}\label{section-2}}

Freytag, Jürgen: Das Forschungsprofil in den 80er Jahren. In:
Zentralblatt für Bibliothekswesen. Heft~9, 1985, S. 390--393.

Freytag, Jürgen: Mythen in der Bibliotheks- und
Informationswissenschaft? Diskussionsbeitrag. In: Zentralblatt für
Bibliothekswesen. Heft~8, 1989, S. 350--356.

Greguletz, Alexander: Bibliothekswissenschaft -- Überlegen zu künftigen
Forschungsstrategien und -inhalten im Zeitalter neuer
Informationstechnologien. In: Zentralblatt für Bibliothekswesen. Heft~1,
1990, S. 4--14.

Greguletz, Alexander: Die Bibliothekswissenschaft und die Informations-
und Kommunikationstechnologien. In: Zentralblatt für Bibliothekswesen.
Heft~8, 1987, S. 290--298.

Knoche, Michael: Bibliothekswissenschaft als spezielle
Informationswissenschaft : Zweites Kölner Kolloquium aus Anlaß des
zehnjährigen Bestehens des Lehrstuhls für Bibliothekswissen- schaft der
Universität zu Köln am 9./10. Mai 1985. In: Bibliothek. Forschung und
Praxis, Heft~3, 1985, S. 308--312.
\url{https://doi.org/10.1515/bfup.1985.9.3.308}.

Kubitschek, Helmut: Forschungsergebnisse und Forschungsstrategien am
Institut für Bibliothekswissenschaft und wissenschaftliche Information.
Heft~9, 1985, S. 387--390.

Marks, Erwin: Nachdenken über ehemalige Professoren und Dozenten der
Berliner Instituts für Bibliothekswissenschaft. In: Zentralblatt für
Bibliothekswesen. Heft~9, 1990, S. 399--406.

Marks, Erwin: Die Gründung des Methodischen Zentrums für
wissenschaftliche Bibliotheken und Informations- und
Dokumentationseinrichtungen. In: Zentralblatt für Bibliothekswesen. Heft
8, 1989, S. 337--342.

Nestler, Friedrich: Entwicklung und Perspektiven des Direktstudiums am
Institut für Bibliothekswissenschaft und wissenschaftliche Information
der Humboldt-Universität zu Berlin. In: Zentralblatt für
Bibliothekswesen. Heft~9, 1990, S. 393--398.

Plassmann, Engelbert; Schmitz, Wolfgang; Vodosek, Peter (Hrsg.): Buch
und Bibliothekswissenschaft im Informationszeitalter : internationale
Festschrift für Paul Kaegbein zum 65. Geburtstag. München u.\,a.: K.G.
Saur, 1990.

Rohde, Renate: Das Bibliothekswissenschaftliche Institut an der Berliner
Universität -- Vorläufer des heutigen Instituts für
Bibliothekswissenschaft und wissenschaftliche Information der
Humboldt-Universität zu Berlin. In: Zentralblatt für Bibliothekswesen.
Heft~1, 1985, S. 24--29.

Rohde, Renate; Klink, Renate: \enquote{Katalographie} -- eine neue
bibliothekswissenschaftliche Disziplin? In: Zentralblatt für
Bibliothekswesen. Heft~2, 1987, S. 72--76.

\hypertarget{section-3}{%
\subsection*{1981--1985}\label{section-3}}

Dube, Werner: Allgemeine Bibliothekswissenschaft. In: Zentralblatt für
Bibliothekswesen. Heft~1, 1985, S. 9--17.

Fröschner, Günter: Methodologische Probleme der Bibliothekswissenschaft.
In: Zentralblatt für Bibliothekswesen. Heft~8, 1983, S. 344--348.

Golczewski, Mechthild: Die Darstellung der russischen
Bibliotheksgeschichte in der sowjetischen Bibliothekswissenschaft nach
dem Zweiten Weltkrieg. In: Bibliothek. Forschung und Praxis. Heft~1,
1983, 32--59. \url{https://doi.org/10.1515/bfup.1983.7.1.32}.

Kołodziejska, Jadwiga: Bibliothekswissenschaft in Polen. In: Bibliothek.
Forschung und Praxis, Heft~1, 1981, 66--70.
\url{https://doi.org/10.1515/bfup.1981.5.1.66}.

Kubitschek, Helmut: Überlegungen zur Stellung der
Bibliothekswissenschaft im System der Wissenschaften. In: Zentralblatt
für Bibliothekswesen. Heft~8, 1983, S. 339--344.

Mittenzwei, Karin: Zur Entwicklung des Verhältnisses der
Bibliothekswissenschaft zur Informa\-tions- und Dokumentationswissenschaft
1968--1978 : eine kritische Analyse. (Veröffentlichungen des
Wissenschaftlichen Informationszentrums der Bergakademie Freiberg, Nr.
96) Freiberg: Bergakademie Freiberg, 1982.

Pannier, Gertrud; Schmidt, Karin: Jugendobjekte im Direktstudium
Bibliothekswissenschaft. In: Zentralblatt für Bibliothekswesen. Heft~3,
1983, S. 120--123.

Schmidmaier, Dieter: Bibliothekswissenschaftliche Bestrebungen an der
Altdorfer Universitätsbibliothek zwischen 1630 und 1800. In:
Zentralblatt für Bibliothekswesen. Heft~1, 1984, S. 17--22.

Schmidt, Karla; Stäber, Peter: Wissenschaftskunde und
Bibliothekswissenschaft. Überlegungen und Fragen zum gegenseitigen
Zusammenhang. In: Zentralblatt für Bibliothekswesen. Heft~8, 1983, S. 348--352.

Stelmach, Valeria D.: Die Organisation und Koordinierung der
Forschungsarbeit auf dem Gebiet der Bibliothekswissenschaft in der
Sowjetunion. In: Bibliothek. Forschung und Praxis. Heft~1, S. 27--31.
\url{https://doi.org/10.1515/bfup.1983.7.1.27}.

Sokolov, Arkadij V.: Vergleichende Analyse des Objektes und des
Gegenstandes von Bibliothekswissenschaft, Bibliografie und Informatik.
In: Zentralblatt für Bibliothekswesen. Heft~11, 1981, S. 493--502.

Wilbert, Gerd: Die Bibliothekswissenschaft in der Sowjetunion: Zentrale
Probleme im Spiegel der Diskussion des letzten Jahrzehnts. In:
Bibliothek. Forschung und Praxis, Heft~1. 1983, S. 5--26.
\url{https://doi.org/10.1515/bfup.1983.7.1.5}.

\hypertarget{section-4}{%
\subsection*{1976--1980}\label{section-4}}

Dietze, Joachim: Bibliotheks- und Informationswissenschaft -- ein
Postulat. In: Zentralblatt für Bibliothekswesen. Heft~9, 1977, S. 411--416.

Dima-Drǎgan, Corneliu: Zur Geschichte der Bibliothekswissenschaft als
Hochschuldisziplin in der SR Rumänien. In: Zentralblatt für
Bibliothekswesen. Heft~9, 1976, S. 394--397.

Ewert, Gisela: Zehn Jahre Erwachsenenqualifizierung am Institut für
Bibliothekswissenschaft und wissenschaftliche Information der
Humboldt-Universität zu Berlin. In: Zentralblatt für Bibliothekswesen.
Heft~2, 1979, S. 54--61.

Fröschner, Günter: Bemerkungen zum Studienplan der Grundstudienrichtung
Bibliothekswissenschaft, Direktstudium. In: Zentralblatt für
Bibliothekswesen. Heft~9, 1979, S. 397--403.

Kaegbein, Paul: Einige Bemerkungen zur gegenwärtigen Situation der
Bibliothekswissenschaft in der Bundesrepublik Deutschland. In:
Bibliothek. Forschung und Praxis, Heft~3, 1978, S. 209--212.
\url{https://doi.org/10.1515/bfup.1978.2.3.209}.

Konferenz \enquote{Entwicklungsprobleme der Bibliothekswissenschaft und
der Informations- und Dokumentationswissenschaft als
Hochschuldisziplinen} : aus Anlass des 20jährigen Bestehens des
Institutes für Bibliothekswissenschaft und wissenschaftliche Information
der Humboldt-Univer\-sität zu Berlin; Berlin, 4.--5.November 1975.
(Wissenschaftliche Schriftenreihe der Humboldt-Universität zu Berlin).
Berlin: Humboldt Universität zu Berlin, 1977.

Kubitschek, Helmut: Zur Wiederaufnahme des Direktstudiums
Bibliothekswissenschaft. In: Zentralblatt für Bibliothekswesen. Heft~9,
1979, S. 393.

Stäber, Peter: Bibliothekswissenschaft, Informationswissenschaft und
Gesellschaft. Eine Polemik.In: Zentralblatt für Bibliothekswesen. Heft
1, 1979, S. 23--29.

\hypertarget{section-5}{%
\subsection*{1970--1975}\label{section-5}}

10 Jahre methodisches Zentrum für wissenschaftliche Bibliotheken. In:
Zentralblatt für Bibliothekswesen. Heft~7, 1974.

Bibliothekswissenschaft und öffentliche Bibliothek : Referate und
Ergebniszusammenfassungen eines Fortbildungsseminars der FHB Stuttgart :
(Stuttgart, 12.--14.Oktober 1972). In: Bibliotheksdienst, Beiheft 1974.

Čubar'jan, Ogan S.; Fröschner, Günther (übers. \& bearb.): Allgemeine
Bibliothekswissenschaft : Entwicklungsergebnisse und Probleme. Berlin:
Bibliotheksverband der DDR, 1975.

Dietze, Joachim: Konferenz über die Methodologie der Bibliotheks- und
Informationswissenschaft am 17. und 18.12.1974 in Poznań. In:
Zentralblatt für Bibliothekswesen. Heft~8, 1975, S. 373--374.

Dube, Werner: Bibliothekswissenschaft und Bibliothekspraxis in der
Sowjetunion. In: Zentralblatt für Bibliothekswesen. Heft~10, 1974, S. 639--642.

Dube, Werner: Informationswissenschaft -- Bibliothekswissenschaft --
Informations- und Dokumentationswissenschaft. Bemerkungen zu einer
Konzeption von Josef Koblitz. In: Zentralblatt für Bibliothekswesen.
Heft~6, 1971, S. 336--347.

Ewert, Gisela: Bibliothekswissenschaft. Versuch e. Begriffsbestimmung in
Referaten u. Diskussionen bei d. Kölner Kolloquium (27.--29. Oktober
1969) {[}Rezension{]} In: Zentralblatt für Bibliothekswesen. Heft~5,
1972, S. 289--290.

Freytag, Jürgen; Kubitschek, Helmut: Zum Stand und zu Problemen der
Entwicklung des Instituts für Bibliothekswissenschaft und
wissenschaftliche Information der Humboldt-Universität zu Berlin. In:
Zentralblatt für Bibliothekswesen. Heft~6, 1975, S. 241--247.

Grundwald, Wilhelm; Krieg, Werner: Die Bibliothekswissenschaft in Lehre
und Forschung. In: Bibliotheksdienst Heft~4, 1970, S. 175--180.

Haake, Rolf: Über die Forschungstätigkeit des Instituts für
Bibliothekswissenschaft und wissenschaftliche Information. In:
Zentralblatt für Bibliothekswesen. Heft~6, 1975, S. 247--252.

Hagelweide, Gert: Das Kolloquium über Bibliothekswissenschaft in Köln.
In: Zeitschrift für Bibliothekswesen und Bibliographie. Heft~1, 1970, S. 58--61.

Krieg, Werner: Bibliothekswissenschaft: Versuch einer Begriffsbestimmung
in Referaten und Diskussionen bei dem Kölner Kolloquium (27.--29.
Oktober 1969). Köln: Greven Verlag, 1970.

Kunze, Horst: 15 Jahre Institut für Bibliothekswissenschaft -- 5 Jahre
Promotionen am Institut für Bibliothekswissenschaft und
wissenschaftliche Information 1970. In: Zentralblatt für
Bibliothekswesen. Heft~10, 1970, S. 608.

Kunze, Horst: 20 Jahre Zentralinstitut für Bibliothekswesen. In:
Zentralblatt für Bibliothekswesen. Heft~3, 1970, S. 163--164.

Kluth, Rolf: Gibt es eine Bibliothekswissenschaft? In: Zeitschrift für
Bibliothekswesen und Bibliographie. Heft~4/5, 1970, S. 227--246.

Roloff, Heinrich: Zum Standort der Bibliothekswissenschaft. In:
Zentralblatt für Bibliothekswesen. Heft~10, 1970, S. 595--602.

Schwarz, Gerhard: Leser -- Bibliothek -- Lektüre. Sowjetische
Forschungsarbeiten zu aktuellen Problemen der Bibliothekswissenschaft.
In: Zentralblatt für Bibliothekswesen. Heft~12, 1972, S. 707--723.

Wendt, Hartmut: Zum Charakter der Bibliothekswissenschaft als
Wissenschaftsdisziplin. In: Zentralblatt für Bibliothekswesen. Heft~9,
1971, S. 529--537.

\hypertarget{section-6}{%
\subsection*{1966--1970}\label{section-6}}

Čubar'jan, Ogan S.: 50 Jahre sowjetische Bibliothekswissenschaft. In:
Zentralblatt für Bibliothekswesen. Heft~12, 1967, S. 705--721.

Koblitz, Josef: Die Stellung der Bibliothekswissenschaft und der
Informations- und Dokumentationswissenschaft in der
Informationswissenschaft. In: Zentralblatt für Bibliothekswesen. Heft
12, 1969, S. 689--720.

Methodischen Zentrum für wissenschaftliche Bibliotheken (Hrsg.):
Entwicklungstendenzen der Bürgerlichen Bibliothekswissenschaft : eine
Sammlung von Aufsätzen. Berlin: Deutsche Staatsbibliothek, 1966.

Nestler, Friedrich: Das bibliothekswissenschaftliche
\enquote{Vollstudium}. Aus der Arbeit der Kommission für Berufsnachwuchs
und Weiterbildung des Deutschen Bibliotheksverbandes. In: Zentralblatt
für Bibliothekswesen. Heft~5, 1966, S. 290--297.

Nestler, Friedrich: III. Konfe­renz über bibliothekswissenschaftliche
Hochschulausbildung in den sozia­listischen Ländern in Leningrad. In:
Zentralblatt für Bibliothekswesen. Heft~12, 1968, S. 677--680.

Tröger, Erika: Ungarische Schnellinformation über Literatur zur
Bibliothekswissenschaft und Dokumentation. In: Zentralblatt für
Bibliothekswesen. Heft~8, 1966, S. 476--477.

\hypertarget{section-7}{%
\subsection*{1960--1965}\label{section-7}}

Drtina, Jaroslav: Zur Klassifikation der Bibliotheks­wissenschaft. In:
Zentralblatt für Bibliothekswesen. Heft~5, 1961, S. 193--206.

Gegenstand und Methoden der Bibliothekswissenschaft unter besonderer
Berücksichtigung der Bibliothekswissenschaft als Hochschuldisziplin:
Referate, Materialien, Diskussionen, Ergebnisse und Empfehlungen.
Leipzig: Verlag für Buch- und Bibliothekswesen, 1963.

Kunze, Horst: II. Konferenz der bibliothekswissenschaftlichen
Hochschulen und Institute in den sozialistischen Ländern. In:
Zentralblatt für Bibliothekswesen. Heft~10, 1962, S. 433--439.

Kunze, Horst: Zur Vorbereitung einer internationalen Konferenz über
Gegenstand und Methode der Bibliothekswissenschaft. In: Zentralblatt für
Bibliothekswesen. Heft~3, 1961, S. 97--110.

Kunze, Horst: Zehn Jahre Institut für Bibliothekswissenschaft der
Humboldt-Universität zu Berlin. In: Zentralblatt für Bibliothekswesen.
Heft~11, 1965, S. 641--698.

Vicentini, Abner Lellis Corrêa: Bibliothekswissenschaft und
Dokumentation. In: Zentralblatt für Bibliothekswesen. Heft~1, 1963, S. 2--8.

Vicentini, Abner Lellis Corrêa: Bibliothekswissenschaft und
Dokumentation : einige grundsätzliche Betrachtungen im Hinblick auf die
Entwicklung in Brasilien. In: Zentralblatt für Bibliothekswesen. Heft~1,
1961, S. 2--8.

\hypertarget{section-8}{%
\subsection*{1946--1959}\label{section-8}}

Vorstius, Joris: Bibliothek, Bibliothekar, Bibliothekswissenschaft. In:
Zentralblatt für Bibliothekswesen. Heft~5/6, 1949, S. 172--185.

%autor
\begin{center}\rule{0.5\linewidth}{\linethickness}\end{center}

\textbf{Ben Kaden} ist Bibliothekswissenschaftler und arbeitet an der
Universitätsbibliothek der Humboldt-Universität zu Berlin.

\textbf{Karsten Schuldt} (Chur / Berlin) ist Wissenschaftlicher
Mitarbeiter am Schweizerischen Institut für Informationswissenschaft,
HTW Chur.

\end{document}
