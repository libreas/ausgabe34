\documentclass[a4paper,
fontsize=11pt,
%headings=small,
oneside,
numbers=noperiodatend,
parskip=half-,
bibliography=totoc,
final
]{scrartcl}

\usepackage{synttree}
\usepackage{graphicx}
\setkeys{Gin}{width=.4\textwidth} %default pics size

\graphicspath{{./plots/}}
\usepackage[ngerman]{babel}
\usepackage[T1]{fontenc}
%\usepackage{amsmath}
\usepackage[utf8x]{inputenc}
\usepackage [hyphens]{url}
\usepackage{booktabs} 
\usepackage[left=2.4cm,right=2.4cm,top=2.3cm,bottom=2cm,includeheadfoot]{geometry}
\usepackage{eurosym}
\usepackage{multirow}
\usepackage[ngerman]{varioref}
\setcapindent{1em}
\renewcommand{\labelitemi}{--}
\usepackage{paralist}
\usepackage{pdfpages}
\usepackage{lscape}
\usepackage{float}
\usepackage{acronym}
\usepackage{eurosym}
\usepackage[babel]{csquotes}
\usepackage{longtable,lscape}
\usepackage{mathpazo}
\usepackage[normalem]{ulem} %emphasize weiterhin kursiv
\usepackage[flushmargin,ragged]{footmisc} % left align footnote
\usepackage{ccicons} 

%%%% fancy LIBREAS URL color 
\usepackage{xcolor}
\definecolor{libreas}{RGB}{112,0,0}

\usepackage{listings}

\urlstyle{same}  % don't use monospace font for urls

\usepackage[fleqn]{amsmath}

%adjust fontsize for part

\usepackage{sectsty}
\partfont{\large}

%Das BibTeX-Zeichen mit \BibTeX setzen:
\def\symbol#1{\char #1\relax}
\def\bsl{{\tt\symbol{'134}}}
\def\BibTeX{{\rm B\kern-.05em{\sc i\kern-.025em b}\kern-.08em
    T\kern-.1667em\lower.7ex\hbox{E}\kern-.125emX}}

\usepackage{fancyhdr}
\fancyhf{}
\pagestyle{fancyplain}
\fancyhead[R]{\thepage}

% make sure bookmarks are created eventough sections are not numbered!
% uncommend if sections are numbered (bookmarks created by default)
\makeatletter
\renewcommand\@seccntformat[1]{}
\makeatother


\usepackage{hyperxmp}
\usepackage[colorlinks, linkcolor=black,citecolor=black, urlcolor=libreas,
breaklinks= true,bookmarks=true,bookmarksopen=true]{hyperref}
\usepackage{breakurl}

%meta
%meta

\fancyhead[L]{S. Schneider\\ %author
LIBREAS. Library Ideas, 34 (2018). % journal, issue, volume.
\href{http://nbn-resolving.de/}
{}} % urn 
% recommended use
%\href{http://nbn-resolving.de/}{\color{black}{urn:nbn:de...}}
\fancyhead[R]{\thepage} %page number
\fancyfoot[L] {\ccLogo \ccAttribution\ \href{https://creativecommons.org/licenses/by/4.0/}{\color{black}Creative Commons BY 4.0}}  %licence
\fancyfoot[R] {ISSN: 1860-7950}

\title{\LARGE{Wie vielfältig kann und sollte Open Access sein?}}% title
\subtitle{Bericht zu den Open-Access-Tagen 2018 in Graz}
\author{Sophie Schneider} % author

\setcounter{page}{1}

\hypersetup{%
      pdftitle={Wie vielfältig kann und sollte Open Access sein? Bericht zu den Open-Access-Tagen 2018 in Graz},
      pdfauthor={Sophie Schneider},
      pdfcopyright={CC BY 4.0 International},
      pdfsubject={LIBREAS. Library Ideas, 34 (2018).},
      pdfkeywords={Open Access, Tagungsbericht, DEAL},
      pdflicenseurl={https://creativecommons.org/licenses/by/4.0/},
      pdfcontacturl={http://libreas.eu},
      baseurl={http://libreas.eu},
      pdflang={de},
      pdfmetalang={de}
     }



\date{}
\begin{document}

\maketitle
\thispagestyle{fancyplain} 

%abstracts

%body
Das Motto \enquote{Vielfalt von Open Access} bildete den Rahmen für die
diesjährigen 12. Open-Access-Tage vom 24. bis 26.09.2018 in Graz. Die
Konferenz richtete sich dabei vor allem an Expert*innen aus dem Bereich
Open Access (OA), \enquote{die sich mit den Möglichkeiten, Bedingungen
und Perspektiven des wissenschaftlichen Publizierens befassen}\footnote{\url{https://open-access.net/AT-DE/community/open-access-tage/open-access-tage-2018-graz/}
  (Abgerufen am 04.10.2018).}. Die Folien der Beiträge können in der
Zenodo-Community \enquote{Open-Access-Tage 2018} abgerufen
werden.\footnote{\url{https://zenodo.org/communities/oat2018/}
  (Abgerufen am 04.10.2018).}

Der langsam fortschreitende Transformationsprozess von Closed zu Open
Access war ein zentraler Gegenstand der Open-Access-Tage 2018. Dieser
gegenwärtig stattfindende Wandel hat eine stark zunehmende Facettierung
zur Folge, welche verschiedenste Modelle zur Finanzierung, Publikation,
Vermittlung und schlussendlich Durchsetzung von Open Access in
Forschung, Politik und Gesellschaft beinhaltet. Zusammengefasst:
\enquote{{[}...{]} vielfältige Strömungen fächern das Themenfeld Open
Access zusehends auf}\footnote{Ebd.}. Einen Ursprung dieser Diversität
könnten womöglich die beteiligten Akteure selbst darstellen:
Entscheidungsträger aus Politik und Gesellschaft, Verlage, Bibliotheken
und anderen Informationseinrichtungen sowie Wissenschaftlerinnen und
Wissenschaftler verfolgen sehr unterschiedliche Ziele und für die
Erreichung von Fortschritten in der gemeinsamen Schnittmenge ist eine
gegenseitige Annäherung dieser Akteure und ihrer Ziele unabdingbar.

Die Vielfalt von Open Access zeichnete sich daher auf den
Open-Access-Tagen insbesondere in folgenden Bereichen ab:

\begin{enumerate}
\def\labelenumi{\arabic{enumi}.}
\item
  Vielfalt von Wissenschaft als Ausgangslage
\item
  Vielfalt von existierenden Publikationsformaten
\item
  Vielfalt von Open Access Policies und Strategien
\item
  Vielfalt in der Finanzierung von Open Access und den daraus
  hervorgehenden Publikationsmodellen
\end{enumerate}

\hypertarget{vielfalt-von-wissenschaft}{%
\section{Vielfalt von
Wissenschaft}\label{vielfalt-von-wissenschaft}}

Open Access muss als fachübergreifendes Modell zur Publikation
wissenschaftlicher Ergebnisse auch auf die diversen Ausgangslagen
eingehen, etwa auf unterschiedliche Publikationskulturen in den
naturwissenschaftlichen sowie geistes- und sozialwissenschaftlichen
Bereichen. Neben fachspezifischen sollten in diesem Kontext auch
gesellschaftliche und kulturelle Einflussfaktoren berücksichtigt werden.
In Bezug auf Open Science, eines der Schwerpunktthemen der diesjährigen
Open-Access-Tage, sollten solche Unterschiede in die entsprechenden
Richtlinien für die Umsetzung einfließen, sodass die damit verbundenen
visionären Vorstellungen nicht als solche stagnieren. In einem Vortrag
wurde exemplarisch der \enquote{Matthew Effect in Open Science}
\footnote{Ross-Hellauer, Tony, \& Wieser, Bernhard. (2018): The Matthew
  Effect in Open Science. \url{https://doi.org/10.5281/zenodo.1441267}.}
besprochen, bei dem es sich um ein ursprünglich von Robert K. Merton
formuliertes Prinzip handelt, nach welchem der Erfolg eines Autors bzw.
einer Autorin von umstrittenen Wertvorstellungen wie Prestige (zum
Beispiel der Affiliation oder der Zeitschrift, in der publiziert wurde)
sowie persönlichen Einstellungen und Eigenschaften abhängig sei. Die
Referenten kritisierten, dass wenngleich neuartige Strömungen wie Open
Science diese Unausgeglichenheit versuchen zu umgehen, die
verschiedenartigen Voraussetzungen der Autor*innen, wie zum Beispiel der
reine Zugang zum Internet oder die historisch bzw. gesellschaftlich
geprägte Bevorzugung oder Benachteiligung einzelner Personen, auch bei
Open Access und Open Science noch nicht in ausreichendem Maße bedacht
werden.

\hypertarget{vielfalt-von-publikationsformaten}{%
\section{Vielfalt von
Publikationsformaten}\label{vielfalt-von-publikationsformaten}}

Bei Open Access stehen nicht mehr nur Zeitschriften und Bücher im
Vordergrund, OA lässt sich inzwischen auf diverse Publikationsformate
beziehen. Ein gutes Beispiel hierfür sind die Open Educational Resources
(OER), welche ebenfalls im Fokus der 12. Open-Access-Tage standen. OER,
das sind \enquote{jegliche Bildungsressourcen (einschließlich
Lehrplänen, Kursmaterialien, Lehrbüchern, Streaming-Videos,
Multimedia-Anwendungen, Podcasts sowie jegliches weitere Material,
welches zu Lehr- und Lernzwecken entwickelt wurde), die Lehrenden und
Lernenden frei zur Verfügung stehen, ohne dass diese für die Anwendung
Nutzungs- oder Lizenzgebühren zahlen müssten}\footnote{Butcher, Neil
  (2013), S. 6.}. Die vielfältigen Arten von Ressourcen, die als OER in
Frage kommen, bedeuten gleichermaßen vielfältige Möglichkeiten der
Entwicklung von und Arbeit mit solchen Ressourcen. Diese müssen aktuell
noch viel kommuniziert und weiterentwickelt werden. In einer Session zum
Projekt \enquote{Open Education Austria} (OEA) wurde insbesondere auf
die Relevanz der Bekanntmachung von OER sowie die entsprechende
Unterstützung Lehrender bei der Produktion und Bereitstellung solcher
Ressourcen eingegangen. Sogenannte E-Producer*innen helfen im Rahmen von
Open Education Austria zum Beispiel bei der Erstellung von
Animationsvideos zu didaktischen Zwecken. Auch das Open Science Training
Handbook\footnote{Vgl.
  \url{https://open-science-training-handbook.gitbook.io/book/}
  (Abgerufen am 04.10.2018).} wurde auf den Open-Access-Tagen
vorgestellt. Dieses präsentiert sich nicht nur als eine idealtypisch
kollaborativ entstandene, frei zugängliche und nachnutzbare Ressource,
sondern fällt auch inhaltlich mit Anregungen zur Vermittlung von
Kompetenzen im Umgang mit offenen Ressourcen auf.

\hypertarget{vielfalt-von-strategien-und-policies}{%
\section{Vielfalt von Strategien und
Policies}\label{vielfalt-von-strategien-und-policies}}

Die Liste an Strategien und Policies zur Umsetzung von Open Access und
Open Science ist lang und zeigt ein gestiegenes Interesse an diesen
Themen, gerade auch aus Sicht der Politik und Trägerinstitutionen. Dabei
sind die bislang entwickelten Strategien keineswegs konträrer Natur,
vielmehr lassen sie sich als einzelne, sich gegenseitig ergänzende
Bausteine auf dem Weg zu einer \enquote{offeneren} Wissenschaft
betrachten.\footnote{Schäffler, Hildegard (2018).} Sowohl nationale als
auch internationale Open-Access-Initiativen standen auf den
Open-Access-Tagen 2018 wiederholt im Fokus, darunter

\begin{itemize}
\item
  die Berliner Erklärung als eine konsensfähige und grundlegende
  Definition von Open Access,
\item
  Open Access im Forschungsrahmenprogramm Horizon 2020,
\item
  die European Open Science Agenda,
\item
  Jussieu Call for Open Science and Bibliodiversity,
\item
  das DEAL-Projekt sowie
\item
  cOAlition S.
\end{itemize}

Auch Kriterien wie FAIR (Findable, Accessible, Interoperable,
Reusable)\footnote{Vgl. \url{https://www.go-fair.org/fair-principles/}
  (Abgerufen am 04.10.2018).} und die Vienna Principles\footnote{Vgl.
  \url{https://viennaprinciples.org} (Abgerufen am 04.10.2018).} wurden
besprochen. Diese können unter anderem zur Evaluierung genutzt werden.
Dass Metriken wie der Journal Impact Factor (JIF) vor dem Hintergrund
solcher Kriterien allein aufgrund fehlender Offenheit überkommen sind
und Alternativen gefunden werden müssen, wurde beispielsweise in dem
Vortrag \enquote{Der Journal Impact Factor auf dem Prüfstand
{[}\ldots{}{]}}\footnote{Wohlgemuth, Michael, Adam, Michaele, \& Musiat,
  Jutta. (2018): Der Journal Impact Factor auf dem Prüfstand: Grenzen
  und Potentiale in der Qualitätsbewertung von OA-Zeitschriften.
  \url{https://doi.org/10.5281/zenodo.1441258}.} sowie in der Keynote
\enquote{Open Science: Where do we go from here?}\footnote{Mayer, Katja.
  (2018): Open Science: Where do we go from here?.
  \url{https://doi.org/10.5281/zenodo.1443171}.} herausgestellt. In
diesem Zusammenhang könnten auch grundlegende Fragen erneut aufgeworfen
werden, zum Beispiel ob quantitative Metriken generell als Maß genutzt
werden dürfen, wenn es über die reine Popularität von Zeitschrift oder
Autor*in hinaus um das Treffen qualitativer Aussagen geht. Was ist dann
textuelle Qualität, und wie kann Wissenschaftlichkeit in diesem Sinne
gemessen werden? An solche Fragen anknüpfend wurde in den Diskursen die
Entwicklung vergleichbarer und aussagekräftiger, wenngleich
kontrastierend zum JIF transparent gemachter Metriken gefordert.

\hypertarget{vielfalt-von-finanzierungs--und-publikationsmodellen}{%
\section{Vielfalt von Finanzierungs- und
Publikationsmodellen}\label{vielfalt-von-finanzierungs--und-publikationsmodellen}}

Ebenfalls als relevant angesehen wurde die Generierung neuer Modelle zur
Finanzierung und Publikation von OA-Ressourcen, denn -- diesbezüglich
waren sich die Teilnehmer*innen der Tagung größtenteils einig -- die
entstehenden Kosten in Form einer sogenannten Article Processing Charge
(APC) sollten weder von den Leser*innen noch den Autor*innen getragen
werden müssen. Einen Ausweg dafür stellen Publikationsfonds dar, sie
gelten als das \enquote{am weitesten verbreitete
Open-Access-Förderinstrument} für Zeitschriftenartikel innerhalb
Deutschlands.\footnote{TU Graz Bibliothek und Archiv (2018), S. 35.}
Auch Services wie DeepGreen\footnote{Vgl.
  \url{https://deepgreen.kobv.de/de/deepgreen/} (Abgerufen am
  04.10.2018).}, welches das automatisierte Hochladen von Dokumenten mit
Zweitveröffentlichungsrecht in OA-Repositorien nach dem Ablauf der
entsprechenden Frist ermöglicht, können einen maßgeblichen Beitrag zum
Wachstum des OA-Anteils und zur Etablierung von OA als Zugriffsparadigma
leisten. Die Vielfalt in den zur Verfügung stehenden Abstufungen von
Grün und Gold OA wurde dabei auf den Open-Access-Tagen eher als Chance
für die Durchsetzungs- und Anpassungsfähigkeit von Open Access und
weniger als Risiko im Sinne eines höheren organisatorischen und
strukturellen Aufwands angesehen.

\hypertarget{ausblick}{%
\section{Ausblick}\label{ausblick}}

Die Vielfalt ist somit ein Aspekt, der vor dem Kontext aktueller
Entwicklungen im Bereich OA nicht zu stark in den Hintergrund rücken
darf. Man könnte diese Vielfalt allerdings auch als Risiko für die
Offenheit sehen, falls die verschiedenen Strömungen und Akteure nicht in
ausreichendem Maße aufeinander eingehen und sich damit in den einzelnen
Entwicklungen zu stark voneinander isolieren. So ruft beispielsweise
Katja Mayer in ihrer Keynote dazu auf, dass sich die OA-Community mehr
in die Entwicklung von Strategien einmischen sollte, da es aktuell nur
wenig der aktiven Arbeit innerhalb der Community in die offiziellen
Policies schafft. Genau deshalb sind Veranstaltungen wie die
Open-Access-Tage so wichtig, da sie in vorbildlicher Weise Personen aus
den verschiedensten Ebenen, Organisationen und Disziplinen
zusammenbringen und Kommunikation über solche Hürden hinweg ermöglichen.

Abschließend möchte ich mich ganz herzlich bei LIBREAS und beim
LIBREAS-Verein für die Unterstützung in Form der Reise zu und Teilnahme
an den \enquote{Open-Access-Tagen 2018} bedanken. Dadurch erhielt ich
nicht nur den Zugang zu aktuellen Diskussionen rund um Open Access,
sondern gleichermaßen die Möglichkeit, mich bereits als Studentin aktiv
in diese einzubringen.

\hypertarget{literatur}{%
\section{Literatur}\label{literatur}}

Butcher, Neil (2013) \emph{Was sind Open Educational Resources? Und
andere häufig gestellte Fragen zu OER}. Deutsche Fassung bearbeitet von
Barbara Marina und Jan Neumann. Bonn: Deutsche UNESCO-Kommission.
Bearbeitete Übersetzung von: Butcher, Neil (2011) \emph{A Basic Guide to
Open Educational Resources (OER)}. Commonwealth of Learning und UNESCO,
S. 1-22.

Schäffler, Hildegard (2018, 25. September) \emph{Projekt DEAL:
Open-Access-Transformation im Publish \& Read-Modell}. Keynote bei den
Open-Access-Tagen 2018, Technische Universität Graz. {[}Eigene
Mitschrift{]}

TU Graz Bibliothek und Archiv (2018) \emph{Open Access Tage 2018;
Vielfalt durch Open Access}. Verfügbar unter:
\href{http://diglib.tugraz.at/open-access-tage-2018-graz-vielfalt-durch-open-access-technische-universitaet-graz-24-26-september-2018--2018}{http://diglib.tugraz.at/open-access-tage-2018-graz-vielfalt-durch-open-access-tech\-nische-universitaet-graz-24-26-september-2018--2018}.

\pagebreak

%autor
\begin{center}\rule{0.5\linewidth}{\linethickness}\end{center}

\textbf{Sophie Schneider} war die erste Stipendiat*in des LIBREAS.
Verein.

\end{document}
