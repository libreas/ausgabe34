\begin{center}\rule{0.5\linewidth}{\linethickness}\end{center}

\textbf{Prof.~Vivien Petras, PhD} ist geschäftsführende
Institutsdirektorin und hat den Lehrstuhl für Information Retrieval am
Institut für Bibliotheks- und Informationswissenschaft seit 2009 inne.
Ihre Forschungsschwerpunkte sind die Evaluation von
Informationssystemen, mehrsprachige Aspekte des Information Retrievals,
Informationssysteme für das Kulturerbe und die Wissensorganisation.

\textbf{Prof.~Dr.~Elke Greifeneder} ist Juniorprofessorin und
stellvertretende geschäftsführende Direktorin am IBIund hat den
Lehrstuhl für Information Behavior am Institut für Bibliotheks- und
Informationswissenschaft seit 2014 inne. Ihre Forschungsschwerpunkte
sind das Informationsverhalten von Menschen in der Interaktion mit
Technik, mit Schwerpunkten in der Methodenforschung, Validitätsforschung
und Studien in natürlichen Nutzerkontexten.

\textbf{Thomas Roesnick} studiert aktuell im sechsten Bachelorsemester
„Bibliotheks- und Informationswissenschaft`` am Institut. Momentan
schreibt er an seiner Bachelorarbeit und würde danach gerne an der
Humboldt-Universität den Master in „Deutsche Literatur`` belegen.

\textbf{Felicitas Härting} studiert zurzeit im zweiten Bachelorsemester
„Bibliotheks- und Informationswissenschaft`` am Institut. 2013 hat sie
die Ausbildung zur „Fachangestellten für Medien- und
Informationsdienste`` in der Stadtbibliothek Steglitz-Zehlendorf
absolviert und arbeitet seitdem auch dort. Nach ihrem Bachelorabschluss
würde sie gerne in einer Kinder- und Jugendbibliothek arbeiten, das
Referat für Jugendliteratur betreuen und Veranstaltungen für und mit
Kindern auf die Beine stellen.

\textbf{Miriam Brauer} studiert aktuell im vierten Bachelorsemester
Bibliotheks- und Informationswissenschaft am Institut. Zudem ist sie als
studentische Mitarbeiterin am Lehrstuhl für Information Processing and
Analytics tätig.
