\documentclass[a4paper,
fontsize=11pt,
%headings=small,
oneside,
numbers=noperiodatend,
parskip=half-,
bibliography=totoc,
final
]{scrartcl}

\usepackage{synttree}
\usepackage{graphicx}
\setkeys{Gin}{width=.4\textwidth} %default pics size

\graphicspath{{./plots/}}
\usepackage[ngerman]{babel}
\usepackage[T1]{fontenc}
%\usepackage{amsmath}
\usepackage[utf8x]{inputenc}
\usepackage [hyphens]{url}
\usepackage{booktabs} 
\usepackage[left=2.4cm,right=2.4cm,top=2.3cm,bottom=2cm,includeheadfoot]{geometry}
\usepackage{eurosym}
\usepackage{multirow}
\usepackage[ngerman]{varioref}
\setcapindent{1em}
\renewcommand{\labelitemi}{--}
\usepackage{paralist}
\usepackage{pdfpages}
\usepackage{lscape}
\usepackage{float}
\usepackage{acronym}
\usepackage{eurosym}
\usepackage[babel]{csquotes}
\usepackage{longtable,lscape}
\usepackage{mathpazo}
\usepackage[normalem]{ulem} %emphasize weiterhin kursiv
\usepackage[flushmargin,ragged]{footmisc} % left align footnote
\usepackage{ccicons} 

%%%% fancy LIBREAS URL color 
\usepackage{xcolor}
\definecolor{libreas}{RGB}{112,0,0}

\usepackage{listings}

\urlstyle{same}  % don't use monospace font for urls

\usepackage[fleqn]{amsmath}

%adjust fontsize for part

\usepackage{sectsty}
\partfont{\large}

%Das BibTeX-Zeichen mit \BibTeX setzen:
\def\symbol#1{\char #1\relax}
\def\bsl{{\tt\symbol{'134}}}
\def\BibTeX{{\rm B\kern-.05em{\sc i\kern-.025em b}\kern-.08em
    T\kern-.1667em\lower.7ex\hbox{E}\kern-.125emX}}

\usepackage{fancyhdr}
\fancyhf{}
\pagestyle{fancyplain}
\fancyhead[R]{\thepage}

% make sure bookmarks are created eventough sections are not numbered!
% uncommend if sections are numbered (bookmarks created by default)
\makeatletter
\renewcommand\@seccntformat[1]{}
\makeatother


\usepackage{hyperxmp}
\usepackage[colorlinks, linkcolor=black,citecolor=black, urlcolor=libreas,
breaklinks= true,bookmarks=true,bookmarksopen=true]{hyperref}
\usepackage{breakurl}

%meta
%meta

\fancyhead[L]{A. Heinrich, A. Runge\\ %author
LIBREAS. Library Ideas, 34 (2018). % journal, issue, volume.
\href{http://nbn-resolving.de/}
{}} % urn 
% recommended use
%\href{http://nbn-resolving.de/}{\color{black}{urn:nbn:de...}}
\fancyhead[R]{\thepage} %page number
\fancyfoot[L] {\ccLogo \ccAttribution\ \href{https://creativecommons.org/licenses/by/4.0/}{\color{black}Creative Commons BY 4.0}}  %licence
\fancyfoot[R] {ISSN: 1860-7950}

\title{\LARGE{GenderOpen -- ein Repositorium für die Geschlechterforschung}}% title
\author{Andreas Heinrich \& Anita Runge} % author

\setcounter{page}{1}

\hypersetup{%
      pdftitle={GenderOpen -- ein Repositorium für die Geschlechterforschung},
      pdfauthor={Andreas Heinrich \& Anita Runge},
      pdfcopyright={CC BY 4.0 International},
      pdfsubject={LIBREAS. Library Ideas, 34 (2018).},
      pdfkeywords={Open Access, fachliche Repositorien, Grüner Weg des Open Access},
      pdflicenseurl={https://creativecommons.org/licenses/by/4.0/},
      pdfcontacturl={http://libreas.eu},
      baseurl={http://libreas.eu},
      pdflang={de},
      pdfmetalang={de}
     }



\date{}
\begin{document}

\maketitle
\thispagestyle{fancyplain} 

%abstracts

%body
Nach veraltetem Wortgebrauch ist ein Repositorium ein Büchergestell oder
ein Aktenschrank, also ein Aufbewahrungsort für geschriebene oder
gedruckte Medien, die nicht permanent in Gebrauch sind. Moderne
Repositorien sind Speicherorte für Daten, also ebenfalls Ablageorte für
die nachhaltige und sichere Verwahrung von Informationen. Insbesondere
fachliche Repositorien, also digitale Speicherorte für wissenschaftliche
Disziplinen oder Felder, erfüllen aber neben der Ablagefunktion weitere
wichtige Aufgaben. Wie Prof.~Wolfgang Schön, Vizepräsident der Deutschen
Forschungsgemeinschaft, in seiner Keynote bei den Open-Access-Tagen 2016
in München betonte, sind mit dem Betrieb fachlicher Repositorien im
besten Falle deutliche Struktureffekte zu erwarten. Im Hinblick auf das
von der DFG geförderte Repositorium für die Geschlechterforschung werde
deshalb nicht nur ein Publikationsforum etabliert. Über die Plattform
\emph{GenderOpen} werde auch das Community Building für ein dezidiert
inter- und transdisziplinäres Forschungsfeld vorangetrieben und die
Modernisierung traditioneller Publikationsmodelle befördert.\footnote{Vgl.
  \url{https://videoonline.edu.lmu.de/en/node/8215}, ab Min. 10.
  {[}Zugriff: 01.10.2018{]}.} Dieses Community Building sei ein
wichtiges Ziel in den DFG-Fördermaßnahmen zur Unterstützung der
wissenschaftlichen Infrastruktur.

Mit dem Aufbau eines fachlichen Repositoriums sind also Chancen und
Herausforderungen verbunden, die über die Sammlung,
(Zweit-)Veröffentlichung und nachhaltige Speicherung von
wissenschaftlichen Ergebnissen weit hinausgehen. Im Folgenden sollen
einige der Herausforderungen und die im Kontext von \emph{GenderOpen}
gefundenen Lösungen dargestellt werden. Ziel ist es, Anregungen und
Hinweise für existierende oder noch aufzubauende Fachrepositorien,
insbesondere in kleineren, inter- und transdisziplinär arbeitenden
Feldern, zu formulieren und die wissenschaftliche Diskussion über Nutzen
und Risiken digitaler Speicherorte voranzutreiben.

\hypertarget{projektbeschreibung}{%
\section{Projektbeschreibung}\label{projektbeschreibung}}

Der Aufbau des Repositoriums \emph{GenderOpen} erfolgt im Rahmen eines
von der DFG-geförderten Verbundprojekts der universitären
Geschlechterforschungszentren Berlins.\footnote{\url{http://gepris.dfg.de/gepris/projekt/286526860}
  {[}Zugriff: 01.10.2018{]}.} Beteiligt sind das
Margherita-von-Brentano-Zentrum (Freie Universität), das Zentrum für
transdisziplinäre Geschlechterstudien (Humboldt-Universität) und das
Zentrum für interdisziplinäre Frauen- und Geschlechterforschung
(Technische Universität). Das Projekt ist auf zwei Jahre angelegt und
startete am 1. Oktober 2016. Gefördert wird GenderOpen von der DFG im
Umfang von knapp 500.000 Euro. Personell umfasst das Projekt drei
Projektleiterinnen aus jeweils einer der beteiligten Einrichtungen, drei
wissenschaftliche Mitarbeiter\_innen in Vollzeit mit jeweils einer
studentischen Hilfskraft sowie eine Informatikerin, die mit 25 \% ihrer
Stelle anteilig für \emph{GenderOpen} arbeitet.

Die einzelnen Projektarbeitspakete sind auf die drei beteiligten
Geschlechterforschungseinrichtungen verteilt. Zu Projektbeginn waren der
Bereich Content Akquise an der Technischen Universität, der Bereich
Metadaten an der Humboldt-Universität und die Bereiche
Projektkommunikation und -koordination, technische Einrichtung sowie
Öffentlichkeitsarbeit an der Freien Universität angesiedelt. Aufgrund
personeller Veränderungen sowie des in der zweiten Projekthälfte noch
stärkeren Fokus auf die Content Akquise hat sich diese Aufteilung leicht
verschoben. Die Arbeitspakete als solche sind aber erhalten geblieben.

Seit dem 4. Dezember 2017 ist \emph{GenderOpen} online und über die
Adresse www.genderopen.de\footnote{\url{https://www.genderopen.de}
  {[}Zugriff: 01.10.2018{]}.} erreichbar. Mit dem Launch im Dezember
wurde der im Projektplan vorgesehene Onlinegang nach einem guten
Projektjahr erreicht. \emph{GenderOpen} wird mit der freien
Repositoriums-Software DSpace in der Version 6.2 betrieben. Als
Kooperationspartner, der den \emph{GenderOpen}-Server hostet und
betreut, konnte im Vorfeld des Projekts die Arbeitsgruppe Elektronisches
Publizieren (AG EPUB) an der Humboldt-Universität gewonnen werden.
\emph{GenderOpen} profitiert dabei von der DSpace-Expertise der
HU-Arbeitsgruppe, da der von der AG EPUB betreute edoc-Server der HU
Berlin ebenfalls unter DSpace 6.2 läuft.

\hypertarget{herausforderungen}{%
\section{Herausforderungen}\label{herausforderungen}}

\hypertarget{rechtliches-und-lizenzen}{%
\subsection{Rechtliches und Lizenzen}\label{rechtliches-und-lizenzen}}

Das Thema \enquote{Rechtliches und Lizenzen} erwies sich hinsichtlich
der Tragweite und des Bearbeitungsaufwandes als erheblich
anspruchsvoller als zu Projektbeginn absehbar. Dies ergab sich vor allem
daraus, dass bei \emph{GenderOpen} -- anders als in vielen anderen
Repositorien -- von Anfang an konsequent auf \enquote{echten} Open
Access gesetzt wurde. Das Team konnte dabei nur bedingt die Strategien
in Rechtsfragen und Verhandlungen anderer Open-Access-Repositorien
übernehmen, wo sehr häufig Lizenzen zur Anwendung kommen, die zwar eine
einfache Nutzung für den privaten Gebrauch, nicht aber eine freie
Nachnutzung erlauben. Es zahlte sich aus, dass sich entsprechend alle
drei Projektmitarbeiter\_innen intensiv mit dieser Thematik
auseinandersetzen mussten, da sich die rechtliche Expertise inzwischen
auf mehrere Teammitglieder verteilt. Über die gemeinsame
Auseinandersetzung mit urheberrechtlichen Fragen, die Zusammenarbeit mit
dem Rechtsamt der Freien Universität sowie die Teilnahme an
Fortbildungen zum Urheberrecht konnten sich die
Projektmitarbeiter\_innen eine sichere Arbeitsgrundlage in diesem
Bereich verschaffen. Im Frühjahr 2017 konnte das \emph{GenderOpen}-Team
schließlich wichtige Informationen zu Rechten und Lizenzen auf dem
Projektblog zur Verfügung stellen\footnote{\url{https://blog-genderopen.de/informationen-fuer-autor_innen/faq}
  {[}Zugriff: 01.10.2018{]}.} und mit der individuellen Beratung von
Autor\_innen beginnen. Zudem fand im März 2017 ein Workshop für
Multiplikator\_innen statt, in dem zu wesentlichen Teilen auch
rechtliche Belange Thema waren.

\emph{GenderOpen} fühlt sich einer Definition von Open Access im Sinne
der Berliner Erklärung\footnote{Berliner Erklärung über den offenen
  Zugang zu wissenschaftlichem Wissen (2003):
  \url{https://openaccess.mpg.de/Berliner-Erklaerung} {[}Zugriff:
  01.10.2018{]}.} verpflichtet. Angestrebt wird also nicht nur der
kostenfreie Zugang zu wissenschaftlicher Literatur als
Minimalvorstellung von Open Access, sondern auch die Möglichkeit der
(freien) Nachnutzung. Dazu zählen beispielsweise die erlaubte
Weiterverbreitung, Nutzung für die Lehre oder Bearbeitungen. Um die Idee
des Libre Open Access möglichst umfassend umzusetzen, empfiehlt
\emph{GenderOpen}, soweit dies urheberrechtlich erlaubt ist, die
Verwendung von Creative-Commons-Lizenzen. Im Idealfall handelt es sich
dabei um die Lizenz CC-BY\footnote{Creative-Commons-Lizenz Namensnennung
  4.0 International:
  \url{https://creativecommons.org/licenses/by/4.0/deed.de} {[}Zugriff:
  01.10.2018{]}.}, die eine freie Nachnutzung weitgehend ermöglicht.
Diese Policy verfolgt das Projektteam bisher erfolgreich sowohl im
Kontakt mit Autor\_innen als auch in den Verhandlungen mit Verlagen.

Eine wesentliche urheberrechtliche Grundlage für die
Zweitveröffentlichung auf \emph{GenderOpen} ist der Paragraph 38 des
Urheberrechtsgesetzes. In den Absätzen eins und zwei sind für
Zeitschriftenartikel beziehungsweise Sammelbandbeiträge wesentliche
Handlungsoptionen geregelt. Darin heißt es, dass etwa ein Verlag im
Zweifel ein ausschließliches Nutzungsrecht zur Vervielfältigung,
Verbreitung und öffentlichen Zugänglichmachung erwirbt, wenn die
Urheber\_innen in eine Publikation einwilligen. Wenn jedoch keine
gegenteilige ausdrückliche Vereinbarung vorliegt, endet die
Ausschließlichkeit des Nutzungsrechts für den Verlag nach 12 Monaten und
die Urheber\_innen können ihr Werk anderweitig vervielfältigen,
verbreiten und öffentlich zugänglich machen.\footnote{Vgl.
  \textsc{Dreier}, Thomas und Gernot \textsc{Schulze}:
  \emph{Urheberrechtsgesetz: Urheberrechtswahrnehmungsgesetz,
  Kunsturhebergesetz\,; Kommentar}, 5. Aufl., München: C.H. Beck 2015,
  Randnotiz 16.\\
  \textsc{Wandtke}, Artur-Axel, Winfried \textsc{Bullinger} und Ulrich
  \textsc{Block} (Hrsg.): \emph{Praxiskommentar zum Urheberrecht}, 4.,
  neu bearb. Aufl., Gesetzesstand 1. April 2014 Aufl., München: Beck
  2014, Randnotiz 8.} Für den Bereich der Geschlechterforschung ist §38
Abs. 1 und 2 des Urheberrechtsgesetzes häufig anwendbar, da gerade bei
älteren Zeitschriftenartikeln oder Beiträgen in Sammelbänden keine
explizite Vereinbarung zwischen Urheber\_innen und Verlag abgeschlossen
wurde oder zum damaligen Zeitpunkt Autor\_innenrichtlinien
beziehungsweise Verlagspolicies nicht vorlagen. Dadurch ist es
Autor\_innen häufig möglich, Zeitschriftenartikel oder
Sammelbandbeiträge auf \emph{GenderOpen} erneut zu veröffentlichen und
zudem durch die Verwendung einer Creative-Commons-Lizenz echten Open
Access für ihr Werk zu realisieren.\footnote{Vgl. \textsc{Steinhauer},
  Eric W.: \emph{Urheberrecht und Wissenschaft in der digitalen Welt:
  ein kurzer Problemaufriss}, Hagen: FernUniversität in Hagen 2013, S.
  7, \url{http://nbn-resolving.de/urn:nbn:de:hbz:708-dh2282}.}

Um Rechtssicherheit herzustellen, verlangt \emph{GenderOpen} die
Einsendung einer unterschriebenen Einverständniserklärung\footnote{\url{https://blog-genderopen.de/wp-content/uploads/2017/03/Einverst\%C3\%A4ndniserkl\%C3\%A4rung-2.pdf}
  {[}Zugriff: 01.10.2018{]}.} beziehungsweise des Vertrags über die
elektronische Veröffentlichung\footnote{\url{https://blog-genderopen.de/wp-content/uploads/2017/11/Depositlizenz_final.pdf}
  {[}Zugriff: 01.10.2018{]}.} in ausgedruckter Papierform. Darin
bestätigen die Autor\_innen unter anderem ihre Berechtigung der
Zweitveröffentlichung und versichern, dass keine Rechte Dritter
entgegenstehen. Während der laufenden Projektphase wird der Content
aktiv eingeworben. Die Kontaktaufnahme und Aufklärung über rechtliche
Fragen sind zeit- und personalintensiv. Daher haben sich in der zweiten
Projekthälfte seit dem Onlinegang die Arbeitsbereiche zugunsten der
Content-Akquise verschoben.

\hypertarget{content-akquise}{%
\subsection{Content-Akquise}\label{content-akquise}}

Im Bereich der Einwerbung von Publikationen verfolgt das Projektteam
eine zweigleisige Strategie. Zum einen werden gezielt Kontakte zu
zentralen Verlagen für die Geschlechterforschung aufgebaut und gepflegt.
Das Spektrum umfasst dabei sowohl große, weltweite operierende Verlage,
mittelständische Unternehmen sowie Kleinverlage. Aus diesem Grund
gestaltet sich die Zusammenarbeit sehr unterschiedlich. So gibt es
beispielsweise hier feste Ansprechpartner\_innen mit definierter
Funktion und Expertise, dort unklare Zuständigkeiten mit teils
lückenhaften Kenntnissen, etwa in urheberrechtlichen Fragen. In der
Zwischenzeit ist es dem Projektteam gelungen, mit zwei Verlagen
Kooperationsvereinbarungen über die regelmäßige Datenlieferung
(Volltexte und Metadaten) einschlägiger Zeitschriften abzuschließen.
Dazu zählen \emph{GENDER}, die \emph{Freiburger Zeitschrift für
GeschlechterStudien}, \emph{Femina Politica} (Verlag Barbara Budrich)
und die \emph{feministischen studien} (Verlag Walter de Gruyter).
Erfreulicherweise konnte mit beiden Verlagen eine Verfügbarmachung der
originalen Verlagsversionen unter einer Creative-Commons-Lizenz
vereinbart werden.

Zum anderen sucht das Projektteam den direkten Kontakt zu Autor\_innen
etwa von Zeitschriftenartikeln und Sammelbandbeiträgen. In personal- und
zeitaufwändiger Arbeit werden Kontaktdaten recherchiert, um die
Autor\_innen anschließend persönlich über die Möglichkeit der
Zweitveröffentlichung auf \emph{GenderOpen} sowie urheberrechtliche
Fragen zu informieren. Den Anschreiben ist eine Einverständniserklärung
über die Zweitveröffentlichung beigefügt, welche die Autor\_innen unter
Angabe einer oder mehrerer ihrer Publikationen sowie der gewünschten
Lizenz an das \emph{GenderOpen}-Team zurücksenden können. Während der
noch laufenden Projektphase kann das Projektteam das Hochladen der
freigegebenen Publikationen übernehmen. Zukünftig sollen die
Autor\_innen im Idealfall ihre Publikationen selbstständig über die
\emph{GenderOpen}-Webseite hochladen, wenn auch die Erfahrung anderer
Fachrepositorien zeigt, dass der prozentuale Anteil selbst eingereichter
Publikationen nicht selten im unteren einstelligen Bereich liegt.

Einen besonderen Werbeeffekt erhofft sich das \emph{GenderOpen}-Team
daher von der Zusammenarbeit mit zentralen Autor\_innen und
\enquote{Pionier\_innen} der Frauen- und Geschlechterforschung. Seit dem
Projektbeginn konnten rund 20 dieser Botschafter\_innen für
\emph{GenderOpen} gewonnen werden.\footnote{Vgl. \textsc{Runge}, Anita:
  \enquote{\emph{Das Botschafterinnen-Modell\,: Auf dem Weg zu einem
  Repositorium für die Geschlechterforschung}}, in:
  \emph{Wissenschaftlerinnen-Rundbrief} \emph{2} (2014), S. 7--9,
  \url{https://refubium.fu-berlin.de/handle/fub188/19417} {[}Zugriff:
  01.10.2018{]}.\\
  Zu einem anderen Testimonial-Modell im Bibliotheks- und
  Informationswesen vgl. bspw. eine Kampagne der Zentralbibliothek für
  Wirtschaftswissenschaften. \textsc{Sigfried}, Doreen:
  \enquote{\emph{Imagekampagne der ZBW\,: Digitale Kompetenz
  kommunizieren}},
  \url{https://www.zbw-mediatalk.eu/de/2016/11/imagekampagne-der-zbw-digitale-kompetenz-kommunizieren/}.
  {[}Zugriff: 01.10.2018{]}.} Herausragende Autorinnen wie etwa
Christina Thürmer-Rohr, Hildegard Nickel oder Christina von Braun haben
ihre Publikationen soweit möglich unter Creative-Commons-Lizenzen auf
\emph{GenderOpen} zweitveröffentlicht beziehungsweise haben dies
zugesagt. \emph{GenderOpen} verspricht sich vom Modell der
Botschafter\_innen zum einen eine verstärkte Nutzung, da nun
einschlägige Literatur teilweise erstmals elektronisch unter überwiegend
freien Lizenzen verfügbar ist. Zum anderen erhofft sich das Team einen
Nachahmungseffekt auf andere Autor\_innen, eigene Texte auf
\emph{GenderOpen} zu veröffentlichen. Auch für den Einsatz in der
universitären Lehre ist die Verfügbarkeit dieser einschlägigen Texte als
Open Educational Resources von zentraler Bedeutung.

\hypertarget{erarbeitung-eines-schlagwort-vokabulars}{%
\subsection{Erarbeitung eines
Schlagwort-Vokabulars}\label{erarbeitung-eines-schlagwort-vokabulars}}

Dass die bibliothekarische Inhaltserschließung mittels etablierter
Prinzipien und Dokumentationssprachen Geschlechterstereotype
reproduziert oder spezifische Sachverhalte aus der Geschlechterforschung
schlicht nicht abbildet, wurde in diversen Publikationen der vergangenen
Jahre auch im Bibliothekswesen wiederholt kritisiert.\footnote{Vgl.
  bspw. \textsc{Aleksander}, Karin: \enquote{\emph{Die Frau im
  Bibliothekskatalog}}, in: \emph{LIBREAS} \emph{25} (2014), S. 8--16,
  \url{https://doi.org/10.25595/400}.\\
  \textsc{Adler}, Melissa: \emph{Cruising the library: perversities in
  the organization of knowledge}, First edition, New York: Fordham
  University Press 2017.\\
  \textsc{Sparber}, Sandra: \enquote{\emph{What's the frequency,
  Kenneth? -- Eine (queer)feministische Kritik an Sexismen und Rassismen
  im Schlagwortkatalog}}, in: \emph{Mitteilungen der VÖB}
  \emph{69}/\emph{2} (2016), S. 236--243,
  \url{https://doi.org/10.25595/93}.\\
  \textsc{Zechner}, Rosa: \enquote{\emph{Zwischen Anspruch und
  Möglichkeit. Frauen*Solidarität: ein Beispiel aus der
  Beschlagwortung}}, in: \emph{Mitteilungen der VÖB} \emph{69}/\emph{2}
  (2016), S. 244--252, \url{https://doi.org/10.25595/96}.} Im
\emph{GenderOpen}-Projektvorhaben ist daher innerhalb des Arbeitspakets
Metadaten die Erarbeitung eines Vokabulars für die geschlechtersensible
Inhaltserschließung auf \emph{GenderOpen} vorgesehen. Eine teaminterne
vierköpfige Arbeitsgruppe nahm zu Jahresbeginn 2017 die Arbeit an der
Entwicklung eines solchen Vokabulars auf.

Entscheidend für die Wahl der Arbeitsweise und des Ansatzes war, dass
das Vokabular möglichst beim Onlinegang von \emph{GenderOpen} zur
Verfügung stehen sollte und dass die Pflege der Schlagwortliste auch
nach Projektende gewährleistet sein muss. Daher kamen weder die
ressourcenintensive Erstellung eines komplexen Thesaurus noch die
Erarbeitung eines sehr umfangreichen Konvoluts an Deskriptoren in Frage.
Als Orientierung wurde ein Umfang von etwa 500 Begriffen festgelegt. Das
Vokabular sollte nicht gänzlich neu erarbeitet, sondern auf Grundlage
dreier vorhandener und für die Arbeitsgruppe leicht verfügbarer Listen
entwickelt werden. Begonnen wurde zunächst damit, dass die
Schlagwortlisten der Genderbibliothek am Zentrum für transdisziplinäre
Geschlechterstudien\footnote{Schlagwortliste der Genderbibliothek am
  Zentrum für transdisziplinäre Geschlechterstudien
  (Humboldt-Universität): \url{http://genderbibliothek.de/Topics/List}
  {[}Zugriff: 01.10.2018{]}.}, der Rezensionszeitschrift
querelles-net\footnote{Homepage der Rezensionszeitschrift für Frauen-
  und Geschlechterforschung \emph{querelles-net}:
  \url{http://www.querelles-net.de} {[}Zugriff: 01.10.2018{]}.} sowie
des Projekts \enquote{Nach Bologna}\footnote{\textsc{Malli}, Gerlinde,
  Susanne \textsc{Sackl} -\textsc{Sharif} und Elisabeth
  \textsc{zehetner}: \emph{Nach Bologna: Gender Studies in der
  unternehmerischen Hochschule. Eine Untersuchung in Österreich und der
  Schweiz}, Graz: Inst. f. Soziologie 2015.
  \url{http://unipub.uni-graz.at/urn:nbn:at:at-ubg:3-2168} {[}Zugriff:
  01.10.2018{]}.} zu einer Wortgutsammlung im Umfang von etwa 16.000
Begriffen zusammengeführt wurden. Anschließend wurden darin anhand
quantitativer Kriterien 100 \enquote{Top-Schlagworte} identifiziert, die
als Grundgerüst in der Folge schrittweise um weitere Deskriptoren
angereichert wurden und schließlich zum \emph{GenderOpen}-Vokabular
anwuchsen. In einem etwa einmonatigen Turnus traf sich die Arbeitsgruppe
ab Februar 2017 regelmäßig, um die Wortgutsammlung schrittweise und Wort
für Wort von A-Z durchzuarbeiten, relevante Begriffe zu identifizieren
und intensiv etwa über Verwendungen und Schreibweisen zu diskutieren.
Orientierung bei der Auswahl und Diskussion möglicher Deskriptoren
lieferten vorhandene Vokabulare und Kataloge aus dem Feld der Frauen-
und Geschlechterforschung sowie der Frauenbewegung. Dazu zählen
beispielsweise der österreichische Thesaurus ‚thesaurA', der Thesaurus
des FrauenMediaTurm, der Women's Thesaurus (Atria), der Gender Equality
Glossary and Thesaurus (EIGE) sowie der META-Katalog des
i.d.a.-Netzwerks. Für bestimmte Einzelthemen oder Themenkomplexe wurden
gezielt Expert\_innen aus dem Feld angesprochen und um kritische
Stellungnahmen gebeten.

Bei der teaminternen Diskussion und letztlichen Auswahl möglicher
Deskriptoren für das \emph{GenderOpen}-Vokabular bildeten mehrere
Prinzipien einen Bezugsrahmen, die in der kritischen Auseinandersetzung
mit Dokumentationssprachen bereits entwickelt wurden. So war es möglich,
Begriffe auf ihr Potential zur sprachlichen Sichtbarmachung von
Sachverhalten oder Personengruppen, zur Herstellung von Symmetrie, zur
Ergänzung fehlender Termini sowie der Vermeidung sprachlicher
Diskriminierung zu prüfen.\footnote{Vgl. \textsc{Klösch-Melliwa}, Helga:
  \enquote{\emph{Frauenrelevante/feministische Inhaltserschließung}},
  in: \textsc{Frida, Verein zur Förderung und Vernetzung
  Frauenspezifischer Informations- \& Dokumentationseinrichtungen in
  Österreich} (Hrsg.): \emph{KolloquiA\,; frauenbezogene, feministische
  Dokumentation und Informationsarbeit in Österreich\,; Lehr- und
  Forschungsmaterialien}, Bd. 11, Wien: Bundesministerium für Bildung,
  Wiss. und Kultur 2001 (Materialien zur Förderung von Frauen in der
  Wissenschaft), S. 450ff.}

Mit Blick auf den großen Arbeitsaufwand für die Erarbeitung des
\emph{GenderOpen}-Vokabulars, das Streben nach Offenheit und Rückmeldung
sowie interessierte Anfragen von verschiedenen Seiten reifte innerhalb
der Arbeitsgruppe die Idee, die Schlagwortliste über \emph{GenderOpen}
hinaus anderen Projekten und Einrichtungen aus der Geschlechterforschung
zur Verfügung zu stellen. Mehrere Zeitschriftenredaktionen
(beispielsweise \emph{GENDER}, \emph{Open Gender Journal}) sowie einige
i.d.a.-Einrichtungen für das Digitale Deutsche Frauenarchiv\footnote{\href{http://www.ida-dachverband.de/ueber-ida/}{http://www.ida-dachverband.de/}
  {[}Zugriff 01.10.2018{]}.} nutzen das \emph{GenderOpen}-Vokabular
inzwischen nach. Um die Einheitlichkeit des Vokabulars zu gewährleisten
und Eigenentwicklungen an den verschiedenen Nutzungsorten zu verhindern,
ist die Gründung einer Redaktion geplant. Im Oktober 2018 wird am
Zentrum für transdisziplinäre Geschlechterstudien an der
Humboldt-Universität ein Schlagwort-Workshop für Interessierte
stattfinden. Dabei soll auch die Redaktion ihre Arbeit aufnehmen. So
konnte die teaminterne Arbeitsgruppe die Entwicklung einer ersten
Version eines geschlechtersensiblen Vokabulars erfolgreich zum Abschluss
bringen und damit einen Impuls zu dessen gemeinschaftlicher
Weiterentwicklung durch die Community geben. Zu denken ist dabei etwa an
die perspektivische Arbeit an einem geschlechtersensiblen
Thesaurus.\footnote{Zu dieser Idee vgl. \textsc{Schenk}, Jasmin:
  \enquote{\emph{Konzept zur Entwicklung eines geschlechtersensiblen
  Thesaurus}}, unveröffentlichte Masterarbeit, Technische Hochschule
  Köln 2018.}

\hypertarget{ausblick}{%
\section{Ausblick}\label{ausblick}}

Die mit dem Aufbau von \emph{GenderOpen} angestrebten Struktureffekte,
das Community Building und die umfassende Modernisierung der
Publikationsmodelle in der Geschlechterforschung sind nicht kurzfristig
zu erreichen. Sie bedürfen einer nachhaltigen Absicherung nach Ende der
ersten Projektphase und der Ergänzung durch weitere moderne
Veröffentlichungsmöglichkeiten nicht nur für Zweitveröffentlichungen
(Green Open Access), sondern auch für Originalbeiträge (Golden Open
Access). Entsprechend wurde das Projekt \emph{GenderOpen} durch den
Aufbau einer Plattform für Zeitschriften ergänzt: Gefördert aus Mitteln
des BMBF wird seit Juni 2018 die \emph{Open Gender Platform} aufgebaut,
mit der ein umfassendes Dienstleistungsangebot für Zeitschriften der
Geschlechterforschung entwickelt und mit dem \emph{Open Gender Journal}
ein echtes Open-Access-Angebot für die qualitätsgesicherte, nicht
ausgabengebundene Publikation von Zeitschriftenartikeln aus der
Geschlechterforschung bereitgestellt wird.\footnote{\url{https://www.opengenderjournal.de/}
  {[}Zugriff: 01.10.2018{]}.}

Repositorium und Zeitschriften-Plattform sind Teil eines neuen
Veröffentlichungskonzepts für die Geschlechterforschung, das mit der
Fachcommunity, aber auch mit der Open-Access-Comm\-unity und
Vertreter\_innen von Bibliotheken, anderen Fachrepositorien und
Publikationsplattformen in regelmäßigem Austausch diskutiert und
weiterentwickelt wird. Im Sinne des Ziels der Offenheit werden dabei
alle erarbeiteten Lösungen -- soweit wie möglich -- frei und nachnutzbar
zur Verfügung gestellt. Das Repositorium \emph{GenderOpen} ist ein
Schritt auf dem Weg zu einer besseren Sichtbarkeit und Anerkennung für
die Geschlechterforschung und zugleich ein Beitrag zu einer neuen
Publikationskultur in inter- und transdisziplinär organisierten
wissenschaftlichen Feldern.

%autor
\begin{center}\rule{0.5\linewidth}{\linethickness}\end{center}

\textbf{Andreas Heinrich}, wissenschaftlicher Bibliothekar, ist
Mitarbeiter im DFG-Verbundprojekt GenderOpen -- Fachrepositorium für die
Geschlechterforschung am Margherita-von-Brentano-Zen\-trum an der Freien
Universität Berlin.

\textbf{Anita Runge}, Dr.~phil., ist Geschäftsführerin des
Margherita-von-Brentano-Zentrums an der Freien Universität Berlin und
seit vielen Jahren in der Publikationsförderung aktiv. Sie ist Teil der
kooperativen Leitung des DFG-Projekts GenderOpen -- Fachrepositorium
für die Geschlechterforschung und leitet das BMBF-Projekt Open Gender
Platform.

\end{document}
