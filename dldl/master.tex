\documentclass[a4paper,
fontsize=11pt,
%headings=small,
oneside,
numbers=noperiodatend,
parskip=half-,
bibliography=totoc,
final
]{scrartcl}

\usepackage{synttree}
\usepackage{graphicx}
\setkeys{Gin}{width=.4\textwidth} %default pics size

\graphicspath{{./plots/}}
\usepackage[ngerman]{babel}
\usepackage[T1]{fontenc}
%\usepackage{amsmath}
\usepackage[utf8x]{inputenc}
\usepackage [hyphens]{url}
\usepackage{booktabs} 
\usepackage[left=2.4cm,right=2.4cm,top=2.3cm,bottom=2cm,includeheadfoot]{geometry}
\usepackage{eurosym}
\usepackage{multirow}
\usepackage[ngerman]{varioref}
\setcapindent{1em}
\renewcommand{\labelitemi}{--}
\usepackage{paralist}
\usepackage{pdfpages}
\usepackage{lscape}
\usepackage{float}
\usepackage{acronym}
\usepackage{eurosym}
\usepackage[babel]{csquotes}
\usepackage{longtable,lscape}
\usepackage{mathpazo}
\usepackage[normalem]{ulem} %emphasize weiterhin kursiv
\usepackage[flushmargin,ragged]{footmisc} % left align footnote
\usepackage{ccicons} 
\setcapindent{0pt} % no indentation in captions

%%%% fancy LIBREAS URL color 
\usepackage{xcolor}
\definecolor{libreas}{RGB}{112,0,0}

\usepackage{listings}

\urlstyle{same}  % don't use monospace font for urls

\usepackage[fleqn]{amsmath}

%adjust fontsize for part

\usepackage{sectsty}
\partfont{\large}

%Das BibTeX-Zeichen mit \BibTeX setzen:
\def\symbol#1{\char #1\relax}
\def\bsl{{\tt\symbol{'134}}}
\def\BibTeX{{\rm B\kern-.05em{\sc i\kern-.025em b}\kern-.08em
    T\kern-.1667em\lower.7ex\hbox{E}\kern-.125emX}}

\usepackage{fancyhdr}
\fancyhf{}
\pagestyle{fancyplain}
\fancyhead[R]{\thepage}

% make sure bookmarks are created eventough sections are not numbered!
% uncommend if sections are numbered (bookmarks created by default)
\makeatletter
\renewcommand\@seccntformat[1]{}
\makeatother


\usepackage{hyperxmp}
\usepackage[colorlinks, linkcolor=black,citecolor=black, urlcolor=libreas,
breaklinks= true,bookmarks=true,bookmarksopen=true]{hyperref}
%URLs hart brechen
\makeatletter 
\g@addto@macro\UrlBreaks{ 
  \do\a\do\b\do\c\do\d\do\e\do\f\do\g\do\h\do\i\do\j 
  \do\k\do\l\do\m\do\n\do\o\do\p\do\q\do\r\do\s\do\t 
  \do\u\do\v\do\w\do\x\do\y\do\z\do\&\do\1\do\2\do\3 
  \do\4\do\5\do\6\do\7\do\8\do\9\do\0} 
% \def\do@url@hyp{\do\-} 
\makeatother 

%meta
%meta

\fancyhead[L]{Redaktion LIBREAS \\ %author
LIBREAS. Library Ideas, 34 (2018). % journal, issue, volume.
\href{http://nbn-resolving.de/}
{}} % urn 
% recommended use
%\href{http://nbn-resolving.de/}{\color{black}{urn:nbn:de...}}
\fancyhead[R]{\thepage} %page number
\fancyfoot[L] {\ccLogo \ccAttribution\ \href{https://creativecommons.org/licenses/by/3.0/}{\color{black}Creative Commons BY 3.0}}  %licence
\fancyfoot[R] {ISSN: 1860-7950}

\title{\LARGE{Das liest die LIBREAS, \\ Nr. \#3 (Sommer / Herbst 2018)}} % title
\author{Redaktion LIBREAS} % author

\setcounter{page}{1}

\hypersetup{%
      pdftitle={Das liest die LIBREAS, Nr. \#3 (Sommer / Herbst 2018)},
      pdfauthor={Redaktion LIBREAS},
      pdfcopyright={CC BY 3.0 Unported},
      pdfsubject={LIBREAS. Library Ideas, 34 (2018).},
      pdfkeywords={Open Access},
      pdflicenseurl={https://creativecommons.org/licenses/by/3.0/},
      pdfcontacturl={http://libreas.eu},
      baseurl={http://libreas.eu},
      pdflang={de},
      pdfmetalang={de}
     }



\date{}
\begin{document}

\maketitle
\thispagestyle{fancyplain} 

%abstracts

%body
Beiträge von Karsten Schuldt (ks), Michaela Voigt (mv), Ben Kaden (bk),
Viola Voß (vv)

\hypertarget{zur-kolumne}{%
\section{1. Zur Kolumne}\label{zur-kolumne}}

Das Ziel dieser Kolumne ist, eine Übersicht über die in der letzten Zeit
erschienene bibliothekarische, informations- und
bibliothekswissenschaftliche sowie für diesen Bereich interessante
Literatur zu geben. Enthalten sind Beiträge, die der LIBREAS-Redaktion
oder anderen Beitragenden als relevant erschienen.

Themenvielfalt sowie ein Nebeneinander von wissenschaftlichen und
nicht-wissenschaftlichen Ansätzen wird angestrebt. Auch in der Form
sollen traditionelle Publikationen ebenso erwähnt werden wie
Blogbeiträge oder Videos beziehungsweise TV-Beiträge.

Gern gesehen sind Hinweise auf erschienene Literatur oder Beiträge in
anderen Formaten. Die Redaktion freut sich über entsprechende Hinweise
(siehe \url{http://libreas.eu/about/}, Mailkontakt für diese Kolumne ist
\href{mailto:zeitschriftenschau@libreas.eu}{\nolinkurl{zeitschriftenschau@libreas.eu}}).
Die Koordination der Kolumne liegt bei Karsten Schuldt. Verantwortlich
für die Inhalte sind die jeweiligen Beitragenden. Die Kolumne
unterstützt den Vereinszweck des LIBREAS-Vereins zur Förderung der
bibliotheks- und informationswissenschaftlichen Kommunikation.

LIBREAS liest gern und viel Open-Access-Veröffentlichungen. Wenn sich
Beiträge doch einmal hinter eine Bezahlschranke verbergen, werden diese
durch \enquote{{[}Paywall{]}} gekennzeichnet. Zwar macht das Plugin
Unpaywall (\url{http://unpaywall.org/}) das Finden von legalen
Open-Access-Versionen sehr viel einfacher. Als Service an der
Leserschaft verlinken wir OA-Versionen, die wir vorab finden konnten,
jedoch nach Möglichkeit auch direkt. Für alle Beiträge, die nicht frei
zugänglich sind, empfiehlt die Redaktion Werkzeuge wie den Open Access
Button (\url{https://openaccessbutton.org/}) zu nutzen oder auf Twitter
mit \#icanhazpdf (\url{https://twitter.com/hashtag/icanhazpdf?src=hash})
um Hilfe bei der legalen Dokumentenbeschaffung zu bitten.

\hypertarget{artikel-und-zeitschriftenausgaben}{%
\section{2. Artikel und
Zeitschriftenausgaben}\label{artikel-und-zeitschriftenausgaben}}

Peter Johan Lor kritisiert die Reaktionen von Bibliotheken auf die
Herausforderungen durch \enquote{Fake-News} und ähnliche Entwicklungen
in den letzten Jahren als naiv. Bibliotheken würden immer wieder
betonen, gesicherte Informationen anzubieten und gleichzeitig
Informationskompetenz zu fördern. Aber das sei keine sinnvolle Antwort
auf Entwicklungen, in denen von immer mehr Menschen Fakten an sich
bezweifelt und Identitäten entwickelt würden, die sich gerade darauf
stützen würden, wissenschaftliches Wissen zu ignorieren. Bibliotheken
müssten sich dieser Realität stellen. {[}Lor, Peter Johan (2018).
Democracy, information, and libraries in a time of post-truth discourse.
In: \emph{Library Management}, 39 (2018), 5. 307--321.
\url{https://doi.org/10.1108/LM-06-2017-0061}{]} {[}Paywall{]}
{[}OA-Version: \url{https://repository.up.ac.za/handle/2263/65140}{]}
(ks)

Die Deutsche Nationalbibliothek (DNB) sammelt nun auch -- unter
Umständen -- Forschungsdaten, wie Dirk Weisbrod berichtet. Ausgehend von
der Pflichtablieferungsverordnung (PflAV) allerdings nur, wenn diese
unmittelbar \enquote{zu den ablieferungspflichtigen Netzpublikationen}
gehören, was im Rahmen des eDissPlus-Projektes zu dissertationsbezogenen
Forschungsdaten beforscht und realisiert wurde. Entscheidend ist hierbei
und laut Forschungsdatenpolicy der DNB das Kriterium der
\enquote{Unverzichtbarkeit}, was bedeutet, dass man sie für die
Nachprüfbarkeit der in einer Dissertation präsentierten Ergebnisse
benötigt. Darüber, ob dies der Fall ist, entscheiden entweder die
Promovierenden selbst oder die abliefernde Institution. {[}Weisbrod,
Dirk (2018): Pflichtablieferung von Forschungsdaten. In: \emph{Dialog
mit Bibliotheken} 2018/1, S. 26f.
\url{https://d-nb.info/115432320X/34}{]} (bk)

In diesem Jahr gab es, anlässlich des 200. Geburtstages, unzählige
Rückblicke und Bewertungen zu Karl Marx und seinem Werk. Insoweit ist es
nur passend, hier auch auf ein originär marxistischen Ansatz im
Bibliothekswesen zu verweisen. Sam Popowich postuliert im \emph{Journal
of Radical Librarianship}, dass der Eindruck ständiger Krisen im
Bibliothekswesen (Berufsbild, Diskussionen um Aufgaben von Bibliotheken,
technische Entwicklung, die Angst vor Konkurrenz und
Deprofessionalisierung) stimmt, aber nicht -- wie das andere, von ihm
angeführten Vertreterinnen und Vertreter der \#critlib-Bewegung tun
würden -- auf kulturelle Fragen zu reduzieren sei. Neoliberalismus, so
Popowich, sei nur eine weitere Spielart der kapitalistischen
Entwicklung, die unterliegenden Strukturen seien immer die gleichen.
Notwendig sei weiterhin eine materialistische Analyse. Die Krisen der
letzten Jahrzehnte seien zu erklären mit einem (weiteren)
Entwicklungsschub der Kommodifizierung von Sphären, die ehemals als
ausserhalb der Märkte stehend begriffen wurden. In dieser Runde würden
Bildung und Kultur kommodifiziert, also \enquote{marktfähig gemacht}.
Bibliotheken ständen in diesem Prozess in einer unauflösbaren
Zwickmühle: Passen sie sich zu langsam an, würden sie als potentielle
Märkte (weil noch Gewinn versprechend) angesehen und bedrängt;
verständen sie sich als cutting-edge, würden sie die Anpassungsleistung
an den Markt selber vollziehen. Der Ausweg sei nur durch die Überwindung
der grundlegenden Verhältnisse möglich: \enquote{In order to truly
change the nature of librarianship and the social relations in which we
find ourselves, we must fundamentally change the way labour, production,
and social life are organized.} (Popowich 2018: 17) Während in vielen
Texten zum 200. Marx-Jubiläum eher Kontextualisierung und Historisierung
vorgenommen wurde, zeigt Popowich, dass sich Marx und die marxistische
Analyse für das benutzen lässt, für das sie (auch) gedacht war: Zum
Nachdenken darüber, warum es nötig sein könnte, eine Revolution zu
machen. Wenn es überzeugt. {[}Popowich, Sam (2018): Libraries, Labour,
Capital: On Formal and Real Subsumption. In: \emph{Journal of Radical
Librarianship} 4 (2018), 6--19,
\url{https://journal.radicallibrarianship.org/index.php/journal/article/view/25}{]}
(ks)

Einen guten Einblick in die Realität und Möglichkeiten professioneller
Entwicklung von Bibliothekarinnen und Bibliothekaren liefert eine Studie
von Ramirose Attebury. Die Autorin führte (in den USA) mit zehn
Kolleginnen und Kollegen längere Interviews dazu durch, wie diese ihre
Möglichkeiten der eigenen Weiterentwicklung sehen. Die grundsätzliche
Haltung ist positiv. Das Personal möchte sich weiterentwickeln -- mit
verschiedenen Ziele und Methoden, nicht nur Training, sondern auch
Beteiligung an professionellen Aktivitäten, Lesezirkeln, informellen
Diskussionen, training on the job -- und muss dazu nicht von der
Administration gedrängt werden. Wichtig ist ihnen, (a) dass sie die
Aktivitäten selber als sinnvoll ansehen, (b) die Freiheit haben, zu
wählen, an welcher Aktivität sie sich beteiligen und wie, (c) dass sie
der Meinung sind, das Gelernte auch im Alltag anwenden zu können.
Administrationen können dies unterstützen oder aber auch abwürgen. Der
(wahrgenommene) Zwang zu Weiterbildungen, insbesondere wenn sie als
unnötig wahrgenommen werden, wirkt negativ. Versuche, das vom Personal
erworbene Wissen zu teilen, werden als grundsätzlich sinnvoll, aber in
der Realität auch dysfunktional angesehen. Interne Vorträge nach
Konferenzbesuchen, bei denen dem gesamten Personal berichtet wird, was
auf der Konferenz gelernt wurde, werden teilweise als Zeitverschwendung
wahrgenommen, wenn es kein Interesse am Thema gibt. Vielmehr werden
Möglichkeiten zum informellen Austausch und der direkten Anwendung am
Arbeitsplatz bevorzugt. Solange das Personal die Möglichkeit hat, aktiv
mitzubestimmen, hat es auch Verständnis für Barrieren, beispielsweise
die eingeschränkten finanziellen Möglichkeiten ihrer Einrichtungen. Der
Artikel zeigt, dass Bibliotheken auch als Einrichtungen funktionieren
können, in denen Personal und Administration diese gemeinsam entwickeln
und sich beide Seite ernst genommen fühlen. {[}Attebury, Ramirose
(2018): The Role of Administrators in Professional Development:
Considerations for Facilitating Learning Among Academic Librarians. In:
\emph{Journal of Library Administration} 58 (2018) 5, 407--433
\url{https://doi.org/10.1080/01930826.2018.1468190}{]} {[}Paywall{]}
(ks)

Inhaltlich zu Fragestellungen passend, welche die LIBREAS-Redaktion im
Call for Papers für die letzte Ausgabe thematisierte, kritisiert
Alexandra Gallin-Parisi -- selber Bibliothekarin und Mutter zweier
Kinder --, dass in der bibliothekarischen Literatur praktisch nicht
vorkommt, dass viele Bibliothekarinnen eben auch Mütter sind. Deren
Erfahrungen als Mütter -- unter anderem Schwangerschaft, Kombination von
Mutterschaft und bibliothekarischen Aufgaben, spezifische sexistische
Annahmen über Mütter -- scheinen nicht thematisiert zu werden, so als
wären es zwei getrennte Welten. Gallin-Parisi führt an ihrem eigenen
Beispiel vor, dass eigentlich viele Fragestellungen möglich (und nötig)
wären. Es scheint ihr aber, dass es in der bibliothekarischen Literatur
kaum um das konkrete Personal geht. Da sie sich an ihrem eigenen
Beispiel orientiert, finden sich auch viele konkrete Fragen und
Probleme, die sich aus der spezifischen Situation in den USA (wo die
Autorin arbeitet) ergeben. Einiges wird also im DACH-Raum anders sein,
aber die grundsätzlichen Feststellungen lassen sich wohl übertragen.
Gallin-Parisi fordert ein, dass sich mehr mit dem Thema beschäftigt
wird, dem ist zuzustimmen. {[}Gallin-Parisi, Alexandra (2018): An
academic librarian-mother in six stories. In: \emph{In The Library With
The Lead Pipe}, 30.05.2018,
\url{http://www.inthelibrarywiththeleadpipe.org/2018/academic-librarian-mother/}{]}
(ks)

OA-Policies etablieren sich an einzelnen wissenschaftlichen
Einrichtungen ebenso wie OA-Man\-date aufseiten von
Forschungsfördereinrichtungen. In der Praxis bedeutet dies mitunter
einen Dschungel an Empfehlungen beziehungsweise Auflagen für
Autor*innen. Insbesondere Bibliotheken (Stichwort Research Support) sind
bemüht, dies mit passenden Serviceangeboten abzufangen, welche jedoch
bei einer hohen Anzahl von Publikationen und der Forderung nach
Steigerung der OA-Quoten vor einem Skalierungsproblem stehen. Die UK
Scholarly Communication Licence (UK-SCL) soll dem entgegenwirken: Ziel
ist in Großbritannien ein Werkzeug in der Hand zu haben, mit dem
Autor*innen (und deren Institutionen gleichermaßen) zu festgesetzten
Bedingungen und flächendeckend Open-Access-Zweitveröffentlichungen
umsetzen können -- unabhängig von in Verlagsverträgen getroffenen
Vereinbarungen oder den vorhandenen Self-Archiving-Policies der Verlage.
Baldwin und Pinfield geben in ihrem Beitrag einen Einblick in die
Entstehungsgeschichte der UK-SCL, erläutern die darin festgeschriebenen
Bedingungen (Zweitveröffentlichung des akzeptierten Manuskripts auf
einem Repositorium, direkt nach Erscheinen und unter einer CC
BY-NC-Lizenz) und präsentieren die Ergebnisse einer qualitativen
Befragung von Personen, die an der Ausgestaltung und Implementierung der
UK-SCL beteiligt waren und sind. {[}Baldwin, Julie, \& Pinfield, Stephen
(2018). The UK Scholarly Communication Licence: Attempting to Cut
through the Gordian Knot of the Complexities of Funder Mandates,
Publisher Embargoes and Researcher Caution in Achieving Open Access. In:
\emph{Publications}, 6 (2018) 3.
\url{https://doi.org/10.3390/publications6030031}{]} (mv)

Jill Emery hat die Zweitveröffentlichungsquoten von fünf ausgewählten
Zeitschriften aus dem LIS-Bereich näher untersucht -- und kommt dabei zu
der ernüchternden, wenn auch (leider) nicht ganz überraschenden
Erkenntnis: Während Bibliotheken Open Access als strategisches
Handlungsfeld besetzt haben, scheint das Selbstverständnis der
publizierenden Bibliothekar*in\-nen noch hinterher zu hinken. Im
Durchschnitt waren weniger als ein Viertel der im Zeitraum 2011--2016
publizierten Artikel auch als Open-Access-Version über ein Repositorium
verfügbar. Der Diskussionsteil des Artikels lässt aufhorchen: Emery
vermutet, unter anderem das sogenannte Imposter-Syndrom (auch als
Hochstapler-Syndrom bekannt) sei Grund dafür. Was auch immer der Grund
sein mag -- sie schließt ihren Beitrag mit der Einschätzung
\enquote{Quite simply, we can and should do better than a 22\% green
deposit rate.} {[}Emery, Jill (2018). How green is our valley?:
five-year study of selected LIS journals from Taylor \& Francis for
green deposit of articles. In: \emph{Insights}, 31 (2018).
\url{https://doi.org/10.1629/uksg.406}{]} (mv)

Mittels Interviews und Beobachtungen versuchten drei Forschende (nicht
aus der Bibliothekswissenschaft, sondern der Kulturwissenschaft) der
University of Melbourne zu eruieren, ob und wie Bibliotheken in
Australien dazu beitragen, gesellschaftliche Veränderungen, die von
digitaler Entwicklungen vorangetrieben werden, zu gestalten. Ihre
Darstellung ist grundsätzlich positiv. Wenn die Bibliothek gross genug
und gut ausgestattet sei, würde sie diese Veränderungen gestalten.
Kritisch sei, wenn die Bibliothek zu klein sei, insbesondere die
Indigenous Knowledge Centres. Der Text vermittelt aber auch den
Eindruck, dass sehr auf die Darstellung der Bibliothekarinnen und
Bibliothekare selber vertraut wird und dass die theoretischen
Begründungen, die in der bibliothekarischen Literatur zu finden sind
(insbesondere Habermas oder \enquote{Dritter Ort}), als nicht
ausreichend für ein Verständnis der tatsächlichen Veränderungen
angesehen werden: \enquote{Rather, our fieldwork suggests that libraries
are transforming public culture by translating technological capacity
into embodied practices through reconfiguring physical and network
spaces, and customising programmes of information-intensive experiences
amenable to diverse uses.} (Wyatt, Mcquire \& Butt 2018:2948) Im Text
kommt auch wieder einmal \enquote{The Edge} vor, der Makerspace in der
State Library of Queensland, welcher wohl als erster dieser Art in der
bibliothekarischen Literatur beschrieben und bis 2013 von Mark Bilandzic
in seiner Promotion erforscht wurde -- falls jemand diese Geschichte
nachverfolgen will. {[}Wyatt, Danielle; Mcquire, Scott; Butt, Danny
(2018). Libraries as redistributive technology: From capacity to culture
in Queensland's public library network. \emph{new media \& society} 20
(2018) 8: 2934--2953. \url{https://doi.org/10.1177/1461444817738235}{]}
{[}Paywall{]} (ks)

Karen Sobel schlägt in einem kurzen Artikel vor, für die Frage, wieso
Nutzerinnen und Nutzer Wissen, das sie in Lehrveranstaltungen zu
Informationskompetenz lernen sollten, nur zum Teil wirklich integrieren
und nutzen, eine Methodik namens \enquote{actor-oriented transfer
perspective} (AOT) nutzen. Diese wurde im Feld der Mathematikdidaktik
entworfen; geht dortdie gleiche Frage, nur für andere Lerninhalte, an.
Grundidee ist, davon auszugehen, dass Lehrende und Lernende Unterricht
oder andere Lehrveranstaltungen aus verschiedenen Blickwinkeln
wahrnehmen. Es geht AOT darum zu verstehen, wie die Lernenden was
lernen, wie sie bestimmte Inhalte wichten und so weiter. So würde der
Transfer oder Nicht-Transfer von Wissen besser verständlich. Relevant an
diesem Vorschlag ist vielleicht gar nicht so sehr die eigentliche
Methodik selber, sondern der Hinweis darauf, dass diese Frage auch
angegangen werden kann, indem die beiden Blickwinkel einbezogen werden.
{[}Sobel, Karen (in print). The Actor-oriented Transfer Perspective in
Information Literacy Instruction. In: \emph{The Journal of Academic
Librarianship} (in print),
\url{https://doi.org/10.1016/j.acalib.2018.07.008}{]} {[}Paywall{]} (ks)

In seinem Editorial zum Themenschwerpunkt \enquote{New Library Science}
fragt Hans-Christoph Hobohm: Warum brauchen wir eine (neue)
Bibliothekswissenschaft? Ursächlich ist für ihn eine \enquote{extreme
Digitalisierung}, der das Fach aus einer reflektierten und verfestigten
(=\enquote{archimedischen}) Position entgegen treten muss. Drei Aspekte
sind für ihn dabei wichtig: erstens \enquote{der historische Blick},
zweitens \enquote{der Blick von außen} und drittens eine
\enquote{wissenschaftliche Neugründung der Fachdiszplinen}, vermutlich
unter den durch Digitalisierung und Digitalität veränderten
Voraussetzungen. {[}Hobohm, Hans-Christoph (2018). Warum brauchen wir
eine (neue) Bibliothekswissenschaft? In: Bibliothek - Forschung und
Praxis. Band 42 Heft 2. S. 333-337.
\url{https://doi.org/10.1515/bfp-2018-0046}{]} {[}Paywall{]} (bk)

\hypertarget{forschungsdaten}{%
\subsection{Forschungsdaten}\label{forschungsdaten}}

Was bedeutet es, jenseits hübscher Informations-Webseiten zum
Forschungsdatenmanagement, eigentlich wirklich für eine Bibliothek, wenn
sie sich auf dieses weite Feld begibt? {[}Töwe, Matthias (2017):
\enquote{Wie Forschungsdaten die Bibliothek verändern. Erfahrungen aus
der ETH-Bibliothek}. In: \emph{B.I.T.online} 20.5:361-370.
\url{http://www.b-i-t-online.de/heft/2017-05/fachbeitrag-toewe.pdf}{]}
Dieser Artikel gibt einen interessanten Einblick in verschiedene Aspekte
wie den beteiligten Parteien oder neue Mitarbeiter, die "ein
erfrischendes Unverständnis für die im direkten Vergleich immer noch
eher starren Strukturen und Hierarchien einer Bibliothek" mitbringen.
Auch wenn man über die verwendeten Metaphern geteilter Meinung sein kann
(vgl. \url{https://twitter.com/bkaden/status/991747346600988672}). :)
(vv)

Der Frage, "warum veröffentlichen Wissenschaftler ihre Forschungsdaten
{[}nicht{]}", sind in letzter Zeit einige Veröffentlichungen
nachgegangen (zum Beispiel Johnson/Steeves 2018, Kaden 2018, Linek u.a.
2017). Über einen Blogpost von Katerina Bohle Carbonell habe ich eine
schöne "sticky metaphor" gefunden, die meines Erachtens dazu gut passt.
Forschungsdatenmanagement ist nämlich manchmal vergleichbar mit Wissens-
und Informationsmanagement: "It is the type of work you need, but nobody
wants to see: Underwear work". Der Ausdruck stammt von Carole Ann Goble,
deren Vortrag man online nach-sehen kann.

Bohle Carbonell, Katerina (2018): \enquote{The underwear of data
science}. In: Enigmas, Networks, and People at Work. 28.1.2018.
\url{https://katerinabc.com/make-open-science-successful/}.

Goble, Carole (2018): \enquote{Building the FAIR Research Commons: A
Data Driven Society of Scientists}. Vortrag auf dem Symposium
\enquote{The Future of a Data-Driven Society} an der Universität
Maastricht, 25.1.2018. Aufzeichnung:
\url{https://www.maastrichtuniversity.nl/events/review-symposium-future-data-driven-society}
(der Vortrag beginnt bei Minute 9:24), Folien:
\url{https://www.slideshare.net/carolegoble/building-the-fair-research-commons-a-data-driven-society-of-scientists}.

Johnson, Kelly; Steeves, Vicky (2018): \enquote{Research Data Management
Among Life Sciences Faculty: Implications for Library Service} (preprint
paper). In: \emph{LIS Scholarship Archive} (2018).
\url{https://doi.org/10.17605/OSF.IO/Q36UV}.

Kaden, Ben (2018): \enquote{Warum Forschungsdaten nicht publiziert
werden}. In: \emph{LIBREAS. Library Ideas} 33 (2018).
\url{http://libreas.eu/ausgabe33/kaden-daten/}.

Linek, Stephanie B.; Fecher, Benedikt; Friesike, Sascha; Hebing, Marcel
(2017): \enquote{Data sharing as social dilemma: Influence of the
researcher's personality}. In: \emph{PLoS ONE} 12 (8) 2017: e0183216.
\url{https://doi.org/10.1371/journal.pone.0183216}. (vv)

\hypertarget{monographien}{%
\section{3. Monographien}\label{monographien}}

Der Abschlussbericht des Projektes OAPEN-CH -- Auswirkungen von Open
Access auf wissenschaftliche Monographien in der Schweiz, durchgeführt
vom Schweizerischen Nationalfonds, zeigt auf der einen Seite die zu
erwartenden Ergebnisse eines solchen Projektes: Wenn
Forschungsfördereinrichtungen die Produktion von Monographien in OA
fördern -- so nachhaltig, dass Verlage ihre Arbeitsabläufe auf dieses
Modell umstellen --, dann fördern diese Monographien die Sichtbarkeit
dieser Publikationen leicht; wenn sie zusätzlich gedruckt und verkauft
werden, hat OA keinen negativen Einfluss. Ist das gegeben, akzeptieren
es die meisten Verlage und die meisten Forschenden. Andererseits ist der
Bericht auch ein Beispiel dafür, wie das Denken und die Terminologie der
BWL für Entscheidungen über OA genutzt wird (und nicht zum Beispiel der
Wissenschaftssoziologie, die wohl mehr dazu zu sagen hätte, wie und
warum publiziert wird). Und auch dafür, wie die Rolle der Bibliotheken
bei solchen Entscheidungen gesehen wird: Sie werden in der Studie fast
nie erwähnt, auch nicht befragt, sondern nur in einer
Informationsveranstaltung \enquote{informiert}, am Ende werden ihnen als
einem der \enquote{wichtigsten Abnehmer von wissenschaftlichen
Verlagspublikationen} Aufgaben zugeschrieben. Mehr nicht.
{[}Schweizerischer Nationalfonds zur Förderung der wissenschaftlichen
Forschung (2018): OAPEN-CH -- Auswirkungen von Open Access auf
wissenschaftliche Monographien in der Schweiz. Ein Projekt des
Schweizerischen Nationalfondes (SNF),
\url{http://www.snf.ch/SiteCollectionDocuments/OAPEN-CH_schlussbericht_de.pdf}{]}
(ks)

Der Konkursbuch Verlag, Tübingen, feiert in diesem Jahr sein
vierzigjähriges Bestehen. Dass kleine Verlage, trotz immer prekärer
finanzieller Lage, solche Alter erreichen, ist ein Grund zum Feiern.
Unter anderem tat der Verlag dies in der 55. Ausgabe des
\enquote{Konkursbuches} unter dem sympathischen Motto \enquote{Bücher}.
Allerdings ist das Motto ein nur sehr lockerer Aufhänger für die
Beiträge, welche vor allem im Verlagsumfeld eingeworben wurden. Viele
gratulieren zum Jubiläum, einige berichten von der Lesesozialisation der
AutorInnen, andere liefern kurze Geschichten oder Gedichte über Bücher.
Interessant ist vor allem die Geschichte der SoVA (Sozialistische
Verlagsauslieferung) von Helmut Richter und die Firmengeschichte des
ehemaligen Druckers des Konkursbuch Verlages, Robert Kump. Grundsätzlich
spannt der Band mit seinen kurzen Texten ein weites Feld zum gesamten
Buchgeschäft, über AutorInnen und LektorInnen zu wie gesagt,
Druckbetrieben und Verlagsauslieferungen, zu BuchhändlerInnen und
reisenden VerlagsvertreterInnen. Das ist charmant. Aber gleichzeitig,
und das mag mit der Einwerbepraxis für die Texte und dem Jubiläum zu tun
haben, hat man immer wieder den Eindruck, einer Generation von
Buchbegeisterten zuzuhören, die langsam das Rentenalter erreichen. Es
sind eher Geschichten von gestern, durchmischt mit Annahmen über das
Leseverhalten junger Leute, die sich mit der Empirie nicht deckt.
Abgrenzungen zum E-Book (oder entschuldigende positive Bezugnahmen), zu
Amazon und Buchladenketten finden sich immer wieder. Bibliotheken kommen
explizit nur in einem Text (zum Bild von Bibliotheken seit der Antike)
vor, BibliothekarInnen schreiben gar nicht (kennt der Verlag keine?),
dafür finden sich über das Buch verstreut immer wieder Erwähnungen von
Bibliotheken, allerdings auch aus einem etwas irritierenden Abstand. Man
liest eher von Vorstellungen einer älteren Generation, als das man
wirklich etwas über Bibliotheken erfährt. {[}Gehrke, Claudia ; Rogge,
Florian (Hrsg.): \emph{Bücher} (Konkursbuch, 55). Tübingen: Konkursbuch
Verlag, 2018{]} (ks)

\enquote{Human Operators: A Critical Oral History on Technology in
Libraries and Archives} besteht aus Interviews mit Bibliothekarinnen und
Bibliothekaren, die in ihrem Berufsalltag mit Technologie zu tun haben,
grösstenteils aus den USA. Die Herausgeberin Melissa Morrone führte die
Interviews und setzte Teile der Interviews so, dass die Aussagen zu
ähnlichen Themen zusammenstehen. Insoweit erreicht es etwas, was die
letzte Ausgabe der LIBREAS. Library Ideas erreichen wollte: Einblicke in
den Alltag in Bibliotheken geben. Inhaltlich werden praktisch alle
Themen -- von der konkreten Infrastruktur über den Umgang von
Nutzerinnen und Nutzern mit Technologie bis hin zu den
Arbeitsbedingungen und -kulturen in Bibliotheken -- angesprochen. Bei
rund 360 Seiten Text ist dafür auch genügend Platz. Wichtig ist
vielleicht die Erkenntnis, dass die Erwartungen, die an die technischen
Fähigkeiten normaler Nutzerinnen und Nutzer gestellt werden, oft weit
übertrieben sind und viel eher Grundlagenarbeit (Hilfe bei der Nutzung
von Computern et ceterea) geleistet werden muss. Während die Methode
eines Interviewbandes zu begrüssen ist (da, wie in der letzten LIBREAS.
Library Ideas postuliert, der Alltag in Bibliotheken viel zu wenig in
der bibliothekarischen Literatur und Forschung vorkommt), ist es
schwierig, die im Untertitel versprochen kritische Perspektive zu sehen.
Sicherlich äussern sich die Befragten zu bestimmten Themen kritisch,
aber nicht nur. Eine gemeinsame kritische Perspektive oder ein
spezifisch kritisches Vorgehen lässt sich nicht erkennen. {[}Melissa
Morrone (Hrsg.) \emph{Human Operators: A Critical Oral History on
Technology in Libraries and Archives}. Sacramento: Library Juice Press,
2018{]} (ks)

Die Universitätsbibliothek der Cornell University hat ein Handbuch
veröffentlicht, das Grundprinzipien des Repositorienbetriebs vorstellt:
Es werden Anregungen gegeben zu übergeordneten (Definition des Services,
Personalplanung, Policy, Dokumentation von technischen und
organisatorischen Aspekten, Infrastruktur und Interoperabilität) und
praktischen Fragestellungen des Repositoryalltags (Autor*innenbetreuung,
Metadatenpflege, Rechtemanagement inkl. Umgang mit Anfragen auf Sperrung
von Dokumenten und vieles mehr). Das Handbuch kann als PDF
heruntergeladen, als lebendes Dokument im öffentlichen Wiki eingesehen
oder auf Basis einer CC BY-NC-Lizenz nachgenutzt werden. {[}Faulder,
Erin, Jim DelRosso, Jenn Colt, Dianne Dietrich, Amy Dygert, Sarah
Kennedy, Jason Kovari, Wendy Kozlowski, Chris Manly, und Michelle
Paolillo. „\emph{Cornell University Library Repository Principles and
Strategies Handbook}". Report. Cornell University Library, März 2018.
\url{http://hdl.handle.net/1813/57034}. Zugang zum Wiki:
\url{https://confluence.cornell.edu/display/culpublic/Cornell+University+Library+Repository+Principles+and+Strategies+Handbook}{]}
(mv)

\hypertarget{social-media}{%
\section{4. Social Media}\label{social-media}}

Die Schreibwerkstatt der Pädagogischen Hochschule der Fachhochschule
Nordwestschweiz (\href{https://twitter.com/SchreibenPH}{@SchreibenPH})
weist auf das neue Tool ZoteroBib (\url{https://zbib.org/}) hin
(\url{https://twitter.com/SchreibenPH/status/1009361274852069381}): Es
ist keine Konkurrenz zum regulären Zotero, sondern als Ergänzung gedacht
-- mit ZoteroBib können ohne Programminstallation und ohne Registrierung
aus dem Stand Bibliografien erstellt werden. Automatisierte
Datenübernahme und Möglichkeiten zur Arbeit im Team gelten als
selbstverständlich. Das FAQ unter \url{https://zbib.org/faq} erläutert
die Unterschiede zwischen Zotero und ZoteroBib. (mv)

\href{https://twitter.com/EllenEuler}{@EllenEuler} hat zusammen mit
Studierenden der FH Potsdam das Brettspiel \enquote{The Publishing Trap}
ins Deutsche übersetzt (deutscher Titel: \enquote{Die
Publikationsfalle}), welches darauf abzielt, Kenntnisse zum Urheberrecht
und Open Access wortwörtlich spielerisch zu vermitteln
(\url{https://twitter.com/EllenEuler/status/1010899917555019776}). Das
Spiel als solches steht unter der Lizenz CC BY. (mv)

Als eine Art Spin-Off des sehr unterhaltsamen Hashtags
\#KunstGeschichteAlsBrotbelag kommt \#BuchcoverAlsBrotbelag daher. Den
Auftakt machte der Tweet von
{[}\href{https://twitter.com/tolino_media}{@tolino\_media} für den Roman
\enquote{Was ich euch nicht erzählte} von Celeste Ng:
\url{https://twitter.com/tolino_media/status/1020658349258543104}. (mv)

Zumindest Artikel, die zurückgezogen (\enquote{retracted}) wurden, sind
bei Elsevier garantiert frei zugänglich -- darauf weisen
@AuthorCarpentry hin
(\url{https://twitter.com/AuthorCarpentry/status/1011976941262364672})
und verlinken den Beitrag bei Retraction Watch, welcher unter anderem
diskutiert, warum der Verlag den Zugriff auf zurückgezogene Beiträge
eher beschränken sollte: \enquote{Retracted papers keep being cited as
if they weren't retracted. Two researchers suggest how Elsevier could
help fix that}
(\url{https://retractionwatch.com/2018/06/27/retracted-papers-keep-being-cited-as-if-they-werent-retracted-two-researchers-suggest-how-elsevier-could-help-fix-that/}).
(mv)

Der Hashtag \#librarylife ist eine Art digitaler Kettenbrief: In
Bibliotheken tätige Menschen twittern eine Woche lang Schwarz-Weiß-Fotos
aus dem Bibliotheksalltag, ohne jeglichen Kommentar. Die Mehrheit
fordert zudem jeden Tag eine weitere Person auf, ebenfalls Bilder zu
\#librarylife beizusteuern
(\url{https://twitter.com/search?q=\%23librarylife}). So kommt einiges
an Bildmaterial zustande -- Naheliegendes (etwa Buchcover oder
Buchrücken, Bücherregale, Katzen, Bibliotheksinnenarchitektur, ...) und
weniger Naheliegendes
(\url{https://twitter.com/h_m17/status/1022157061302435840}). Schade,
dass offenbar nur wenige daran denken, Informationen zu freien Lizenzen
mit ihren Bildern zu verknüpfen... (mv)

\hypertarget{konferenzen-konferenzberichte}{%
\section{5. Konferenzen,
Konferenzberichte}\label{konferenzen-konferenzberichte}}

Unter einem provokanten Titel hat Torsten Reimer bei der Tagung Open
Repositories im Juni 2018 grundlegende Fragen zu
Repositorieninfrastrukur diskutiert: Er mahnt dazu, den Fokus auf
Services zu legen, statt auf infrastrukturelle Fragen und Systeme lokal
nur dann selbst zu betreiben und entwickeln, wenn man es selbst besser
könne als andere. Er stellte die These auf, dass sich Nutzer*innen nicht
für die zugrundeliegenden Strukturen interessieren -- für sie gehe es
primär um Zugang zu Inhalten und Funktionalität. Für das Vereinigte
Königreich schlägt die British Library daher vor, Daten in verteilten
Strukturen zu halten, aber eine gemeinsame Plattform für den Zugang zu
und die Archivierung von Inhalten zu erarbeiten. {[}Reimer, Torsten
(2018). \enquote{For repositories to succeed they have to end.
Reflections on (not just) the UK repository scene}. Open Repositories
2018.
\url{https://www.slideshare.net/TorstenReimer/for-repositories-to-succeed-they-have-to-end-reflections-on-not-just-the-uk-repository-scene/TorstenReimer/for-repositories-to-succeed-they-have-to-end-reflections-on-not-just-the-uk-repository-scene}
(mv)

\hypertarget{populuxe4re-medien-zeitungen-radio-tv-etc.}{%
\section{6. Populäre Medien (Zeitungen, Radio, TV
etc.)}\label{populuxe4re-medien-zeitungen-radio-tv-etc.}}

In der Ausgabe vom 07.07.1978 berichtet das Neue Deutschland über die
Bibliotheksnutzung in Berlin (Ost) und kommende Ferienprogramme der
öffentlichen Bibliotheken. Statistisch, so die Aussage, nutzen 20 \% der
Berliner \enquote{die Ausleihmöglichkeit einer staatlichen
Allgemeinbibliothek}, bei der Altersgruppe 14--17 sind es 70 \%, was
über besondere Angebote zum Beispiel Bibliotheksführungen für 8. Klassen
und regelmäßige Lesungen für 9. und 10. Klassen sowie konkret
\enquote{Literaturjugendclubs} (Pablo-Neruda-Bibliothek in
Berlin-Friedrichshain, Caragiale-Bibliothek in Berlin-Pankow) begründet
wird. Diese bleiben nicht auf Bibliotheken beschränkt, sondern
organisieren auch Exkursionen in Museen und
\enquote{Tanzveranstaltungen}. Angekündigt werden zudem mehrere Um- und
Neubauten, zum Beispiel die Wohngebietsbibliothek an der Greifswalder
Straße. Illustriert wird der Beitrag mit einem Foto aus der
Urlauberbibliothek am Zeltplatz Zeuthener See II. {[}Funke, Gisela: Das
Lesen gehört zu unserem Alltag. In: Neues Deutschland, 07.07.1978,
S.8{]} (bk)

Seit zwanzig Jahren schon gibt es in einer ländlichen Gegend Kolumbiens
statt Bücherbus zwei Bücheresel, darüber berichtete bereits im April
2018 die BBC. Mit den \enquote{Biblioburros} sorgt ein Lehrer dafür,
dass auch Kinder in abgelegenen Orten Zugang zu Büchern haben. Wer der
Geschichte hinterher recherchiert, findet in der New York Times von 2008
bereits einen Beitrag dazu. {[}Romero, Simon: Acclaimed Colombian
Institution Has 4,800 Books and 10 Legs. In: New York Times, 19.10.2008,
\url{https://www.nytimes.com/2008/10/20/world/americas/20burro.html};
Peñarredonda, José Luis (Text), Fabara, Sergio (Film): Biblioburro: The
amazing donkey libraries of Colombia. In: BBC 11.04.2018, Kurztext und
vierminütiges Video,
\url{http://www.bbc.com/culture/story/20180410-biblioburro-the-amazing-donkey-libraries-of-colombia}{]}
(mv)

\hypertarget{weitere-medien}{%
\section{7. Weitere Medien}\label{weitere-medien}}

Danny Kingsley (\href{https://twitter.com/dannykay68}{@dannykay68} auf
Twitter) thematisiert in ihrem Vortrag die zunehmende Kommerzialisierung
des wissenschaftlichen Publikationswesens ebenso wie die Ausweitung des
Portfolios von Verlagen auf andere Geschäftszweige -- etwa Evaluation
von Forschungsleistungen, Literaturverwaltung und Annotationswerkzeuge,
Repositoryservices, Forschungsdatenmanagement und vieles mehr. Sie
hinterfragt die Rolle von Bibliotheken und der Verwaltung: Sie würden
dem Wandel der Aufgabenfelder nicht schnell genug folgen. Doch mit
dieser Feststellung, die heutzutage wie ein Gemeinplatz wirkt, endet der
Vortrag zum Glück nicht. Sie ruft zum Handeln auf und benennt mehr oder
weniger Konkretes unter der Überschrift \enquote{What can YOU do?}. Den
im Bereich Open Science tätigen Bibliothekar*in\-nen werden die Punkte
geläufig sein, doch ist der Vortrag vor der Cambridge Libraries Group
als solcher wohl ein Indiz dafür, dass die Botschaft wohl noch immer
nicht im Kern angekommen ist: Bibliotheken müssen ihrer Meinung nach
noch stärker im Bereich Forschungsunterstützung aktiv werden.
{[}Kingsley, Danny (2018). \emph{Librarians -- we need to talk}. Vortrag
für die Cambridge Libraries Group.
\url{https://doi.org/10.17863/cam.24096}{]} (mv)

Am 5. September 2018 erscheint \enquote{Paywall -- The Business of
Scholarship} (häufig auch kurz mit \enquote{Paywall -- The Movie}
betitelt). Der Film thematisiert unter anderem das Geschäftsfeld der
Wissenschaftspublikation und die Profitmargen großer
Wissenschaftsverlage, die Bedeutung des Zugangs zu wissenschaftlichen
Publikationen für einzelne Wissenschaftler*innen und die Rolle von
Schattenbibliotheken. Die unter
\url{https://paywallthemovie.com/trailers} verfügbaren Trailer lassen
vermuten: Die Transformation zu Open Access ist unausweichlich, um das
Funktionieren des Gesamtsystems in der Zukunft sicherstellen zu können.
Da der Film unter der Lizenz CC BY erscheint, ist eine Nachnutzung
möglich: Auf der Seite \url{https://paywallthemovie.com/screenings}
werden geplante, dezentrale Aufführungstermine gelistet -- darunter auch
einige in Deutschland. {[}Schmitt, Jason et al: \emph{Paywall the
movie}, erscheint am 5.9.2018,
\url{https://paywallthemovie.com/our-team}{]} (mv)

Eine bemerkenswerte Formulierung, die weniger mit der
Bibliothekswissenschaft an sich, wohl aber viel mit der Umbruchsituation
am Institut an der Humboldt-Universität zu tun hat, fand Erwin Marks in
seiner durchaus persönlich gehaltenen Reflexion über ehemalige Dozenten
und Professoren sowie immerhin einer Professorin (Ruth Unger) aus dem
Jahr 1990. Deutlich wird, wie er die Wertschätzung für den ehemaligen
Hochschullehrer Georg Schmoll auch in der Wendezeit aufrecht erhalten
wollte, zugleich aber dessen Engagement in einer offenbar schockierend
plötzlich problematisch gewordenen Partei erwähnen musste. Diese
Herausforderung löste er so: \enquote{Selbst ein Opfer des Faschismus,
vertrat er bis zu seiner Emeritierung 1989 weltanschauliche Positionen,
die ihm die aktive Mitwirkung in der nach ihrem Programm
antifaschistische Einheitspartei geraten erschienen ließen {[}sic!{]}.}
(Marks, Erwin: Nachdenken über ehemalige Professoren und Dozenten der
Berliner Instituts für Bibliothekswissenschaft. In: Zentralblatt für
Bibliothekswesen. 1990, S.399-406) (bk)

%autor

\end{document}
