\documentclass[a4paper,
fontsize=11pt,
%headings=small,
oneside,
numbers=noperiodatend,
parskip=half-,
bibliography=totoc,
final
]{scrartcl}

\usepackage{synttree}
\usepackage{graphicx}
\setkeys{Gin}{width=.4\textwidth} %default pics size

\graphicspath{{./plots/}}
\usepackage[ngerman]{babel}
\usepackage[T1]{fontenc}
%\usepackage{amsmath}
\usepackage[utf8x]{inputenc}
\usepackage [hyphens]{url}
\usepackage{booktabs} 
\usepackage[left=2.4cm,right=2.4cm,top=2.3cm,bottom=2cm,includeheadfoot]{geometry}
\usepackage{eurosym}
\usepackage{multirow}
\usepackage[ngerman]{varioref}
\setcapindent{1em}
\renewcommand{\labelitemi}{--}
\usepackage{paralist}
\usepackage{pdfpages}
\usepackage{lscape}
\usepackage{float}
\usepackage{acronym}
\usepackage{eurosym}
\usepackage[babel]{csquotes}
\usepackage{longtable,lscape}
\usepackage{mathpazo}
\usepackage[normalem]{ulem} %emphasize weiterhin kursiv
\usepackage[flushmargin,ragged]{footmisc} % left align footnote
\usepackage{ccicons} 

%%%% fancy LIBREAS URL color 
\usepackage{xcolor}
\definecolor{libreas}{RGB}{112,0,0}

\usepackage{listings}

\urlstyle{same}  % don't use monospace font for urls

\usepackage[fleqn]{amsmath}

%adjust fontsize for part

\usepackage{sectsty}
\partfont{\large}

%Das BibTeX-Zeichen mit \BibTeX setzen:
\def\symbol#1{\char #1\relax}
\def\bsl{{\tt\symbol{'134}}}
\def\BibTeX{{\rm B\kern-.05em{\sc i\kern-.025em b}\kern-.08em
    T\kern-.1667em\lower.7ex\hbox{E}\kern-.125emX}}

\usepackage{fancyhdr}
\fancyhf{}
\pagestyle{fancyplain}
\fancyhead[R]{\thepage}

% make sure bookmarks are created eventough sections are not numbered!
% uncommend if sections are numbered (bookmarks created by default)
\makeatletter
\renewcommand\@seccntformat[1]{}
\makeatother


\usepackage{hyperxmp}
\usepackage[colorlinks, linkcolor=black,citecolor=black, urlcolor=libreas,
breaklinks= true,bookmarks=true,bookmarksopen=true]{hyperref}
%URLs hart brechen
\makeatletter 
\g@addto@macro\UrlBreaks{ 
  \do\a\do\b\do\c\do\d\do\e\do\f\do\g\do\h\do\i\do\j 
  \do\k\do\l\do\m\do\n\do\o\do\p\do\q\do\r\do\s\do\t 
  \do\u\do\v\do\w\do\x\do\y\do\z\do\&\do\1\do\2\do\3 
  \do\4\do\5\do\6\do\7\do\8\do\9\do\0} 
% \def\do@url@hyp{\do\-} 
\makeatother 

%meta
%meta

\fancyhead[L]{Th. Roesnick, M. Brauer, A. Erbe\\ %author
LIBREAS. Library Ideas, 34 (2018). % journal, issue, volume.
\href{http://nbn-resolving.de/}
{}} % urn 
% recommended use
%\href{http://nbn-resolving.de/}{\color{black}{urn:nbn:de...}}
\fancyhead[R]{\thepage} %page number
\fancyfoot[L] {\ccLogo \ccAttribution\ \href{https://creativecommons.org/licenses/by/3.0/}{\color{black}Creative Commons BY 3.0}}  %licence
\fancyfoot[R] {ISSN: 1860-7950}

\title{\LARGE{Interview mit \\ Gertrude Pannier}}% title
\author{Thomas Roesnick (Interviewer) \\ Miriam Brauer \& Andreas Erbe (Transkription und Korrektur)} % author

\setcounter{page}{1}

\hypersetup{%
      pdftitle={Interview mit Gertrude Pannier},
      pdfauthor={Thomas Roesnick (Interviewer) \\ Miriam Brauer \& Andreas Erbe (Transkription und Korrektur)},
      pdfcopyright={CC BY 3.0 Unported},
      pdfsubject={LIBREAS. Library Ideas, 34 (2018).},
      pdfkeywords={Bibliothekswissenschaft, Institut für Bibliotheks- und Informationswissenschaft, Humboldt-Universität zu Berlin, Projektseminar, Buchprojekt, Studium},
      pdflicenseurl={https://creativecommons.org/licenses/by/3.0/},
      pdfcontacturl={http://libreas.eu},
      baseurl={http://libreas.eu},
      pdflang={de},
      pdfmetalang={de}
     }



\date{}
\begin{document}

\maketitle
\thispagestyle{fancyplain} 

%abstracts

%body
\textbf{{Interview mit Gertrud Pannier, geführt von Thomas Roesnick
(09.07.2018)}}

\textbf{TR: Vielen Dank, Frau Pannier, dass Sie sich die Zeit für dieses
Interview nehmen. Wir feiern 90 Jahre IBI, das ist ein langer Zeitraum
und nicht ganz so lange, aber beinahe die Hälfte dieser Zeit, haben Sie
am Institut verbracht. Wie war denn Ihre erste Begegnung mit dem IBI?
Wie sind Sie überhaupt auf das Institut aufmerksam geworden?}

GP: Ich habe mich hier zum Studium beworben. Von 1969 bis 1973 habe ich
hier studiert und das Studium dann mit dem Diplom abgeschlossen. Meine
Mutter war Bibliothekarin, insofern war mir die Nähe zu diesem Fach
vertraut und angenehm. Ich hatte mich erkundigt und es gab dieses
universitäre Studium in der DDR nur in Berlin an der
Humboldt-Universität. Ansonsten gab es noch die Fachschule in Leipzig,
aber das war eben die Fachschulebene. Da wir als Abiturienten alle
angehalten wurden, zu studieren, habe ich mich hier beworben und bin
dann auch angenommen worden. Damals waren wir 30 Studierende am IBI. Es
war ein interessantes Studium, obwohl kurz zuvor die Hochschulreform von
1968 dazu geführt hatte, dass der Studiengang nicht mehr zwei Fächer,
sondern nur noch ein Fach umfasste. Ein Studium, das früher nur die
Hälfte eines Vollzeitstudiums gebildet hatte, war plötzlich ein
Vollstudium und wir hatten daher sehr verschiedene interessante Fächer,
aber zu Beginn nur relativ wenig Stunden. Das wurde dann erst später
erweitert. So kam ich also ans IBI.

\textbf{TR: Welche inhaltliche Ausrichtung hatte das Institut denn zu
Ihrer Anfangszeit? Module gab es ja noch nicht, aber erinnern Sie sich
noch an die Lehrinhalte?}

GP: Wir hatten damals Bestandserschließung, Bestandsaufbau, wie auch
jetzt die Formal- und Inhaltserschließung. Wir hatten
Bibliotheksgeschichte und Buchkunde, Schriftgeschichte, Bibliographie,
Bibliotheksverwaltung und Planung von Bibliotheksprozessen. Marxismus
und Leninismus, das war ein Pflichtfach für jeden Studierenden, und vier
Sprachen hatten wir auch: Englisch, Französisch, Russisch und Latein. Es
gab damals noch die Auffassung, dass man sagte: „Bibliothekare müssen
viele Sprachen können, um die Literatur erschließen zu können". Auch
Sport hatten wir noch als Pflichtfach. Das war sozusagen der Druck, dass
man sich sportlich betätigte als junger Mensch und ein Angebot der
Sportfakultät der Humboldt-Universität.

\textbf{TR: Sie mussten also 4 Sprachen lernen?\\
}

GP: Wenn man vier Sprachen gelernt hat, kann man damit auch andere
Sprachen soweit übersetzen, dass man eine Titelaufnahme machen kann. Die
Ausrichtung war eigentlich gar nicht so schlecht. Es war von Anfang an
so, dass nicht zwischen wissenschaftlichen und öffentlichen Bibliothek
getrennt wurde und auch die Informationsdokumentation wurde immer mit
bedacht. Das ist etwas, das in den Fachschulen anders war und auch in
der alten Bundesrepublik ganz anders gelaufen ist. Es war eine
Universität, an der studiert wurde, und damit gab es auch immer eine
deutliche wissenschaftliche Ausrichtung. Man kann natürlich immer
streiten und den Streit gibt es ja bis heute: Ist
Bibliothekswissenschaft überhaupt eine Wissenschaft? Diese Dinge, die
jetzt diskutiert werden, waren durchaus auch damals Gegenstand der
Diskussion. Wir hatten zum Beispiel auch Bibliothekswesen anderer
Länder, eine Art Landeskunde. Da wurden dann auch die sozialen
Verhältnisse verglichen, natürlich auch mit einer sozialistischen
Ausrichtung. Man sagte, dass Bibliotheken auch progressive Gedanken und
Literatur propagieren sollen. Sie sollten aktiv sein und Leser zur
Lektüre anregen. Sowas gab es natürlich, aber es war auch die erklärte
Ausrichtung, dass man erziehen will.

\textbf{TR: Und dann haben Sie nach dem Studium direkt eine Stelle am
Institut angenommen?\\
}~\\
GP: Genau, wissenschaftliche Assistentin hieß das damals. Es gab relativ
wenig Stellen und wenn es Stellen gab, dann wurden Studierende
angesprochen, die man für geeignet hielt. Ich gehörte durchaus zu den
Studierenden, die ganz gut waren, und es gab eine freie
Assistentenstelle für das Fach Bibliographie. Da war ich gut drin. Ich
war zeitweise auch Hilfsassistentin, also studentische Hilfskraft, was
damals nicht bezahlt wurde und habe die bibliographischen Übungen
begleitet. Es war also schon ein Interesse von meiner Seite da. Die
Lehrstuhlausrichtung war damals nicht so stark, jedenfalls nicht bei uns
am Institut. Vor dem Ende des dritten Studienjahres, also noch vor dem
Ende des Studiums, gab es die sogenannte Absolventenlenkung, bei der den
Studierenden Arbeitsplätze angeboten wurden. Es gab auf der einen Seite
die gesetzliche Verpflichtung zwei oder drei Jahre dort zu arbeiten,
wohin man nach dem Studium vermittelt wurde, und es gab andererseits
aber auch für jeden ein Arbeitsplatzangebot. Dafür gab es eine
Lenkungskommission, in einer solchen war ich später auch selbst
Mitglied, die dann schauten: Ist jemand verheiratet, wohnt der Mann in
Halle? Ist jemand sehr interessiert an Naturwissenschaften? Dann wurde
nach Stellenangeboten in einer entsprechenden Bibliothek oder
Universitätsbibliothek geschaut. Es wurde natürlich auch ein bisschen
geguckt, wer ist geeignet, ein Kollektiv zu leiten und eine andere
wichtige Aufgabe zu übernehmen. Das war natürlich eine paradiesische
Situation. Wir hatten alle am Ende des dritten Studienjahres einen
Arbeitsvertrag, obwohl noch keiner wusste, wie gut wir das Diplom machen
würden. Da habe ich dieses Angebot am Institut eben angenommen. Es war
üblich, dass solche Stellen auf 4 Jahre befristet waren und in dieser
Zeit sollte man auch promovieren. Das habe ich nicht geschafft, also
wurde ich verlängert. Ich habe dann nach dem Studium noch ein Jahr
Auslandsstudium in Leningrad in der Sowjetunion gemacht und dann habe
ich eben später promoviert, auch in der Fachrichtung Bibliographie.

\textbf{TR: Jetzt machen wir mal einen größeren Sprung in die Wendezeit,
also speziell zum Jahr 1989 mit dem Mauerfall und in die 1990er Jahre.
Wie wirkte sich denn die Wendezeit auf die Bibliotheks- und
Informationswissenschaft in Berlin aus?}

GP: Erst einmal gab es natürlich ein ziemliches Chaos und eine große
Unsicherheit, wie es weitergeht. Es gab schon ziemlich früh Stimmen, die
meinten, dass man dieses Institut nicht brauchen würde, weil es in
Westdeutschland ja auch keines in dieser Art gab. Es gab auch die
Diskussion innerhalb des Kollegiums zum politischen Engagement wegen
restriktiver Maßnahmen gegen Studierende aus politischen Gründen.
Einerseits war das schon sehr demokratisch und spannend, andererseits
war es zum Teil auch eine schmerzhafte Erfahrung, zu sehen, wie
„Wendehälse" sich outeten. Dass also Leute, die immer ihre politische
Überzeugung vor sich hergetragen hatten, dann plötzlich sagten, sie
wären eigentlich nie so gewesen. Dazu kam die persönliche Angst. Ich
hatte zum Beispiel eine Kündigung erhalten, denn es wurden ja nun
westdeutsche Maßstäbe angelegt. Das hieß zum Beispiel, dass der
Personalschlüssel ein ganz anderer war und es war ganz klar, dass darauf
hingearbeitet werden sollte, dass eine neue Soll-Struktur kommt und man
nicht bei der Ist-Struktur bleiben kann. Deshalb gab es zahlreiche
Kündigungen und vorzeitige Berentungen. Es war also manches friedlich
und manches sehr kämpferisch. Ich habe dann gegen diese Kündigung
geklagt und auch Recht bekommen. Es gab dann sofort viel Arbeit durch
die neuen Studienordnungen nach Magisterstrukturen, die aus der alten
Bundesrepublik kamen. Ich glaube 1991 wurde schon die erste neue
Studienordnung eingeführt, mit einer Fächerkombination und dem Wegfall
von Marxismus und Leninismus als Fach. Die klassischen Inhalte sind aber
geblieben und einiges Neues kam hinzu, Verlagswesen oder
Literaturseminare zum Beispiel.\\
Auch die Etats wurden hinterfragt, da wussten wir auch nicht, wie es
weitergeht und dann gab es natürlich die Auseinandersetzung mit der
Vergangenheit, das war vielleicht das Wichtigste.\\
Wie sich das auf die Wissenschaft am Institut auswirkte? Es war anfangs
eigentlich ganz konstruktiv, weil die Berufsvereinigungen gleich Kontakt
aufnahmen. Da gab es fachlichen Austausch, die Literatur hatte man ja
vorher schon zur Kenntnis genommen. Es gab auch Treffen mit anderen
Ausbildungseinrichtungen und das ist in meiner Erinnerung noch sehr
deutlich, weil dabei immer wieder die Föderalismusdiskussion aufkam. Ich
sagte damals: \enquote{Warum gibt es denn so viele verschiedene
Profilierungen oder auch Ansprüche an diese Ausbildung?}. Zu dem
Zeitpunkt gab es ja vor allem die Fachhochschulen. Es waren auch
Fachhochschulen mit Universitäten fusioniert, aber deshalb waren es
natürlich noch keine universitären Ausbildungsgänge. Mir wurde dann
gesagt: \enquote{Ist doch toll, wenn Föderalismus herrscht, dann kann
jedes Bundesland das Beste aus der Ausbildung machen und muss sich nicht
nach der Mehrheit oder nach einem Durchschnitt richten}, was auch
erstmal ganz gut klang. Wie sich dann aber herausstellte, wussten die
meisten nicht, wie woanders eigentlich ausgebildet wurde, sodass dieser
Faktor \enquote{Wir suchen uns den besten Weg} eigentlich gar nicht so
zum Tragen kam. Aber gut, das war eben so.

\textbf{TR: Was sind denn Ihre Gedanken zur Zusammenlegung der Institute
von Humboldt-Universität und Freier Universität Mitte der 1990er Jahre?}

GP: Also im Nachhinein ist es ja ganz gut gelaufen \emph{{[}lacht{]}}.
Aber das war schon dramatisch. Ich bin froh, dass die Zusammenlegung
unter dem Dach der HU erfolgte. Denn man merkte sehr schnell, dass die
FU mit diesem Institut nicht so ganz glücklich war. Da gab es natürlich
wie immer das Beharrungsvermögen der Westprofessoren, wie es bei uns
wahrscheinlich auch gewesen wäre, die wollten das nicht. Einige der
Alteingesessenen der FU wollten also weder von ihrem Hohenzollerndamm
weg, noch an die HU, was politische Gründe hatte. Die HU galt als eine
rote Uni. Aber hätte es keine Fusion gegeben, dann wäre unser Institut
bestimmt geschlossen worden. Die Machtverhältnisse waren ja ansonsten
so, dass man eher im Osten etwas schloss als im Westen, das war eben
auch einfacher. Insofern denke ich, dass das insgesamt ganz gut war. Und
es zeigte sich auch, also ich will jetzt nicht die wissenschaftliche
Qualität des alten IBI zu hoch loben, aber ich hatte das Gefühl, dass
das Institut der FU eher dem Fachhochschulniveau entsprach. Sie waren
sehr praxisorientiert, also nicht so sehr auf Selbständigkeit aus und
die Ansprüche an die Studierenden und auch die Ansprüche der Dozentinnen
und Dozenten an sich selbst waren niedriger. Wir hatten damals zum
Beispiel auch schon Fernstudiengänge am IBI, das wurde völlig abgelehnt:
\enquote{Wie kann man denn nur? Sowas lässt sich nicht organisieren}.
Und später ist das Fernstudium dann sogar zum Aushängeschild des
Instituts geworden. Es gab also durchaus einen Selbstfindungsprozess für
dieses neue, gemeinsame Institut.

\textbf{TR: Sechs Jahre später gab es dann den sogenannten
Bologna-Prozess. Wie wirkte sich dieser denn auf das Institut
beziehungsweise die Lehrsituation aus?}

GP: Erst einmal gab es den Druck von außen, also von der Universität,
die Studienordnungen entsprechend diesem Bologna-Prozess umzustellen.
Ich habe das eigentlich für hilfreich gehalten. Denn ich weiß nicht, wie
sehr diese Erneuerung aus dem Institut selbst hätte kommen können. Dabei
ging es ja nicht nur um das formale Aufsplitten in Module und ein
Punktesystem und anderes, sondern auch um das Verkürzen der
Studiengänge. Der Bachelorstudiengang umfasst eben nur drei Jahre und
wir hatten vorher viereinhalb Jahre, die allerdings auch kaum jemand
eingehalten hat. Man musste sich also genau überlegen, welche Inhalte in
diesen Studiengang übernommen werden und was man sich vielleicht für
einen späteren Master aufhebt. Zur Festlegung wurden dann Kommissionen
gebildet, die zu 50 \% aus Studierenden und zu 50 \% aus Lehrenden
bestanden und dort haben die Studierenden auch eine ganze Menge an
Forderungen eingebracht. Es ging so weit, dass manche Vorschläge
abgelehnt wurden und es dann zu Konflikten innerhalb der Kommission kam.

\textbf{TR: Ich höre da heraus, dass Sie die Verkürzung der Studienzeit
nicht für schlechter halten?\\
}~\\
GP: Ich bin keine die sagt: \enquote{Dieser ganze Bologna-Prozess ist
verkehrt}. Das Problem ist, und das zieht sich in manchen Studienfächern
wohl noch bis heute hin, dass man versucht hat, die gesamte Fülle der
Inhalte und Anforderungen in diese drei Jahre zu pressen und damit ist
es wirklich eine Überforderung geworden. Man kann eben nicht alles mit
aufnehmen. Bei uns am IBI hat es dazu geführt, dass viele von den
historischen Blickwinkeln verschwunden sind, dass es keine
Buchgeschichte, keine Schrift- und Illustrationsgeschichte und keine
Bibliotheksgeschichte mehr gibt. Das ist schon schade. Aber den Vorteil
sehe ich darin, dass das Studium nun strukturierter ist, als es der
Magister war. Es sind insgesamt vielleicht 5 \% der Magisterstudierenden
gewesen, die die Freiheiten in der Wahl der Inhalte wirklich genutzt
haben, um sich zu profilieren. Es hat aber viele auch hilflos gemacht.
Insofern denke ich, dass das neue System ein besseres System ist.

\textbf{TR: Es gab Anfang der 2000er erneut eine Diskussion um eine
mögliche Schließung des Instituts. In welche Richtung hat sich das
Institut dadurch verändert?\\
}~\\
GP: Die Aufgabe der Universität war es, zu sparen. Es gab daher großen
Druck von außen und dann haben alle geschaut: \enquote{Auf was könnten
wir denn verzichten?}. Und da ist bei so einem kleinen Institut mit bloß
drei Professoren, die man woanders unterbringen muss, natürlich weniger
Widerstand zu erwarten, als bei einem großen Institut. Eines ist auf
jeden Fall passiert: Es gab einen Politisierungsprozess. Die
Studierenden waren damals so interessiert und engagiert. Bis heute ist
es wichtig zu sagen: Ohne diese engagierten Studierenden wäre das
Institut damals geschlossen worden. Die haben wirklich viel organisiert,
waren im Fernsehen, in der Zeitung, zusammen mit anderen Protestierenden
natürlich. Es gab ja an der ganzen Universität eine große Unruhe. Eine
wichtige Frage war: \enquote{Warum brauchen wir die
Bibliothekswissenschaft? Was ist das und warum braucht die Universität,
die Bundesrepublik das?} Es gab bei den Studierenden natürlich auch
Unterschiede. Es gab welche, die sagten: \enquote{Wir müssen das hier
jetzt inhaltlich begründen} und andere haben gesagt: \enquote{Mensch,
hört doch mal auf. Ich will einfach nur meinen Abschluss hier haben}.
Diese Konflikte wurden unter den Studierenden ausgetragen, aber die
Fachschaft hat sich formiert. Ich war immer dafür, dass Studierende sich
engagieren sollen, dass sie auch ihre Gremienmöglichkeiten wahrnehmen
sollen. Insofern habe ich da auch gerne mitgemacht und habe eine
öffentliche Vorlesung am S-Bahnhof Friedrichstraße gehalten
\emph{{[}lacht{]}} oder mich in den Gremien eingebracht, in denen ich
war. Ich glaube, das war schon ein sehr wichtiger Prozess. Mit der
Berufung von Professor Seadle hieß es: \enquote{Wenn das Institut weiter
besteht, dann muss es sich reformieren und dann muss es auch neue
Personen geben} und das ist dann auch geschehen. Die Studierenden haben
da den größten Elan gezeigt.\\
~\\
\textbf{TR: Sie sagten, das Institut musste sich reformieren. Was folgte
war dann eine eher technische Ausrichtung des Instituts. Können Sie
darauf etwas näher eingehen?\\
}~\\
GP: Technisch ja, aber ich glaube auch sehr anwendungsbezogen. Das ist
aber vermutlich dem Hintergrund geschuldet, dass wenn man der
Gesellschaft sagen will \enquote{Ihr braucht uns}, dann will man auch
Praxisbereiche aufzeigen. Die Absolventen müssen mit den Kenntnissen,
die sie hier erworben haben, nicht in die Bibliothek, sie können auch in
viele andere Bereiche gehen. Es ist, glaube ich, eine praxisnahe
Ausrichtung dabei herausgekommen.\\
~\\
\textbf{TR: Der Studiengang wurde auch umbenannt in Bibliotheks- UND
Informationswissenschaft. Bei Informationswissenschaft denke ich zuerst
an das Internet. Können Sie den Zeitpunkt nennen, an dem sich das
Internet bei den Lehrinhalten am Institut etabliert hat?\\
}~\\
GP: Den genauen Zeitpunkt kann ich Ihnen nicht sagen. Wir hatten schon
Ende der 1980er Jahre in der DDR die ersten Computerkurse und PCs. Ich
denke, angekommen ist das Internet dann auf jeden Fall so Mitte/Ende der
1990er Jahre und zwar als Gegenstand. Da konnte sich mein alter Kollege
Herr Michael Heinz sehr viel einbringen und wir haben das dann
angewendet. Wir haben damals ein gemeinsames Fernstudium mit
Koblenz-Landau eingeführt. Die haben also Bibliothekswissenschaft dort
als Zweitfach im Magister angeboten und wir haben die Kurse angeboten.
Wir saßen hier vor Videokameras und es gab auch einen Prozess von
handgeschriebenen Notizen hin zu Word-Dokumenten, die dann auch in die
Onlinekurse eingeflossen sind.

\textbf{TR: Da sprechen Sie schon die Veränderung der Lehrmaterialien
an. Inwiefern hat sich denn das Lernen und Lehren am IBI zwischen Ihrer
eigenen Studienzeit im Vergleich zu heute verändert? }

GP: Zu meiner Zeit als Studentin gab es noch die klassische Vorlesung.
Da hat einer gelesen und wir haben mitgeschrieben und haben auch gelesen
oder auch nicht, haben exzerpiert, alles natürlich mit der Hand und
haben uns dann für die Prüfung vorbereitet. Heute packt man als Dozentin
oder Dozent das Material in einen Moodle-Kurs, das ist natürlich
komfortabel. Für die Dozentinnen und Dozenten macht es Arbeit, aber man
kann den Studierenden sagen: \enquote{Es ist alles da, ihr könnt alles
noch einmal angucken}. Also natürlich nicht alles, ich habe nicht das
volle Material reingestellt, aber doch genug. Wer nicht da war und auch
wer da war, konnte anhand der eingestellten Materialien durchaus
verfolgen, was Schwerpunkt war, was behandelt wurde und wo man sich
weiterklicken kann. Das halte ich für einen Vorteil, das muss ich schon
sagen. Dazu noch die Möglichkeit der Kommunikation über Email oder
andere Wege. Dass sowohl die Studierenden sich niedrigschwellig an die
Dozentinnen und Dozenten wenden können und dass diese unabhängig von
Sprechzeiten antworten können, das finde ich gut.

\textbf{TR: Eine letzte Frage noch zu den Veränderungen. Haben Sie eine
Veränderung bei den Studierenden gespürt? Man sagt ja, beim
Bachelorstudium hat man mehr Stress und viele Studierende müssen neben
dem Studium arbeiten gehen. War das früher auch schon so?\\
}

GP: Zu meiner Studienzeit hat in der DDR jeder Studierende ein
Stipendium bekommen. Das war niedrig, aber die Lebenshaltungskosten
waren ja auch niedrig. Also war der Stress nicht so groß. Ich habe
während des Studiums immer mal wieder gearbeitet, aber das war mehr, um
sich noch etwas kaufen zu können. Ich hatte eine kleine
Studentenwohnung, die Miete waren 23 Mark \emph{{[}lacht{]}} und es gab
ein Leistungsstipendium, was noch zum Grundstipendium hinzukam. Das
waren 80 Mark, also damit hätte ich dreimal meine Miete bezahlen können.
Insofern kam das Arbeiten müssen erst nach der Wende, aber dann auch
sehr massiv. Ich war später Studienfachberaterin und da waren es
bestimmt 70 bis 80 \% der Studierenden, die arbeiten gehen mussten. Das
verursacht schon Stress und das muss man tolerieren. Ich denke da haben
wir auch im Institut und im Prüfungsausschuss immer Möglichkeiten
gefunden, um die Studierenden zu unterstützen, beispielsweise durch eine
Stückelung des Praktikums. Familiäre Belastung gab es bei
DDR-Studierenden auch, da haben ja sehr viele während des Studiums
Kinder bekommen. Da wurden dann Sonderstudienpläne gemacht und
Patenschaften organisiert. Die meisten haben dadurch die
Regelstudienzeit trotzdem einhalten können.\\
~\\
\textbf{TR: Noch eine abschließende Frage: Im Winter feiern wir 90 Jahre
IBI. Was wünschen Sie dem Institut zum 90. Geburtstag?}

GP: Auf jeden Fall wünsche ich dem IBI, dass es weiter als akzeptierter,
sichtbarer Teil der Universität existiert. Ich wünsche mir, dass immer
wieder auch innovative Prozesse stattfinden. Dass durch neue Leute,
durch neue Fragestellungen, durch neue Projekte und durch Kooperationen
immer wieder auch etwas bewegt und neu erforscht wird und neu in die
Ausbildungsinhalte kommt. Darüber hinaus wünsche ich dem Institut, dass
ein gutes Klima herrscht, eine Kollegialität unter den Kolleginnen und
Kollegen und eine Solidarität mit den Studierenden, so dass dieses
offene und studierendenfreundliche Niveau erhalten bleibt. Und dann
wünsche ich dem IBI natürlich auch, dass immer ausreichend personelle
und materielle Ausstattung da ist \emph{{[}lacht{]}}. Es wäre schöner,
es gäbe mehr Stellen, so dass sich die Arbeit mehr verteilt. Ein
besserer Zustand der Räumlichkeiten wäre auch gut. In meinem ganzen
Berufsleben habe ich immer wieder Umzüge und Bauarbeiten erlebt
\emph{{[}lacht{]}.}

\textbf{TR: Frau Dr.~Pannier, vielen Dank für das Interview.}

GP: Ich danke Ihnen, es war sehr interessant das noch einmal zu
spiegeln.

%autor
\begin{center}\rule{0.5\linewidth}{\linethickness}\end{center}

\textbf{Thomas Roesnick} studiert aktuell im sechsten Bachelorsemester
„Bibliotheks- und Informationswissenschaft`` am Institut. Momentan
schreibt er an seiner Bachelorarbeit und würde danach gerne an der
Humboldt-Universität den Master in „Deutsche Literatur`` belegen.

\textbf{Andreas Erbe} studiert derzeit im 3. Bachelorsemester
Bibliotheks- und Informationswissenschaft an der Humboldt-Universität zu
Berlin. Vor seinem Studium absolvierte er erfolgreich eine Ausbildung
zum Fachangestellten für Medien- und Informationsdienste in der
Fachrichtung Bibliothek und ist seit dem an der Staatsbibliothek zu
Berlin tätig. Den damit eingeschlagenen Berufsweg möchte er auch nach
seinem Studienabschluss weiter fortsetzten.

\textbf{Miriam Brauer} studiert aktuell im vierten Bachelorsemester
Bibliotheks- und Informationswissenschaft am Institut. Zudem ist sie als
studentische Mitarbeiterin am Lehrstuhl für Information Processing and
Analytics tätig.

\end{document}
