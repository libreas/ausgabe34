\documentclass[a4paper,
fontsize=11pt,
%headings=small,
oneside,
numbers=noperiodatend,
parskip=half-,
bibliography=totoc,
final
]{scrartcl}

\usepackage{synttree}
\usepackage{graphicx}
\setkeys{Gin}{width=.4\textwidth} %default pics size

\graphicspath{{./plots/}}
\usepackage[ngerman]{babel}
\usepackage[T1]{fontenc}
%\usepackage{amsmath}
\usepackage[utf8x]{inputenc}
\usepackage [hyphens]{url}
\usepackage{booktabs} 
\usepackage[left=2.4cm,right=2.4cm,top=2.3cm,bottom=2cm,includeheadfoot]{geometry}
\usepackage{eurosym}
\usepackage{multirow}
\usepackage[ngerman]{varioref}
\setcapindent{1em}
\renewcommand{\labelitemi}{--}
\usepackage{paralist}
\usepackage{pdfpages}
\usepackage{lscape}
\usepackage{float}
\usepackage{acronym}
\usepackage{eurosym}
\usepackage[babel]{csquotes}
\usepackage{longtable,lscape}
\usepackage{mathpazo}
\usepackage[normalem]{ulem} %emphasize weiterhin kursiv
\usepackage[flushmargin,ragged]{footmisc} % left align footnote
\usepackage{ccicons} 
\setcapindent{0pt} % no indentation in captions

%%%% fancy LIBREAS URL color 
\usepackage{xcolor}
\definecolor{libreas}{RGB}{112,0,0}

\usepackage{listings}

\urlstyle{same}  % don't use monospace font for urls

\usepackage[fleqn]{amsmath}

%adjust fontsize for part

\usepackage{sectsty}
\partfont{\large}

%Das BibTeX-Zeichen mit \BibTeX setzen:
\def\symbol#1{\char #1\relax}
\def\bsl{{\tt\symbol{'134}}}
\def\BibTeX{{\rm B\kern-.05em{\sc i\kern-.025em b}\kern-.08em
    T\kern-.1667em\lower.7ex\hbox{E}\kern-.125emX}}

\usepackage{fancyhdr}
\fancyhf{}
\pagestyle{fancyplain}
\fancyhead[R]{\thepage}

% make sure bookmarks are created eventough sections are not numbered!
% uncommend if sections are numbered (bookmarks created by default)
\makeatletter
\renewcommand\@seccntformat[1]{}
\makeatother


\usepackage{hyperxmp}
\usepackage[colorlinks, linkcolor=black,citecolor=black, urlcolor=libreas,
breaklinks= true,bookmarks=true,bookmarksopen=true]{hyperref}
%URLs hart brechen
\makeatletter 
\g@addto@macro\UrlBreaks{ 
  \do\a\do\b\do\c\do\d\do\e\do\f\do\g\do\h\do\i\do\j 
  \do\k\do\l\do\m\do\n\do\o\do\p\do\q\do\r\do\s\do\t 
  \do\u\do\v\do\w\do\x\do\y\do\z\do\&\do\1\do\2\do\3 
  \do\4\do\5\do\6\do\7\do\8\do\9\do\0} 
% \def\do@url@hyp{\do\-} 
\makeatother 

%meta
%meta

\fancyhead[L]{K. Schuldt\\ %author
LIBREAS. Library Ideas, 34 (2018). % journal, issue, volume.
\href{http://nbn-resolving.de/}
{}} % urn 
% recommended use
%\href{http://nbn-resolving.de/}{\color{black}{urn:nbn:de...}}
\fancyhead[R]{\thepage} %page number
\fancyfoot[L] {\ccLogo \ccAttribution\ \href{https://creativecommons.org/licenses/by/3.0/}{\color{black}Creative Commons BY 3.0}}  %licence
\fancyfoot[R] {ISSN: 1860-7950}

\title{\LARGE{Zu disparat als Debattenbeitrag}}% title
\subtitle{Rezension zu: Schöpfel, Joachim ; Herb, Ulrich (edit.). Open Divide: Critical Studies on Open Access. Sacramento, CA: Library Juice Press, 2018}
\author{Karsten Schuldt} % author

\setcounter{page}{1}

\hypersetup{%
      pdftitle={Zu disparat als Debattenbeitrag. Rezension zu: Schöpfel, Joachim ; Herb, Ulrich (edit.). Open Divide: Critical Studies on Open Access. Sacramento, CA: Library Juice Press, 2018},
      pdfauthor={Karsten Schuldt},
      pdfcopyright={CC BY 3.0 Unported},
      pdfsubject={LIBREAS. Library Ideas, 34 (2018).},
      pdfkeywords={Open Science, Kritik},
      pdflicenseurl={https://creativecommons.org/licenses/by/3.0/},
      pdfcontacturl={http://libreas.eu},
      baseurl={http://libreas.eu},
      pdflang={de},
      pdfmetalang={de}
     }



\date{}
\begin{document}

\maketitle
\thispagestyle{fancyplain} 

%abstracts

%body
Dass Open Access durch die Etablierung kommerzieller Formen, also vor
allem die Durchsetzung von Open Access Gebühren, vulgo Article
Processing Charges, bei Publikation in Zeitschriften der grossen
Wissenschaftsverlage und die Übernahme dieser Gebühren durch
Bibliotheken, Forschungseinrichtungen und Forschungsförderer, nicht mehr
das ist, was am Anfange dieser \enquote{Bewegung} diskutiert wurde, ist
eine weit verbreitete Einschätzung. Egal, was man als den Beginn dieser
Bewegung ansieht (zumeist wird bekanntlich auf die Budapest Open Access
Initiative von 2002 oder die Berlin Declaration on Open Access von 2003
verwiesen), ist dies auch einfach ersichtlich. Das Versprechen -- um
daran nochmal zu erinnern --, mit Open Access einen weltweiten Zugang zu
wissenschaftlichem Wissen zu schaffen, damit die Wissensunterschiede
zwischen dem globalen Norden und Süden aufzuheben, eine bessere und
fairere gesellschaftliche Nutzung dieses Wissens zu ermöglichen und
grundsätzlich zu einer besseren Welt beizutragen, scheint verschwunden
zu sein. Auch diese Zeitschrift thematisierte diesen Eindruck in einem
Schwerpunkt in der Ausgabe \#32 (2017).

Das nun zu besprechende, dünne Buch tritt an, diese Einschätzung weiter
zu vertiefen und vor allem auch Blickwinkel aus dem globalen Süden
einzubeziehen. Notwendig ist dies. Das Unbehagen mit der jetzigen
Situation zu thematisieren, Möglichkeiten zu schaffen, dieses Unbehagen
zu diskutieren und zu analysieren, ist wohl der einzige Weg, eine
Position zu entwickeln, um den grossen Wissenschaftsverlagen die
Deutungshoheit im Bereich Open Access wieder zu nehmen. Dass das
notwendig ist, sollte spätestens nach der Lektüre der besseren Artikel
in diesem Buch verständlich sein. Insoweit ist das Buch zu begrüssen. Es
bedürfte mehr solcher Publikationen.

Allerdings: Das Buch ist ein Herausgeberwerk und als solches erstaunlich
divergent, sowohl was den Inhalt und die Ziele der Texte angeht als auch
die Textformen und die Qualität der Texte. Die ordnende Hand des
Herausgeberteams -- die an sich beide inhaltlich durch vorhergehende
Publikationen gut als kritische Experten im Themenfeld positioniert sind
--, welche in einem solchen Werk zu erwarten wäre, fehlt. Die Texte
stehen, wie in einer Zeitschrift, ohne jeden Bezug nebeneinander.
Teilweise widersprechen sie sich direkt. Dies ist direkt bei den beiden,
hintereinanderstehenden Texten von Florence Piron und Iryna Kuchma zu
sehen. Im ersten wird die neokoloniale Ausrichtung des Open
Access-Diskurses (dazu unten mehr) kritisiert, vor allem die Idee, der
globale Süden müsste die Wissenschaftsentwicklung des globalen Nordens
\enquote{aufholen}, im zweiten wird praktisch in einem Jubelartikel eine
Initiative vorgestellt, die dem globalen Süden genau dieses
\enquote{Aufholen} ermöglichen soll. Diese Divergenz zieht sich durch
das ganze Buch und macht es schwierig zu lesen. Es gibt auch einige
Texte, bei denen der Bezug zum Thema des Buches selber gar nicht klar
ist, besonders der von Soenke Zehle, dessen Inhalt dem Rezensenten auch
bei zweifachen Lesen nicht verständlich wurde, der aber mit Open Access
offenbar nicht wirklich etwas zu tun hat, und der von Beatriz de los
Arcos und Martin Weller, welcher gar nicht über Open Access, sondern
über Open Educational Ressources spricht.

Es soll aber nicht der Eindruck entstehen, alle Texte im Band wären
unwichtig oder beliebig. Sie sind einfach nicht als Gesamtheit
wahrnehmbar. Im ersten von zwei Teilen finden sich vor allem Texte, die
sich mit der Geschichte von Open Access beschäftigen. Hier wird auch
offen diskutiert, ob sich die ursprünglichen Versprechen eigentlich
halten lassen oder ob sie nicht eher schon von Beginn an als neoliberal
zu bezeichnen wären. Hat Open Access nicht eher den Effekt, dass nicht
die Gesellschaft -- die ja, so die verbreitete Argumentation, für die
Produktion wissenschaftlichen Wissens zahlt -- profitiert, sondern vor
allem die Wirtschaft, die so noch direkter und ohne eigene Kosten auf
dieses Wissen zurückgreifen kann, während die Gesellschaft wenig mit
diesem Wissen anzufangen weiss, da zur sinnvollen Nutzung von Wissen
mehr gehört, als das es frei ist? Ist Open Access nicht zu einem
weiteren Werkzeug geworden, um die Arbeit von Wissenschaftlerinnen und
Wissenschaftlern zu kontrollieren? Und sicher: Ist der Fakt, dass die
grossen Wissenschaftsverlage Open Access als Geschäftsmodell integriert
haben -- aber was sollen Verlage als Wirtschaftsunternehmen auch sonst
tun -- ein starker Hinweis dafür, dass Open Access sich in eine falsche
Richtung entwickelt hat? Es ist, wie gesagt, wichtig, solche Fragen zu
stellen und zu diskutieren, wenn zum Beispiel Forschende oder
Bibliotheken (wieder) zu proaktiven Beteiligten der Debatten werden
wollen.

Der zweite Teil des Buches beschäftigt sich mit dem globalen Süden.
Hierzu hatte einer der Herausgeber, Joachim Schöpfel, vor drei Jahren
schon ein Buch vorgelegt (vergleiche Schöpfel 2015), allerdings werden
jetzt nicht nur, wie noch 2015, die BRICS-Staaten betrachten, sondern
weitere Staaten einbezogen. In diesem Teil finden sich, wie schon
erwähnt, sehr disparate Texte. Gewichtig sind hier die Texte, welche
explizit diskutieren, ob Open Access in seiner jetzigen, kommerziellen
Form nicht zu einen \enquote{neocolonial tool} geworden ist und ob dies
nicht auch schon in der frühen Bewegung angelegt war. Wenn Open Access
heute vor allem daran gebunden wird, ob Universitäten oder Staaten die
allfälligen Open Access Gebühren zahlen (können) oder nicht, ist es dann
nicht so, dass die eh schon vorhandenen Unterschiede in der
Dissemination von Wissen verstärkt werden? Also: Ist es dann nicht nur
so, dass Forschende aus dem globalen Süden ihre Forschungsergebnisse
nicht nur nicht mehr sichtbar veröffentlichen können, sondern dass zudem
das Wissen, welches sie selber wahrnehmen, das aus dem globalen Norden
ist? Werden dann nicht Forschungsinteressen und -perspektiven aus dem
globalen Norden als \enquote{das gesamte mögliche Wissen} präsentiert,
noch mehr als jetzt schon? Und: Ist dies erst ein Ergebnis der
Kommerzialisierung von Open Access oder war dies schon angelegt in der
Vorstellung, eine Lösung für die Ungleichheit in der Welt wäre, wenn das
Wissen des globalen Nordens auch dem globalen Süden zugänglich gemacht
würde? Waren die eher technischen Debatten -- wie das Wissen zugänglich
machen, welche Technik nutzen -- der frühen Open Access-Bewegung, in
denen kaum über eine systematische Änderung der Wissensproduktion selber
nachgedacht wurde, nicht schon ein Problem? Wurde hier nicht auch schon
vertreten, dass der Rest der Welt den Forschungstraditionen des globalen
Nordens folgen sollte, ohne dass sich dieser selbst verändern müssten?
Einige der Texte im zweiten Teil des Buches diskutieren, wie angesichts
der Realität in verschiedenen Staaten und schon vorhandener Versuche von
Open Access im globalen Süden, dieses grundsätzlich an den Überzeugungen
zu Open Access ansetzenden Probleme anzugehen seien. Zu erwähnen sind
der Text von Florence Piron, welche auch fordert, über andere Modelle
von Wissenschaft nachzudenken, und das Interview mit Leslie Chan, einem
der Gründer von Bioline, welche die Publikation von Wissen aus der
Biologie aus dem globalen Süden ermöglicht. Chan, der aus der
Perspektive eines kanadischen Forschenden, welcher Wissen aus dem
globalen Süden nutzt, spricht, erwähnt aber auch -- und das ist ein
wichtiges Ende dieses Buches, welches zeigt, wie die Situation gerade
tatsächlich ist -- das Beispiel von Zeitschriften aus Indien, die über
Bioline so gewichtig wurden, dass sie von einem der grossen
Wissenschaftsverlage aufgekauft, in dessen System integriert und dem
ursprünglichen Redakteur, der jetzt verschwunden ist, entrissen wurden.
Neokolonialismus in seiner am einfachsten zu fassenden Form.

Das Buch ist zu empfehlen als Teil einer notwendigen Debatte, solange
das Ziel dieser Debatte ist, Open Access sinnvoll zu entwickeln, also
nicht nur den grossen Verlagen abspenstig zu machen, sondern auch über
die impliziten Annahmen im Diskurs um Open Access nachzudenken und diese
zu verändern (was auch heisst, über die Ungerechtigkeiten der heutigen
Form von Wissensproduktion nachzudenken). Aber nur Teile des Buches
tragen tatsächlich zu dieser Debatte bei. Andere lenken davon ab. Dies
sollte beim Lesen beachtet werden. Es ist aber zu hoffen --
insbesondere, wenn immer noch das Ziel gilt, dass die Welt durch
Wissenschaft und Wissenschaftskommunikation besser werden soll -- dass
die Debatte selbst weitergeführt wird.

\section{Literatur}

Schöpfel, Joachim (edit). Learning from the BRICS: Open Access to
Scientific Information in Emerging Countries. Sacramento, CA: Litwin,
2015

%autor
\begin{center}\rule{0.5\linewidth}{\linethickness}\end{center}

\textbf{Karsten Schuldt} (Chur / Berlin) ist Wissenschaftlicher
Mitarbeiter am Schweizerischen Institut für Informationswissenschaft,
HTW Chur und Redaktor der LIBREAS. Library Ideas.

\end{document}
