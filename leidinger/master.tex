\documentclass[a4paper,
fontsize=11pt,
%headings=small,
oneside,
numbers=noperiodatend,
parskip=half-,
bibliography=totoc,
final
]{scrartcl}

\usepackage{synttree}
\usepackage{graphicx}
\setkeys{Gin}{width=.4\textwidth} %default pics size

\graphicspath{{./plots/}}
\usepackage[ngerman]{babel}
\usepackage[T1]{fontenc}
%\usepackage{amsmath}
\usepackage[utf8x]{inputenc}
\usepackage [hyphens]{url}
\usepackage{booktabs} 
\usepackage[left=2.4cm,right=2.4cm,top=2.3cm,bottom=2cm,includeheadfoot]{geometry}
\usepackage{eurosym}
\usepackage{multirow}
\usepackage[ngerman]{varioref}
\setcapindent{1em}
\renewcommand{\labelitemi}{--}
\usepackage{paralist}
\usepackage{pdfpages}
\usepackage{lscape}
\usepackage{float}
\usepackage{acronym}
\usepackage{eurosym}
\usepackage[babel]{csquotes}
\usepackage{longtable,lscape}
\usepackage{mathpazo}
\usepackage[normalem]{ulem} %emphasize weiterhin kursiv
\usepackage[flushmargin,ragged]{footmisc} % left align footnote
\usepackage{ccicons} 

%%%% fancy LIBREAS URL color 
\usepackage{xcolor}
\definecolor{libreas}{RGB}{112,0,0}

\usepackage{listings}

\urlstyle{same}  % don't use monospace font for urls

\usepackage[fleqn]{amsmath}

%adjust fontsize for part

\usepackage{sectsty}
\partfont{\large}

%Das BibTeX-Zeichen mit \BibTeX setzen:
\def\symbol#1{\char #1\relax}
\def\bsl{{\tt\symbol{'134}}}
\def\BibTeX{{\rm B\kern-.05em{\sc i\kern-.025em b}\kern-.08em
    T\kern-.1667em\lower.7ex\hbox{E}\kern-.125emX}}

\usepackage{fancyhdr}
\fancyhf{}
\pagestyle{fancyplain}
\fancyhead[R]{\thepage}

% make sure bookmarks are created eventough sections are not numbered!
% uncommend if sections are numbered (bookmarks created by default)
\makeatletter
\renewcommand\@seccntformat[1]{}
\makeatother


\usepackage{hyperxmp}
\usepackage[colorlinks, linkcolor=black,citecolor=black, urlcolor=libreas,
breaklinks= true,bookmarks=true,bookmarksopen=true]{hyperref}
%URLs hart brechen
\makeatletter 
\g@addto@macro\UrlBreaks{ 
  \do\a\do\b\do\c\do\d\do\e\do\f\do\g\do\h\do\i\do\j 
  \do\k\do\l\do\m\do\n\do\o\do\p\do\q\do\r\do\s\do\t 
  \do\u\do\v\do\w\do\x\do\y\do\z\do\&\do\1\do\2\do\3 
  \do\4\do\5\do\6\do\7\do\8\do\9\do\0} 
% \def\do@url@hyp{\do\-} 
\makeatother 

%meta
%meta

\fancyhead[L]{G. Leidinger \\ %author
LIBREAS. Library Ideas, 34 (2018). % journal, issue, volume.
\href{http://nbn-resolving.de/}
{}} % urn 
% recommended use
%\href{http://nbn-resolving.de/}{\color{black}{urn:nbn:de...}}
\fancyhead[R]{\thepage} %page number
\fancyfoot[L] {\ccpd\ \href{https://creativecommons.org/publicdomain/mark/1.0/}{\color{black}Creative Commons Public Domain Mark 1.0}}  %licence
\fancyfoot[R] {ISSN: 1860-7950}

\title{\LARGE{Was ist Bibliothekswissenschaft}}% title
\author{Georg Leidinger} % author

\setcounter{page}{1}

\hypersetup{%
      pdftitle={Was ist Bibliothekswissenschaft},
      pdfauthor={Georg Leidinger},
      pdfcopyright={CC Public Domain Mark 1.0},
      pdfsubject={LIBREAS. Library Ideas, 34 (2018).},
      pdfkeywords={Bibliothekswissenschaft, Geschichte, Selbstverständnis, Berufsbild},
      pdflicenseurl={https://creativecommons.org/publicdomain/mark/1.0/},
      pdfcontacturl={http://libreas.eu},
      baseurl={http://libreas.eu},
      pdflang={de},
      pdfmetalang={de}
     }



\date{}
\begin{document}

\maketitle
\thispagestyle{fancyplain} 

%abstracts
\begin{abstract}
\noindent Erstabdruck: Zentralblatt für Bibliothekswesen 45 (1928) 8, S. 440--455\\
\href{http://resolver.sub.uni-goettingen.de/purl?PPN338182551\_0045}{http://resolver.sub.uni-goettingen.de/purl?PPN338182551\_0045}
\end{abstract}

%body
\vspace*{1em}
Referent: Geh. Reg.-Rat Prof.~Dr.~Georg Leidinger -- München
\vspace*{1em}

Hochgeehrte Festversammlung!

Göttingen ist für den wissenschaftlichen Bibliothekar \emph{geweihter
Boden}. Hier haben wahrhafte Bibliothekare gewaltet, denen
Bibliothekswesen und Bibliothekswissenschaft viel zu verdanken haben.
Wenn heute die deutschen Bibliothekare zu ihrer Jahresversammlung hier
zusammengekommen sind, um sich über die Sorgen und Fragen ihres Berufes
zu besprechen, haben sie das Gefühl, dem genius loci verpflichtet zu
sein. In diesem Sinne hat der Vorsitzende unseres Vereines an mich die
Bitte gerichtet, über einen Gegenstand zu reden, der, wie er jenen
Männern am Herzen lag, auch unser aller Interesse in höchstem Grade
beanspruchen darf.

Ich habe das Thema meines Vortrages in Frageform gekleidet: \enquote{Was
ist Bibliothekswissenschaft?} Mancher unter Ihnen mag sich darüber
gewundert haben. Hätte ich nicht ebensogut sagen können: \enquote{Über
Wesen und Inhalt der Bibliothekswissenschaft}? Gewiß.

Aber gerade die Frage: \enquote{Was ist denn eigentlich
Bibliothekswissenschaft?} ist schon so oft an mich gestellt worden, daß
sie mir zeigte, hier liege bei gar Vielen eine Unkenntnis vor, die zu
beheben der Mühe lohnen dürfte. Waren es doch nicht etwa nur dem
Bibliothekswesen Fernerstehende, die so fragten, sondern sogar
Berufsgenossen: diese hatten offenbar über jene ihr eigenes Fach
betreffende Frage noch so wenig nachgedacht, daß ihnen eine Antwort
darauf nicht ohne weiteres zur Hand war.

In der kurzen Zeit, die zur Verfügung steht, wird es unmöglich sein, das
Thema erschöpfend zu behandeln. Wenn ich Ihre Aufmerksamkeit und Geduld
nicht allzu lange in Anspruch nehmen soll, werde ich mich darauf
beschränken müssen, Ihnen einzelne Gedanken vorzutragen, die mir im
Laufe meiner nun mehr als zwanzigjährigen bibliothekarischen
\emph{Lehr}tätigkeit aufgestiegen sind. Nur skizzenhaft kann ich mich
äußern.

Was ich vortragen will, sind kaum neue Dinge; das Meiste davon ist schon
einmal und öfter gesagt worden. Aber es sind Betrachtungen, die man
nicht oft genug vorbringen kann, da sie immer wieder -- bald da, bald
dort -- außer Acht gelassen werden.

Ich darf Sie kurz daran erinnern, daß wir, wenn wir von \emph{der}
Wissenschaft sprechen, einen \emph{allgemeinen} Begriff vor uns haben,
welchem ein sondernder und teilender Begriff, \emph{die}
Wissenschaft\emph{en} untergeordnet ist. Das Ziel der Wissenschaft ist
das Wissen, das heißt das Auffassen dessen, was überhaupt der
menschlichen Erkenntnis zugänglich ist. Um ihr Ziel zu erreichen und zum
Wissen durchzudringen, \emph{gruppiert} die Wissenschaft die sachlich
zusammengehörigen Gegenstände, auf die sich ihre Untersuchungen
beziehen. Sie teilt ihr ganzes Gebiet, wie die Philosophie sagt,
adäquat, das heißt: eben dem Wesen der Gegenstände angeglichen oder
angemessen ein. So entstehen die einzelnen Wissenschaften.

Der englische Schriftsteller \textsc{George Henry Lewes} (1817--1878),
der am bekanntesten durch eine Biographie Goethes geworden ist, hat
vielleicht die beste Definition des Begriffes Wissenschaft gegeben als:
die \emph{Systematisierung unserer Erfahrungen}. Eine Einzelwissenschaft
ist demnach die Systematisierung unserer Erfahrungen auf einem
bestimmten Gebiete.

Wir pflegen die großen Gruppen der Naturwissenschaften und der
Geisteswissenschaften zu unterscheiden. In neuerer Zeit sind zu
letzteren noch einzelne Wissensgebiete getreten, die früher nicht als
eigentliche Geisteswissenschaften angesehen worden sind, die aber im
Laufe der Zeit sich dazu entwickelt haben.

Es gibt \emph{Gebiete menschlicher Tätigkeiten}, welche eben dadurch,
daß über sie gedacht worden ist, und dadurch, daß das Gedachte in
lebendige Worte gefaßt und mündlich oder schriftlich verbreitet worden
ist, Wissenschaften hervorgebracht haben. Denken Sie an
Finanzwissenschaft, Handelswissenschaft, Verkehrswissenschaft,
Theaterwissenschaft, Zeitungswissenschaft, Missionswissenschaft usw. Vor
wenigen Jahrzehnten hat man noch nicht gewagt, von diesen Fächern als
von Wissenschaften zu reden: man sprach von Finanzlehre, Handelslehre,
Verkehrslehre oder Finanzkunde, Handelskunde, Verkehrskunde oder
Verkehrswesen, Theaterwesen, Zeitungswesen, Missionswesen usw. Jetzt
besitzen diese Fächer als vollberechtigte Wissenschaften Lehrstühle an
den Hochschulen.

Hier handelt es sich um Beobachtungen und Erfahrungen über praktische
menschliche Arbeit, über kulturelle Tätigkeit, Erfahrungen, die infolge
der Denkvorgänge über sie, durch geistige Prüfung und Kritik sich zu
Wissenschaften gestalten und ihre Grundlagen auf eine höhere Stufe ihres
Wesens emporheben können. Ich möchte sie zusammenfassend als
\emph{Kulturwissenschaften} bezeichnen.

Ich bin der Letzte, der es bestreitet, daß jene Kulturwissenschaften von
der eigentlichen philosophischen Wissenschaft sich wesentlich
unterscheiden.

Der Begriff des Wortes Wissenschaft ist mit dem Erstehen dieser
Wissensgebiete eben verändert, hat sich erweitert oder, wenn man will,
verengert.

Aber es geht nicht mehr an, der Pflege dieser Gebiete die
\emph{Berechtigung}, das Wort Wissenschaft für sich zu beanspruchen,
nehmen zu wollen.

Zu dieser Art von Wissenschaften gehört auch die
Bibliothekswissenschaft.

Am nächsten von den vorhin genannten erst in der Neuzeit in die
Erscheinung getretenen Kulturwissenschaften stehen ihr
Zeitungswissenschaft und Theaterwissenschaft. So gut sich um das
Zeitungswesen die Zeitungswissenschaft bilden konnte und jetzt weiter
entwickeln kann, so gut aus dem Theaterwesen die Theaterwissenschaft
entsprossen ist und weiterstrebt, ebensogut können die Bibliotheken ein
Kulturgebiet sein, das seine eigene Wissenschaft pflegen will. Mit
Zeitungs- und Theaterwissenschaft hat die Bibliothekswissenschaft die
Eigenschaft gemeinsam, daß sie jung ist und alle Merkmale jungen Wesens
an sich trägt. Sie steht, wie ich von vornherein betonen möchte, erst in
ihren \emph{Anfängen}.

Wo eine Büchersammlung vorhanden ist oder entstehen soll, braucht man
Menschen, um die Büchermassen, welche die Bibliothek bilden sollen, in
überlegter und sinnvoller Weise zu erwerben, sie in noch sinnvollerer
Weise aufzustellen und zu ordnen, dann in Ordnung zu halten, Menschen,
die ferner darüber nachdenken, wie ihre Benutzung zu ermöglichen ist,
wie die Bücher zu verzeichnen oder zu katalogisieren sind, wie dann ihre
Benutzung zu überwachen und die Erhaltung zu sichern, wie denn überhaupt
die Sammlung so zu verwalten ist, daß sie keinen Schaden leidet,
vermehrt wird und in guter Erhaltung bleibt, damit sie ihrer Gegenwart
dient und für ferne Zukunft nützen mag. Vor allen Dingen müssen jene
Menschen von den Büchern, von ihrem Wesen \emph{und} Inhalt, etwas
verstehen. So stellt sich uns das Bibliotheks\emph{wesen} dar.

Die Summe aller Erfahrungen aber bei dieser Tätigkeit, welche der Mensch
im Bibliothekswesen entfaltet, bildet, logisch-philosophisch behandelt,
den Inhalt der heutigen Bibliothekswissenschaft, wobei, um sie zur
vollen wissenschaftlichen Erkenntnis zu führen, die geschichtliche
Betrachtung des Bibliothekswesens der Vergangenheit nicht fehlen darf.

Ich unterscheide also das Bibliotheks\emph{wesen} und die
Bibliotheks\emph{wissenschaft}, insoferne das erstere die Grundlage der
letzteren bildet.

Die Bibliothekswissenschaft baut ihre Behauptungen und Lehren auf dem
Bibliothekswesen auf.

Sie entfaltet (ich gebrauche ungern das folgende Fremdwort, aber es ist
sehr bezeichnend) eine abstrahierende Tätigkeit. Sie entnimmt dem
Bibliothekswesen Abstraktionen, sie entzieht für sich diesem alles, was
gedanklich zu erfassen und zu gestalten ist.

Mag man die Bezeichnung der neuen Kulturwissenschaften auch als einen
Konventionsbegriff ächten zu müssen geglaubt haben, als einen Begriff
des Übereinkommens, so ist die Konvention eben eine Notwendigkeit, ein
Zwang, den das \emph{Vorhandensein} dieser Wissensgebiete ausübt.

Ich will nicht über die Geschichte der Bibliothekswissenschaft reden,
will nicht zeigen, welche Leute der Bibliothekskunde den Weg zur
Anerkennung als Lehrfach gebahnt haben. Starke Widerstände galt es zu
überwinden, merkwürdigerweise auch aus dem eigenen Fach heraus, aber
langsam und stetig ist die Erkenntnis durchgedrungen, daß das
Bibliothekswesen von höherer Warte als der des täglichen Dienstes aus
überblickt und überdacht werden muß.

Das Bibliothekswesen ist gewissermaßen ein Rohstoff, der erst zur
Bibliothekswissenschaft veredelt werden muß. Diese Veredelung erfolgt
durch das Denken, welches die durch Beobachtung gemachten Erfahrungen
dann erst geistig verarbeitet und ihnen ihre Bedeutung innerhalb des
ganzen unserer Erkenntnis zugänglichen Seins anweist, sie würdigt und
wertet, vor Über- oder Unterschätzung bewahrt und der richtigen Nutzung
zuführt.

Ein Beamter der Münchener Zentralbibliothek, \textsc{Martin
Schrettinger}, ist es gewesen, welcher die Bezeichnung
Bibliothekswissenschaft (er sagte allerdings fälschlich
Bibliothekwissenschaft) zuerst in die Fachliteratur eingeführt hat durch
sein 1808 erschienenes \enquote{Lehrbuch der Bibliothekwissenschaft}.

\textsc{Schrettinger} war jedenfalls einer der tüchtigsten Männer
älterer Zeiten gewesen, die je über das gesamte Bibliothekswesen
\emph{nachgedacht} hatten. Aus diesem Nachdenken heraus hatte sich ihm
das Gefühl entwickelt, daß die \emph{Ergebnisse} seiner \emph{Denkarbeit
wissenschaftlich} waren, und dadurch kam er dazu, den Inhalt seiner sich
mit dem Bibliothekswesen beschäftigenden Darlegungen als
Bibliothekswissenschaft zu bezeichnen. Nicht Stolz oder Überhebung oder
Einbildung waren es, die ihn hierbei leiteten. Auch wir
Gegenwartsmenschen sind weit entfernt von solchen Eigenschaften. Heute
kann die Behauptung, der Name \enquote{Bibliothekswissenschaft} sei
\enquote{allzu prätentiös und mit Recht bestritten}, die
\textsc{Schrettinger} gegenüber gefallen ist, nicht mehr aufrecht
erhalten werden.

Zu der auf dem Wege philosophischer Methode erfolgenden Behandlung einer
Wissenschaft muß sich notwendigerweise, soll eine völlige Erkenntnis
ihres Wesens sich erschließen, auch die geschichtliche Betrachtung ihrer
Entwickelung gesellen. Gerade die Historie einer Wissenschaft von jener
Art, die sich, wie ich darlegte, auf eine bestimmte menschliche
Tätigkeit gründet, läßt schließlich erkennen, wie lange jene Tätigkeit
gebraucht hat, um sich zur Wissenschaft zu entfalten, läßt erkennen, in
welchem Umfange sie in der Gegenwart Wissenschaft geworden ist, wie weit
sie berechtigt ist, sich solche zu nennen, und welche Wege sie
einschlagen muß, um sich ferner zu entwickeln, weiterzuschreiten und zu
höherer Ausbildung und bis zur Vollkommenheit zu gelangen. Ist doch jede
Einrichtung der Gegenwart, wie ja auch jeder Mensch, bedingt und
abhängig von zahllosen Einflüssen der Vergangenheit.

Gerade das Bibliothekswesen aber ist ja nicht eine Erscheinung nur der
neueren Zeiten, wie etwa das die Grundlage der Zeitungswissenschaft
bildende Zeitungswesen, sondern reicht in graue Vorzeit zurück. Und
seine Geschichte ist verbunden mit der Geschichte der höchsten geistigen
Interessen der gesamten Kulturmenschheit.

Eine Einzelwissenschaft zeigt ihren Inhalt und auch die Höhe ihres
Standes am besten dort, wo sie \emph{gelehrt} werden muß, wo ihr Inhalt
nicht bloß \emph{gefühlt} wird, sondern Anderen in klaren Darlegungen
geoffenbart werden muß. Die Kenntnis von allen Einzelheiten, welche man
zur Anlage und Verwaltung von Büchersammlungen nötig hat, die Kenntnis
des Bibliothekswesens, wird zur Wissenschaft erst durch die Lehre.

An unseren großen Bibliotheken ist diese Lehre entstanden und gepflegt
worden, sie gehört aber jetzt in den Rahmen der Universität, der
universitas litterarum. Sie hat ein \emph{Recht} darauf, dort
vorgetragen zu werden.

Für den höheren Beamten unserer Bibliotheken genügt heutzutage die bloße
Ausbildung im praktischen Betrieb nicht mehr. Er muß die großen
Zusammenhänge seines Berufes, in die das Bibliothekswesen nun einmal
hineingestellt ist, kennen lernen, muß von den Erscheinungen wissen, die
ihm erst die Berechtigung geben, sich über unteren und mittleren
Beamtengruppen als eine zu höherer geistiger Tätigkeit verpflichtete
Persönlichkeit zu fühlen und von allen den Wissensgebieten einen
Überblick zu erhalten, die für seinen Beruf in Frage kommen. Er muß von
den Quellen erfahren, aus denen er eine zu immer höherer Vollkommenheit
sich steigernde Ergänzung seiner praktischen Kenntnisse schöpfen kann.
Hierzu gelangt er am besten durch den Besuch entsprechender
bibliothekswissenschaftlicher Vorlesungen, durch die Ausbildung in
bibliothekswissenschaftlichen Kursen und Instituten.

Der Lehrbetrieb der Bibliothekswissenschaft an den Hochschulen hat zu
einem großen Teil durch Vorlesungen zu erfolgen (hierüber brauche ich
wohl kaum Worte zu verlieren), zu einem wichtigen Teil aber auch durch
übungsmäßigen Unterricht in bibliothekswissenschaftlichen Instituten.
Solche besitzen wir überhaupt noch nicht oder höchstens in rudimentären
Ansätzen an den größeren Bibliotheken, an denen überhaupt Unterricht im
höheren Bibliothekswesen erteilt wird.\footnote{Während des Druckes
  dieses Vortrages kommt die erfreuliche Nachricht von der Errichtung
  des bibliothekswissenschaftlichen Instituts an der Universität Berlin.}

Die Grundlage solcher bibliothekswissenschaftlicher Institute wären
Handbibliotheken, in welchen die wichtigste bibliothekarische Literatur
zusammengestellt ist. Hier müßten alle Werke zu finden sein, die für
sämtliche Zweige der Bibliothekswissenschaft notwendig sind. Für unsere
Münchener Bibliothekskurse besteht eine solche Handbibliothek, die sich
mit leichter Mühe noch weiter ausbauen ließe und zweifellos das
Handwerkszeug für ein bibliothekswissenschaftliches Institut darbietet.

Ähnliche bibliothekswissenschaftliche Handbibliotheken müßten an allen
Orten geschaffen werden, die für bibliothekswissenschaftlichen Betrieb
überhaupt in Betracht kommen.

Denn ich bin nicht der Meinung, daß nun an \emph{jeder} Universität die
Bibliothekswissenschaft vertreten sein müßte. An den kleineren
Universitäten mit ihrem bescheidenen Bibliotheksbetrieb kann die gesamte
bibliothekswissenschaftliche Literatur unmöglich angeschafft werden;
dort mögen die Bibliotheksanwärter in der Praxis soweit ausgebildet
werden, als es je nach den örtlichen Verhältnissen möglich ist. Darnach
müssen sie jedoch zur Vermehrung ihrer Kenntnisse an größere
Bibliotheken kommen, an denen ihr Gesichtskreis sich erweitern soll.

Nur in Verbindung mit größeren Bibliotheken wird Bibliothekswissenschaft
gedeihen. Und auch da nur, wenn entsprechende Persönlichkeiten vorhanden
sind, welche eine wahrhaft wissenschaftliche Auffassung ihres Berufes
haben.

Denn die Pflege der Bibliothekswissenschaft ist in erster Linie eine
Persönlichkeitsfrage.

Ein Bibliotheksdirektor, dem Bücher nichts anderes sind als Objekte der
Statistik, wird geringen bibliothekswissenschaftlichen Sinn haben und
demgemäß auch nicht entsprechend auf seine Beamten, besonders nicht auf
die erst auszubildenden Anwärter einwirken können. Ihm einen
bibliothekswissenschaftlichen Lehrauftrag zu geben, ist verfehlt.

Wenn Bibliothekswissenschaft richtig betrieben werden soll, so erfordert
sie die volle Arbeitskraft eines Mannes. Man wird aus dieser Behauptung
in bezug auf den Betrieb der Bibliothekswissenschaft an den
Universitäten eine nächste und notwendige Folgerung ziehen müssen: diese
lautet dahin, daß an Universitäten, an welchen Bibliothekswissenschaft
gelehrt werden soll, eine ordentliche Professur dafür vorhanden sein
müßte.

Zur Zeit ist die einzige in Deutschland vorhandene ordentliche Professur
nicht besetzt, wir haben allenthalben nur Honorarprofessuren, an denen
viel geleistet wird, aber der richtige Zustand ist das nicht.

Bibliothekswissenschaft muß vom Nebenamt zum Hauptamt fortschreiten.

Wenn ich eben gesagt habe, die volle und gesamte Arbeitskraft eines
Mannes sei dazu nötig, um Bibliothekswissenschaft richtig zu treiben,
die Vertretung dieses Wissenschaftsgebietes fülle also das
Tätigkeitsfeld eines Mannes vollständig aus, so gehe ich sogar noch
weiter für den Fall, daß Bibliothekswissenschaft nun einmal in den
Unterrichtsrahmen einer Universität hineingestellt und aufgenommen ist.

Und ich sage: \emph{ein} Mann \emph{allein} vermag die
Bibliothekswissenschaft im ganzen nicht zu beherrschen und zu vertreten,
da sie bereits viel zu umfangreich geworden ist. Der ordentliche
Professor wird noch Helfer heranziehen müssen, sofern und soweit er an
dem betreffenden Orte sie erlangen kann.

Daß Professuren für Bibliothekswissenschaft eine hohe Bedeutung für die
Kultur haben werden, wird wohl niemand bestreiten. Wenn wir unsere
deutschen Verhältnisse betrachten, möchte man daher zunächst den Wunsch
aussprechen, daß in allen Städten, in denen sowohl eine große Bibliothek
-- die notwendige Voraussetzung für gedeihlichen Betrieb der
Bibliothekswissenschaft -- wie eine große Universität vorhanden ist,
eine ordentliche Professur für unser Fach errichtet werden möge.

Die Errichtung von Professuren für Bibliothekswissenschaft ist aber
nicht bloß für Deutschland zu wünschen, sondern muß eine internationale
Forderung sein und muß sich allmählich in allen Ländern geltend machen,
die den Stand ihrer Kultur durch das Vorhandensein öffentlicher
Büchersammlungen dartun. Es gibt noch genug Staaten, denen nennenswerte
Bibliotheken fehlen oder deren Bibliothekswesen auf sehr niedriger Stufe
steht. Bis das Bibliothekswesen in diesen wissenschaftlich behandelt
werden kann, dazu wird oft noch lange Zeit verfließen. Aber in den
großen Kulturstaaten, besonders solchen mit alten Büchersammlungen, wird
Betrieb und Ausbildung der Bibliothekswissenschaft nicht ausbleiben.

Ansätze dazu sind hier und dort vorhanden; doch will ich darauf hier
zunächst nicht eingehen.

Dringend nötig hätten wir ein Büchlein, das ich als \enquote{Ratgeber
für das Studium der Bibliothekswissenschaft} bezeichnen möchte. Es müßte
eigentlich alle Semester neu erscheinen, müßte ersehen lassen, was im
Inland und im Ausland an bibliothekswissenschaftlichen Vorlesungen und
Übungen geboten wird, so daß diejenigen, die nach einer höheren
Ausbildung im Bibliothekswesen streben, die Möglichkeit hätten, ihre
Studien entsprechend einzurichten, sich einen Studienplan
zurechtzulegen.

Sollte nicht der Verein Deutscher Bibliothekare in der Lage sein, einen
solchen \enquote{Ratgeber} ins Leben zu rufen, selbst wenn er nicht
selbständig als Büchlein, sondern als Zusammenstellung im Zentralblatt
für Bibliothekswesen erscheinen würde, aber regelmäßig, in sachgemäßen
Zeiträumen?

Notwendig ist, daß Bibliothekswissenschaft als Promotions- bzw.
Prüfungsfach anerkannt wird. Leider haben bisher die zopfigen
Promotionsordnungen der philosophischen Fakultäten das Vordringen der
Bibliothekswissenschaft verhindert und bibliothekswissenschaftliche
Arbeiten, die von den zur Zeit lehrenden Honorarprofessoren in die Wege
geleitet werden, müssen unter der Flagge anderer Wissenschaften
(Geschichte, Literaturgeschichte, Kunstgeschichte, Philologie usw.)
segeln.

Sie werden öfter den Ausdruck Bibliothekswissenschaft\emph{en} lesen
können. Diejenigen, welche ihn gebrauchen, sind der Meinung, daß bei der
wissenschaftlichen Behandlung des Bibliothekswesens verschiedene
Wissenschaften Stoff liefern, wie z. B. Paläographie,
Literaturgeschichte, Sprachwissenschaft, Architekturwissenschaft,
Rechtswissenschaft usw. Dies ist ja bis zu einem gewissen Grade richtig.

Aber es werden bei der Erhebung des Bibliothekswesens zu einer dieses
wissenschaftlich erfassenden Betrachtung doch nur einzelne Teile jener
Wissenschaften übernommen, dann aber in ihrer ganz besonderen Beziehung
zu bibliothekarischen Dingen umgearbeitet und schließlich alle
zusammengefaßt, so daß aus dem Vielerlei und der Mannigfaltigkeit eine
Einheit wird. Diese jedoch dürfen wir dann mit voller Berechtigung
Bibliothekswissenschaft nennen, wie ich es denn auch tue. So halte ich
den Ausdruck \enquote{Bibliothekswissenschaften} für überflüssig.

Bibliothekswissenschaft behandelt in allen ihren Teilen zunächst das
Bibliothekswesen unseres deutschen Vaterlandes, dieses genommen, soweit
die deutsche Zunge klingt, soweit auf diesem Gebiete Bücher gesammelt
werden. Darüber hinaus verfolgt sie aber die bibliothekarischen
Verhältnisse in allen Kulturländern der Welt, gibt sich also über den
nationalen Inhalt hinaus einen internationalen.

Damit arbeitet sie wieder der nationalen bibliothekarischen Leistung
anderer Länder und Völker in die Hand.

Bibliothekswissenschaft (wie ich sie betrieben wissen möchte) zerfällt
in vier große Teile (man kann die Einteilung auch anders treffen):

\begin{enumerate}
\def\labelenumi{\Roman{enumi}.}
\item
  Buchkunde.
\item
  Literaturkunde.
\item
  Die Lehre vom Bibliothekswesen der Vergangenheit (Geschichte des
  Bibliothekswesens).
\item
  Die Lehre vom Bibliothekswesen der Gegenwart, soweit es
  wissenschaftlich erfaßt werden kann.
\end{enumerate}

Die erste große Hauptabteilung der Bibliothekswissenschaft ist die
Buchkunde.

Ich scheide sie für die bibliothekswissenschaftliche Behandlung in zwei
große Unterabteilungen, wobei ich von vornherein bemerken möchte, daß
ich -- entgegen der Meinung von Anderen -- keiner von beiden (eben vom
Standpunkte der Bibliothekswissenschaft aus) irgend einen Vorzug vor der
andern einräumen möchte. Beide sollen vielmehr als völlig
gleichberechtigt im Bibliothekswesen gelten:

\begin{enumerate}
\def\labelenumi{\arabic{enumi}.}
\item
  Die Kenntnis des \emph{alten} Buches, des Buches der Vergangenheit;
\item
  Die Kenntnis des \emph{neuen} Buches, des Buches der Gegenwart.
\end{enumerate}

Die erste Unterabteilung umfaßt die Kenntnis vom Buch aller vergangenen
Zeiten.

Hier könnte man eine Einschränkung machen durch den Zusatz: soweit es in
Bibliotheken aufbewahrt wird. Dieser Zusatz mag infolge praktischer
Äußerlichkeiten Manchem als angebracht erscheinen. Das Buch der Assyrer
und Babylonier z. B., das auf Tontafeln, Tonzylindern usw. überliefert
ist, wird meist in Museen aufbewahrt, und die Kenntnis von ihm wird uns
durch die Assyriologie überliefert und erläutert.

Nichtsdestoweniger gehört die Kunde von ihm zur Bibliothekswissenschaft,
und diese hat eben aus der Assyriologie alles herüberzuholen, was das
assyrische und babylonische Buch betrifft

Ähnlich steht es mit dem älteren ägyptischen Buch, soweit es in der
Schrift der Hieroglyphen und demotisch geschrieben ist. Die Kenntnis
dieser Schriftarten wird man dem Bibliothekar nicht zumuten; seine
Stellung diesen Spezialitäten gegenüber wird sich darauf beschränken
können, die Ergebnisse der Ägyptologie in Bezug auf das Gebiet in sein
Wissensgebiet herüberzuholen.

Und so steht er auch späteren Spezialgebieten und besonders der
orientalischen und exotischen Bucherzeugung gegenüber. Mexikanische
Bilderhandschriften oder alte chinesische Holztafeldrucke lassen wir uns
durch die entsprechenden Spezialwissenschaften erklären, und wir nehmen
von dort alles in unsere Bibliothekswissenschaft herüber, was geeignet
ist, die Kenntnis vom Buch jeder Art uns zu überliefern und insbesondere
buchtechnische und buchkünstlerische Zusammenhänge begreifen zu lassen
und zu erläutern oder uns Aufschlüsse über die Geschichte der Schrift zu
geben.

Zur Bibliothekswissenschaft gehören alle diese Dinge. Eine Frage ist
nur, ob es einem Einzelnen möglich sein wird, sie sich anzueignen oder
sie zu beherrschen, bzw. wie weit das Jemand können wird.

Wir dürfen jedoch folgende Behauptung aufstellen:

Alle buchkundlichen Kenntnisse, die sich auf unserer höheren Bildung,
insbesondere der humanistischen, aufbauen lassen, sollte der
wissenschaftliche Bibliothekar sich aneignen.

Kein Bibliothekar sollte insbesondere eine Kenntnis oder Fertigkeit, zu
der bei ihm in der Mittelschule der Grund gelegt wurde oder die er auf
der Hochschule sich anzueignen Gelegenheit hatte, verfallen lassen, wie
es so häufig vorkommt, sondern sie stets im Zusammenhänge mit seiner
bibliothekarischen Berufstätigkeit weiterpflegen.

Wie häufig sind Bibliotheksvorstände gar nicht richtig unterrichtet über
die Kenntnisse, die ihre Untergebenen besitzen. Wenn es auch auf der
einen Seite Aufgabe der Vorgesetzten sein mag, sich auf irgend eine
Weise zu vergewissern, über welche Talente und Fähigkeiten ihre Beamten
verfügen, so ist es doch umgekehrt Pflicht der Beamten -- schon in ihrem
eigensten Interesse -- ihren Vorgesetzten Behörden mitzuteilen,
anzudeuten oder merken zu lassen, was sie können, damit sie darnach
verwendet werden.

Auf der sprachlichen Grundlage, welche unsere Mittelschulen uns
mitgeben, ist dem wissenschaftlichen Bibliothekar schon weitgehende
Buchkunde möglich. Das ganze ehemalige Gebiet des griechischen Buches
vermag er zu übersehen (unsere Vorbedingungen für den höheren
Bibliotheksdienst verlangen mit Recht von denjenigen, welche Griechisch
nicht können, das Nachlernen dieser Sprache und die Ablegung einer
Prüfung daraus). Vom griechischen Kulturkreis abgesehen ist dem
akademisch gebildeten Bibliothekar das große Gebiet des lateinischen
Buches und des Buches der lateinischen Tochtersprachen beherrschbar.

Von der Vorgeschichte der deutschen Sprache bringt er auch so viel
Kenntnisse mit, daß er das alte deutsche Buch würdigen kann.

Wenn auch \textsc{Harnack} der Meinung war, ein Bibliothekar brauche
sich nicht um Handschriftenkunde zu kümmern, nur eine gewisse Anzahl von
Bibliothekaren müsse sie gründlich kennen, so bin ich anderer Meinung.

Gewiß, Volksbibliothekare und solche an Bildungsbibliotheken brauchen
nichts davon zu wissen.

Aber es sollte keinen Bibliothekar an einer wissenschaftlichen
Bibliothek geben, der nicht wenigstens einen übersichtlichen Kurs über
Handschriftenkunde mitgemacht hat.

Handschriftenkunde ist eben die geschichtliche Grundlage der
wissenschaftlichen Buchkunde, und diese letztere sollte jeder
wissenschaftliche Bibliothekar im allgemeinen beherrschen.

Jede größere und kleinere wissenschaftliche Bibliothek besitzt heute
mehr oder minder umfangreiche Handschriftensammlungen. An den größeren
Bibliotheken gibt es eigene Handschriftenabteilungen, und es wird darauf
gesehen, daß in diesen Beamte amtieren, die von Handschriftenkunde etwas
verstehen.

Schon bei diesen Handschriftenabteilungen liegt zur Zeit noch vieles im
Argen, was die Kenntnisse der Beamten anlangt, und es wird zu wenig Zeit
vorgesehen, um einen richtig geschulten und mit wirklich notwendigen
Kenntnissen versehenen Nachwuchs heranzubilden.

Wie steht es aber in dieser Hinsicht an kleineren wissenschaftlichen
Bibliotheken? Oft tieftraurig und jammervoll. Es gibt solche, in denen
der Vorstand zittert, wenn einmal eine Anfrage über eine zur dortigen
Sammlung gehörige Handschrift einläuft.

Neben der Handschriftenkunde sind dem Bibliothekar notwendig Kenntnisse
über die alte Druckgeschichte, die Geschichte der Erfindung der
Buchdruckerkunst und der damit zusammenhängenden Fragen, besonders jener
über die Blockbücher; es folgt dann Inkunabelkunde, Kenntnis vom Wert
und der Bedeutung besonderer Stücke der älteren Literatur und
schließlich Vertrautheit mit den Gepflogenheiten des Altbuchhandels, des
sog. Antiquariates. Gerade hier eignet sich vieles hervorragend zur
mündlichen Überlieferung.

Zum Hauptinhalt der modernen Bibliothekswissenschaft gehört dann die
Buchkunde der Gegenwart, die Kenntnis vom Buchwesen der Gegenwart.

Hier muß ich eine Einschränkung machen, indem ich sage: Die Kenntnis vom
Buchwesen der Gegenwart, soweit es mit dem Bibliothekswesen
zusammenhängt.

Denn Buchkunde hat ein anderes Gesicht, je nachdem sie dem
Buch\emph{drucker}, dem Buch\emph{verleger}, dem Buch\emph{händler}, dem
Buch\emph{liebhaber} und schließlich dem Bibliothekar gegenübertritt.

Von gewisser Seite aus, namentlich von Kreisen des Buchdruckes und des
Buchhandels her, sind Bestrebungen im Gange, um die Buchkunde in den
Kreis der Gegenstände zu bringen, welche an den Universitäten behandelt
werden.

Da die Bibliothekswissenschaft ihre Grundlagen in der Bibliothek hat und
diese wieder aus einzelnen Büchern sich zusammensetzt, ist es
selbstverständlich, daß Alles, was unter den Begriff Buchkunde fällt,
denen geläufig sein muß, welche das Bibliothekswesen betreiben, und
damit erst recht denen, welche die Bibliothekswissenschaft pflegen.

Wenn man Buchkunde im weitesten Sinne als die Kenntnis vom Buch, vom
geschriebenen und gedruckten Buch und allem, was mit ihm zusammenhängt,
bezeichnet hat, so ist es klar, daß diese Buchkunde ein \emph{Teil} der
Bibliothekswissenschaft ist und demgemäß in deren Rahmen zu behandeln
ist.

Die Interessen des Bibliothekars der Buchkunde gegenüber gehen aber
weiter als die des Buchdruckers und des Buchhändlers, nehmen das Buch
von höheren Gesichtspunkten aus als jene.

Demgemäß ist die Buchkunde des Bibliothekars, wenn sie auch sachlich
sich auf die gleichen Tatsachen stützt wie die Buchkunde des
Buchdruckers und des Buchhändlers, doch anders zu behandeln als jene der
letzteren Interessenten. Insbesondere wird die Bibliothekswissenschaft
die geschichtlichen Teile der Buchkunde viel tiefgründiger und
eingehender vorzunehmen haben.

Und wer demnach Buchkunde an den Hochschulen behandelt wissen will, der
unterstütze alle Bestrebungen, welche geeignet sind, die
Bibliothekswissenschaft an den Hochschulen fördern zu helfen.

Ich meine Buchkunde hier immer vom höchsten möglichen Standpunkte aus
unter Zurückstellung der technischen Einzelheiten, deren allgemeine
Kenntnis aber doch auch zu Grunde liegen muß.

Insbesondere muß der Bibliothekar auch die kommerzielle Seite der
Buchkunde (wieder vom höchst möglichen Standpunkte aus) kennen lernen.

Hier handelt es sich z.B. darum: was muß der Bibliothekar vom Buchhandel
wissen? Durchaus nicht die gesamte Buchhandelslehre, aber doch einen
gewissen Überblick über diese, die sich zerteilt in die Lehre vom
Verlagsbuchhandel, vom Selbstverlag, von Buchgemeinschaften usw.
Vorauszuschicken und am besten hier einzufügen wäre ein kurzer Überblick
über die Geschichte des Buchhandels, wenn man das nicht schon bei der
alten Buchkunde, von der ich vorhin sprach, behandeln will.

Nach der Kenntnisnahme vom Wesen des Buchhandels an sich kommt eines der
schwierigsten, aber auch interessantesten und für das Bibliothekswesen
wichtigsten Kapitel, die Lehre von dem Verhältnis des Bibliothekswesens
zum Buchhandel.

Unter die Kenntnis vom Buchwesen der Gegenwart, soweit die
Bibliothekswissenschaft sie behandeln muß, fällt auch die Kenntnis vom
Zeitungs- und Zeitschriftenwesen, also ein Teil der neuerstandenen und
wie die Bibliothekswissenschaft in ihrer ersten Entwickelung begriffenen
Zeitungswissenschaft. Der wissenschaftliche Bibliothekar wird die
Aufgabe haben, die Leistungen der letzteren aufmerksam zu verfolgen und
aus ihr herüberzuholen, was er für sich brauchen kann. Ergebnisse von
drüben wird er für seine Zwecke zu bearbeiten haben.

Es ist ja von vornherein ein großer Unterschied zwischen
Zeitungswissenschaft und Bibliothekswissenschaft in ihrer Stellung der
Zeitung gegenüber. Die erstere schenkt ihre Aufmerksamkeit vor allem dem
Entstehen der Zeitung, die Bibliothekswissenschaft aber hat es mit dem
fertigen Zeitungsblatt als Sammlungsobjekt zu tun.

Beiden Wissenschaften gemeinsam ist höchstens die Kenntnis von
Bedeutung, Einfluß und Wirkung der Zeitungen und Zeitschriften.

Notwendig wird also auf gewissen Gebieten, von denen ich eben gesprochen
habe, ein Zusammenarbeiten dieser Wissenschaften sein.

Von Seite der Bibliothekswissenschaft wird hier zur Zeit kaum etwas
versäumt, wohl aber begehen einzelne Vertreter der Zeitungswissenschaft
Fehler darin, daß sie die Bedeutung der Sammeltätigkeit der
Bibliotheken, die schon in frühen Zeiten die hier vorhandenen Aufgaben
erkannt und an einzelnen Stellen musterhaft durchgeführt haben, beinahe
übersehen zu können glauben, wodurch sie sich ins Unrecht setzen und
ihre eigene Grundlage schädigen.

Neben die große I. Abteilung der Bibliothekswissenschaft, die Buchkunde,
die sich mit dem Buch als Objekt beschäftigte, tritt nunmehr die II.
große Abteilung der für den Bibliothekar notwendigen Kenntnisse, die
Literaturkunde, die von dem Inhalt der Bücher ausgeht. Hat unsere I.
Abteilung die Kenntnis vom Wesen des einzelnen Buches vermittelt, so
will die II. Abteilung, die Literaturkunde, über die inhaltlich
zusammengehörigen Bücher unterrichten.

Von der Mittelschule und der Hochschule bringt auf diesem Gebiete Jeder,
welcher sich dem Bibliothekarberufe widmet, ausgedehnte Kenntnisse mit.
Die Bibliothekswissenschaft hat nur dafür zu sorgen, daß diese weiter
ausgebaut werden. Sie sagt zunächst dem Einzelnen, daß er ein um so
brauchbarerer Bibliothekar sein wird, je mehr er Bücher kennen gelernt
haben wird, je bessere Auskunft er über diese zu geben imstande sein
wird, überhaupt ein je kenntnisreicherer Mensch er ist.

Hängt diese Eigenschaft im Bibliothekswesen aber besonders von einem
guten Gedächtnis ab, so wird der bibliothekarische Betrieb doch nicht
allein auf dieses und seine Leistungen sich verlassen, sondern er hat
sich für die bibliothekarische Literaturkunde mechanische Hilfsmittel
geschaffen in den Bücherverzeichnissen jeglicher Art und besonders in
den über einzelne Wissensfächer systematisch angelegten
Bücherverzeichnissen, die man als Bibliographien bezeichnet.

Im bibliothekarischen Betrieb kann mit ihrer Hilfe gewissermaßen auf das
Gedächtnis des Bibliothekars in bezug auf die Büchertitel der einzelnen
Fächer verzichtet werden; es wird von ihm nun aber andererseits
verlangt, daß er die in der Gegenwart auf eine beträchtliche Zahl
angewachsenen Bibliographien kennt.

Diese Hilfsmittel werden in der Gegenwart immer mehr vollendet. Schon
gibt es eine Bibliographie der Bibliographien. Und der schon ziemlich
alte Spruch: Ein Bibliothekar braucht keinen Büchertitel zu merken, aber
er muß wissen, \emph{wo} er ihn findet, wird immer wahrer.

Nichtsdestoweniger wird ein tüchtiger und richtiger Bibliothekar seinen
Stolz darein setzen, sowohl möglichst viele Büchertitel im Kopfe zu
haben, als auch über den Inhalt von möglichst vielen Büchern Auskunft
geben zu können. Er wird sich also zum lebendigen Gefäß
bibliothekarischer Literaturkunde ausbilden, so daß man ihm nicht den
Spruch entgegenzuhalten braucht, den ein Freund von mir -- oft
allerdings paradox -- zu verwenden pflegte: \enquote{Mehr lesen!} Auch
muß im Kopfe des Bibliothekars die Literaturkunde \emph{geordnet} sich
vorfinden, damit er nicht Gefahr läuft, wie der alte Professor
\textsc{Johann Nepomuk Sepp} als \enquote{umgestürzter Bücherkasten}
bezeichnet zu werden.

Neben der Kenntnis der Bibliographien ist von dem wissenschaftlichen
Bibliothekar zu verlangen, daß er über den Inhalt und die Abgrenzung der
einzelnen Wissenschaften unterrichtet ist insbesondere auch über die
wissenschaftlichen Vereinigungen und Organisationen sowohl des eigenen
Landes, wie auch wenigstens über die bedeutendsten des Auslandes.

Die III. sehr umfangreiche Abteilung der Bibliothekswissenschaft ist der
Geschichte des Bibliothekswesens gewidmet. Hierüber Einzelheiten zu
sagen, ist in diesem Rahmen wohl unnötig.

Der IV. und letzte Teil der Bibliothekswissenschaft umfaßt die
wissenschaftliche Behandlung des Bibliothekswesens der Gegenwart. Er
setzt sich aus vier Unterabteilungen zusammen, die behandeln sollen:

\begin{enumerate}
\def\labelenumi{\arabic{enumi}.}
\item
  Alle Fragen, die sich mit dem Wesen von Bibliotheken im allgemeinen,
  sowie mit der Übersicht, Einteilung und Abgrenzung der Bibliotheken
  befassen.
\item
  Die Lehre von den Bibliotheksbauten.
\item
  Die Lehre von den Bibliotheksbeständen, und zwar

  \begin{enumerate}
  \def\labelenumii{\alph{enumii})}
  \item
    von ihrer Erwerbung,
  \item
    ihrer Aufstellung, Ordnung, Einteilung,
  \item
    ihrer Katalogisierung (wobei Sprach- und Transkriptionskunde
    einschlägig ist),
  \item
    ihrer Schützung, Erhaltung und Benützung.
  \end{enumerate}
\item
  Die Lehre von der Bibliotheksverwaltung, darunter

  \begin{enumerate}
  \def\labelenumii{\alph{enumii})}
  \item
    Finanzfragen,
  \item
    Personalfragen,
  \item
    Statistik usw.
  \end{enumerate}
\end{enumerate}

Zu diesem IV. Hauptteil und seinen Unterabteilungen nur noch einige
kurze Bemerkungen.

Bibliothekswissenschaft beschäftigt sich mit \emph{allen} Arten von
Bibliotheken der Gegenwart. Sie zieht in den Kreis ihrer Betrachtung
also nicht nur die wissenschaftliche Bibliothek, sondern auch die
Bildungsbibliothek und die Volksbibliothek, andererseits nicht nur die
öffentliche Bibliothek, sondern auch die Privatbibliothek, die
Bibliophilensammlung, die Fachbibliotheken, die Lesehallen usw. bis
herab zur Kinderlesehalle.

Der Beamte der wissenschaftlichen Bibliothek soll alle Fragen der
Bibliothekswissenschaft kennen. Den Beamten der Volksbibliothek brauchen
in der Regel nur die Fragen zu interessieren, welche eben die
Volksbüchereien betreffen.

Von der größten Wichtigkeit ist das Zusammengehen der
Bibliothekswissenschaft mit der Architekturwissenschaft auf dem Gebiete
der Lehre von den Bibliotheksbauten. Schon für die Geschichte der
Bibliotheksbauten kann die Bibliothekswissenschaft die Mitarbeit der
Architekturgeschichte, die an den technischen Hochschulen zur Zeit in
einem erfreulichen Aufblühen sich befindet, nicht entbehren. Bei der
Betrachtung der Lehre vom modernen Bibliotheksbau aber müssen
Bibliothekar und Architekt einträchtig zusammenarbeiten, wie das die
grundlegende Forderung für die Herstellung der Bibliotheksbauten selbst
ist. Wie viel Unheil wäre vermieden worden, wenn diese Forderung stets
beachtet worden wäre.

Über die Wichtigkeit der Bibliotheksverwaltungslehre brauche ich wohl
kein Wort zu verlieren.

Die Überlieferung der Erfahrungen auf diesem bibliothekarischen Gebiet
immer wieder an das kommende Geschlecht dünkt mich die wichtigste
Aufgabe der Bibliothekswissenschaft zu sein.

Hier erstehen ihr unendliche Aufgaben. Auf die Einzelheiten hier
einzugehen, ist in dieser Stunde unmöglich.

Jedenfalls: die Bibliothekswissenschaft ist vorhanden und sie
marschiert. Von hervorragender Bedeutung ist, daß sie ihr
Veröffentlichungsorgan hat. Eigentlich ist es ja schon vorhanden, leider
aber zeugt es gegen sie durch seinen Titel. Ich bin der festen Meinung,
daß aus unserem \enquote{Zentralblatt für Bibliothekswesen} eines
schönen Tages notwendigerweise ein \enquote{Zentralblatt für
Bibliothekswissenschaft} werden muß. Es braucht dazu ja kaum eine innere
Umänderung vorzunehmen. \textsc{Dziatzko}s »Sammlung
bibliotheks\emph{wissenschaftlicher} Arbeiten" hat längst den Weg
gewiesen.

Fassen wir zusammen, so erkennen wir, daß die Aufgaben der
Bibliothekswissenschaft schon so ungeheuer groß sind, daß ihre
gleichmäßige Bearbeitung noch lange nicht möglich sein wird. Aber es ist
schon viel gewonnen, wenn der Tummelplatz der jungen Wissenschaft
zunächst einmal abgesteckt und diese Abgrenzung anerkannt wird. Die
Hauptsache ist, daß man Bibliothekswissenschaft als eine Einheit gelten
läßt. Wie viel der Einzelne von ihr alsdann zu bearbeiten und zu
beherrschen imstande sein wird, hängt von seinen Kräften ab. Immer aber
muß -- so hat \textsc{Harnack} einmal sehr richtig bemerkt -- das Ganze
aus dem Teil, der bearbeitet und vorgetragen wird, hervorleuchten.

Sehr verehrte Herren Kollegen! Dem deutschen Bibliothekar haftet seit
langem ein \emph{Grundfehler} an: er ist immer viel zu
\emph{bescheiden}.

Er sagt immer nur: \enquote{Aliis inserviendo consumor} und verzehrt
sich tatsächlich im steten Dienste für Andere. Er hat gar keine Zeit, um
recht an \emph{sich} zu denken. Er vergißt gewöhnlich, daß er
Kulturleistungen vollbringt, für die er ganz andere Aufmerksamkeit
verlangen darf, als er bisher getan hat, für die er auch anderen Lohn
fordern muß als den ihm bisher meist zugemessenen.

Er muß mehr Zeit gewinnen, um über sich und seinen Beruf nachzudenken
und aus diesem Nachdenken Entschlüsse für die Gestaltung des
Bibliothekswesens zu fassen. Das heißt: er muß Bibliothekswissenschaft
treiben.

Sehen Sie doch um sich, welche Ansprüche die Zeitungswissenschaft
erhebt, mit welch gewaltigem Lärm sie ihre Bestrebungen bekannt macht.
Nun will ich die Bibliothekswissenschaft durchaus nicht zur bloßen
Nachahmung dieses Vorgehens aufforde{[}r{]}n, aber etwas mehr muß der
wissenschaftliche Bibliothekar doch die allgemeine Aufmerksamkeit auf
sich und seine Leistungen lenken, als es bisher geschehen ist. Und
hierzu kann die Bibliothekswissenschaft zweifellos helfen.

Der wichtigste Erfolg aber, den die Pflege der Bibliothekswissenschaft
bringen muß, ist die höhere Erziehung unserer bibliothekarischen Jugend.
Wenn es sich um die Besetzung wichtiger Stellen handelt, dürfen nicht
mehr wirklich oder angeblich Leute fehlen, die dorthin berufen werden
können, die dann auch das sittliche \emph{Recht} haben und es verfolgen
können, dorthin zu gelangen.

Sie alle, hochverehrte Herren Kollegen, sind berufen, an dem Ausbau und
der Festigung der Bibliothekswissenschaft mitzuhelfen. Noch liegt das
Kindlein in der Wiege. Noch stehen wir in der Inkunabelzeit der
Bibliothekswissenschaft. \textsc{Fritz Milkau} ist der Name, der
zunächst führen wird. Sein Handbuch der Bibliothekswissenschaft ist im
Entstehen und wird eine erste Grundlage bilden. Helfen aber auch
außerdem Sie Alle mit, daß das Kindlein Bibliothekswissenschaft
großgezogen wird, wächst, blüht und gedeiht zum Besten unseres schönen
Berufes!

%autor
\begin{center}\rule{0.5\linewidth}{\linethickness}\end{center}

\textbf{Dr.~Georg Leidinger}, Leiter der Handschriftenabteilung der
Bayerischen Staatsbibliothek, später stellvertretender Generaldirektor
derselben Bibliothek. Ab 1922 Honorarprofessor für
Bibliothekswissenschaften (Universität Müchen). († 1945)

\end{document}
