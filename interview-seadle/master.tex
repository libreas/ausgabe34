\documentclass[a4paper,
fontsize=11pt,
%headings=small,
oneside,
numbers=noperiodatend,
parskip=half-,
bibliography=totoc,
final
]{scrartcl}

\usepackage{synttree}
\usepackage{graphicx}
\setkeys{Gin}{width=.4\textwidth} %default pics size

\graphicspath{{./plots/}}
\usepackage[ngerman]{babel}
\usepackage[T1]{fontenc}
%\usepackage{amsmath}
\usepackage[utf8x]{inputenc}
\usepackage [hyphens]{url}
\usepackage{booktabs} 
\usepackage[left=2.4cm,right=2.4cm,top=2.3cm,bottom=2cm,includeheadfoot]{geometry}
\usepackage{eurosym}
\usepackage{multirow}
\usepackage[ngerman]{varioref}
\setcapindent{1em}
\renewcommand{\labelitemi}{--}
\usepackage{paralist}
\usepackage{pdfpages}
\usepackage{lscape}
\usepackage{float}
\usepackage{acronym}
\usepackage{eurosym}
\usepackage[babel]{csquotes}
\usepackage{longtable,lscape}
\usepackage{mathpazo}
\usepackage[normalem]{ulem} %emphasize weiterhin kursiv
\usepackage[flushmargin,ragged]{footmisc} % left align footnote
\usepackage{ccicons} 

%%%% fancy LIBREAS URL color 
\usepackage{xcolor}
\definecolor{libreas}{RGB}{112,0,0}

\usepackage{listings}

\urlstyle{same}  % don't use monospace font for urls

\usepackage[fleqn]{amsmath}

%adjust fontsize for part

\usepackage{sectsty}
\partfont{\large}

%Das BibTeX-Zeichen mit \BibTeX setzen:
\def\symbol#1{\char #1\relax}
\def\bsl{{\tt\symbol{'134}}}
\def\BibTeX{{\rm B\kern-.05em{\sc i\kern-.025em b}\kern-.08em
    T\kern-.1667em\lower.7ex\hbox{E}\kern-.125emX}}

\usepackage{fancyhdr}
\fancyhf{}
\pagestyle{fancyplain}
\fancyhead[R]{\thepage}

% make sure bookmarks are created eventough sections are not numbered!
% uncommend if sections are numbered (bookmarks created by default)
\makeatletter
\renewcommand\@seccntformat[1]{}
\makeatother


\usepackage{hyperxmp}
\usepackage[colorlinks, linkcolor=black,citecolor=black, urlcolor=libreas,
breaklinks= true,bookmarks=true,bookmarksopen=true]{hyperref}
%URLs hart brechen
\makeatletter 
\g@addto@macro\UrlBreaks{ 
  \do\a\do\b\do\c\do\d\do\e\do\f\do\g\do\h\do\i\do\j 
  \do\k\do\l\do\m\do\n\do\o\do\p\do\q\do\r\do\s\do\t 
  \do\u\do\v\do\w\do\x\do\y\do\z\do\&\do\1\do\2\do\3 
  \do\4\do\5\do\6\do\7\do\8\do\9\do\0} 
% \def\do@url@hyp{\do\-} 
\makeatother 

%meta

%meta

\fancyhead[L]{Th. Roesnick, A. Erbe, M. Brauer\\ %author
LIBREAS. Library Ideas, 34 (2018). % journal, issue, volume.
\href{http://nbn-resolving.de/}
{}} % urn 
% recommended use
%\href{http://nbn-resolving.de/}{\color{black}{urn:nbn:de...}}
\fancyhead[R]{\thepage} %page number
\fancyfoot[L] {\ccLogo \ccAttribution\ \href{https://creativecommons.org/licenses/by/3.0/}{\color{black}Creative Commons BY 3.0}}  %licence
\fancyfoot[R] {ISSN: 1860-7950}

\title{\LARGE{Interview mit \\ Prof. Michael Seadle, PhD}}% title
\author{Thomas Roesnick (Interviewer) \\ Andreas Erbe \& Miriam Brauer (Transkription und Korrektur)} % author

\setcounter{page}{1}

\hypersetup{%
      pdftitle={Interview mit Prof. Michael Seadle, PhD},
      pdfauthor={Thomas Roesnick (Interviewer), Andreas Erbe \& Miriam Brauer (Transkription und Korrektur)},
      pdfcopyright={CC BY 3.0 Unported},
      pdfsubject={LIBREAS. Library Ideas, 34 (2018).},
      pdfkeywords={Bibliothekswissenschaft, Institut für Bibliotheks- und Informationswissenschaft, Humboldt-Universität zu Berlin, Projektseminar, Buchprojekt, Studium},
      pdflicenseurl={https://creativecommons.org/licenses/by/3.0/},
      pdfcontacturl={http://libreas.eu},
      baseurl={http://libreas.eu},
      pdflang={de},
      pdfmetalang={de}
     }



\date{}
\begin{document}

\maketitle
\thispagestyle{fancyplain} 

%abstracts

%body
\textbf{{Interview mit Michael Seadle, geführt von Thomas Roesnick
(13.07.2018)}}

\textbf{TR: Vielen Dank Herr Seadle, dass Sie sich eingefunden haben, um
mit mir ein Interview für das Projektseminar 90 Jahre IBI zu führen. Sie
sind in Detroit geboren und haben dann verschiedene Stationen im
bibliothekarischen Bereich durchlaufen: \emph{Northwestern University},
\emph{Cornell University}, und \emph{Michigan State University} und
haben dort verschiedene Tätigkeiten innegehabt. Welche unterschiedlichen
Sichtweisen gibt es denn in Deutschland und Amerika auf die Bibliotheks-
und Informationswissenschaft beziehungsweise wie ist Ihre Sicht auf
diese beiden Welten?}

MS: Es gibt ziemlich viele Unterschiede und viele Ähnlichkeiten. Wenn
man den Fokus auf die Unterschiede legt, und ich rede wirklich nur über
die wissenschaftlichen Bibliotheken, dann sind die amerikanischen
Bibliotheken viel, viel besser finanziert und ausgestattet. Wir reden
nicht über 10, 20, 50, 100 Mitarbeiter, sondern über 500 bis 1000 und
auch die technische Ausstattung ist manchmal anders. Als ich eine
Leitungsstelle an der \emph{Michigan State University} hatte, hatte ich
etwa 20 Leute nur im informationstechnischen Bereich. Die Dimensionen
sind da einfach anders. Heißt das, dass die Universitäten Bibliotheken
ernster nehmen? Ja, in einem Sinn schon. Bibliothekare sind nicht nur
Angestellte, sondern haben je nach Universität einen ähnlichen Status
wie Professoren. Das ist ein nicht unwichtiger Unterschied. Die
Unabhängigkeit der Bibliothekare ist auch Thema. Ich hatte viele
amerikanische Kollegen, die wirklich nie verstanden haben, dass man,
wenn man ein gemeinsames Projekt mit einem deutschen Bibliothekar
durchführen möchte, erst die Leitung fragen muss.

\textbf{TR: Sie haben dann 2006 die Stelle als Institutsdirektor hier am
IBI in Berlin angenommen. Wie kam es zu dieser Entscheidung?}

MS: Das fing nicht erst 2006 an. Peter Schirmbacher habe ich bereits
einige Jahre früher kennengelernt. Auch Elmar Mittler kannte ich schon
ziemlich lange. 2003 war ich Mitglied der Findungskommission, die der
Universität eine Empfehlung geben sollte, ob das Institut geschlossen
werden sollte oder nicht. Wir haben uns mit Studierenden und dem
Mittelbau getroffen und die Empfehlung ausgesprochen, das Institut zu
behalten, aber neu aufzubauen in der Richtung der \emph{University of
Michigan}, der \emph{University of Illinois} oder einer der damaligen
\emph{iSchools}. Als ich schon in der Kommission war, fing man an mich
zu fragen: \enquote{Wenn es eine Ausschreibung für die neue Professur
gäbe, würdest du Interesse daran haben?} Ich war davon überrascht, aber
ich sagte, ich würde nicht sofort Nein sagen. Und meine Frau sagte:
\enquote{Wir wollten immer nach Berlin kommen.} Ich habe Familie hier,
mehr in Berlin als in den Vereinigten Staaten.

\textbf{TR: Sie haben erwähnt, dass Sie sich an der Ausrichtung der
\emph{University of Michigan} orientiert haben. Wie hat sich denn die
inhaltliche Ausrichtung am IBI verändert, nachdem Sie dann angefangen
haben?}

MS: Erst muss ich sagen, die Universität hat etwas Ungewöhnliches getan:
Sie hatte mir Mittel gegeben, die niemand anderes hatte, das war die
Kontrolle über das Fernstudium. Das Fernstudium war damals nicht optimal
verwaltet, es gab zum Beispiel zwei Sekretärinnen. Ich habe einige
Mittel erst gespart und dann für den Aufbau des Instituts benutzt.
Gemeinsam mit Peter Schirmbacher haben wir neue Lehrveranstaltungen
konzipiert, die viel technischer waren. Es war auch die Übergangszeit
zwischen dem Magister und dem Bachelor/ Master. Wir haben eine
Internationalisierung eingeführt, Peter Schirmbacher mit dem Blick aus
der deutschen Perspektive, ich etwas mehr international. Wir haben uns
für die \emph{iSchools} beworben. Wenn wir das früher getan hätten, wäre
das etwas schwieriger geworden, weil wir nicht so viele Drittmittel
hatten und auch nicht so bekannt waren. Es war eine ziemlich schnelle
Entscheidung, dass wir eingeladen werden konnten. Dadurch sind wir die
erste europäische \emph{iSchool} geworden. Dieser Schritt Richtung
Internationalisierung, Richtung neue Mitarbeiter, war wichtig. Das war
der Anfang. Mein Ziel war es, junge Leute zu holen, Leute, die
internationale Beziehungen haben und Frauen.

\textbf{TR: Warum speziell Frauen?}

MS: Es ist ein weiblicher Beruf und wir hatten nur Männer. Das fand ich
schade. Ich freue mich, dass jetzt Vivien Petras und Elke Greifeneder
hier sind.

\textbf{TR: Sie waren nicht nur Institutsdirektor, sondern haben auch
gelehrt.} \textbf{Welche Lehrveranstaltung haben Sie denn besonders
gerne unterrichtet?}

MS: Ich habe so viele verschiedene Veranstaltungen unterrichtet. Die
Forschungsmethoden-Vorlesung ist eine Lehrveranstaltung, die ich sehr
lange und immer mit Freude gemacht habe. Als Elke Greifeneder als
Professorin aus Kopenhagen zurückkam, habe ich die Veranstaltung an sie
abgegeben, weil sie es hervorragend macht. Aber das war eine von meinen
Lieblingslehrveranstaltungen. Langzeitarchivierung, ein Gebiet auf dem
ich Forschung betreibe und zu dem ich immer noch ein DFG-Projekt habe,
ist auch ein Thema, das ich gerne gelehrt habe. Insgesamt habe ich eine
große Auswahl unterschiedlicher Lehrveranstaltungen gemacht. Man muss
einfach das finden, was interessant ist. Vor einigen Semestern habe ich
zum Beispiel die Medien-Vorlesung von Konrad Umlauf geerbt. Mit dem
Thema Medien hatte ich zuvor nicht viel zu tun. Aber Medien waren auch
ein Thema, als ich an der \emph{Michigan State University} tätig war.
Ich hatte dort persönlich nicht viel damit zu tun, aber ich wusste
Einiges davon. In der Lehre habe ich es dann völlig anders gemacht als
Konrad Umlauf, aber trotzdem konnte ich interessante Aspekte daran
finden.

\textbf{TR: Sie sind auch Historiker, haben promoviert in moderner neuer
deutscher Geschichte und haben sich auch als kunstaffin gezeigt. Mir ist
aufgefallen, dass Sie in den Lehrveranstaltungen gerne Anekdoten oder
auch persönliche Erfahrungen erzählen und es heißt immer, das Lernen
geschieht auch immer über Emotionen. Würden Sie dem zustimmen, dass man
gut über Geschichten lernt?}

MS: Ja, das ist etwas Amerikanisches, war aber auch der Stil meines
Vaters, der Professor war -- wenn auch nicht Amerikaner. Sein Gebiet war
deutsche Literatur und darin gibt es viele Geschichten
\emph{{[}lacht{]}}. Ich finde, Geschichten sind wirklich eine Methode,
mit der man die abstrakten Ideen der Wissenschaft konkret machen kann.
Wenn es konkret ist, die Leute ein Beispiel sehen, ein Beispiel
miterleben können, dann ist das wirklich effektiver, als einfach
theoretisch was zu erzählen, das alle auswendig können sollen, wenn sie
aufwachen \emph{{[}lacht{]}.}

\textbf{TR: Welche Projekte am Institut, die Sie gemacht haben, haben
Ihnen denn besonders gefallen oder sind Ihnen in Erinnerung geblieben?}

MS: \emph{Library Hi Tech} war ein Projekt, mit dem ich lange zu tun
hatte. Elke Greifeneder und ich haben dort lange gearbeitet und auch
Studierende involviert, damit sie Erfahrungen mit Peer-Reviewing sammeln
konnten. Der Aufbau des \emph{HEADT Centres} war ein völlig unerwartetes
Projekt. Der \emph{Senior Vice President} {[}Anmerkung des Interviewers:
von Elsevier{]} kam zu mir, saß in meinem Büro und sagte: \enquote{Wir
interessieren uns für eine Zusammenarbeit mit der Universität und wir
dachten vielleicht an ein Projekt, in dem Sie die Schnittstelle des
neuen Systems bewerten könnten}. Da sagte ich: \enquote{Interessiert
mich nicht. Wenn Sie mich finanzieren möchten, dann lieber etwas mit
wissenschaftlichem Fehlverhalten -- \emph{Research Integrity}} und seine
Augen wurden groß. \enquote{Oh, das ist für uns sehr interessant} und
aus diesem Gespräch ist das \emph{HEADT Centre} entstanden. Ich war
schon Vorsitzender der Kommission Wissenschaftliches Fehlverhalten.
Deshalb hatte ich etwas Erfahrung in diesem Bereich. Das passte auch gut
zu \emph{Library Hi Tech}, denn in \emph{Library Hi Tech} habe ich
natürlich Erfahrung mit Fehlverhalten gehabt, mit Plagiaten. Das spielte
eine Rolle und war ein Grund, warum ich zu dieser Kommission eingeladen
wurde. Also eines baut auf dem nächsten auf.

\textbf{TR: Sie haben ja schon viele Namen genannt, Peter Schirmbacher
oder auch Vivien Petras, Elke Greifeneder. Welche Kollegen und
Kolleginnen haben Sie in Ihrer Zeit am Institut denn besonders geprägt?
Gibt es da noch weitere, die Sie nennen könnten? Oder mit denen Sie
gerne zusammengearbeitet haben?}

MS: Stefan Gradmann. Er hatte bei uns eine befristete W2-Professur, was
ungewöhnlich ist. Und ich hatte schon Überlegungen, wie das geändert
werden könnte, sagte ihm aber vorsichtshalber, er solle auch andere
Angebote einholen, falls es nicht funktioniert. Das hat er getan und
eine Stelle in Belgien angeboten bekommen, die ihm besser bezahlt wurde
und wo er Professor und Bibliotheksdirektor sein konnte. Bevor wir eine
ordentliche Professur einrichten konnten, hat er die Stelle angenommen.
Es war spannend mit ihm, auch weil er so viele Sprachen sprechen konnte:
Flämisch, Französisch, Deutsch und Englisch. Er ist wirklich
sprachbegabt. Konrad Umlauf spielt eine ganz wichtige Rolle, besonders
im Fernstudium. Er war hervorragend effizient, ist immer noch
hervorragend effizient. Und Engelbert Plassmann, mit dem ich auch immer
engen Kontakt hatte. Meine Perspektive war für einige Jahre weniger
intern im IBI, sondern extern an der Universität. Und das war für das
IBI auch kein Nachteil, aber ich war für einige Zeit etwas ferner vom
täglichen Institutsleben.

\textbf{TR: Blicken wir mal auf die Gesamtzeit zurück, also die zwölf
Jahre von 2006 bis jetzt. Welche Veränderungen haben Sie in der Lehre am
Institut wahrgenommen? Sie haben die technische Ausrichtung schon
angesprochen.}

MS: Ja, also Elke Greifeneder hat etwas in ihrer Laudatio {[}Anmerkung
des Interviewers: Bei der offiziellen Verabschiedung von Prof.~Seadle{]}
gesagt: Wir machen nicht traditionell Vorlesungen und Seminare. Wir
haben einen Mischtyp, wo auch in den Vorlesungen viel gesprochen wird,
viel Interaktion mit Studierenden erfolgt. Die Seminare sind etwas
lockerer gestaltet. Es ist nicht so, dass ein Student kommt, einen
Vortrag hält und dann ein nächster kommt. Wir versuchen es mit
verschiedenen Methoden, mit Teamarbeit. Teamarbeit ist auch etwas, das
wir, Elke Greifeneder besonders und ich in der frühen Phase, eingeführt
haben. Ich habe vieles von ihr gelernt. Sie ist eine wirklich gute
Lehrerin. Inhaltlich haben wir auch viel mehr auf Englisch eingeführt,
weil wir diese enge Beziehung mit Kopenhagen haben. Viele deutsche
Studierende sagen \enquote{Oh, mein Englisch ist schlecht. Ich kann das
nicht.} Eigentlich stimmt das nicht. Ein großer Teil unserer deutschen
Studierenden können besser Englisch als die gewöhnlichen amerikanischen
\emph{Undergraduates} {[}Anmerkung des Interviewers: amerikanische
Bezeichnung für Studierende, die noch keinen Abschluss haben{]}. Sie
glauben es nur nicht.

\textbf{TR: Also die englische Sprache ist sehr wichtig im Bachelor und
im Master sowieso. Der Masterstudiengang wurde ja auch umbenannt, jetzt
heißt er nicht mehr Bibliotheks- und Informationswissenschaft.}

MS: Genau, das war ein heikles Thema, aber es ist geschehen. Ich habe es
unterstützt, aber es war eine Entscheidung von Vivien Petras, Robert
Jäschke und Elke Greifeneder.

\textbf{TR: Also das ist praktisch schon die Handschrift der neuen
Institutsleitung?}

MS: Es ist die neue Handschrift, ja. Es macht Sinn in unserem Fach. Die
wissenschaftlichen Fachzeitschriften sind, mit Ausnahme von
\emph{Bibliothek, Forschung und Praxis}, englischsprachig.\footnote{{[}Anmerkung
  der Redaktion: Das ist die Meinung des Interviewten, nicht die der
  Redaktion.{]}} Die Tagungen sind englischsprachig und wenn wir uns
international positionieren möchten, dann muss es Englisch sein oder
Mandarin. Aber Mandarin ist schwer.

\textbf{TR: Sind Sie denn in Ihrer Position als Direktor in den zwölf
Jahren auch auf Probleme gestoßen? Also gibt es da etwas, was Sie gerne
anders gemacht hätten?}

MS: Hm, sicher, aber es fällt mir jetzt nicht ein. Nichts, was ich hier
sagen dürfte \emph{{[}lacht{]}}.

\textbf{TR: Aber das ist ja schön, wenn es keine Probleme gab,
großartig.}

MS: Es gab kaum Probleme hier. Wir haben einen großen Vorteil als
Institut: Wir reden miteinander. Wie Elke Greifeneder gesagt hat: Die
Sekretärin, die Mitarbeiter im Mittelbau, die Professoren gehen zusammen
essen und wir besprechen da Sachen, machen einen Konsens, was wir tun
sollten. Studierende kommen auch ab und zu zum Mittagessen mit uns, das
ist nicht ausgeschlossen. Es passiert zwar nicht oft, aber es passiert
und alle sind willkommen. Diese Atmosphäre, dass wir nicht sehr formal
sind, dass wir hier offene Türen haben, dass wir auch für Gedanken von
anderen offen sind, finde ich sehr wichtig. Es war nicht genauso vor
meiner Zeit, aber Peter Schirmbacher und ich haben einen neuen Wind
eingeführt.

\textbf{TR: Ein familiärer Kreis, kann man schon fast sagen. Jetzt
blicken wir mal noch in die Zukunft. Sie sind jetzt als \emph{Senior
Researcher} tätig. Inwiefern bleiben Sie denn der Bibliotheks- und
Informationswissenschaft erhalten?}

MS: Ich bin immer noch hier. Formal führe ich noch zwei Projekte: Das
\emph{HEADT Centre} und mein DFG-Projekt zu National Hosting. Ich mache
nächstes Semester Lehre. Ich verschwinde nicht.

Man muss laut Gesetz mit 65 Jahren aufhören mit der Ausnahme, dass man
dreimal eine Hinausschiebung der Rente beantragen kann. Das habe ich
gerne gemacht, aber irgendwann kommt man gesetzlich an ein Ende. Aber
ich bin noch nicht bereit zu sterben. Mein Vater hat sein letztes Buch
mit 96 veröffentlicht. Eine Tätigkeit, die ich vielleicht von ihm geerbt
habe.

\textbf{TR: 96 Jahre wird das Institut noch nicht alt, wir feiern den
90. Geburtstag und deshalb die abschließende Frage: Was wünschen Sie dem
Institut zum 90. Geburtstag?}

MS: Weiterhin Erfolg. Wir sind in einer ungewöhnlichen Lage. Wir sind
wirklich klein, unglaublich klein im Vergleich zu Michigan, Illinois
oder ähnlichen \emph{iSchools} und ich bin nicht überzeugt, dass die
Mitglieder dieser \emph{iSchools} verstehen, wie viel kleiner wir sind,
weil unsere Leistung so viel größer ist und das nicht nur international.
Wir sind aktiv. Wir spielen innerhalb der Universität eine
außerordentlich große Rolle. Vivien Petras ist Vorsitzende der
Medienkommission, ich bin Vorsitzender der Kommission zur Überprüfung
von Vorwürfen wissenschaftlichen Fehlverhaltens. Zwei von einem
Institut, sehr ungewöhnlich. Und ich hoffe, dass das so bleibt. Es würde
mich überraschen, wenn Vivien Petras und Elke Greifeneder nicht
irgendwann Dekan werden. Ich denke, dass das für uns als Institut
wirklich gut ist, dass wir immer noch aktiv innerhalb unserer
Universität bleiben, die Sichtbarkeit erhalten. Das war eines der
ursprünglichen Probleme. Die Professoren damals waren eher nach innen
orientiert und nicht nach außen. Sie dachten, dass sie ihre eigene
Forschung machen und mehr nicht. Das geht heute nicht mehr. Es
funktionierte vielleicht an einem Punkt im 19. Jahrhundert, aber das ist
nicht mehr die Realität.

\textbf{TR: Also diese Sichtbarkeit, dass hier etwas geschieht, ist ganz
wichtig?}

MS: Man muss sichtbar sein und zwar positiv sichtbar. Also nicht nur
schimpfen und gegen Sachen sein, sondern wirklich aktiv etwas aufbauen.

\textbf{TR: Vielen Dank für das Interview.}

MS: Nichts zu danken.

%autor
\begin{center}\rule{0.5\linewidth}{\linethickness}\end{center}

\textbf{Thomas Roesnick} studiert aktuell im sechsten Bachelorsemester
„Bibliotheks- und Informationswissenschaft`` am Institut. Momentan
schreibt er an seiner Bachelorarbeit und würde danach gerne an der
Humboldt-Universität den Master in „Deutsche Literatur`` belegen.

\textbf{Miriam Brauer} studiert aktuell im vierten Bachelorsemester
Bibliotheks- und Informationswissenschaft am Institut. Zudem ist sie als
studentische Mitarbeiterin am Lehrstuhl für Information Processing and
Analytics tätig.

\end{document}
